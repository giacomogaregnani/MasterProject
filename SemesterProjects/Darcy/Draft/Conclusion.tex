\section{Conclusion}

We investigated the stochastic particle transport problem in the framework of underground flow. In particular, we analysed three numerical schemes that allow estimating the exit time of the solution of a general SDE from a bounded domain. Numerical experiments show that CEM seems to be the most appropriate choice in order to fulfill this purpose. In fact, estimating the probability of exit at each timestep using the Brownian bridge approach implies a small computational cost and improves dramatically the precision of DEM. On the other hand, an adaptivity procedure we analysed based on the position of the trajectory in the domain succeeds in restoring the weak order of the Euler-Maruyama scheme in an unbounded domain but does not imply a considerable advantage in computational time. Then, we managed to prove the properties of convergence of the analytic and numerical solution of a perturbed SDE to its non-perturbed solution. This is fundamental when an interpolation of the transport field is needed to increase computational speed. The convergence is confirmed by numerical results on a test case, where we succeeded in tuning the parameters of interpolation and time integration in order to have good performances. Finally, we applied the numerical schemes to the Darcy case, providing the details of a double Montecarlo simulation that can be exploited to estimate the exit time of a pollutant particle in the frame of the underground flow problem. Future improvements of the procedure explained in this work could be done employing Multi Level Montecarlo techniques for estimating the exit time in the Darcy case. Moreover, the modeling of extraction wells has not been taken into account, since we feel that the integration of the SDE in presence of singularities in the velocity field would have implied a deeper theoretical investigation and different numerical techniques.  
