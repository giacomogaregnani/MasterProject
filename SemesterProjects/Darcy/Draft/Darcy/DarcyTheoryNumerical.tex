\subsection{Analysis of numerical convergence}

We now consider the Euler-Maruyama method applied to the perturbed equation \eqref{eq:AppSDEPer}. We would like to find a balance between the error due to the numerical integration of the SDE with step size $h$ and the approximation of the transport field $f$ with $f^\epl$. In this way, knowing $\epl$ one can choose wisely the step size $h$ in order to avoid extra computational time.

\noindent Let us consider $X(t), X^\epl(t)$ the solution of \eqref{eq:AppSDE} and \eqref{eq:AppSDEPer} respectively, and let us denote with $X_n, X_n^\epl$ the numerical solution obtained with Euler-Maruyama method applied to the two equations at time $t_n = hn, t_N = T$. It is known that if $f$ satisfies the assumptions of Lemma \ref{lem:Lemma1}, and if $\sigma$ is a constant, then the strong error for $X_n$ approximating $X(t)$ is given by (\textit{e.g.}, \cite[Chapter 10]{Kloeden1992})
\begin{equation}\label{eq:StrongEM}
	\sup_{n = 0, \dots, N} \E \|X(nh) - X_n\| \leq C h,
\end{equation}
for a constant $C$ independent of $h$. It is possible to obtain a similar estimate for $X_n^\epl$ estimating $X(t)$. 
\begin{theorem}\label{thm:StrongConv} Let us consider \eqref{eq:AppSDE}, \eqref{eq:AppSDEPer} with $W_1 = W_2 = W$ almost everywhere. Given $h > 0$ such that $hN = T$ for $N \in \N, N > 0$, let us consider $X_n^\epl$ the numerical solution given by the Euler-Maruyama method applied to \eqref{eq:AppSDEPer}, \textit{i.e.},
\begin{equation*}
\left \{
\begin{aligned}
	X^\epl_{n+1} &= X^\epl_n + f^\epl(X^\epl_n)h + \sigma (W(t_{n+1}) - W(t_n)), && n = 0, \dots, N - 1, \\
	X^\epl_0 &= X_0.
\end{aligned} \right .
\end{equation*}
Then, if $f, f^\epl$ satisfy the assumptions of Lemma \ref{lem:Lemma1}
\begin{equation}\label{eq:StrongConvEpl}
	\sup_{n = 0, \dots, N} \E \|X(nh) - X^\epl_n\| \leq C h +  \|f^\epl - f\|_{\infty} \frac{e^{KT} - 1}{K}, 
\end{equation}
with $C$ a real constant independent of $h$ and depending only on the final time $T$ and the Lipschitz constant $K$ of $f$.
\end{theorem}

\begin{proof} We consider $X_n$ the numerical approximation of $X(t)$ obtained using Euler-Maruyama with the same initial condition $X_0$. If we add and subtract $X_n$ and apply the triangular inequality we obtain
\begin{equation*}
	\E \|X_n^\epl - X(nh)\| \leq \E \|X_n^\epl - X_n\| + \E \|X_n - X(nh)\|. \\
\end{equation*}
Since $f$ is regular enough and $\sigma$ is a constant, for the second term it is known that
\begin{equation}\label{eq:EMStrongProof}
	\sup_{n = 0, \dots, N} \E \|X_n - X(nh)\| \leq C h.
\end{equation}
We can then make a recursive analysis of the first term. For almost all $\omega$ in $\Omega$
\begin{equation*}
\begin{aligned}
	\|X^\epl_n - X_n\| &\leq \|X_{n-1}^\epl - X_{n-1}\| + h\|f^\epl(X^\epl_{n-1}) - f(X_{n-1})\|  \\
	&\leq \|X_{n-1}^\epl - X_{n-1}\| + h\|f^\epl(X^\epl_{n-1}) - f(X^\epl_{n-1})\| + h\|f(X^\epl_{n-1}) - f(X_{n-1})\| \\
	&\leq h  \|f^\epl - f\|_{\infty} + (1 + hK) \|X_{n-1}^\epl - X_{n-1}\| \\
	&\leq h  \|f^\epl - f\|_{\infty} + (1 + hK) [h  \|f^\epl - f\|_{\infty} + (1 + hK)\|X_{n-2}^\epl - X_{n-2}\|] \\
	&\leqtext{(\cdots)} h  \|f^\epl - f\|_{\infty} \sum_{i = 0}^{n-1} (1 + hK)^i + (1 + hK)^n \|X_0^\epl - X_0\|
\end{aligned}
\end{equation*}
Since $X_0^\epl = X_0$ and using the geometric series
\begin{equation*}
\begin{aligned}
	\|X^\epl_n - X_n\| &\leq h  \|f^\epl - f\|_{\infty} \frac{(1 + hK)^n - 1}{hK} \\
	&\leq  \|f^\epl - f\|_{\infty} \frac{(1 + hK)^N - 1}{K} \\
	&\leq  \|f^\epl - f\|_{\infty} \frac{e^{KT} - 1}{K},
\end{aligned}
\end{equation*}
where the last inequality is valid since $N = T/h$ and $K, T, h$ are all positive real numbers. Since the bound we found is independent of $\omega$ and $n$, we can take the expectation and the supremum, obtaining
\begin{equation}\label{eq:DiffXeplnXn}
	\sup_{n = 0, \dots, N} \E \|X^\epl_n - X_n\| \leq  \|f^\epl - f\|_{\infty} \frac{e^{KT} - 1}{K}.
\end{equation}
This result combined with \eqref{eq:EMStrongProof} concludes the proof.
\end{proof}
\noindent As far as the weak convergence is concerned, it is known that the Euler-Maruyama method is of weak order one \cite[Chapter 14]{Kloeden1992}. The term due to the perturbation of the transport field can be treated as for the strong error, therefore we get without any further assumption for a constant $C$ independent of $h$ 
\begin{equation*}
	\sup_{n = 0,\dots,N} \|\E(X^\epl_n - X(nh))\| \leq Ch + \|f^\epl - f\|_{\infty} \frac{e^{KT} - 1}{K}.
\end{equation*}
\begin{remark} \label{rmk:InterpNum} \normalfont{If $f^\epl$ is the interpolation of $f$ on a regular grid of size $\epl$, the result of Proposition \ref{thm:StrongConv} allows to balance the interpolation error and the error due to numerical integration. In fact, we reported in Remark \ref{rem:Remark2} some results on interpolation of $f$ with piecewise polynomials $f^\epl$, which we can plug in \eqref{eq:StrongConvEpl} as follows 
\begin{enumerate}
	\item if $f$ is Lipschitz continuous of constant $K$ or of class $\mathcal{C}^1$ and $f^\epl$ is a piecewise constant interpolation of $f$, then 
	\begin{equation*}
		\sup_{n = 0, \dots, N} \E \|X^\epl_n - X(nh)\| = O(h) + O(\epl).
	\end{equation*}
	\item if $f$ is of class $\mathcal{C}^2$ , and $f^\epl$ is a piecewise linear interpolation of $f$, then 
	\begin{equation*}
		\sup_{n = 0, \dots, N} \E \|X^\epl_n - X(nh)\| = O(h) + O(\epl^2).
	\end{equation*}
\end{enumerate}
Therefore, in the first case one should choose the step size $h$ to be of the same order of magnitude as $\epl$, while in the second case it should scale as $\epl^2$.}\end{remark}
