\subsubsection{Numerical experiment - Likelihood}

We consider the FitzHug-Nagumo problem defined in \eqref{eq:FitzNag} and the parameters fixed to the true value $\bar\theta = (0.2, 0.2, 3.0)$. We generate ten equispaced observations from initial time $t = 0$ to final time $T = 1$ adding a zero-mean Gaussian perturbation with variance equal to $10^{-2}$ to the two components of a numerical solution computed with a small time step. Then we generate a reference solution using the same small time step without noise in order to have an approximation of $\diffL(\mathcal{Y}|\bar \theta)$ with negligible error. Then we compute 300 realizations of $\diffL_h^M(\mathcal{Y}|\bar \theta)$ using Euler Forward or RK4 as a deterministic integrator with time steps $h$ in the set $h = 0.1\cdot 2^{-i}$ with $i = 0, \ldots, 11$, for Euler Forward and $h = 0.1\cdot2^{-i}$ for $i = 0, \ldots, 6$, for RK4. Hence, we estimate the bias of $\diffL(\mathcal{Y}|\bar \theta)$ with respect to the true value of the likelihood and its variance.

\begin{figure}
	\centering
	\resizebox{0.6\linewidth}{!}{% This file was created by matlab2tikz.
%
%The latest EFupdates can be retrieved from
%  http://www.mathworks.com/matlabcentral/fileexchange/22022-matlab2tikz-matlab2tikz
%where you can also make suggestions and rate matlab2tikz.
%
\definecolor{mycolor1}{rgb}{0.00000,0.44700,0.74100}%
\definecolor{mycolor2}{rgb}{0.85000,0.32500,0.09800}%
\definecolor{mycolor3}{rgb}{0.92900,0.69400,0.12500}%
\definecolor{mycolor4}{rgb}{0.49400,0.18400,0.55600}%
%
\begin{tikzpicture}

\begin{axis}[%
width=4.65in,
height=3.685in,
at={(0.78in,0.497in)},
scale only axis,
xmode=log,
xmin=1e-05,
xmax=0.2,
xminorticks=true,
xmajorgrids,
ymajorgrids,
xlabel={$h$},
xlabel style = {font=\Large},
ymode=log,
ymin=1e-25,
ymax=100000,
yminorticks=true,
mark size = 3,
axis background/.style={fill=white},
legend style={at={(0.03,0.03)},anchor=south west,legend cell align=left,align=left,draw=white!15!black},
ticklabel style={font=\Large},legend style={font=\Large},title style={font=\Large}
]
\addplot [color=mycolor1,solid,mark=o,mark options={solid}]
  table[row sep=crcr]{%
0.1	152.40285890697\\
0.05	15.0297723366959\\
0.025	1.622064894367\\
0.0125	0.245883345545411\\
0.00625	0.0419951733022408\\
0.003125	0.00863645199805523\\
0.0015625	0.00214939046093994\\
0.00078125	0.000473378166729877\\
0.000390625	0.00012171354847693\\
0.0001953125	3.61539589880665e-05\\
9.765625e-05	7.35843899567676e-06\\
4.8828125e-05	2.00603661030911e-06\\
};
\addlegendentry{$\Var(\diffL^h_M)$, EE};

\addplot [color=mycolor2,solid,mark=asterisk,mark options={solid}]
  table[row sep=crcr]{%
0.1	6.04599135651302e-06\\
0.05	3.40675451330916e-08\\
0.025	1.50790252227762e-10\\
0.0125	5.16324189950121e-13\\
0.00625	1.9467469819001e-15\\
0.003125	6.763689931673e-18\\
0.0015625	2.9811655337e-20\\
};
\addlegendentry{$\Var(\diffL^h_M)$, RK};

\addplot [color=mycolor3,solid,mark=+,mark options={solid}]
  table[row sep=crcr]{%
0.1	1489.94500365821\\
0.05	135.895613310291\\
0.025	17.7752309641098\\
0.0125	2.81110597557495\\
0.00625	0.531377491242425\\
0.003125	0.115688833138768\\
0.0015625	0.0281828582477648\\
0.00078125	0.00629035903035823\\
0.000390625	0.00165537979125842\\
0.0001953125	0.00037965972386549\\
9.765625e-05	8.69516182542916e-05\\
4.8828125e-05	2.37754855783276e-05\\
};
\addlegendentry{$\E[\diffL - \diffL^h_M]^2$, EE};

\addplot [color=mycolor4,solid,mark=triangle,mark options={solid,rotate=90}]
  table[row sep=crcr]{%
0.1	0.00019144049901202\\
0.05	5.82638410963337e-07\\
0.025	2.11531160318506e-09\\
0.0125	8.23470525936426e-12\\
0.00625	2.81351837106436e-14\\
0.003125	1.09523279390305e-16\\
0.0015625	4.14205891877e-19\\
};
\addlegendentry{$\E[\diffL - \diffL^h_M]^2$, RK};

\addplot [color=black,dashed,forget plot]
  table[row sep=crcr]{%
0.1	1\\
0.05	0.25\\
0.025	0.0625\\
0.0125	0.015625\\
0.00625	0.00390625\\
0.003125	0.0009765625\\
0.0015625	0.000244140625\\
0.00078125	6.103515625e-05\\
0.000390625	1.52587890625e-05\\
0.0001953125	3.814697265625e-06\\
9.765625e-05	9.5367431640625e-07\\
4.8828125e-05	2.38418579101563e-07\\
};
\addplot [color=black,solid,forget plot]
  table[row sep=crcr]
	\caption{Approximation of the likelihood at a fixed value $\bar \theta$.}
	\label{fig:MonteCarloVarianceH}
\end{figure}