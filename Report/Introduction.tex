\section{Introduction}

In recent years, probabilistic numerics has been a growingly relevant field within numerical mathematics. Many classic problems of numerical analysis, such as the approximation of Ordinary and Partial Differential Equations, have been recently reinterpreted from a probabilistic standpoint. In this work, we consider methods to perform Bayesian inference in the frame of parametric Ordinary Differential Equations (ODE's), as well as techniques used to give a probabilistic interpretation of the numerical solution of a differential equation. 

Firstly, we present a survey of some of the existing techniques to perform Bayesian inference. In particular, we focus on the Markov Chain Monte Carlo methods (MCMC), a class of algorithms whose variants allow to perform inference in a wide range of different contexts. 

Secondly, we consider a probabilistic numerical method for ODE's which has been recently introduced in \cite{CGS16}. In this work, we analyze its properties, presenting results of convergence showed in \cite{CGS16}, as well as considering the behavior of Monte Carlo estimators produced by the realizations of the solution. In particular, we show how the Monte Carlo approximation of the probability measure introduced by the probabilistic solver collapses to the punctual true solution of the ODE with respect to the time step chosen for the numerical integration. This aspect has only been partially analyzed before, and clarifying this convergence property theoretically and with numerical experiments is the main contribution of this work.

Finally, we show how it is possible to integrate the probabilistic solver of ODE's in a MCMC algorithm. In \cite{CGS16}, the authors show that using the probabilistic numerical method the estimation of the posterior distribution reflects better the uncertainty introduced by the numerical error. On the other hand, it is unclear whether the chosen number of trajectories influences the approximation of the posterior. Following the results on Monte Carlo estimations, we derive convergence rates for the posterior distribution, as well as for the Monte Carlo approximation of the parameter values obtained with MCMC.

The structure of this work is the following. In Section \ref{sec:ONE} we introduce basic notions of Bayesian inference and focus on the implementation of MCMC algorithms, taking into account some of the issues that could arise from the considered inferential problem. In Section \ref{sec:TWO}, we consider the probabilistic solver of ODE's and present its properties. Finally, we show in Section \ref{sec:THREE} how to integrate the probabilistic solver in an inferential context, thus drawing our final considerations in the conclusion.