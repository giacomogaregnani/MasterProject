\subsubsection{Weak convergence}

Another relevant feature of stochastic numerical methods is their weak convergence. The following analysis has been carried out in \cite{CGS16}, and we report the main steps in this work. Let us recall the definition of weak order of convergence
\begin{definition} Define the weak order of convergence. 
\end{definition}
\noindent For the numerical method introduced in \eqref{eq:probMethod}, a result of weak convergence can be proved using a technique of \textit{backward error analysis}. The main idea behind this technique is finding a \textit{modified equation} that the numerical method solves exactly or with a higher accuracy than the original equation. \\
Let us consider \eqref{eq:ODE} and the numerical method \eqref{eq:probMethod}. Using the Lie derivative notation, it is possible to find the differential operators $\mathcal{L}$ and $\diffL^h$ such that for all $\phi \in \mathcal{C}^\infty(\R^d, \R)$ 
\begin{equation}\label{LieNotation}
\begin{aligned}
	\phi(\Phi_h(u)) &= \left(e^{h\diffL} \phi\right)(u), \\
	\E\phi(U_1 | U_0 = u) &= \left(e^{h\diffL^h}\phi\right)(u).
\end{aligned}
\end{equation}
In particular, $\diffL = f \cdot \nabla$ and the explicit definition of $\diffL^h$ is not needed in this scope. \\
We now introduced a modified ODE
\begin{equation}\label{modifiedODE}
	\dv{\hat u}{t} = f^h(\hat u), 
\end{equation}
and a modified SDE
\begin{equation}\label{modifiedSDE}
	\dd{\tilde u} = f^h{\tilde u}\dd{t} + \sqrt{h^{2p} Q} \dd{W},
\end{equation}
where $p$ has been introduced in Assumption \ref{assumption_1}. We rewrite the solution of these equations in terms of Lie derivatives as for \eqref{LieNotation} introducing the differential operators $\hat \diffL$ and $\tilde \diffL$, \textit{i.e.},
\begin{equation}\label{LieNotationModif}
\begin{aligned}
	\phi\left(\hat u(h) | \hat u(0) = u\right) &= \left(e^{h\hat\diffL^h}\phi\right)(u), \\
	\phi\left(\tilde u(h)|\tilde u(0) = u\right) &= \left(e^{h\tilde\diffL^h}\phi\right)(u).
\end{aligned}
\end{equation}
Therefore,
\begin{equation}
\begin{aligned}
	\hat \diffL^h &= f^h \cdot \nabla, \\
	\tilde \diffL^h &= f^h \cdot \nabla + \frac{1}{2} h^{2p} Q : \nabla^2,
\end{aligned}
\end{equation}
where $\tilde \diffL^h$ is the \textit{generator} of \eqref{modifiedSDE}. (\ldots all the passages to get to (28) in \cite{CGS16} \ldots).
\begin{assumption} \label{assumption_3}The function $f$ in \eqref{eq:ODE} is in $\mathcal C^\infty$ and all its derivatives are uniformly bounded in $\R^n$. Furthermore, $f$ is such that for all functions $\phi$ in $\mathcal C ^\infty(\R^n, \R)$ 
\begin{equation}
\begin{aligned}
	\sup_{u\in\R^n} \left| e^{h\diffL} \phi(u) \right| &\leq (1 + Lh) \sup_{u\in\R^n} \left|\phi(u)\right|, \\
	\sup_{u\in\R^n} \left| e^{h\diffL^h} \phi(u) \right| &\leq (1 + Lh) \sup_{u\in\R^n} \left|\phi(u)\right|, \\
\end{aligned}
\end{equation}
for some $L > 0$.
\end{assumption}
\noindent We can now state the following result about weak convergence.
\begin{theorem}\label{thm:weakorder} Consider the numerical method \eqref{eq:probMethod} and Assumptions \ref{assumption_1}, \ref{assumption_2} and \ref{assumption_3}. Then for any function $\phi$ in $\mathcal{C}^\infty$ endowed with the properties of Assumption \ref{assumption_3},
\begin{equation}
	\left|\phi(u(T)) - \E^h\left(\phi(U_k)\right)\right| \leq Kh^{\min\{2p, q\}}, \quad kh = T,
\end{equation}
and 
\begin{equation}
\left|\E\phi(\tilde u(T)) - \E^h\left(\phi(U_k)\right)\right| \leq Kh^{2p + 1}, \quad kh = T,
\end{equation}
with $u$ and $\tilde u$ solutions of \eqref{eq:ODE} and \eqref{modifiedSDE}.
\end{theorem}
\begin{proof} proof in \cite{CGS16}.
\end{proof}

\subsubsection{Numerical verification of weak order}

Let us consider for \eqref{eq:ODE} the FitzHugh-Nagumo model, defined by
\begin{equation}\label{eq:FitzNag}
\begin{aligned}
	\dv{x_1}{t} &= c\left(x_1 - \frac{x_1^3}{3} + x_2\right), \\
	\dv{x_2}{t} &= -\frac{1}{c}(x_1 - a + bx_2),
\end{aligned}
\end{equation}
where $a, b, c \in \R$. In particular, we choose $a = 0.2, b = 0.2, c = 3$. We provide the system with the initial condition $x_1(0) = -1, x_2(0) = 1$. We integrate numerically this equation with \eqref{eq:probMethod}, using RK4 as $\Psi$. Therefore, Assumption \ref{assumption_2} holds with $q = 4$. Moreover, we consider $Q$ in Assumption \ref{assumption_1} to be $Q = \sigma I$ with $\sigma = 0.1$. We approximate the solution up to time $T = 10$ with $p$ in Assumption \ref{assumption_1} equal to $1, 1.5, 2, 2.5$  and $h$ vary in the range $0.25 / (2^i), i = 0, \ldots, 3$. We approximate $\E^h\left(\phi(U_k)\right)$ using a Monte Carlo simulation over $50000$ trajectories and compare it with the solution computed on a fine grid to obtain an estimation of the weak error. Results (Figure \ref{fig:weakorder}) show that the predicted order $\min\{2p, q\}$ applies in this example. In particular, since $q = 4$, it is possible to notice that no difference in order is detected between the cases $p = 2$ and $p = 2.5$. 

\begin{figure}
\centering
\resizebox{0.6\linewidth}{!}{% This file was created by matlab2tikz.
%
%The latest updates can be retrieved from
%  http://www.mathworks.com/matlabcentral/fileexchange/22022-matlab2tikz-matlab2tikz
%where you can also make suggestions and rate matlab2tikz.
%
\definecolor{mycolor1}{rgb}{0.00000,0.44700,0.74100}%
\definecolor{mycolor2}{rgb}{0.85000,0.32500,0.09800}%
\definecolor{mycolor3}{rgb}{0.92900,0.69400,0.12500}%
\definecolor{mycolor4}{rgb}{0.49400,0.18400,0.55600}%
%
\begin{tikzpicture}

\begin{axis}[%
width=4.717in,
height=3.721in,
at={(0.791in,0.502in)},
scale only axis,
xmode=log,
xmin=0.01,
xmax=1,
xminorticks=true,
xlabel={$h$},
ymode=log,
ymin=1e-06,
ymax=1,
yminorticks=true,
ylabel={error},
axis background/.style={fill=white},
legend style={at={(0.97,0.03)},anchor=south east,legend cell align=left,align=left,draw=white!15!black}
]
\addplot [color=mycolor1,solid,mark=o,mark options={solid}]
  table[row sep=crcr]{%
0.25	0.631246661321072\\
0.125	0.271954994254564\\
0.0625	0.0934646255275224\\
0.03125	0.0288825692070494\\
};
\addlegendentry{$p = 1$};

\addplot [color=mycolor2,solid,mark=+,mark options={solid}]
  table[row sep=crcr]{%
0.25	0.343449960636189\\
0.125	0.0562182182303921\\
0.0625	0.00706611887813939\\
0.03125	0.000811541742610916\\
};
\addlegendentry{$p = 1.5$};

\addplot [color=mycolor3,solid,mark=triangle,mark options={solid,rotate=90}]
  table[row sep=crcr]{%
0.25	0.160147596847408\\
0.125	0.00949041732485985\\
0.0625	0.000560514049779203\\
0.03125	8.000000000008e-06\\
};
\addlegendentry{$p = 2$};

\addplot [color=mycolor4,solid,mark=asterisk,mark options={solid}]
  table[row sep=crcr]{%
0.25	0.0855526336473634\\
0.125	0.00334580468646921\\
0.0625	0.00015008331019791\\
0.03125	2.06155281282234e-05\\
};
\addlegendentry{$p = 2.5$};

\addplot [color=black,dashed]
  table[row sep=crcr]{%
0.25	0.625\\
0.125	0.15625\\
0.0625	0.0390625\\
0.03125	0.009765625\\
};
\addlegendentry{slope 2};

\addplot [color=black,dashdotted]
  table[row sep=crcr]{%
0.25	0.15625\\
0.125	0.01953125\\
0.0625	0.00244140625\\
0.03125	0.00030517578125\\
};
\addlegendentry{slope 3};

\addplot [color=black,dotted]
  table[row sep=crcr]
\caption{Weak order of convergence of \eqref{eq:probMethod} applied to \eqref{eq:FitzNag}.}
\label{fig:weakorder}
\end{figure}