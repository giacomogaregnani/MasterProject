\documentclass{scrartcl}

\usepackage{vmargin}
\setmarginsrb{28mm}{25mm}{28mm}{25mm}{0pt}{0mm}{0pt}{0mm}
\setlength{\footskip}{20pt}
\usepackage{amssymb}
\usepackage{amsmath}
\usepackage{amsthm}
\usepackage{pgfplots} 
\usepackage{graphicx}
\usepackage[utf8x]{inputenc}
\usepackage{tikz}
\usepackage{bbm}
\usepackage{subcaption}
\usepackage[boxruled]{algorithm2e}
\usepackage{mathtools}
\usepackage{lipsum}
\usepackage[title,titletoc]{appendix}
\usepackage{array,booktabs}
\usepackage{here}
\usepackage[hidelinks]{hyperref}
\mathtoolsset{showonlyrefs}
\DeclarePairedDelimiter{\ceil}{\lceil}{\rceil}
\renewcommand{\phi}{\varphi}
\newcommand{\eqtext}[1]{\ensuremath{\stackrel{#1}{=}}}
\newcommand{\leqtext}[1]{\ensuremath{\stackrel{#1}{\leq}}}
\newtheorem{theorem}{Proposition}[section]
\newtheorem{lemma}{Lemma}[section]
\newtheorem{remark}{Remark}[section]
\newcommand{\N}{\mathbb{N}}
\newcommand{\R}{\mathbb{R}}
\newcommand{\E}{\mathbb{E}}
\newcommand{\epl}{\varepsilon}
\newcommand{\defeq}{\coloneqq}


\begin{document}
	
\section{Random time stepping}

Let us consider $f\colon\R^d\to\R^d$ and the ODE 
\begin{equation}
	y' = f(y), \quad y(0) = y_0 \in \R^d.
\end{equation}
Given a time step $h > 0$, the Explicit Euler method applied to this ODE reads
\begin{equation}
	y_{n+1} = y_n + h f(y_n).
\end{equation}
Let us consider the ODE to be chaotic, i.e., integrating the equation with different time steps leads to different numerical solutions. The idea is to choose at each time $t_n$ the time step as the realization of a random variable $H_n$ in order to quantify the uncertainty introduced by the numerical method. The following assumptions are needed in the following.
\begin{assumption}\label{as:assumptionH} The i.i.d. random variables $H_n$ satisfy 
	\begin{enumerate}
		\item $H_n > 0$ a.s.,
		\item there exists $h > 0$ such that $\E[H_n] = h$ for all $n = 0, 1, \ldots$,
		\item there exists a function $A\colon\R\to\R$ with $A(h) \to 0$ when $h \to 0$ such that the first three integer moments of the random variables $Z_n$ defined as
		\begin{equation}
			Z_n \defeq \frac{1}{A(h)}(H_n - h)
		\end{equation}
		satisfy
		\begin{equation}
			\E[Z_n] = 0, \quad \E[Z_n^2] = h, \quad \E[Z_n^3] = 0.
		\end{equation}
	\end{enumerate}
\end{assumption}
\noindent Integrating with variable time step $H_n$ gives the discrete stochastic process $\{Y_n\}_{n\geq 0}$ defined as
\begin{equation}\label{eq:numHSto}
	Y_{n+1} = Y_n + H_n f(Y_n).
\end{equation}
We can write equivalently $Y_n$ as 
\begin{equation}
\begin{aligned}
	Y_{n+1} &= Y_n + H_n f(Y_n) \\
	&= Y_n + hf(Y_n) + (H_n - h)f(Y_n) \\
	&= Y_n + hf(Y_n) + A(h)f(Y_n)Z_n,
\end{aligned}
\end{equation}
where $Z_n$ are the r.v. of Assumption \ref{as:assumptionH}. Let us consider $\tilde Y$ the solution of the SDE
\begin{equation}\label{eq:modifiedSDE}
	\dd \tilde Y = f(\tilde Y) \dd t + A(h) f(\tilde Y) \dd W(t), \quad \tilde Y(0) = y_0,
\end{equation}
where $W(t)$ for $t > 0$ is a standard one-dimensional Wiener process. Then, the following result of local weak convergence holds.
\begin{theorem} If the sequence $\{H_n\}_{n\geq 0}$ satisfies Assumption \ref{as:assumptionH}, then for any smooth function $\phi$ and $n = 1, 2, \ldots$, the numerical solution $Y_n$ given by \eqref{eq:numHSto} and the solution $\tilde Y$ of \eqref{eq:modifiedSDE} satisfy 
\begin{equation}\label{eq:weakOrder}
	\abs{\E[\phi(Y_n) - \phi(\tilde Y(nh))]} \leq Ch,
\end{equation}
where $C$ is a positive real constant.
\end{theorem}
\begin{example}\label{ex:uniformH} Given $h > 0$, let us consider the random variables $H_n$ to be drawn from an uniform distribution 
	\begin{equation}
		H_n \iid \mathcal{U}(0, 2h).
	\end{equation}
	Then, points 1. and 2. of Assumption \ref{as:assumptionH} are trivially verified. Point 3. is verified considering
	\begin{equation}
		A(h) = \sqrt{\frac{h}{3}},
	\end{equation} 
	as in this case the random variables $Z_n$ are distributed following
	\begin{equation}
		Z_n \iid \mathcal{U}(-\sqrt{3h}, \sqrt{3h}),
	\end{equation}
	and therefore the odd moments are trivially equal to zero and
	\begin{equation}	
		\E[Z_n^2] = \frac{1}{12}(2\sqrt{3h})^2 = h. 
	\end{equation}
\end{example}

\subsection{Numerical examples}
\subsubsection{Weak order}
\begin{figure}[t]
	\centering
	\resizebox{0.6\linewidth}{!}{% This file was created by matlab2tikz.
%
%The latest EFupdates can be retrieved from
%  http://www.mathworks.com/matlabcentral/fileexchange/22022-matlab2tikz-matlab2tikz
%where you can also make suggestions and rate matlab2tikz.
%
\definecolor{mycolor1}{rgb}{0.00000,0.44700,0.74100}%
%
\begin{tikzpicture}

\begin{axis}[%
width=4.717in,
height=3.721in,
at={(0.791in,0.502in)},
scale only axis,
xmode=log,
xmin=0.001,
xmax=1,
xminorticks=true,
xlabel={$h$},
xlabel style = {font = \Large},
xmajorgrids,
ymajorgrids,
yminorgrids,
ymode=log,
ymin=0.001,
ymax=1,
yminorticks=false,
axis background/.style={fill=white},
legend style={at={(0.97,0.03)},anchor=south east,legend cell align=left,align=left,draw=white!15!black},
ticklabel style={font=\Large},legend style={font=\Large},title style={font=\Large},
mark size = 3
]
\addplot [color=mycolor1,solid,mark=o,mark options={solid}]
  table[row sep=crcr]{%
0.5	0.486196033858052\\
0.25	0.317096456934064\\
0.125	0.186608048115359\\
0.0625	0.102288496105347\\
0.03125	0.0536560915748252\\
0.015625	0.0272008305415445\\
0.0078125	0.0138350975923238\\
0.00390625	0.0069150656036725\\
};
\addlegendentry{weak error};

\addplot [color=black,dashed]
  table[row sep=crcr]
	\caption{Weak order}
	\label{fig:WeakOrder}
\end{figure}
In this first example we consider the one-dimensional ODE
\begin{equation}
	f'(y) = \lambda y, \quad y(0) = 1. 
\end{equation}
with $\lambda = 1$, and final time $T = 1$. Let us remark that in practice we adapt the last time step so that the numerical solution is computed exactly up to time $T$. We consider the random variables $H_n$ and the function $A(h)$ given in example \ref{ex:uniformH}. The SDE \eqref{eq:modifiedSDE} then reads
\begin{equation}
	\tilde Y = \lambda \tilde Y \dd t + \sqrt{\frac{h}{3}} \lambda \tilde Y \dd W(t),
\end{equation}
and therefore the exact solution is given by
\begin{equation}
	\tilde Y(t) = \exp(\left(\lambda - \frac{h}{6}\lambda^2\right)t + \sqrt{\frac{h}{3}}\lambda W(t)).
\end{equation}
We consider $\phi$ in \eqref{eq:weakOrder} to be the identity function and compute the weak order for $h = 0.5 \cdot 2^{-i}$, with $i = 0, \ldots, 7$ with $M = 10^4$ trajectories of the numerical solution and $10^6$ realizations of the exact solution $\tilde Y(T)$. Results (Figure \ref{fig:WeakOrder}) show that the weak order of convergence one seems to be verified.

\subsubsection{Distribution of the solution}

We consider the Lorenz system 
\begin{equation}\label{eq:Lorenz}
\begin{aligned}
x' &= \sigma(y - x), \quad &&x(0) = -10,\\
y' &= x(\rho - z) - y, \quad &&y(0) = -1,\\
z' &= xy - \beta z, \quad &&z(0) = 40.
\end{aligned}
\end{equation}
with $\sigma = 10$, $\rho = 28$, $\beta = 8/3$, so that the system has chaotic behavior. Moreover, we consider final time $T = 100$ and the function $\phi(x) = x^Tx$. We consider $h = 0.01$ and $5\cdot 10^3$ trajectories of the following numerical solutions
\begin{itemize}
	\item ODE solver with random solver, time steps $H_n$ distributed as in example \ref{ex:uniformH},
	\item SDE \eqref{eq:modifiedSDE} with Euler-Maruyama and fixed time step $h$,
	\item probabilistic ODE solver \cite{CGS16} with fixed time step $h$.
\end{itemize}  
Thus, we consider the value of $\phi$ applied to the numerical solution at final time $T$. Empirical results (Figure \ref{fig:distributionPhi}) show that the distribution of $\phi(\tilde Y)$ and of $Y_N$ present the same behavior for the three methods above.

\begin{figure}[t]
	\begin{subfigure}{0.49\linewidth}
		\centering
		\resizebox{1.0\linewidth}{!}{% This file was created by matlab2tikz.
%
%The latest EFupdates can be retrieved from
%  http://www.mathworks.com/matlabcentral/fileexchange/22022-matlab2tikz-matlab2tikz
%where you can also make suggestions and rate matlab2tikz.
%
\definecolor{mycolor1}{rgb}{0.00000,0.44700,0.74100}%
%
\begin{tikzpicture}

\begin{axis}[%
title = {ODE, random time-stepping},
width=4.717in,
height=3.721in,
xtick = {0, 400, 800, 1200, 1600, 2000, 2400, 2800},
xlabel = {$\phi(Y)$},
xlabel style = {font = \LARGE},
at={(0.791in,0.502in)},
scale only axis,
xmin=0,
xmax=3000,
ymin=0,
ymax=0.0012,
axis background/.style={fill=white},
ticklabel style={font=\LARGE},legend style={font=\LARGE},title style={font=\LARGE}
]
\addplot[fill=mycolor1,fill opacity=0.5,draw=black,ybar interval,area legend] plot table[row sep=crcr] {%
x	y\\
0	0.0001\\
100	0.000372\\
200	0.000516\\
300	0.000618\\
400	0.000682\\
500	0.00074\\
600	0.000862\\
700	0.000986\\
800	0.001082\\
900	0.000966\\
1000	0.000638\\
1100	0.000502\\
1200	0.000376\\
1300	0.000294\\
1400	0.00024\\
1500	0.000194\\
1600	0.00018\\
1700	0.000136\\
1800	0.000114\\
1900	9.2e-05\\
2000	7.6e-05\\
2100	6.8e-05\\
2200	5e-05\\
2300	1.6e-05\\
2400	2.8e-05\\
2500	2.6e-05\\
2600	6e-06\\
2700	1.4e-05\\
2800	1e-05\\
2900	6e-06\\
3000	2e-06\\
3100	6e-06\\
3200	2e-06\\
3300	2e-06\\
};
\end{axis}
\end{tikzpicture}%}
	\end{subfigure}
	\begin{subfigure}{0.49\linewidth}
		\centering
		\resizebox{1.0\linewidth}{!}{% This file was created by matlab2tikz.
%
%The latest EFupdates can be retrieved from
%  http://www.mathworks.com/matlabcentral/fileexchange/22022-matlab2tikz-matlab2tikz
%where you can also make suggestions and rate matlab2tikz.
%
\definecolor{mycolor1}{rgb}{0.00000,0.44700,0.74100}%
%
\begin{tikzpicture}

\begin{axis}[%
title = {SDE, Euler-Maruyama},
width=4.717in,
height=3.721in,
at={(0.791in,0.502in)},
xtick = {0, 400, 800, 1200, 1600, 2000, 2400, 2800},
xlabel = {$\phi(Y)$},
xlabel style = {font = \LARGE},
scale only axis,
xmin=0,
xmax=3000,
ymin=0,
ymax=0.0012,
axis background/.style={fill=white},
ticklabel style={font=\LARGE},legend style={font=\LARGE},title style={font=\LARGE}
]
\addplot[fill=mycolor1,fill opacity=0.5,draw=black,ybar interval,area legend] plot table[row sep=crcr] 
	\end{subfigure}
	\begin{subfigure}{0.49\linewidth}
		\centering
		\resizebox{1.0\linewidth}{!}{% This file was created by matlab2tikz.
%
%The latest EFupdates can be retrieved from
%  http://www.mathworks.com/matlabcentral/fileexchange/22022-matlab2tikz-matlab2tikz
%where you can also make suggestions and rate matlab2tikz.
%
\definecolor{mycolor1}{rgb}{0.00000,0.44700,0.74100}%
%
\begin{tikzpicture}

\begin{axis}[%
title = {ODE, Probabilistic method \cite{CGS16}},
width=4.717in,
height=3.721in,
at={(0.791in,0.502in)},
xtick = {0, 400, 800, 1200, 1600, 2000, 2400, 2800},
xlabel = {$\phi(Y)$},
xlabel style = {font = \LARGE},
scale only axis,
xmin=0,
xmax=3000,
ymin=0,
ymax=0.0012,
axis background/.style={fill=white},
ticklabel style={font=\LARGE},legend style={font=\LARGE},title style={font=\LARGE}
]
\addplot[fill=mycolor1,fill opacity=0.5,draw=black,ybar interval,area legend] plot table[row sep=crcr] 
	\end{subfigure}
	\begin{subfigure}{0.49\linewidth}
		\centering
		\resizebox{1.0\linewidth}{!}{% This file was created by matlab2tikz.
%
%The latest EFupdates can be retrieved from
%  http://www.mathworks.com/matlabcentral/fileexchange/22022-matlab2tikz-matlab2tikz
%where you can also make suggestions and rate matlab2tikz.
%
\definecolor{mycolor1}{rgb}{0.00000,0.44700,0.74100}%
\definecolor{mycolor2}{rgb}{0.85000,0.32500,0.09800}%
\definecolor{mycolor3}{rgb}{0.92900,0.69400,0.12500}%
%
\begin{tikzpicture}

\begin{axis}[%
title = {Scatter plot of the solution at $t = T$},
width=4.717in,
height=3.721in,
at={(0.791in,0.502in)},
scale only axis,
xmin=-20,
xmax=20,
tick align=outside,
xlabel={$x$},
xlabel style = {font = \LARGE},
xmajorgrids,
ymin=-40,
ymax=40,
ylabel={$y$},
ylabel style = {font = \LARGE},
ymajorgrids,
zmin=0,
zmax=60,
zlabel={$z$},
zlabel style = {font = \LARGE},
zmajorgrids,
view={145.7}{27.6},
axis background/.style={fill=white},
axis x line*=bottom,
axis y line*=left,
axis z line*=left,
ticklabel style={font=\LARGE},legend style={font=\LARGE},title style={font=\LARGE}
]
\addplot3[only marks,mark=*,mark options={},mark size=0.5000pt,color=mycolor1] plot table[row sep=crcr]{%
-6.06583450242061	-9.54728673458569	17.1051179325151\\
-11.0422449212367	-18.9475065668965	17.8275433872171\\
-5.76481434863356	-4.9286417750943	25.0958679055087\\
8.09163713815784	8.33911344894302	26.183471632082\\
14.9079690245228	12.3432035240174	38.3424748071157\\
-5.97871967964217	-4.40604373817637	26.3293085013767\\
-5.86369367919328	-3.79462355372003	26.8604316456132\\
-8.70692964326059	-2.74485725030165	33.5368213165793\\
7.81763690312753	-4.14977384288575	37.1795300162655\\
12.2615690201751	17.989964903046	24.4374040546449\\
-2.21496146155809	-4.64961796530626	23.1279396904115\\
-11.3031940093092	-17.6636332977675	21.4675085064787\\
0.804692244328018	-4.92556402642826	28.1884794339753\\
8.86866994233043	13.4722067045694	20.0341172823127\\
-13.0226704638298	-8.57285091365721	37.2861969864022\\
-9.32949486871052	-11.9696378038516	24.3249411512977\\
8.76696270939643	10.1709945437147	25.4063687549472\\
-12.7289359362058	-22.0236831246174	19.1939891285237\\
-9.55285357463033	-16.2105550868762	16.9060706107275\\
13.9780583906861	8.26380327782081	39.6620186111441\\
-0.400216257612513	-2.24231043182149	22.1387702685109\\
9.45161197090837	10.3769895455173	27.0068262792969\\
0.483014287055022	-4.20275725859511	26.9185278831291\\
2.83337450513	4.78024965915613	21.4751367010147\\
2.15132221786654	3.13444752830655	15.9551351523657\\
-2.39036469615756	-4.21627690913789	11.4169715534697\\
-10.1772755128508	-9.99685789667384	29.3548911245385\\
-5.41683677107924	-9.92063035111935	11.0257570776427\\
-8.35143167339388	-13.3578221038197	18.0242156729834\\
3.54726796153132	0.840206458175113	25.8226102095468\\
13.6422031481633	6.54545234022446	40.2223031832675\\
9.10168161358742	3.59068241738003	33.5439286886183\\
-7.4926263770467	-7.03159540896021	26.4399514106407\\
9.62628859226231	3.14041191764604	34.8937922988319\\
-1.86220019376334	3.23179247831027	27.8371902485314\\
6.54644830712336	5.74576803440524	25.8602328546543\\
-14.5783255488531	-23.5206400530685	24.1495613853241\\
-8.12247080701012	-8.42075340179051	26.15097046155\\
-13.5399604312679	-12.1784193182059	35.2383156729119\\
15.3664709499752	9.42599451070667	41.8181927830745\\
3.79274057063277	5.75675639772296	18.7040586060135\\
6.58716858942197	9.10927530289544	27.4023760526186\\
-10.6841037251895	-8.67055185923327	32.0160400372715\\
10.7210121349377	13.9108482532245	25.6630326530801\\
7.87345473867682	7.48094938928522	26.7866735383804\\
4.7460060393257	6.92956296216336	19.6930748490638\\
8.85835532421375	8.53246783480898	27.8747961071461\\
14.0511633659849	17.5422064812791	30.729552189825\\
-11.0945325235301	2.97084688825016	41.7088437901633\\
-2.19440961826629	-6.11137331825144	25.828153432227\\
8.01928939488175	10.5972535368339	22.4040100605841\\
-8.68925697285511	-9.86425556311719	25.639230707964\\
12.360082272591	14.8402116371307	29.2452817285829\\
0.0811914931339175	1.52541865961599	21.2192579931717\\
-9.53540385999807	-10.0890671806391	27.6088560160805\\
-14.7766646127678	-23.0441531811132	25.6094500338105\\
0.196815546064429	0.0196146699455372	16.3706586815268\\
-7.63374078677209	-8.5712817843241	24.599479735954\\
-10.0294224299283	-9.15976122051464	29.963874001066\\
6.64910819626644	9.6677398829956	19.1847120416162\\
-20.2931825984369	-24.500651531302	41.4783013252564\\
-11.9634623949131	-7.94243036202208	35.5450295754768\\
-7.06208672352265	-8.97416251611996	22.1909429236988\\
-8.24278718353161	-8.37962599644173	26.5238484892703\\
-14.7518247624636	-18.8254319651968	31.2198914337972\\
-0.422287394551561	4.75228596428349	27.4825457183553\\
-8.00906126593601	-1.64296990995459	33.1623145235295\\
11.4289121589935	11.2453468159745	31.0362416958485\\
6.41196831450319	6.89421970080783	23.7893875818328\\
3.72543372883629	5.6266820123888	16.3231389939302\\
14.989789195574	21.4867700961929	28.5442813252076\\
0.161178614597427	-1.24888739450206	21.3618240452874\\
8.83402969282314	9.31825852222602	26.7914891486028\\
-2.79374247573068	-2.16720093692513	21.8167713475973\\
3.25022310443553	6.58189618359635	5.15097641206106\\
6.95965148712988	2.16989489055052	30.8232921664405\\
12.3208772383806	21.3209385547027	18.7592454141459\\
2.40827072010522	3.627546434522	15.567725305302\\
2.24396377530559	0.726002867215101	23.1076607662667\\
-1.85942742313665	-3.15157568547532	13.4501945459334\\
2.74761263012997	5.03399179809919	10.0940724061543\\
-0.741061449974612	-5.8533611494843	27.2623503551635\\
2.43304101775927	4.09476371799977	12.9667232811429\\
14.0284281611856	4.62213413621564	42.3838579994595\\
-8.23614157646609	-8.13014260000843	26.8374458263826\\
2.56669423619913	-1.69962319039009	27.2057105466492\\
11.0338697871165	15.9610353601624	23.4516398930104\\
15.4532392460337	17.0811869108691	35.2007853812785\\
-7.33898021482287	-10.8312934859164	19.3939049229325\\
-3.31718009639451	-5.50537019271563	22.4151333335194\\
1.76416938065359	3.06700407120204	12.3223261205049\\
-0.279285010726979	0.132991856303946	18.0773384962422\\
-15.4025924290208	-20.8082953455984	30.8231706325812\\
-7.87309125350948	-8.84033318796936	24.8646742483572\\
7.6925290195969	7.18549595224579	26.9410600365586\\
7.22439123562167	6.25016406215295	27.1246957240902\\
-11.3243910542623	-13.6979363391693	27.7462215658546\\
7.99904097573197	7.34316892571447	27.2746069700027\\
-5.54980117978229	-9.05734989492512	15.3775366498451\\
9.87972068553894	13.8840964439443	23.0397200110858\\
4.38068023373482	4.56076635657475	21.9235433552844\\
-8.28746822540923	-10.6566579494446	23.6726392405503\\
-8.38732923591528	-1.3699210547563	34.0850019973928\\
-2.419318478224	-3.72079661443304	15.1623764098682\\
-6.11744450171477	-11.2570634372291	11.1569493366232\\
8.04781321807319	14.1918357362936	14.1962560863746\\
9.23479394302086	2.11524398981726	35.0346634838438\\
1.62194820035958	2.90979640956587	10.867829567372\\
-0.659102482843543	7.78676388967676	30.6328976940883\\
7.37845622128101	2.84519577387502	30.9440362571753\\
-0.985937858397573	-1.72872588166167	12.3315415516213\\
9.18737330405753	7.82877812940515	29.5001725380648\\
-1.89386511137903	2.66524526423785	27.1965638411692\\
-4.41141681971922	4.66962495431043	32.6548767355289\\
4.71019746781186	6.85774989570136	22.5183814558523\\
2.82052464491813	2.67680057255919	20.8100471328245\\
-6.418239911478	-10.7688262966657	14.9924037893219\\
-11.6075778434687	-20.0909838312952	18.0236670377823\\
10.6145286837073	12.9046945904056	26.7888614233682\\
10.5599476858123	6.00930457994362	34.2940125188181\\
7.22229141163165	13.795808747017	10.0051328768592\\
4.88129686703515	2.62696056639248	26.231405058811\\
0.161851319201855	-4.08591754540189	26.3117046361964\\
-9.36933558875582	-13.2155103755083	22.4630989285862\\
-11.3221360469687	-12.4404975848017	29.341798761835\\
4.71768112814005	7.02213930571789	18.8357041756474\\
1.64526276006629	5.01131432988555	24.9558548989328\\
-1.832970090218	0.258045897566148	23.8400724488659\\
-8.11453941984785	-6.77838588273893	28.2473857724077\\
9.20302264439843	11.1768131998904	25.4077365671855\\
2.25708763499201	6.15024272719668	25.8615776164967\\
4.40656468060317	4.51546877108005	22.0911900891784\\
-11.1700516454801	-11.5445020345066	30.0380228100569\\
-8.30597659166012	-8.10417510945438	27.2074470591266\\
8.69070144782541	10.6910350351877	24.3885361254265\\
8.08529671136741	11.1169810869117	21.6700690022369\\
-8.43045992545421	-8.16640941496478	27.2781332162665\\
-9.59847697710516	-10.284364071982	27.5216936451214\\
9.48528665476151	3.90106925783153	34.0038586895071\\
-9.17765584885325	-9.49589794999161	27.451325427457\\
-5.28561063402908	-2.32804809624911	27.4490798767213\\
6.54002289504191	2.17777578362105	30.02205175185\\
-4.12273432666282	-2.34981585776648	24.9531058485925\\
3.29339518636248	1.80320415530782	23.854341932154\\
-11.7662708697602	-13.867730722264	28.8134810898643\\
-9.11211332129797	3.28178891181522	38.5903724678376\\
7.08798360333912	6.65616635916092	25.9418773729787\\
2.00298301051894	-4.9382208514715	29.7174233438151\\
16.8171108926014	6.23835408358926	47.3074758752163\\
6.41007202587945	-1.15364627875633	32.8179037583547\\
0.590566660555043	1.04528418081458	12.5816647763651\\
8.76376140996798	16.1494133582441	12.7850962627348\\
-9.24390823768478	-14.0743812328034	20.2937398634243\\
10.175669547035	2.22499527994602	36.6394788683005\\
-10.3491366371507	-13.9621000167426	24.4066217892871\\
9.71942085380336	8.34381178038124	30.1699060783666\\
8.01044634090384	5.97801010478622	28.9459839976116\\
15.7240949872665	17.0421757247524	35.943935426894\\
-8.91309820097649	-10.4529032063422	25.407688580875\\
-9.44488653301185	-9.34380998503226	28.3230820080605\\
0.46991492130518	3.64888430494047	24.6755545582818\\
15.495966544565	22.6523317554095	28.7303057045361\\
-7.32885236706764	-7.49560070481791	25.3756753861887\\
-1.88761674445137	-1.84561629730888	19.3721790595681\\
13.3872987117579	15.3737151542088	31.4514368380972\\
13.1108980106785	21.3360899780067	22.0492986649253\\
-0.952752277160769	2.28366551827474	25.1970929808182\\
-8.91084124100143	-10.1577324063023	26.1641334678983\\
7.21304847383982	6.60785142011239	26.3173454598396\\
2.98748974209981	-3.14894995334462	29.3662898367438\\
9.55655295392143	4.29358971740587	33.7940548479386\\
9.5928335293833	1.88244612107406	35.8408483019265\\
-15.3295154893258	-18.1867351942297	33.6218850676917\\
-4.86950513616928	-6.0933417017109	20.3808419736809\\
8.80836244399019	6.18005541085862	30.4638332617288\\
1.7122183613202	1.7589588646965	18.8276122646799\\
-7.96258210816525	-7.61639641231722	26.8322107953017\\
8.50097346243781	7.3010927880786	28.5277619292607\\
3.17075696248782	2.59749010988936	22.1167398696766\\
6.55034712877497	-3.82507138784223	35.0808012774223\\
-13.4966064190071	-21.5318249299767	23.2172740445019\\
9.1755559549275	9.60103697763396	27.309009456716\\
4.12138525774506	7.96359697987109	7.94694345125093\\
10.1464803907891	8.47970509146283	31.0032239539518\\
4.28346280220028	6.34620604297262	19.6417268069111\\
-7.62508684042637	5.96523561590557	38.0043913531685\\
13.1668356324797	16.0057782970121	30.0330869571615\\
7.74033616699679	14.3305316549509	11.851762536148\\
-7.9607681627138	-8.42296583474309	25.7187482751238\\
10.242514429394	9.79134464338582	29.7867936652596\\
-8.84326035536975	-8.00616407868798	28.4989636104204\\
0.229715815373392	-1.62065100149199	22.4562995423582\\
7.85105628381321	10.4020580646386	22.1905391712393\\
1.20374655274942	-3.23885910757785	26.8305115408484\\
9.77796374872848	10.2568094065584	28.0246683395577\\
1.86326453280406	0.0508723662935627	23.3739799985246\\
-10.1189695701407	-10.7137429658776	28.3315363605722\\
0.982753246441996	1.69142351213102	11.7360885468652\\
-12.5665695770936	-22.6862078530851	17.175458368816\\
-1.41274264539243	-3.11598908758244	21.2362795991598\\
-0.665177221308396	-1.31230757349829	16.7191895906924\\
-11.9242093818514	3.63447929223073	43.5205392343624\\
-9.62579041087891	-9.86550580452201	28.1299376638427\\
-7.61677527510363	-6.92029217107174	26.8918035417856\\
8.32746737835452	11.0721347616488	22.593038966249\\
-5.70322264889102	-9.16474723769173	16.2534737109101\\
-0.632317742466033	0.727338335394468	21.6202286245199\\
-7.72939299617832	-7.86082246929944	25.9036444600008\\
9.44930347223302	9.41901601941846	28.2403318125154\\
9.25746040347554	9.50603265943683	27.6440769963789\\
-5.0889966224662	-8.5739613677807	14.2842010686025\\
-3.9349182755328	-7.04092569752987	11.3754233343831\\
-8.7013695504026	-10.2148412420072	25.1521168894607\\
-4.77852069695194	0.595193608859542	29.6543226336708\\
-9.29430940015474	-8.80842789007445	28.6423710163723\\
6.46842398283259	9.70349348303708	18.7844969607895\\
5.06514408717472	7.87936609222216	25.6316126258988\\
6.37478115169369	-0.582097801116048	32.2945675712585\\
-1.89431859793682	-3.27510642510714	11.9500846310529\\
1.57371571566984	0.857407114879828	20.8447202819324\\
1.87745374858943	3.78960748902512	5.0389837539198\\
-2.84665641314911	-4.71298722433335	14.4915578605639\\
2.00848285162177	-4.00085222213486	28.8499975325472\\
14.1881516866711	20.9936426681558	26.5548557148988\\
-10.0715374157514	-10.9387460675421	27.922342187445\\
7.99159423606691	7.25886077991207	27.6192727309107\\
14.7619966850145	11.9645428122185	38.3427629901331\\
2.22871320039806	3.72219807863901	19.0475115866401\\
-3.61411140911316	-5.77931037027062	14.9024010920237\\
-3.13627211258656	-6.42418494678989	4.44857519941174\\
-7.94071384027626	-1.91852593220203	32.8340226034703\\
9.1830960556922	9.71396470973315	27.1792133314583\\
-6.32672532082317	-9.76478328565347	17.8156325536763\\
7.56449312950765	9.44567627850462	22.9601068885869\\
-5.39670038709749	-3.68289168730678	25.9658134766535\\
-5.9242171251537	-8.23645952337883	25.2668126582574\\
4.58987594279695	8.74268357912363	8.84793202699596\\
-6.15045264388964	1.47090312748465	32.6725453698292\\
8.2730846628394	8.3207452387033	26.7100112929636\\
-6.32729177476998	-10.2105724390293	16.4147553884999\\
-6.34057331653025	-10.9107548672839	14.0915705047406\\
9.01615051858709	9.72144117580829	26.9755891000983\\
7.48269431274733	10.8923107327225	20.190477458815\\
-2.28708495993844	-4.01703301994003	20.7658478819614\\
-9.13602490324663	-7.9151137090535	29.2952478018251\\
5.39816292826633	2.25532741564738	28.4868640392456\\
0.825878895082541	1.56739491936009	8.11175228641183\\
6.47824266709449	12.0943319661971	10.6948518908049\\
4.41423155183901	3.86546048039954	23.5216573266904\\
9.73603023699009	10.1359587358297	28.071322261241\\
-4.33665003703941	-8.88443976427881	27.8039695020298\\
9.60432880954265	1.03182676710562	36.5067773568605\\
12.3230823733498	11.6092923597292	32.8322267328224\\
-0.714967447059671	-1.40289475360133	16.8450472429897\\
-5.91864517567922	-2.99828503825038	27.9335658186431\\
-11.4218614951524	-11.702765105882	30.4831977912372\\
-5.6966509391184	-8.50097042761866	18.7466595473034\\
13.6911839180337	8.13752907618497	39.1116576852954\\
12.2557729369399	4.09774787572809	39.2179777601705\\
12.4398895632986	11.6560794512186	33.0605776900055\\
10.9645526913546	14.4535797728213	25.6576207034065\\
-8.82866334813918	-8.10831683184459	28.3432984471478\\
-10.7668587790472	-7.57240857417594	33.2932372847738\\
11.93080948361	13.6789063818552	29.4691492624418\\
14.1945079089031	10.1264723809854	38.5690644054634\\
-0.325971452302796	2.7855897128972	24.8159950334144\\
2.60346374847242	12.3741740540055	31.6741391668417\\
-9.87752116217081	-16.795841019723	17.1087531653173\\
7.98043589081377	5.90965936267492	28.9573502848184\\
0.964294837669596	1.40202771840193	14.8297456194355\\
-11.8671323587861	-12.7049893227568	30.4877599718423\\
6.11610178213432	8.15152538627217	20.4843793090285\\
0.0597839147113053	-6.48344294242047	28.809159491495\\
9.87871450109492	8.65597365829908	30.1769892455119\\
-9.02138242925017	-6.20465204237668	30.8910909071945\\
14.7820577625388	19.0068157355126	31.0773019060693\\
-7.91787128207447	-6.15069304494042	29.4051968239719\\
13.5248911378631	19.3377336986914	26.687387728657\\
-9.16967535053397	0.311313325918754	36.7390231990586\\
-3.64746402791876	-5.54413009801216	19.1197733137363\\
-12.6257766989619	-10.7242630007892	34.4611726137013\\
1.62458371813801	3.04552176591025	8.78920381275133\\
-4.08010874347064	-6.42546509133414	23.2126922413742\\
12.6367334287202	12.3193617597923	32.8506025138086\\
1.49499420034809	2.89256917301086	7.24340557071341\\
7.28675973515692	12.0377721046917	16.1743545312941\\
3.48788169407926	5.5074666656904	16.5595126083264\\
-6.33247562267493	-8.39375064695568	23.2402971935823\\
3.34770902245642	1.65225409282114	24.7653772066467\\
9.12492887019264	16.3697825091584	14.275352658381\\
4.49483085997221	3.19915646149535	24.569764706256\\
5.41685547148693	6.54320393107873	21.3532867802904\\
3.65886606023498	-8.82812096155534	34.542782975076\\
-12.3798208244133	-6.07753947349607	37.972977230492\\
-2.37351801158733	-3.81127167012369	16.3885144836598\\
6.55084310602339	12.3621697566749	10.2113373123505\\
-7.7939173516615	-7.26121714158151	27.0601757260348\\
4.85602023861485	2.94469620926055	25.7655055559628\\
-13.672833783457	-8.48425665139513	38.7853347179255\\
4.72310036257744	8.08257068027535	13.5163232946798\\
-19.3834920594247	-9.86498621357175	50.6001875624326\\
-9.92789290679225	-11.1743533672932	27.3340725480098\\
-4.81937963263965	-7.5915826767761	16.6368329211982\\
8.41725020457201	7.9721796881212	27.5039484232025\\
-8.17634927469164	-7.13788727806517	27.9556764150227\\
-8.54990740082219	-10.6790721000971	23.9773306563971\\
5.66004344540043	6.89443385482998	21.4705046146581\\
11.0242039087727	15.5885556449572	24.058795379948\\
-13.0029108786995	-7.70419777418461	37.9694045184553\\
0.220197161026433	4.51588782520638	26.3873555583613\\
-9.0301672827792	-9.67086156851486	27.0725990895832\\
-0.828014732872671	-5.39777369455898	26.6126294202671\\
8.10946122727861	5.07187640717083	30.1458589679837\\
-8.85445205925241	-1.9781651607887	34.4478994089026\\
-15.8371384064181	-17.9324386997498	35.2530927867607\\
10.8825373739623	11.7778027027486	29.0102876082105\\
-1.5537945925043	9.14940958805107	32.5862188792932\\
-5.49847658386698	-9.44494006526669	28.1307665271172\\
1.6619451157307	7.1067763838106	27.6977176830439\\
8.53861899071839	8.23415763987584	28.907686418344\\
-1.32661652423429	-2.26101591763588	13.8156241578633\\
7.39305552514997	5.90838776102687	27.6438837790154\\
-10.6116632601345	-17.4050020769495	19.1365833872469\\
-9.34160517311019	-8.68313211957313	28.902928352814\\
-18.6395206079687	-2.80997019804926	53.1525185324116\\
8.51731071285228	8.06888093433039	27.630684335407\\
11.6349496291449	13.0316730202385	29.4735770889848\\
-9.67174293103603	-9.95908784136927	28.1295400748897\\
0.265308415572155	0.0272432181399749	16.9552334005285\\
9.50356740446498	9.9990486906426	27.6438079246192\\
-3.8152072730311	-0.918058979996431	26.252118192577\\
-1.8164414826873	-3.04245410168948	14.5029884349232\\
-3.17116515555904	-7.54635000532439	26.7538107078583\\
7.19309171299557	12.8633510406083	13.0486132801069\\
11.3753481060156	-2.69527298965891	42.0455935749039\\
-12.5499992881743	-15.6634738769902	28.732486610337\\
-5.20339027914954	1.28446791836148	31.020041338133\\
6.81291815361863	7.12084071172616	24.5481360400019\\
-2.74814022518393	2.40568056174204	28.2582641495221\\
-5.47206233741178	3.52540108988978	33.2323566765121\\
-6.74962841366244	-9.95037114382695	18.957884638103\\
-14.9087976453127	-8.40209796741562	41.5580591360931\\
-8.69696256345712	-10.3656724523874	25.3166698281903\\
-8.09062383416109	5.28877819560712	38.2127934446881\\
9.06492609525758	4.9248884167298	32.2542825916775\\
9.9971854708474	5.40776009537903	33.6783328602464\\
7.58799861102417	9.93226891038337	22.9403066895877\\
8.55616384434764	13.9441467834653	17.562166659671\\
-10.4979006704723	-8.49014705415994	31.7825039815106\\
-7.11074675492855	-6.61867837893199	26.2201008751288\\
-8.82903159779453	-5.36034670717295	31.3462883547153\\
-4.96622489595801	-7.39680900322466	17.4069707907073\\
5.73974420264117	-0.452199199985585	31.0982671279036\\
4.64847406594392	8.80880922393304	9.09508654222477\\
-5.86218814066577	-6.40427748766183	23.1045241698999\\
1.15871480583855	2.00076762653734	13.0546276518526\\
-8.59764472344646	-6.89319378807998	29.2155859268492\\
-7.85764382953223	-11.3934986664068	20.5622088984206\\
-7.10625800869715	-10.5569129081719	19.0453209270724\\
5.18477424485786	1.96248297534917	27.6754725696211\\
-3.31199063433582	-5.73207291876207	23.2746602518709\\
-16.2462447173702	-11.7668925942954	41.9229553169876\\
5.29106390945161	7.01926542187605	19.861695931476\\
-13.3613676525154	-7.70357540872143	38.7216228451346\\
-10.1642036817523	-7.95700874543922	31.5853604579084\\
6.34588977674247	2.54687945352089	29.2787644318331\\
-0.275073803669144	-1.28348665936107	19.7435236886998\\
7.5950734785729	11.3578551745922	19.276934044953\\
11.9101372114528	5.97580171006482	37.0931279055514\\
-9.17248849674728	-9.69091745932614	27.3725442395934\\
-8.02775143155859	-9.38200366288188	24.4734238708276\\
10.7545990351688	15.0048668027874	24.0630999442613\\
-2.57271075062045	-4.216305006166	13.7814854572577\\
10.801250427819	6.77750701108114	34.1055374133387\\
7.42520683050371	7.62700951278845	25.4376464186201\\
-12.1571915386073	-11.7113523793026	32.3079368464271\\
6.76750962813998	6.51008741064206	25.3363516075296\\
8.79174652880031	10.005633778908	26.0958985319796\\
-11.4455149858308	-11.7707846636205	30.4762405797656\\
6.94691507967855	4.96723916299293	27.7876631038066\\
4.63500753041224	5.47713315930228	20.8877865898283\\
-7.04519569429946	-5.65052720059203	27.5564354780142\\
-10.3719189291908	-11.4771081203279	28.0242585644937\\
-7.41500647173478	-3.96406569795255	29.8826823963177\\
4.4337657298469	2.0223841258061	26.0857670150569\\
10.4147727771907	18.9473399230798	14.7019539874951\\
-2.07425277406371	-3.47359073725094	18.7606874887884\\
-8.81177851621911	-9.12958087085084	27.2839645962413\\
-12.4724223955785	-13.7229402554829	30.8962992074844\\
-9.2673988460202	-2.88740314920155	34.4255731536304\\
-12.8488009611942	1.34572931527299	43.8205803232299\\
2.42858468709709	2.10862087231319	20.8041473561676\\
-15.1920760272616	-18.4063042216246	32.9692593349371\\
10.0009485502914	10.0898196984613	28.8032115730091\\
7.24976287437382	12.0027611772443	16.0854860816575\\
-1.68580953717401	-2.82613254937393	14.4430855656039\\
2.83320954944496	0.605978034420448	24.6445236265003\\
-3.50299399962088	-6.29920385787579	24.2677778253647\\
9.18714487214853	11.8126366308943	24.1285889809424\\
10.633479526076	8.45314005627218	32.1220979618816\\
11.7356182995473	-1.13556616301661	41.7631094067837\\
8.93836439708376	8.31839420320858	28.33505480448\\
-10.644146672762	-15.8363718099234	22.2661454453249\\
-13.2346543404206	-16.0611288454709	30.20935563101\\
-16.07037418426	-19.1212761830031	34.7014515715811\\
-10.820717661243	-9.82240645904633	31.1350987771061\\
5.84660070373096	9.11070743073939	16.8857045218792\\
6.28103159753542	6.47785803693417	24.0901218939897\\
-12.9233824578949	-20.4358001452652	22.8830812316692\\
4.1983244872077	8.08806778953896	26.7270784440583\\
-7.18313619678462	-13.0121979108857	12.4806127513125\\
-9.8687376703318	-11.1190562509862	27.1393464961399\\
-3.80515942372377	-9.35079983775565	28.6709643451049\\
-14.2254858143528	-9.53712273008682	39.1009071220082\\
6.07978246404508	1.24008998914637	30.1103529261875\\
13.9606181204466	20.1551763392287	26.9683442507478\\
-5.85799217373294	-0.430807427572579	30.4980876668852\\
-1.51733250144026	-2.91728516108329	7.57361566721602\\
-3.07811572425461	-5.19465619142539	13.1411336728126\\
-6.26018962036177	-11.2794798979895	30.5295642227029\\
-7.48598703227027	-2.30949320496857	31.6569329677918\\
-10.5643343004755	-4.67839867201175	35.4434784083893\\
11.6807389434071	9.19189090034627	33.7477937546789\\
13.9764088059614	23.0210742161981	22.6426144260987\\
-6.41440952142022	-7.16802365479889	23.3982899901199\\
-5.22433575259504	-2.45373050251157	27.1684301138105\\
-7.39864449792202	-10.4998216751742	20.8606016938574\\
10.0105908772189	14.9605483724466	21.5386080847781\\
-3.31186886473522	-5.91096497212088	11.37368713149\\
17.0161869807577	13.8526727267493	42.1775262053206\\
-8.18343472859725	-7.54816567478215	27.4614526231672\\
7.92952087764733	7.45688805866257	27.1253044281301\\
4.65912790409028	-1.24590943853742	30.1095478888519\\
-2.16946894257154	-3.95165226037796	10.1188766674563\\
1.9992088384291	5.51917387217461	25.2175232149553\\
7.01126381523684	7.37767884781761	24.753220127626\\
-14.2626562898234	-6.32158580596959	41.7103121009466\\
7.0701924468613	7.13060870164321	25.2226839978735\\
-8.68748200476064	-4.38897362809756	31.9971311276216\\
-8.88364530279788	-9.75790853117057	26.6651219316935\\
-2.53233580258884	-3.13351596772226	18.3766475548175\\
-12.590208173589	-1.44720607469491	41.6886472500084\\
3.8116211161274	17.8487727454092	34.5117409536603\\
-7.81665060240801	-4.16944896883591	31.9831428975728\\
6.8122271719881	4.96691113833506	27.489519485972\\
5.57080067861335	8.40925848475514	17.5154111538368\\
-4.89332394379569	14.1910777117762	38.721721773543\\
-6.63437966433905	-6.20507253740734	25.432667589619\\
9.14036909964818	10.0597492422476	26.8249266542322\\
-0.545555159087748	1.60687276454124	23.2445056231603\\
8.21600674311199	7.9352129377189	27.2261278403077\\
9.13892717532531	13.1069576465106	21.8470771661602\\
-8.27979146678583	-9.18522424093145	29.4704211464194\\
13.6224396948273	23.6275102484056	20.1019154686159\\
17.0726863194784	7.84198460515937	46.7252335184023\\
-7.92611955545536	-9.45733883867358	25.2210941170833\\
-5.10507356374517	-2.62920565439999	26.719115146973\\
-8.23531018117276	-8.26834043659873	26.6645756285769\\
14.8873078121464	23.1551534574419	25.8298858518619\\
10.715363196607	16.4184926708665	21.4819832014874\\
-9.9333791952331	-9.92907139783221	28.8289989410215\\
1.29259615154363	5.47834917529466	26.1272123931721\\
-3.5513751831505	-3.79036451040602	20.8046723032321\\
7.08725093399393	10.870360796858	18.2050598350176\\
-7.55746707920733	-2.99436997404792	31.1347678641274\\
-3.11309277217122	-4.88287516028174	19.9035555570353\\
-10.2654645144439	-11.2740906959821	28.0022906070352\\
15.0186600410954	8.42181995739537	41.7466354077573\\
8.4479468698936	10.8958326498347	23.2954145722113\\
-1.07940447486433	-2.0107666716061	9.07720105706431\\
0.0964912990020721	0.170985217618587	9.80007859587108\\
10.9776816617948	13.0318143315848	27.6610662876544\\
10.9406463003388	1.47111796263132	38.567178856217\\
-8.44320418019049	-13.0852568449034	19.0107092233843\\
9.43200378629234	9.63300027732764	27.9271928911669\\
-1.46794403663591	-0.751862789336987	20.7511331117321\\
0.387663444250807	0.672044912588579	11.0969580322649\\
-12.5643800850058	-19.5734447095643	22.93235531056\\
-6.39578641937994	-6.65897515866281	24.1272335599151\\
4.16506208321782	6.72949498800855	14.9162742804346\\
0.177122659727724	-9.63215071385318	31.6103651707327\\
-12.1320072535236	-18.9698882906812	22.3385033724879\\
8.46243257420111	11.1822887256902	22.8490892843264\\
-4.13345064948703	-6.32880944642541	16.2511515555147\\
-3.03120899782981	-3.35590446083921	19.9527249100884\\
1.54298198503585	2.68535034726236	12.2533617408108\\
-7.71114079534845	-7.90508485801127	25.794763553818\\
-7.87977247355957	-9.37873930292004	25.2230118569284\\
-0.628700153117724	-2.81992201176514	22.8228623364334\\
-17.2017323150362	-11.9674317156576	44.0431388552029\\
-8.37433680716976	-7.47027496160477	28.0259608502618\\
-6.52788581051488	-8.27722521128931	21.6983935811144\\
11.7476959531883	8.41995664187803	34.6338400704421\\
8.06005546301747	6.57823293257795	28.3606477924722\\
-5.5582267455719	-1.95111624002217	28.4144275442468\\
-10.1309955991011	-12.8993311658401	25.4667513158144\\
7.12613034475173	7.19685360476706	25.3479106448795\\
-8.49721891999389	-8.89821547624866	26.4774367264724\\
-8.68938628230188	-11.6852849985652	23.0027055177567\\
-14.990608351672	-16.1019457184634	34.9090314731185\\
4.03519045459002	7.38661090661953	10.4776133468688\\
3.47774912380864	5.59930347763055	22.0533374394474\\
-11.0103335835006	-19.0200573182718	17.5113794776003\\
-3.62371853666655	-0.589568036185434	26.2842964825336\\
-17.5768203295615	-20.7019754945163	37.3422764279395\\
3.6803880638819	6.06770441290675	14.657676975257\\
-8.82798204535556	-13.3965005752777	19.9447052888364\\
1.86929575246127	2.29530265566961	17.741019443436\\
2.37058466275428	4.40211067321216	9.35091886716326\\
-2.32011675848858	-4.69835464610074	4.94011330429686\\
2.18442023489801	2.22566771045702	19.5062956758615\\
-10.8693154520874	-10.5529986544252	30.4474769925454\\
16.6398353396452	-7.34054867302716	54.1644344687836\\
1.89664719956937	3.02332623600965	14.0964701219946\\
1.70468241376229	-2.57335346923669	26.8421023793474\\
5.63682459615544	1.26157436608312	29.2769124290857\\
-9.06540520284147	-8.09102322015314	28.9294883976177\\
-11.5942369902762	-12.3945393494969	30.1315568616909\\
5.42481625557573	7.39124457650064	19.5182966874769\\
5.66531350826533	7.4094833666736	20.4143738343877\\
-7.58231513061319	-11.5385533953915	19.025444630204\\
-7.18212263383629	-10.2027328004797	20.1573655516596\\
-6.91072936777145	-9.01336574960513	21.606097068246\\
-8.42514196074279	-5.0293023323412	30.8480064163379\\
-7.95829115985512	-6.22912969365054	28.5461063970624\\
-5.48339385420633	-5.4411974985593	23.5670391355848\\
-1.52866865555093	-0.0631685839077058	22.4525557400277\\
11.05985561455	3.45153499374071	37.3637868673352\\
-4.31761952836521	-3.8532941110456	23.0624573713661\\
1.40084734866512	-2.78151889794924	26.5864131379642\\
1.97055899595893	2.58898035927468	17.1796668524257\\
2.01023913886079	4.65090404724511	23.5555280398993\\
8.81419642763548	8.35854036663848	27.9830190292426\\
-9.90429452802318	-14.7634943293463	21.4644362262534\\
1.56205098750316	2.6336153287537	14.2580518610674\\
-1.01255982116625	-1.75191942297888	11.6270648408617\\
7.21778030849864	8.20244669033255	24.0151336361126\\
-3.05889004905358	-1.58508160847124	23.6553636234944\\
3.66140599429598	1.62663282057349	24.9812136263858\\
8.29935593497834	10.0322914778355	24.8835471334331\\
-9.36178906590624	-11.6101473549451	25.203556589582\\
-9.65835729615615	-10.8101918629751	26.9814938888202\\
3.51331165301289	3.91608616185791	20.4008616827234\\
-3.28587536363912	-3.46338399068327	20.5996083535965\\
-9.2274525742612	-10.3766327522957	26.3985793288742\\
4.38498722759698	3.1212492043675	24.4275254523071\\
-2.13664069306323	-3.55676347052232	18.9083925563172\\
8.51803022772257	8.19774224866294	27.4556504257888\\
-5.47372150064025	-8.47760031254241	16.8040102662685\\
4.76576668950928	7.17504616462574	18.3305283847839\\
-7.29322374997238	-6.50156345602121	26.6577640415141\\
-8.99178777971633	-7.66477183201372	29.2348087925481\\
-3.50041883073119	-5.62119116598453	14.7410514153674\\
-14.4152325338357	-22.0131328129868	25.8997819793249\\
-11.0702947425589	-16.8892621794597	21.9639021334908\\
15.0311549656447	7.2109377412266	42.6797742343073\\
8.73113269803867	4.279664337491	32.1917597684244\\
-6.94960261018661	-8.71231589792285	25.2661427400765\\
10.3090708088469	18.2370882554246	15.8741958557294\\
10.6972231368231	16.0781499629733	22.0262219490276\\
-9.92772585056195	-8.6751686157084	30.2703158042596\\
3.02317521375798	4.76825056686313	17.1690542497972\\
-13.2375314369749	-8.20553528500995	38.0745518520832\\
4.95543002781375	6.58698820095228	19.5423444607383\\
-0.111306759602005	-0.249419578388133	13.3191948728394\\
10.140783256208	13.3933109242436	24.726865323675\\
-3.30463990551255	-6.68081291231385	5.26218676821597\\
2.31747104982745	0.919893644825937	22.9499814043139\\
10.5315518156783	16.9524258870796	19.7137306786384\\
6.66499877347042	10.0270922526935	18.7751012443363\\
0.0576375325728731	1.33594913631685	20.8146815879174\\
8.39078417286323	7.70282523959723	27.7691443421681\\
-4.63379513897743	-5.49231584072318	20.8401960317718\\
-5.61270374683694	-9.25044066015113	15.1058684032585\\
7.97245159049704	12.755377975006	17.7471269657743\\
6.0288786669881	-13.4836455018555	39.7932357063915\\
9.65065556198192	0.548331807326099	36.9072670654834\\
-0.102436784774023	-0.26605989987147	14.1134012261156\\
-9.07354062131916	-8.20113983936129	28.8008843879744\\
-6.82137566805617	-6.21065356675968	25.8941267943489\\
4.93637575391123	8.72208173484092	12.3512744464463\\
3.43738823196139	6.99384069570598	4.9165765250698\\
-8.29743958735541	-7.95339796007912	27.7061200611256\\
10.6318800493052	10.3116410304781	30.1400935128207\\
-1.76731388013068	-3.66347470717608	21.6296219458034\\
13.540260782758	13.0522830719919	34.3046129173485\\
3.24429838115636	5.90670707761712	10.4961179793766\\
-8.4317487346184	-8.01458797932364	28.7255692687509\\
0.812749902306106	2.10643536806064	20.1314331053686\\
13.1907184067879	14.410933853719	32.0139929594673\\
-8.85057395844032	-9.05085582579199	27.1901580455493\\
7.23931768701955	10.1671875621929	20.4525199411962\\
-4.5883043675409	-8.498653918149	10.1731746644959\\
-6.8626656093552	-10.2703330054205	19.0606643057492\\
-9.96625645666107	-6.48390460169555	32.6151530365563\\
-9.59309529740034	-10.2274033506906	27.5813281322959\\
-6.47254790187616	-5.8293493668525	25.5609946370486\\
7.18336730704219	8.45437635104081	23.4705477941265\\
-3.44930288819988	-4.67407855383266	18.0471875355267\\
-6.20233493769163	-10.440689208567	14.8307040488639\\
-10.1366442898938	-4.56763288619631	34.690652101519\\
0.596903240937818	0.856304703510218	14.3724982335361\\
2.41100394602822	4.25698716673273	11.3897754811679\\
-6.78210336032857	-9.55575233486606	19.9931753930933\\
5.23835269109851	-3.69110041361029	33.0537342065543\\
-7.45575567882229	-9.40455249638617	23.817670725301\\
-6.99822528974563	-8.6397635219643	22.5747345954757\\
-7.76337725892653	-9.12904148761025	25.8721538914412\\
1.0730893465138	2.1850963662281	18.9622544304873\\
-7.58090818268562	-11.1992959344813	19.5806497677723\\
-0.307463894453778	-0.76477908272038	16.4700098155839\\
-2.03693459836873	-3.39900278091166	14.3300085985504\\
-2.3197706816012	-4.28713036370186	9.59910996778764\\
-12.6024944802645	-16.9744485658487	27.1017642412467\\
1.68619114619753	2.99961798719765	11.2583497947327\\
-6.09055700928611	-8.18453536197147	22.8957478105653\\
11.4264026366064	15.437959825826	25.5862336041072\\
13.5514542567534	8.66598322906391	38.3455240709436\\
-4.87057361570619	-2.01171031956477	26.9849963901475\\
-1.63223483758735	-3.00876545206976	9.62934352201776\\
-7.89434513137288	-11.5585573807761	20.0654079599535\\
-6.43147213624178	-8.06397035704989	21.7808334628924\\
10.191146942351	17.4704813634244	17.0697001969226\\
12.745565300212	6.02102639243159	38.7766208488253\\
-6.9583963685529	-8.08773148411771	23.4093749733633\\
-7.22035215733461	-9.92407952142223	20.8841130971013\\
-3.87064099196029	-6.15142915987887	16.1186379675388\\
7.34052444609662	7.52003109285534	25.3691437514682\\
-8.65072153209591	-7.62363419731236	28.4865249075012\\
-2.87336309799419	-5.8344113738589	4.89858327358364\\
-11.4381656607261	-19.2503516912099	18.9724416877321\\
6.87422163056238	0.830045115624828	31.8895762413891\\
-5.71693993383459	-0.966730678660997	29.7564573942749\\
3.87921577965936	7.18434647505085	9.90140728131223\\
0.425363841985311	0.532244075021714	14.8746745117396\\
8.35286242303832	11.8873729773717	21.1571845991124\\
-10.4579217775297	-13.3926365965258	25.6956183744285\\
-6.33795095766359	-5.07427342534169	26.2747541192899\\
-8.53255603825191	-10.2610140553794	24.5897098557322\\
5.82075830292079	4.73195533123855	25.5195299564889\\
-9.13572033410303	-10.1586278229975	26.4505127895074\\
-11.180035534736	-17.609121601619	21.0804604777235\\
-12.93608077237	-5.53967078172141	39.518642904381\\
2.23971171792327	5.10083674625355	24.0097077371326\\
8.81794208642718	10.9705054531002	24.3331231119974\\
12.323677156249	18.9360643026208	23.1241746779303\\
-9.37379674407877	-8.70669093797627	28.9477994913176\\
-6.14857087320902	-11.8643336048087	8.89046603713123\\
15.2438909018229	4.69998347915589	44.8538761674709\\
4.8531651614355	7.80094665807059	15.9079557240983\\
-4.79478705415974	-7.42732087744229	16.3680797308963\\
-6.03509159228124	-9.72631310110743	16.2747403857577\\
-12.4254345122454	-3.79612362789934	39.7658360023454\\
-6.40193504868688	-9.7530770088883	18.2636623218386\\
0.745936092511994	1.0203598529736	15.0988872304613\\
8.88942492085709	8.8160739538939	27.65431751824\\
10.8212807945839	13.5967192359534	26.417045291667\\
-8.60387125121263	-13.7002011884916	18.319643040476\\
8.98144191162234	11.58923966397	24.1081414236405\\
9.35985402529453	7.4791059177452	30.2828109649224\\
-9.00649114849876	-11.1227152605095	24.9580920515476\\
-0.68202584429446	-3.4235540703893	23.8380376214187\\
-2.83113413661013	-2.54171687206442	21.1634108406484\\
5.46463294696528	4.05259695955403	25.6244575273964\\
11.3326166735299	9.52782655608346	32.6291680386089\\
17.5692568007453	23.9553863502629	33.9615190217449\\
-2.22657890125554	-3.13193100631478	16.5270049317446\\
8.84684560803243	14.2521615427426	18.1907754505966\\
-7.06460725292763	-7.44249996004752	24.7431708560865\\
6.55720627554864	1.68499242135789	30.5293585544517\\
-9.27158334950882	-15.1634548631035	18.0040138219915\\
4.65442556269531	1.54366851657748	27.1040677640592\\
6.05450699228887	-2.7925527144803	33.500289662368\\
8.87227824344701	8.01672222680388	28.5412775210972\\
14.1261514750823	7.81637411415606	40.2372729717362\\
-3.43725519965479	3.38418422630335	30.1943757359896\\
16.6925087717675	17.1316356759623	38.4418321615128\\
-6.8553639965679	-2.77019089228361	30.0182758465502\\
-12.0173101514506	5.35060696254088	44.6666574655283\\
-4.56383975970967	-0.44825877472518	28.2039499040614\\
-10.7108159286125	-2.50424505315474	37.4421534818343\\
-10.1502160633016	-9.61685163759182	29.7366101196557\\
-4.2313920939601	-3.77014481109644	22.9769744626507\\
-10.1621153273537	-16.4350582942995	19.2080591042549\\
0.454757793041956	0.793836105133267	11.0109356979181\\
7.86500376784672	11.3998154842104	20.5658410392626\\
-1.12341545654175	2.18141822439728	25.2983911390137\\
-13.0329762144077	-15.5608966684117	30.213378033231\\
8.85849004759162	8.87232404458079	27.5477434323572\\
-2.53254805691067	-3.15125451049156	18.327376637355\\
10.7108076575905	11.3153758333973	29.122182365024\\
9.67982267486722	10.2467182248871	27.7845080390151\\
14.5138596466834	21.8646478999236	26.4429722618003\\
-15.0755248333551	-13.581579293561	37.6424835956686\\
2.96359772010259	5.46160939718798	9.85671627383035\\
11.6257523401757	18.2005864469973	21.7304056438017\\
10.2727795476572	12.2089534984589	26.7774974628635\\
5.02000252509382	5.91206842849563	21.2850123444025\\
4.5816380761498	-1.27541873888734	29.9559746479355\\
-0.105564412165415	-0.285506107550686	14.4107540730132\\
-9.79703466167802	-2.17421515509007	35.9992082559444\\
0.776145654450146	1.02059251207091	15.5208966020137\\
-1.55541369046042	-2.64950550978501	12.2993298970742\\
3.25020789481621	5.4246784654397	22.3540829658827\\
-13.042513258482	-16.9879120855415	28.4014865310519\\
10.0421664160766	15.7531601160293	20.0869527120744\\
5.36508305816316	3.82673832921358	25.7061401796173\\
-15.2387663475324	-4.29478295985464	45.1642565905725\\
3.75852293152066	6.65816610995583	11.6985793094071\\
-0.239779609030434	-0.367047923356344	12.5565026138826\\
8.39167246855593	5.18383600543073	30.6213433539407\\
-3.62157491881136	7.67131829275694	33.8248702187118\\
11.1124022749159	19.0827508906171	17.8556252819804\\
-1.612511594909	-4.18554204025044	23.4398960932542\\
8.1207935465472	8.32526704961487	26.2780079032983\\
8.58790862299137	7.08970480361145	28.9687860462121\\
-6.68835433457482	-2.96251866979377	29.5052026723409\\
-16.6739639147064	-21.2998508652052	34.0635811361086\\
-0.420349817932619	0.579351371777568	20.4696130916412\\
7.75845995019922	9.73166269091691	23.0811656474545\\
-4.91557915788757	-6.48215961393252	19.6384415260176\\
1.7113412771626	0.406571896912479	22.30661886673\\
-12.8933502697396	-23.0120833538306	18.0319940195245\\
5.56332611316321	-0.446657344140108	30.8020620590177\\
-17.0290808928803	-10.8166539447504	44.7167696888773\\
-4.75471322443644	-4.0914882647246	23.8184370880998\\
8.18233611299416	8.34977950659124	26.4150562727245\\
-7.05028478997661	-1.63541875963661	31.4841155478797\\
0.910097186749663	-2.74994753065098	25.6867238684833\\
-1.24302347271203	-4.674085899065	25.0554632390152\\
8.40250626627291	10.8638225856264	23.2077962161439\\
-6.40110729290381	-6.54862597105757	24.3121877771758\\
-5.89955792883889	-8.36112308841675	19.1560987385515\\
2.24767446616942	2.25850036542467	19.6851749181814\\
-5.70434404569528	-9.22527701962672	16.0063135858666\\
10.0960745225192	17.2514407463524	17.1151982987136\\
17.0908057642111	9.02188132001793	46.0363795656204\\
2.674445775646	-3.83934173563952	29.585091073975\\
8.79982380379011	9.7510244715096	26.105975026454\\
2.13301619559743	2.76277855718367	17.4924210765675\\
-4.66458266882286	-1.2317041710589	27.5156815779508\\
-10.4082194448648	-6.2989532400856	33.7138807227553\\
4.14898165836333	7.93944548384767	8.44678050374653\\
-1.18639402503428	-2.139304627636	17.5759205740139\\
2.772667011627	6.18176226428251	25.2391467612908\\
0.1094072573103	0.252234599683986	13.4709383063666\\
12.6486931802019	12.1935172418839	33.0339031431058\\
13.9361327875974	1.10281478870816	44.4947095704372\\
-0.530702350515059	-3.23120552140122	23.7752404494397\\
1.35118433283592	2.62969077881655	6.93927122516532\\
7.34288064239896	9.86064654279872	21.4503953936848\\
-6.74347624862042	-9.00390491753077	22.3520327315681\\
9.444305759127	3.49617005973088	34.2772614577529\\
-5.86947664480835	-8.21707142835133	20.9309858740892\\
5.89148218814205	8.32733708680417	25.6879419110834\\
-4.44653537985388	-4.0220482713835	23.1018654598683\\
1.00203746362694	1.88398769264884	17.5121378575756\\
0.977098680780645	-0.569866165268946	22.2090972169533\\
-7.55121393609887	-2.15500985723056	31.912673731301\\
-0.912940647250457	5.80107582921547	29.1795335084356\\
-0.838356548347446	-1.47872967207254	12.0351792642006\\
-11.026282452965	-9.04366003438555	32.4157255875434\\
8.26232457737111	7.91884055582691	27.3666365902691\\
-8.27271548708706	-5.95330992434132	29.5564187830906\\
-7.87239995335967	-8.27865094229563	25.7079150926551\\
-2.28558290150368	-2.88734752296658	17.9189051025235\\
6.87180812659838	8.71167637392734	24.3829983157522\\
-8.04284732755444	-6.8142338556217	28.7377058781571\\
12.3767638454394	9.19758426824018	35.3219858808535\\
-8.25554028451872	-10.3692148964153	23.5719894024992\\
-11.444908447866	-13.898562602079	27.8314494794437\\
-8.6327640129578	-8.98213022698106	27.0825428432349\\
5.91518939380635	8.14172347306428	19.7341192057497\\
8.35377343298829	9.61234420605018	25.0670846187822\\
-8.78522594859315	-9.75102020388826	26.0635991989438\\
-0.109360107385815	-7.38587115042539	29.4305406039167\\
-3.66608109101261	-6.9637050372935	8.64099528140777\\
-10.2169801083336	-4.68284454291394	34.7556560016924\\
3.49947075757094	4.7139562193856	18.1441272180078\\
18.0371546352923	21.5261903427724	37.8288887219422\\
-7.72563417495227	-10.389429665537	21.7949725058379\\
2.24655156742973	5.66055821391319	25.0293411786397\\
-8.04306191947667	-6.55666382459549	28.346848978201\\
-6.31555228423792	-7.67012226099152	22.1449456910866\\
10.9123629008694	10.4101962517905	30.6911331789462\\
-5.64919869922297	-8.56544567611177	18.1574705881011\\
-0.430571561185967	-1.96974118397985	21.3648376842842\\
-7.43930483131257	-6.69752683390877	26.7539454736082\\
7.0434268666034	6.05444050641644	26.645175916109\\
10.6777667352308	11.9290716976157	28.3262756657441\\
-13.5792172503799	-18.2590629399296	28.3793029124164\\
2.56953306994501	4.10639575936551	14.3870370928204\\
-6.85419987448152	-5.86165984260207	26.4483457522677\\
7.24316909314098	12.8905768815721	13.2869842420912\\
3.52192649342346	3.20243698217738	21.9973385304281\\
3.63143885300217	6.57423250835624	10.8623775875556\\
0.97722205423238	4.13831092198269	24.5700025711078\\
10.5284594395182	7.15505189742429	33.1738664465004\\
-2.34162126615148	-4.43912342709137	22.1043262714874\\
-11.8083489747978	-14.0422175759584	28.6710200593478\\
-6.00444954745539	-8.51084028252209	19.2254702269297\\
-10.2103937822682	-10.8244930559959	28.4299522422439\\
-7.68872483178395	-6.72392866356336	27.319542296738\\
5.95923262266821	8.40141114862394	19.3027486841404\\
-7.39479039574537	-6.44373666676886	26.9779485026147\\
5.61056448600009	4.73398821721844	25.0066220979289\\
-6.99934429844305	-5.54049074388455	27.5912752782048\\
4.966116693734	8.67635217648677	12.8285892833067\\
2.11636586000241	2.8145736244192	17.1621678200734\\
-4.76349090441118	-5.43473288884466	21.4045616842286\\
4.41735272976684	3.56236071758569	23.802883983254\\
13.2329030465377	16.8359364431649	29.1983505013945\\
-13.4570989515787	-16.4617379980062	30.3049226137006\\
-2.09887056687407	-3.30026320334181	14.5064610682903\\
3.91208020781041	6.13464736721239	15.5535084570321\\
-9.99040669928535	6.24744546799485	41.723409296584\\
-16.9406060791256	-21.4965712900026	34.5998174593846\\
7.15537289804524	6.58295310633961	26.2095346772115\\
-6.19356522623978	-5.99373843067176	24.6074867034528\\
-9.67039764775767	-3.72980305027573	34.4994557083523\\
-3.50264431574056	-3.5861596778985	21.0939629696243\\
-15.2255354699121	-15.5568720446828	36.1220155069121\\
-1.53801062815218	1.38520793648553	24.9825300282105\\
-5.7581402135488	-2.15239743953239	28.5655739915245\\
-11.0269872138622	-12.2615906916699	28.7822030719133\\
-6.71273568218142	-9.20471325013726	20.4787281198369\\
-8.90634980875979	-11.7990940890029	23.4935225545672\\
-7.88601996265676	-10.3410509824012	22.4229108075423\\
-5.71686631020564	-5.90703776660618	23.4461178388686\\
-4.88502743658021	-8.48154806616238	12.927390111483\\
-8.11008861540103	-8.98232028618867	28.334485377094\\
2.12998636556104	4.06219993115704	8.07463976434464\\
8.13633249385884	4.35721016700407	32.5329595489447\\
3.67184402787138	3.04034646124449	22.695197334279\\
12.6261484894455	10.3383578631949	34.8249952020347\\
-1.27591039105224	-1.72324992074212	16.1010769627069\\
15.305540178519	19.1727585971998	32.4114560814968\\
-5.36169725633864	-4.31024994002426	25.0166425016351\\
8.20930139701957	4.24479923132514	33.1284761570452\\
-9.27991196386002	-15.4299351041757	17.4433730722508\\
9.88290240468221	6.00622950762543	32.9054274106384\\
12.1396744449669	7.71418819485011	36.128223536569\\
-6.93405671611061	5.77439514413261	36.7622311270703\\
8.50032996897564	6.78871449352186	30.7806528883857\\
10.2547623328004	15.9703375110094	20.515693542339\\
-6.07412236643239	-7.4704190682256	21.7309770149431\\
-12.8909177340057	-4.35453449900987	40.2768219926514\\
19.5032201714144	12.1990319627215	49.5581655509312\\
-5.6910905807596	-9.98462615045566	12.9823711120739\\
-8.63890523575154	-11.6968737301266	22.5321423914563\\
-6.85317073751816	-8.71343414033717	25.3582765825391\\
-8.77972306965729	6.88791269093671	40.2495893130986\\
6.31088324834286	6.74535892830914	23.7399040811859\\
-8.8455691236119	-2.50214520619574	33.9791106020917\\
4.88381418940106	3.8699364847513	24.4883039570366\\
-5.34657679788559	-8.08757310034441	25.6012426713979\\
2.99042484015455	9.51039838290022	29.3148924540013\\
8.18935815166492	-0.975518181472272	35.5907064000654\\
4.41671015227389	-3.27047781764439	31.4619669276885\\
6.72352475774924	9.32831566857844	21.1637868286477\\
7.20582365861026	10.7403463663272	19.0468237089482\\
-10.8420406962166	-5.23900520718686	35.5270915494187\\
-7.40718411791758	-8.70851101467989	23.7209846854296\\
-6.49452753009133	-3.30611572435781	28.7524039492666\\
-6.28078840549532	-9.54854469355446	17.8375866362149\\
6.99531190403019	10.0552729622432	19.735837419316\\
5.67454376956128	8.88837530356637	17.0833479451116\\
-5.94668054476051	-10.3124381519657	13.518107325204\\
7.52644437200595	1.15944548022816	32.7575292812011\\
-8.36912553579219	-7.74763404040309	27.673034619818\\
1.71281637748863	-4.76750180323122	29.1565198100579\\
0.0385539442196972	-0.199197475156068	16.2232860944304\\
-9.44506185129604	-10.8448190386116	26.3465232159241\\
-1.93870755509112	-3.26121497320904	18.3595385705279\\
-0.495452399334326	-6.39612271965796	28.1500493996484\\
-7.07310371368042	-9.50959949703351	21.1837835599707\\
3.4046541426901	5.0454573575909	16.4829233090092\\
-14.1239730539949	-13.7573685436614	35.0660863077462\\
9.37161120449148	9.57071917034793	27.8521839684049\\
-18.3749680685506	-25.0477246731645	35.0771492319737\\
-8.468626574782	-7.97450550667176	27.628844668265\\
4.41257694607151	5.04599406686109	21.03852857998\\
-5.52588478235213	-9.88557333064533	12.0960299802754\\
5.42586455770536	8.30576009417532	17.777375280907\\
-4.01012702970497	-4.42443673331745	21.0133333301356\\
-5.99658753539128	-9.53226145789161	16.7325322785089\\
6.69197730099178	-7.71602386957557	37.5955901659841\\
5.95692121117885	9.18720601633514	27.3932550566704\\
-5.77460259042229	-9.04407661172653	17.1263752799251\\
11.1092173654658	11.4548270718373	29.9782951093472\\
-2.73743884825626	-3.68223723499209	17.5479894422007\\
8.175966398938	13.283316214746	17.3239317438169\\
7.33125737952099	6.50432782945737	26.7467422274395\\
9.52489630047582	4.83115721849171	33.2568064373607\\
5.39529605636057	6.18767589691481	21.9805332679399\\
5.58192564130215	5.39513784113193	23.9164348226539\\
9.28494051810788	9.69619019978549	27.4685213094493\\
-2.61833083822455	-3.22073477903705	18.4729536146994\\
6.57019790652933	9.14409087798137	20.0668419638217\\
-8.37512213688946	-8.02032991038431	27.7861094274326\\
-11.0664229734322	-15.0369010287852	25.0474069241164\\
16.1395022353553	11.0351443064752	42.284244163987\\
-8.18220276271759	-13.1727097387689	17.6274216134587\\
14.3462713143991	8.85298436620563	39.8980171824327\\
10.6190502128333	17.6018536702255	18.7625785937743\\
-8.37314238971197	-8.22057366434569	27.1180265966448\\
10.5361093972132	14.7117300237084	23.8030895599249\\
-10.5117235918773	-11.2240540743388	28.720378264318\\
-0.215419858495375	3.51455726683169	25.5825567999652\\
2.30092751280809	4.03962269075015	11.8968616253866\\
3.16232356839262	5.21081412154174	14.6132979930311\\
2.88456660699057	4.81760279106523	14.0747223781063\\
3.56779186807007	6.36053692360183	11.3782290578076\\
-8.54588627176126	-7.93546812498003	27.8707269109585\\
1.65026680266641	3.97515770832336	22.8512746711583\\
4.76978392390239	7.00822808415626	17.5733433618675\\
7.43670922297845	10.0850565822528	21.3475212878414\\
-4.10114885651128	-6.53121805699986	16.1264041966024\\
-15.7318141395692	-19.4928440043432	33.3555518018428\\
-11.9010090568558	-1.27388622797316	40.5425401225674\\
-5.70368483389398	-8.61780692253515	17.5798515313571\\
-4.79763440994871	-3.55679940804103	24.7518831307645\\
-8.80835111909201	-7.50884591432471	29.0018443145904\\
6.83390316692605	9.50121647291426	21.1147653192744\\
-10.4518058005893	-13.5444235715927	25.4602220973061\\
3.85322286340185	5.94171474312472	17.4415855320568\\
15.1893444654026	12.0561116682649	39.2495264254458\\
10.4755983173098	17.2297214229268	18.903560430365\\
4.7541546076961	9.13478753276698	8.47269922507987\\
12.0691905137767	8.18054693399952	35.5393107041673\\
12.4535162176243	20.8738800529248	20.2529127978636\\
-14.0360257487367	-3.67299049854639	43.035650962907\\
-0.997735818186892	2.31805038042111	25.305233560104\\
-2.64243446193739	-3.9196924755906	15.9991741590912\\
-10.0874355088306	-3.75782538927038	35.2803669818018\\
-8.55668913075875	-14.7101669862366	15.6051029718786\\
-8.41574194702947	-5.27946228181731	30.5677043908474\\
0.675062459342478	-1.79223338895777	23.8832553890885\\
-5.67461041321014	-8.39069912239262	18.9783311725478\\
8.29258739368635	6.95760132333412	28.4498132288924\\
8.35810126572961	5.75956348097252	29.9488280176437\\
3.40112039887231	5.30035868565879	20.6064433879811\\
-4.17648213300602	-4.63683046758235	21.1105422443916\\
7.78414549025444	1.29794565799862	33.0819301808061\\
-13.7276284214344	-17.0853234784695	30.3468904394067\\
-7.11121693289524	-6.70690985294714	26.090335066114\\
-13.3901511316921	-9.84057063051174	37.038889484403\\
9.91824079571483	10.5271219381631	28.1290357256899\\
4.02745793564956	6.6087959283536	14.9017196257686\\
7.85353569025683	7.3923904977252	26.8541759003644\\
-5.46012698999198	-0.695134385456841	29.5474049144474\\
1.41833476608247	1.85301060355959	16.6031001129078\\
-8.25981252461179	-4.15423822356698	31.3808613290709\\
-7.01307361351484	-8.20480481632172	23.4023086581639\\
-9.63267027115519	-12.6148683694258	24.3965173193356\\
-13.0242880213095	-22.2133107734968	20.1691177724635\\
10.6247169290646	0.943317513646807	38.3948193814188\\
-6.85827592948823	-1.97744449421887	30.8105720666728\\
-14.9861580021727	-21.7280332495257	28.2274890724768\\
-8.31998888864112	-8.05719903939317	27.4006690063415\\
8.72239387030409	6.73578629102916	29.6729625453282\\
-9.39691412822779	-10.346342597409	26.90176631891\\
5.06832066712238	7.79099182881726	17.5762309228005\\
10.3279680276418	11.7306235033191	27.5767049887595\\
-7.32200234248092	2.26274818122757	35.0756755857312\\
3.90514673794518	4.31440935934173	20.8676288869166\\
-7.54744494511825	-3.31071040231937	30.8160210068183\\
-4.65944013628422	-3.66331386987854	24.2500859452926\\
-4.48239577913071	-8.94351816144499	27.7452001125831\\
9.54804268904143	-8.50004805237372	42.2467320866799\\
-11.8227658525241	-19.9765177271409	19.2281796982932\\
-4.83764438220088	-3.49912396220356	24.9390126979315\\
9.30611946680546	14.751671153833	19.0728038417814\\
-5.74332071008672	-8.03102896624075	25.115091675454\\
3.20181586033533	5.84587124165567	10.3469398322865\\
2.28483151534831	5.16579324288726	24.0762931002552\\
-8.14678998657482	-2.47899520969483	32.7158153214511\\
0.0915925972911856	-0.273009830260591	17.2700957380081\\
-5.94901064860673	-10.1437768522366	14.2491273841051\\
5.22023385266181	5.36490122348551	22.9445677228824\\
7.15852507262714	3.90410159445645	29.4316958627506\\
3.00144256270291	4.97111406588235	14.5070497414714\\
-13.3123461663533	-20.0538067830097	24.8839599056728\\
1.31643248290664	2.51366718300455	7.88523090513066\\
12.06246732213	8.00934408316729	35.7011783970927\\
-6.95116176977057	-12.6933481566371	11.9762633340816\\
5.96241018453133	9.24300114614275	17.1088180500748\\
-6.73604406829907	-7.14026157440187	24.3687057940605\\
1.57249246223487	0.990793774385696	20.5361679256511\\
8.39535812649229	11.7728646226304	21.5323490525224\\
-13.9115230873811	-13.6254798769492	34.683620920972\\
-2.9134509471192	2.58283744777345	28.7104455330879\\
-9.96892391482371	-9.3976082899552	34.5301548023048\\
-0.983189129149872	-1.69691991993697	13.8712148840299\\
-11.956264379526	-13.0072656995568	30.3733980285567\\
9.22502535444816	9.12543421490801	28.0447135354083\\
9.35702988188428	5.35717919549374	32.4307369648262\\
-7.92906247775997	-7.2494356583888	27.2247846241767\\
4.20143349418901	-7.17766945548785	34.0598825850889\\
7.07446486678072	8.83780595068661	22.4630242516382\\
};
\addplot3[only marks,mark=*,mark options={},mark size=0.5000pt,color=mycolor2] plot table[row sep=crcr]{%
8.96143523522992	12.8044214890083	21.6423545615764\\
11.5119183884796	11.5405394491242	30.9351090545481\\
-8.16311155340391	-7.05627643602893	28.0309048847667\\
13.7697295975118	19.3582866776343	27.5074245852375\\
-14.6263167684744	-5.40892039387672	43.0906389359968\\
1.55494804688436	1.29639054487076	19.554893884876\\
-7.02309734119773	-8.81790849455489	22.3354606336817\\
-5.5421974195286	-4.44123526070645	25.5758886762583\\
-11.7323494007668	-10.6894186679192	32.3676722878876\\
-11.2205624892254	-10.565245087234	31.2829767418052\\
-8.62437053309439	-8.95917612078031	26.7266147635497\\
-2.65550152510862	-4.48305240730408	20.817530212999\\
7.75297128931923	5.93923565730831	28.4247260335516\\
-9.61396739332687	-10.2701939053082	27.5786746489167\\
1.88132378449065	0.0985287852807718	23.3472052868956\\
10.1019728059863	9.11714527250552	30.2121500176371\\
8.69117867929577	6.24521805565766	30.1376179570183\\
10.7753633278171	18.0247780620134	18.5594094013189\\
-4.5590963580767	-5.02673808178508	21.5684852410443\\
9.10925830392847	11.2731604985111	24.7421869061118\\
-12.4567420675998	-15.2698145747973	28.9327941316155\\
-5.71812196835033	-5.82529402223481	23.7012761149821\\
-8.79137366024514	-8.43847804380253	27.8269728563139\\
7.75301377009749	12.6395939642667	16.9568731868444\\
15.5797117529413	0.708455908371067	47.8666781117695\\
9.70640382218576	5.52293629507706	32.9989425381687\\
-12.6351901231181	-13.3739093688835	31.6980515943615\\
-6.53550573259715	-10.5681006257707	16.4589402311807\\
6.05331310081393	5.69570604540868	24.6907505469875\\
-7.65051580073601	-5.76383491994185	28.4010637708447\\
-8.34602703019829	-8.36906133083179	26.8072646854016\\
9.4440397557127	11.2248861886904	25.9878564640964\\
3.09485914639347	-7.63074555031437	33.1314397605523\\
7.67974679766089	-7.18670054006189	38.8001029367976\\
-4.86590193251698	-7.30543800625646	17.151427085523\\
-4.5230317552734	-5.19877026563144	21.1122726441773\\
-3.90520858381733	-9.53032154164663	28.7446844116144\\
8.50564411344738	7.82008848376311	27.9112540903174\\
-0.541312608468847	-2.7125898400226	22.8288054686818\\
-0.917308012099759	-1.6131906331184	12.1478139783335\\
-12.3856212179424	-19.0004604128031	23.2413042657543\\
2.96658855833676	4.81511937282548	15.419813172162\\
1.95210548120198	2.02469434718474	19.1000624853442\\
8.29151826955182	10.3921298692008	24.1733824266224\\
-9.51478990784018	-12.1539179430219	24.7651697722441\\
-4.67392250871736	-5.46065941874219	21.0462563696397\\
-5.59946989095417	-10.1785183020617	11.4384598880749\\
-3.18716712667587	-6.24592978604708	6.92533512993048\\
7.84123206490444	13.6615747283737	14.5447876377551\\
2.79520128929693	0.0797811820802928	25.3602579778775\\
-3.45181214874095	-6.14063153678228	11.5060671197703\\
-9.44838091824752	-9.76389902247348	27.892020979532\\
-13.9756060782986	-4.99715474120895	42.0686655056011\\
-4.61123449602931	5.23031145744521	33.3418163941295\\
10.8215730660533	19.0000668953959	16.6805958167862\\
-7.78581346226377	-5.78366384102832	28.675952286571\\
8.86277047993863	-1.36104814880591	36.9217691917407\\
12.4981617356263	9.734375320631	35.0763116954143\\
5.04757337557859	0.907978747971707	28.5828462403751\\
-6.41581171393793	-11.4396850874703	12.7028299020824\\
8.75958534274585	15.6963451005701	14.0412264035021\\
-4.31530308889636	-6.34921144967443	17.2289096930809\\
2.41938199363608	3.92730982746831	15.7061254626954\\
10.5381924283748	12.8831385842017	26.5972799327093\\
-0.568152881576998	-1.01578406970106	11.4717524164085\\
-8.07119032241687	-8.58545286872729	25.7826890868282\\
9.89321236777264	10.3447108048177	28.2974499636975\\
-9.94388083839095	-8.3607542700518	30.655939594714\\
-6.25376982095802	-7.91236464003267	21.4755631265034\\
8.54302423197809	7.82950237076612	27.9795183526614\\
0.783343461877941	2.11142722719081	20.3722511506823\\
-9.6541326230372	-8.19392037861521	30.1668218084318\\
-9.08462277899536	-14.1682460670968	19.4390476103172\\
-14.9188999490264	-12.7285004486264	38.0509839973393\\
-7.72332711541004	-11.1566363593708	20.2314969720498\\
12.4828141679758	20.7965451989807	20.4449392538398\\
-5.81146304585823	-7.89063718927825	23.2882124989122\\
2.11543149567439	3.94395589605846	9.15378657199283\\
-10.5614311673673	-11.9541846987104	27.9945118534316\\
7.78467405528641	5.29077834160077	29.2286114560365\\
-0.714170421879768	7.29678101601696	30.4072521048589\\
-6.08352795697826	-8.27338705186859	24.966638430672\\
5.52814347567213	5.63141633239705	23.3829772382758\\
7.78754246526568	-4.27373323420605	37.2226058661363\\
-4.08340125416588	-5.56180866819291	18.4630654076426\\
8.20577176314065	9.57530174303842	24.6918632440149\\
9.43176479431276	11.7813433464514	24.9385693790626\\
-6.51931294241947	-8.34311249214307	21.5432119636128\\
-8.24230749168232	-12.38638582532	19.8521861305281\\
3.02078957617377	4.73930554883388	19.2688806293406\\
-12.4015547652286	-13.0727121434091	31.4714844035997\\
9.06875751655171	16.7067143854125	13.0203376219864\\
-4.75880861018005	-2.35644674281927	26.3317191036532\\
11.5674931682743	2.43799395860846	39.0851605669359\\
-7.50648213024752	-9.82442810763326	22.8943942953324\\
7.15055499636691	9.9459794958725	29.1385948978059\\
12.0616195221474	16.7659466259401	25.6882218046139\\
9.25156725951948	2.99292542062752	34.3413113455577\\
-5.23451816238752	-3.17041484485597	26.2834292455892\\
1.91161748767506	2.75905564692938	15.8854551653157\\
-4.48243758321448	-5.1972506732966	20.9569419238518\\
-8.92899706735038	-7.72531380736991	29.0196895757609\\
10.0557783568667	10.9920696874377	27.9013131306312\\
-6.57441488506522	-9.23387378705904	20.8199073121627\\
1.50471197100656	2.65003617981024	11.768957361556\\
-6.43949277336019	-8.42237660904566	21.1008760694393\\
2.85612107640255	1.50431191976812	23.2392049661848\\
-9.23721680595077	-14.8678905370216	18.5206297017539\\
6.26969024875267	9.61284071672579	17.6145853272713\\
-9.45266218082553	-10.9835706260192	26.3462991887877\\
3.70112944635081	-3.77276977051393	30.953862444955\\
3.87826014057224	1.54295085248039	25.5942602767591\\
10.7072496485415	10.3897826418021	30.2061874752669\\
1.52803411902923	1.61022806303143	18.3857397027579\\
-9.65773501949403	-8.80043143083797	29.491758213765\\
8.24859876877247	7.63374072645738	27.5117447386653\\
-6.7246999848903	-8.31945856510487	22.3018980199822\\
-3.87231761675073	-3.92870237191637	21.5686924582827\\
0.600413419778667	-10.7661270012727	32.7490882252524\\
-4.31129779954735	-7.55016859809338	12.415088750169\\
-3.27010779323966	0.990926022105596	27.5474499709967\\
3.04005405586855	0.681620239296415	24.9789249683397\\
1.26216445140188	0.658705795617556	20.1967702187424\\
8.25108793806415	13.6478602020502	16.7768668224649\\
-16.3556808288398	-22.4971931430715	31.7583554939747\\
-9.34344315695672	-13.6479713301378	21.5531448890406\\
-9.99327245913131	-14.2602541048554	22.7000966244239\\
3.93161436551832	1.72411851935223	25.4336448093358\\
-8.80747198119293	-10.3796363856909	25.2121272740572\\
-4.55802508244975	-3.06856398835627	24.9022250259383\\
-8.28130989554047	-8.18909802852297	26.9298240332473\\
-3.40023340975885	-3.20250625272972	21.5693504467976\\
-1.19877335391717	-2.49797473232543	19.8685380063116\\
0.261154419287486	-2.65199356086034	24.4392885332826\\
9.03362507715209	11.393532095652	24.5960626193868\\
11.3704514741923	12.4129661503712	29.550987916542\\
14.5579224791315	20.1845249421273	28.8957698949655\\
-12.3398797665738	-3.34701057630288	39.9007688352345\\
10.4685219024769	10.2931412251056	29.7306291568182\\
0.527151061767223	6.59248371094024	28.3863888950077\\
-13.3461813451725	-2.62132742730184	42.4176074779447\\
-1.77510454967538	-3.08610199434357	12.3361224487018\\
10.779712068239	13.7759269928448	26.0383373158261\\
3.82077297352103	5.70518166303409	16.6006767554229\\
-2.6329007763276	-1.81960786351857	22.0763850049541\\
10.9278182070304	16.6947514221665	21.7993624240525\\
8.13166736498834	8.43357266650113	26.1573587547387\\
12.3746894666128	16.7120954759056	26.7515152378113\\
-10.6918098349472	-8.99779656492272	31.6951567001547\\
5.48985576164804	6.60510621339768	21.5354707430775\\
8.57427620227604	6.09310120205013	31.8840009768349\\
5.47550581429727	5.12205975647206	24.0733959524715\\
17.5942568101677	5.9476227421607	49.0426012257166\\
-3.32467860300918	-5.45569972266357	14.8429611114685\\
0.877442868983329	1.59299024031214	10.4060973374683\\
9.15585371266027	12.3496937692346	23.1381759123611\\
9.14327062589248	6.45076293669533	30.8999028716681\\
7.1728154857093	5.73274733135771	27.3579910700717\\
-9.255202344853	-6.30546592375251	31.2837263794178\\
-2.35235195319224	-5.24901353903505	24.1288420657236\\
-8.17662520136872	-8.8265242291013	27.0555787511293\\
8.36205710628559	11.5218048464924	21.8842720803101\\
8.27307101210551	9.40292061546065	26.0709477239939\\
9.69282606625983	9.01615667276776	29.356110194941\\
10.5362284529083	-0.622258381439438	39.2446414933685\\
-10.6228485890255	-13.8203243730422	25.4864275310348\\
-4.68288820160924	-7.59276032792912	15.5482675924453\\
3.8527615688243	5.26399375321163	18.2216876592904\\
5.77315277312369	2.2664019685176	28.4741164838153\\
-10.445431539135	-12.5636701315432	26.7751823599756\\
-7.26146137800998	-1.44371356109289	32.020549210747\\
11.4558209786042	15.1528245166512	26.1150626547989\\
6.19075400620962	9.974881483761	28.8069342539897\\
-6.5277842998954	-9.61573405192838	19.344325932886\\
8.93570017802771	9.7528136307728	26.4695446589396\\
3.10409079771977	5.97034088556158	7.87776036016203\\
-13.7132892196922	-4.64029831224137	41.6833142525633\\
5.44220028377098	10.2344183346144	9.82259614139382\\
-5.87663364602174	-10.1676570149566	13.6348416017829\\
0.195100905947257	5.25242766123121	27.2560053204981\\
-9.04982141003757	-10.5863328762671	25.6049148027778\\
2.08554365401273	8.83365104534875	28.7926592834487\\
10.6686266647715	13.8511914550129	25.6539103168198\\
-11.9221879617856	-4.73896742480064	38.0712155001108\\
6.05027454460189	8.60320564993152	19.1857024639496\\
14.9960957546217	20.4872241691106	30.0313294239478\\
-8.30168635325425	-9.0867209405453	28.0018876082215\\
-12.7155780235752	-19.2304494180255	24.060821225066\\
13.2858434463191	17.7149612695839	28.2312553255483\\
-8.21867117481774	-2.63727623884386	32.714555255294\\
-2.6766516779151	2.21284833203096	27.9691256785321\\
11.3205863138493	14.2196641582722	27.0299496219117\\
9.18204806104408	16.3105835927412	14.7479795400635\\
-6.46893939898264	-7.26013432156748	23.3495845290442\\
13.9060776108116	13.7808057996677	34.5306022763971\\
-8.69090599616063	-7.90725079165268	28.2396169437233\\
-7.15310509310779	-9.51813695953824	21.447061915029\\
-9.4969182263152	-8.93417449571342	28.981463483789\\
5.93070433638039	8.18755270581499	19.6900559922184\\
-4.4156069997876	-2.63357398556772	25.201197478785\\
-3.43210556075384	-5.65213629424515	22.5391422906972\\
-10.2499240682149	-7.33302361063305	32.4046568787461\\
7.62073684966528	10.0314893689819	22.0970147408905\\
10.4224519742805	9.21662861594067	30.8554502453357\\
-2.27118880330641	-3.67408152607717	17.5305965888246\\
-3.20531111465442	0.317333528229487	26.6405755126446\\
-2.30467279151097	-4.5679795100894	6.13803422887043\\
-12.4764477009464	-13.2972919087321	31.3924597979159\\
-11.2507271512044	-10.362471500367	31.572005988026\\
5.40262373964353	0.514568852380521	29.6471420193203\\
-1.23244357948763	-2.13924123054458	16.393121068831\\
-0.784803867638535	-2.22772190251725	20.7993216375223\\
-14.8536263077399	-19.4214277530888	30.7914716076526\\
7.79421099105929	6.93408747939633	27.3029059360681\\
-10.1826451347174	-15.961951553116	20.2408717275134\\
-8.36807360468774	-9.95362942199286	24.5814869358214\\
12.5608490862761	3.97150924246411	39.8660698859673\\
5.98937630052106	8.475848430109	19.2495020834834\\
1.32072552555223	2.27013656447057	16.8642982986262\\
-11.0278033043875	-9.96709769305444	31.4576355289604\\
-6.09523486330799	-9.60511390970525	17.0407306564982\\
-5.15534757716675	-8.64710676751676	14.482819590719\\
7.88345112952045	7.62203736737972	26.7214668364111\\
-10.277612870617	-8.90178366005926	30.8560121076969\\
16.3967762709468	-0.565259322095161	50.3570276572896\\
1.28795601862657	2.30054734945258	11.1158933667201\\
1.47114214978416	3.04802327972696	20.887136704999\\
-6.54855200631841	-8.50550592856494	25.3639790657621\\
6.80542168182312	-5.35889874041986	36.4681704426176\\
8.39719494295886	7.98627246927502	28.296200877514\\
16.9129944289629	22.1503776292345	33.927770560814\\
10.7691197032072	12.7845288985819	27.3964309157647\\
3.12403128677218	4.77533883227482	15.6874598737893\\
3.98302793117249	6.19686456998498	15.7875203013681\\
-11.5990584834272	-2.6955399833628	38.9246304605766\\
-3.65144020583449	-5.9952256515551	14.888399168848\\
0.342209989314347	1.28514270770334	19.4631396206949\\
-14.2964606460227	-8.85572425943662	39.8408436022894\\
5.49065472235685	8.04649914129702	18.1380720415024\\
-7.4918375743722	-6.2021866173836	27.5118857714881\\
8.37373348417267	8.2079447503229	27.094849899346\\
-8.77316290462712	-8.29807135238607	27.9576853645535\\
0.454979736781297	2.86123160378264	23.2888503488262\\
-17.0137168948082	-17.7300391252544	38.8046021885452\\
9.98304922270758	14.3640041268769	22.4866925364645\\
13.1701406081723	5.81703682293466	39.7780899335722\\
8.36074624317824	8.5069517812497	26.6539776859455\\
10.2412874574801	16.2047287395691	20.0102257118157\\
-1.01512873390521	1.77284528794745	24.5645805185643\\
-11.3945261994181	-8.75790753668992	33.5304086296856\\
3.83901456881315	-0.877033613491021	28.3619388928053\\
-8.90926242138149	-9.17785968206958	27.3196803934108\\
8.60632851080841	11.0166382672402	23.5965205089188\\
-8.36477719033521	-1.49636830135796	33.9357679970378\\
-10.5096219110419	-10.1654081035654	29.9765877824582\\
15.7542233946647	21.5295378122671	30.9919526306052\\
12.0347294913593	11.0236048516772	32.7718230291159\\
5.16018411411319	5.3565190219893	22.7876029975463\\
1.58475326765342	-3.88120823198652	28.091214432187\\
1.83042609557664	12.014618050321	31.7813841453681\\
4.11803743878168	7.24073403912906	12.2455351512803\\
0.0248892276738268	7.12610427779477	29.3461687844812\\
-7.20066389128995	-8.16016754575572	24.0047995362073\\
2.84225312475772	4.77405808231219	13.8741956049721\\
-15.3846350926998	-20.633077143428	30.9979084279989\\
6.54977945042942	5.62962039372004	26.0308243585083\\
8.61759030832111	10.3325514918636	24.7337368255031\\
3.2190831902	6.12296245156258	24.3866806563385\\
-5.87115193546564	-0.800259735026052	30.1808635153831\\
10.6165766775091	13.2788920259609	26.3476066588529\\
-11.6225978955145	-12.9145444306502	29.5733942173014\\
-7.21702318221189	-6.16727491607508	26.9142398229109\\
13.4523734083802	14.7097039894424	32.3543594403985\\
1.98000108642378	-3.41629391830851	28.2066140924981\\
-7.25583945681963	-3.8608057438989	29.6777649321054\\
-8.81264563841949	-9.27211078102847	26.7973928122087\\
-4.47110071637724	-6.61615632796457	19.7362726054599\\
4.51570698145665	2.04117966121281	26.2167128299245\\
-8.21168724160234	-6.46894490504259	28.8317581901299\\
-10.4610528096367	-19.3594843710427	13.9806025610528\\
-14.755565528153	-22.0838602193092	26.9056397785711\\
-9.22007001133611	-15.4325963659088	17.1521826863875\\
-9.27705655379629	-14.4645020759315	19.6073530253739\\
-0.782791434952404	-1.36821568576436	13.7477705343164\\
4.60490072523526	1.86666761523899	26.6301913485309\\
12.3329160975745	6.75125735279735	37.3209885955141\\
-10.4441454342369	-7.25101051195167	32.9017240034299\\
3.65987595092273	4.34705858370117	19.8706174638488\\
-4.71989047553372	-7.33808755273	24.5119042499469\\
-3.65590671473311	-3.99442618459997	20.7152717498073\\
9.19074673654541	9.94557797620765	26.8901982798113\\
7.91794925658165	5.96959601890521	28.7492723863296\\
-7.27564109220843	-9.02874310211544	24.5750526657536\\
-11.7150670187472	-12.1496800434433	30.7245214010842\\
-8.9140458614046	-15.3868405111671	15.7206386649094\\
-1.57319843370145	-5.3154121364507	25.5946311732187\\
10.5113582908569	12.3980963015863	27.189427668652\\
-8.6540943514798	-9.28692356277987	29.9977591181898\\
-5.03891241982306	-0.944714446784034	28.5445615906536\\
-3.44174557343907	3.51393285976037	30.3561615866078\\
6.9270715577646	9.65662145924992	21.1344377749954\\
10.3296561888496	10.7978715269952	28.7745511393016\\
8.64188594992799	9.61440086282271	25.8643873486074\\
-11.2355546189492	-8.54752888891687	33.3859228465523\\
-11.7943281266634	-5.48349627990004	37.2407602780123\\
-11.1189154452737	-8.38296991798066	33.2775877872813\\
0.499665993519674	1.23981259413855	18.191206885063\\
0.925056666925942	1.59957996807328	13.6041553416129\\
-5.12051871212942	4.26226798847027	33.3306387747137\\
6.87504029279577	10.2047892711378	19.3188267646612\\
5.38302534720742	8.71189445628875	15.5410585112673\\
-3.57433344245307	-6.95909747853141	7.38345567430493\\
-8.95064925995596	-9.38929803071427	27.0018212483971\\
3.55148964607904	-4.80980586945912	31.541213592029\\
10.774311774508	11.9728740556874	28.5223444610106\\
10.9349551846836	4.35001025120637	36.4519392754274\\
-3.67406193112581	-4.60102031993092	19.3163510486352\\
4.60408646252294	8.04655057295743	12.6483511632907\\
-6.36496664743327	-7.79908889674754	22.066896514846\\
-10.0965716859872	-8.04598911217974	31.3371985339549\\
8.13020079887372	7.76760130412016	27.0487113540373\\
2.38335467763542	9.47462417632698	29.6983559108454\\
-2.92779875905254	0.896387915978657	26.8526883063303\\
9.47127731797162	10.0735629937516	27.4619932551317\\
8.75589165960917	7.68252019317535	28.6642368752548\\
-1.31746150346555	-1.2530421073002	18.6352998671158\\
-3.92683149076818	-6.40963387044682	15.0902738656137\\
-8.18071167384552	-5.63078903219548	29.701176004929\\
6.2196899820738	7.29185321052906	22.531081006228\\
0.507664614345573	2.38577085152218	22.2173410551276\\
9.91945903513273	9.52670378406673	29.2800647286125\\
15.0414162357605	4.04379031767001	44.9108801164403\\
7.19462161567507	3.90846612931023	30.6751191004342\\
6.03214233096986	7.33605950138408	21.8605001111052\\
-4.29442281298607	-5.11563825149829	20.465725087978\\
-4.88670120016008	-2.24470955249185	26.7292823170573\\
1.41611929994705	2.77311223886916	6.61270663115613\\
3.21333156542548	7.89854430035436	27.1734332947409\\
-4.03059678022415	-6.11720803568134	18.5214413583283\\
-11.4283253391464	-14.6723131325683	26.709854078606\\
2.39133989467724	4.07015103174386	12.5995642335737\\
-8.88490045751962	-9.55311203749251	26.6075374503963\\
8.25661551280512	6.86460260677826	28.4772300124188\\
-6.65782976447492	-8.56068074820963	25.0394240635211\\
-4.23829767301812	-7.13500514161851	13.9887274359239\\
17.0807311978822	22.9224935126759	33.5197906371311\\
1.50070645915645	2.56496175279873	13.2920909834745\\
-11.196093729676	-9.71340418458495	32.1347271466435\\
-4.3238202378151	-7.6300229360606	12.0948212087478\\
-1.4095392060738	-1.54576349712645	17.9183642631264\\
-12.867619441276	-17.9295023556201	26.5540208681207\\
-1.88634299471135	0.898283260010892	24.928299723313\\
3.72280224414083	4.31311598719887	20.2084544929656\\
-10.8532678237159	-4.46260066970647	36.1799637079774\\
-1.94362231247548	-3.19782978649889	13.3686180140456\\
3.73282214054105	6.15526229874545	14.1241274381064\\
-10.2105776673135	-16.1754067120103	19.9422009142921\\
0.651592119395983	1.17145470128286	14.9953958116945\\
-12.3593346016271	-21.2606455183612	19.0698560888806\\
-6.56681919755812	-9.18745497027194	27.4507704611148\\
0.679907748541265	1.20432798129268	13.6202035241834\\
7.32100934352422	5.00756651281101	28.5529647640202\\
-13.2998467558309	-16.8420516138003	29.383199140126\\
-10.5800562139556	-10.0259948378065	30.3401285731164\\
4.35898700131249	6.64642521625675	17.9265282894527\\
3.80501034718261	2.23971234364556	24.4099958484344\\
8.84702530561094	11.4076103251762	23.7129500502547\\
11.0526274667189	12.2321900921406	28.8883247383896\\
-2.67168936175448	-4.49346267237314	13.8252291521637\\
8.72924566685283	8.89088767154573	27.0887776987261\\
9.66772838262517	-2.71050035354028	39.2351675570602\\
-7.34795382554193	1.11462281360268	34.3107231754672\\
6.69654676975568	8.73025286205397	23.2035242720395\\
-0.528959504720475	-0.963210178750227	10.4069854478032\\
9.22937686318885	7.17728524262906	32.9385620461429\\
10.0580034415605	10.3341448564495	28.651453057268\\
1.88686069452677	-0.195116845026306	23.862307563346\\
13.8223666905879	11.9006872079781	36.2015770521278\\
7.95244132840746	6.08662702779392	29.5475362151344\\
-0.993032969720313	-2.17592557666127	19.6386649439715\\
17.0534425121007	21.1252497006241	35.5112024020044\\
2.81213305690047	0.773550478630193	24.3692045743219\\
-3.03695247493737	-4.33600061057227	16.9106437840437\\
-8.73492068518544	-12.6996875912074	20.9883708290733\\
-7.89277136019936	-11.7987143167915	19.5343435755117\\
5.2287080668119	7.74256077251595	25.0659358959165\\
13.1239022750675	15.5268010376391	30.4943379149125\\
6.33555719353499	-9.4368784215968	38.12550637255\\
-7.5886380888227	-1.84663574769358	32.2396055473166\\
2.52903749965742	-1.09171603316686	26.3937822552437\\
-5.64924894403377	-9.31596878986257	15.1148473913503\\
-5.10588877790366	-8.64263942414129	13.9541141327792\\
17.8291407971758	20.2997144711181	38.4433347747304\\
0.696486306879424	0.981949918888994	14.7666890271457\\
-4.17922746875683	-6.5471578927088	23.4986958530251\\
12.2364647152986	15.0100387457419	28.6502899325635\\
-11.4715881550358	-18.0838725714584	21.3779905020304\\
-8.31475770771935	-10.4087323337731	24.2377346668978\\
11.3418873088727	15.1743380779133	25.7203912235641\\
10.003447416587	12.0310532424169	26.3733805169076\\
-6.82338831559592	-7.26721511117581	24.3480536060508\\
-0.464915210374519	-0.924549395516757	5.7401085658527\\
-8.31632131465274	-9.49821046244526	25.1253557252556\\
3.26042001059451	4.36437333944119	18.0538862200714\\
-10.4042945586913	-16.0638212941399	20.9336552014401\\
16.6844802222978	6.48820434499456	46.6491595589887\\
-5.95380406319949	-3.0859041223975	28.7325212397843\\
15.2359704400207	7.12835229290898	43.2737427134024\\
-11.659046224851	-15.3955493024229	26.376862920749\\
4.61524907507593	4.71353260466418	22.3670836925313\\
-11.757599335679	-18.5460048386627	21.6626359184472\\
2.48759602531813	2.62437791446299	19.6662096817492\\
-1.24445245721813	-5.801182116824	26.6935431528726\\
13.5850775600559	18.4812930384507	28.1200904531001\\
12.2507158821968	9.52647081377807	34.7045298415404\\
-8.11051881807708	-9.44339520059964	24.6191771783125\\
7.2272192929306	6.34266242834354	26.7057739319979\\
15.3752710212928	7.58822364461901	43.1707280288343\\
1.54430434170958	3.09013105103242	5.4759136291759\\
-12.999962128915	-17.1613664029042	28.0433250323248\\
3.55821929141922	1.56817639170803	24.8398415938938\\
-9.38019924886441	-12.2277473979134	24.0699960511314\\
-7.54423879641197	-9.05332590314643	23.5643859582222\\
-5.20730084160911	-4.90463646803633	23.6940443126155\\
6.65641905424053	7.65894861775137	23.2663372199101\\
-6.75804005929129	-10.5295424066812	17.5299241696729\\
7.7231952258107	12.6501481219323	16.7217121410519\\
0.995223942505174	1.96074935816855	6.25445413878624\\
-6.99450451744158	-8.8013193050563	24.6286221823149\\
14.0803422942991	7.62481301366671	40.310517167366\\
5.71054739372529	6.09950842224434	23.1080650960999\\
-6.92609436316121	-10.1195534992814	19.805728892416\\
-7.96586415773077	-8.25107309395038	25.9753122068389\\
-3.57254585548802	8.51427636465282	34.2865076152139\\
9.54100316712737	8.85581257202563	29.1483969174006\\
5.3253367795841	3.12264908892642	26.5507570075717\\
-8.3721668854404	-6.07847312664074	29.6331615214418\\
8.07386760266061	9.13884654717245	26.0796798375443\\
8.16603264824237	9.24256698220931	26.0186670892364\\
-11.9324294997772	-18.3608389425152	22.663758612896\\
2.2092802787805	3.60621818247381	18.2808939933295\\
-10.1614926623411	-9.91434260998206	29.4129805185114\\
5.13898963985159	-10.3813739043624	37.3041147808609\\
-7.8318106150899	-8.07698124670625	25.8678830074537\\
0.869170507847852	1.35110498470641	13.8327430877516\\
8.57033134646416	8.3460518487316	27.4589820149538\\
10.2477821197342	2.17113175260478	36.7985350398233\\
-1.14975467836484	-2.19451422189206	18.3127742265746\\
-4.91285716563723	-4.73527675030307	23.2032891218246\\
13.1293084773821	14.9919408924339	31.1282162730688\\
7.17227175485953	8.59395881505569	23.2012844917119\\
2.28735934300968	3.01472804350985	17.4082509022275\\
-7.52786626828184	-12.3711755174809	16.4594983846493\\
10.2990985346725	6.63428359816202	33.1697376133341\\
-10.1383673094273	-11.7170622222553	27.07136669915\\
-8.71416349585569	-8.96111588944523	26.9569683643341\\
-7.46638788773661	-7.19867676709342	26.1463818042215\\
-8.31710828859482	-8.2186494083892	26.9569809290627\\
-9.46534328972407	-9.65923701700061	27.9808209345767\\
3.67972260601342	2.6309433695259	23.4330232837922\\
-0.76349918623678	-1.48724209330918	17.1157525863351\\
10.8816918502039	9.15973911351083	31.9682021696794\\
1.76224497135879	3.77844682127209	21.8815191587461\\
-6.46556368405342	-8.4720119813162	24.0049398876646\\
15.281867543627	18.728312412528	32.8660777134766\\
4.67747809315314	8.15269126379434	12.6870675696374\\
-12.9865795802936	-12.676480311479	33.3755431882675\\
6.23485938626221	6.86948617524307	23.3854578490651\\
2.59201744461182	4.13538478445033	17.3626627435124\\
-1.47022158542004	2.7792234692602	26.7133425132483\\
7.97886903822477	9.65991424685198	23.886308117106\\
-10.3100319637818	-17.4778572548836	17.6039376604792\\
5.01520977193321	-0.887187222149236	30.2973256447788\\
-15.8647939997881	-15.0480843481794	38.2496305887945\\
1.17693759127404	0.70640458922053	19.6986652099898\\
-1.54741578828826	-2.62344491389247	13.908459853727\\
-4.38562483239328	-6.4193521709988	17.3879571058825\\
3.51271070501671	5.63733582963312	15.9037839161795\\
6.42026472152616	5.56839369110796	25.79785265931\\
0.997088076589661	-0.354730822145592	21.8229690854691\\
-7.24687186369478	-10.6526696368731	19.4124051411001\\
8.56494664641487	8.82822363726687	26.7483802378305\\
2.85803596299384	8.60360685764308	28.2444490361776\\
-6.01749572637155	-5.80657667816615	24.5837723471739\\
-17.7571301393626	-1.75200726619215	52.1063640146535\\
8.89391945998733	5.57791091021926	31.2677069656605\\
7.44667838446564	11.030026375357	19.3904904103289\\
-11.2764531971656	-11.0885869909368	30.8171166580517\\
-0.452966437174476	-0.979557545030134	16.3908505187601\\
-10.238450606727	-15.7823192783609	20.8108602035094\\
-8.79534562745409	-10.1158947061655	25.5675282876765\\
-9.39846584960761	-10.3001392420888	26.9690584152109\\
3.50531962156881	4.6870671045111	18.2567934739538\\
4.77062682621761	6.85834495680593	21.1828461493108\\
-10.0339694707548	-13.4837009982843	24.1470749357181\\
-1.03490367801083	-1.5600089581855	14.4611842909794\\
-1.46891899604932	-2.19917238331245	15.0043268697695\\
-6.86453641447823	-8.32967217817753	22.704254678403\\
10.0170132609603	11.3964889880337	27.1728677082007\\
8.08767669837492	8.26942163260711	26.2691823573193\\
-10.7676261870443	-10.4885899812711	30.2409146100695\\
-4.44340518593628	-6.55203397647798	22.0052665705396\\
-9.97977727763042	-12.535344685433	25.5689544798852\\
7.20194563757286	7.70149554872964	24.7267359505828\\
-7.17109036852288	-6.48735513836087	26.377416830577\\
-8.51371075706586	-8.85847056592439	26.9228737461223\\
5.69017119252607	6.38331840569272	22.5464720396671\\
0.890714614244136	1.54280688960355	14.1996676537691\\
7.09289607553505	12.5793738740455	13.3435914899581\\
-8.75296420064138	-9.12681055970104	27.1191299955201\\
-7.89237114423764	-8.05539270728556	26.0562779829371\\
-11.1767312406833	-18.3311532624847	19.6623132248463\\
-6.25777922999505	-1.22713893575514	30.4249956997376\\
13.7878954203987	14.332407415503	33.6485421963098\\
-3.61376735633216	-4.48765556258876	19.3680252062898\\
-11.5759085808846	-17.6717719247328	22.4792526420779\\
-8.06028746195335	-9.72886826469127	24.7845035524043\\
-6.0967146601471	-8.46494234758381	21.3812591720527\\
-9.51147769356117	-7.01802297223176	31.1117520492769\\
1.74956624554674	-6.00915147784208	30.4121992009591\\
-10.9835821742553	-13.264712108306	27.3523381345408\\
6.44280416604774	4.88397761533692	26.7594540766501\\
7.82413454045228	0.474991196844909	33.847186291301\\
-7.37183795761119	-7.69729716322569	25.1956011412458\\
-0.00508493692384954	-0.0125928845796288	9.05438223138892\\
-8.88229639556942	-1.2198895215404	35.0797071758578\\
-4.8694840255194	-7.82259727862986	15.4575362392732\\
-14.0410201043588	-12.5425382552459	36.1632045838004\\
11.7939712738605	-5.44755433789339	44.2866631364433\\
8.04692980418152	-1.36178768629222	35.6037565566813\\
-7.89796538691558	-9.49825270120775	25.0452154673569\\
-1.37042168400178	-2.70082920163409	6.29166460014417\\
0.566825651174148	0.721562205495971	15.183529539652\\
11.2426172853912	14.2585239452494	26.7485787019936\\
9.61754969236083	14.5961915726309	20.6740569107578\\
8.98064757298961	15.2357627709898	16.4540392521912\\
15.6901492151235	15.57734022757	37.2890558104979\\
8.80684882595259	8.93806365872987	27.2257029914535\\
-4.98642494647582	-4.01940066883231	24.5174564542465\\
-4.12525798940244	-1.48110983612721	26.1578072339006\\
-0.771984665850312	0.163469614341747	20.6436391323135\\
-0.0482851785483943	-0.14154563339036	13.3468924858458\\
-6.37202715332489	-9.9192274491073	17.5701558140933\\
9.17081890820497	12.5738613455551	22.9790084598436\\
-9.22307655455072	-14.657416612108	18.9175501466506\\
4.94611067977569	5.77060391940459	21.3329311006075\\
2.18954459410898	0.156306889242923	23.9367665422513\\
6.30221327684729	7.3231782309649	22.7338941357405\\
-10.0993091915059	-11.1902385585455	27.7642495410806\\
-8.23644915855476	-9.001310761238	28.592519294939\\
8.2637204628313	7.50298385823779	27.7123071148278\\
-5.42542070587421	-8.36700789777636	16.9025349236469\\
-8.17052103820278	-7.30757992723128	27.7380663330137\\
7.0488814937138	9.17625946307429	21.7460443415319\\
1.59220894367616	-3.01353981558358	27.1796522787117\\
-7.34186655066715	-6.05996689216617	27.339885193582\\
3.26183524767142	-1.81554137933605	28.416793998386\\
-8.7281299842228	-7.76200160159963	28.5033774151788\\
5.70963752666358	8.81233456691368	17.6191737675676\\
8.63562846624931	7.78762256389593	28.2533020847451\\
-9.50268811028833	-5.86417257319246	32.2517664751454\\
-8.88656838840352	-3.7918553912597	32.937804163707\\
-16.515866861017	-16.8397553922177	38.3474153314104\\
7.895288458051	7.9810041230703	26.1668844641818\\
-0.868686729965319	-5.28140468115871	26.3997067657617\\
-7.13284763885961	-6.94869967139874	25.6462168889909\\
-7.84941932162177	-11.8530373748866	19.4241424434135\\
-3.27569629752004	-5.70144582618155	12.1880193537064\\
2.49428847378613	4.51892414141884	10.4394106850011\\
-9.36565985349118	-8.68971706542374	28.9208990146801\\
-1.55101645618838	-2.7117961280623	12.170843411402\\
-8.20135581912831	-8.02918825903428	26.9470456254639\\
2.19979230992442	3.74341090258858	19.6778597923426\\
-1.73226117168043	-1.5417666229989	19.5959101963077\\
-11.1122076310243	-5.28422289787774	36.0395691550264\\
-7.8988940706039	-8.44598305502652	25.5373836107373\\
7.45289371067952	10.6871016577648	20.1759122582965\\
-7.35725970478403	-10.1296954066676	21.6058294593166\\
10.4827816072321	9.00449929832651	31.2179226865401\\
-2.61955104532825	-4.42322284037733	21.016909287959\\
-2.77048737941298	-3.1208066305959	19.486221743677\\
-8.69730995991494	-8.82041114587555	27.0997269954351\\
-1.68515978299296	0.960477544399324	24.6089841903789\\
-1.82700318608164	-1.73556990112339	19.7048482744894\\
14.5066614993302	13.8158520515163	36.0332803970688\\
-7.96437649088708	-8.09958711580599	26.1824679599751\\
-9.85841363509193	-12.5189584314784	25.1036766965761\\
7.31680799842639	13.8729618845306	10.4362800641245\\
-5.56250582673898	-7.66893810111072	22.3869754337741\\
6.8567894493146	4.15701151052145	28.5402541416804\\
10.569454081226	12.9721723420345	26.5616809436534\\
9.07825529986296	12.3040236840429	23.1465469846993\\
6.92898188535398	8.68542636168308	25.2170082257505\\
-8.36260146494001	-10.1800451867064	24.8027502753192\\
-4.77791175054699	-6.98595766970044	19.7259996171167\\
8.94545458019707	9.24563518152625	27.1785701529816\\
-15.318831378706	-5.44099465780816	44.5136494408896\\
5.68102931202057	8.01662968260179	20.6871838413312\\
7.88908464549375	7.58551609264529	26.6880112617566\\
8.20500215717276	7.8946862580096	27.0747944778289\\
-9.50106589387573	-8.67308258198783	29.2872779248324\\
-8.70714682514316	-8.29815463482749	27.7955774573904\\
-8.99994726747675	-9.94991058212895	31.8102653616002\\
-7.87401586098148	-12.3236414180083	18.3744102927177\\
-2.13879457920336	3.38952245383907	28.3936476973918\\
15.0387959616935	15.3236046279103	35.9171153435978\\
8.740295288491	10.8197077863581	24.7081146789676\\
6.688543442494	7.49917229473326	23.5966538779116\\
2.80980446715782	4.40162692155079	15.0011549198315\\
-8.38724764575973	-10.8403139147373	23.6094510601029\\
-7.22423836353994	-10.3763225869627	20.424486705235\\
-4.20214185337811	0.293754707347581	28.3840607840417\\
-0.583125330134173	-1.05388599910851	10.7906018480595\\
-10.3039104291033	-15.03236735768	22.4518859461683\\
6.92720126577922	8.86489957395213	21.9256887647649\\
-13.6376209335624	-13.4101006253433	34.2602177570336\\
2.09193471088803	3.54832005759367	13.4455470524356\\
-7.90094637863526	-6.74284122513134	27.7938982140059\\
-10.4495061718356	-14.2746948809526	24.2307936098076\\
8.20388908325624	7.98493685599208	26.9606374564502\\
12.1471515951315	-3.54712900976439	43.8080927235231\\
8.86002419919356	7.33265707675154	29.309993598406\\
-9.34127383252759	-9.90313664967766	27.4957192496011\\
3.02150651397936	4.89479524108713	20.8542652247329\\
17.3954924686784	22.8377285760706	34.5966223829106\\
-2.04724143562531	0.216239467622673	24.2124225725018\\
-0.513290891027789	-2.94551362303019	23.2811342952218\\
6.97470791599836	12.4805843747082	12.8909332700488\\
2.5679029171426	4.25813847172055	13.4454644634089\\
9.14179017842606	9.17911954392994	27.8532849432994\\
3.37342779586213	6.35205639440329	8.99084090581858\\
4.71907235214083	8.03146812040986	13.7470283886646\\
5.67710510274232	7.7951331825335	23.1521119705103\\
7.10147133898073	7.41870316348134	24.8782977307584\\
-10.9597319997804	-10.4374013504994	30.7766005009115\\
-4.86119852721953	-5.03859767307486	22.6145197527365\\
-9.05166642694945	-12.1361169690095	23.1515111451425\\
-11.2924919473854	-9.58306259698007	32.4797019599442\\
8.25073316546554	7.19302050907394	28.0641690922592\\
10.9304346354849	11.0139097800426	30.0779083334594\\
6.65282656004586	9.23963006803462	21.1045288286775\\
-8.25633711968404	-10.0630765453196	24.0742814126305\\
-6.38804803318947	-4.26280178095149	27.4234031036556\\
-0.661819868202999	0.91945616814714	22.1677029976743\\
-12.2602022110935	-8.63515626398817	35.5380421968997\\
8.46703517687311	12.4660640265046	20.6071872367081\\
-4.38344995997736	-8.26971024995932	9.20841792151167\\
-14.1531974191906	-20.3040121722456	27.4248867137495\\
5.17307448545665	4.17598588770893	24.7431274279704\\
10.3705072442895	16.5444830586124	19.8475617854171\\
-10.0434163294364	-16.4525319364965	18.6444025711543\\
-9.1468853312097	-10.7303675161464	25.6687119654001\\
-7.70932437352461	-4.46051905095241	29.9681977715873\\
12.5530546447572	9.81657892586041	35.1372445570062\\
-8.03298426161553	-0.0159288703141272	34.5937694222353\\
-0.397124330004972	-0.666925864556227	11.7688204860046\\
15.895216658165	25.226663223763	26.4342540694151\\
-0.0658605394829657	-0.272184367546932	15.2655702707626\\
-2.70408669666106	-3.81574429764362	16.8404671436921\\
-3.0719314988798	0.553878133411367	26.6985241546927\\
12.1946907528536	11.5443109248156	32.5770890306626\\
-1.48245286730093	6.77372000215651	30.4918191915646\\
1.4860253858386	-0.838484685506264	24.0110884700715\\
6.68621713457841	7.83236932891297	23.0216565097135\\
-2.57726891012038	-2.57571693501212	20.1567516658325\\
-8.36553320286939	-8.55485751828083	26.599958603766\\
-4.20282569145595	-6.58868182006683	15.7123156250907\\
5.20227399569472	0.00675526629918496	29.8069638701353\\
-7.39757293387954	-10.9344326624744	19.4182032359036\\
2.37680343604637	-2.44212741140245	27.7544807094259\\
1.91998322071935	3.25152942980832	12.6076310496852\\
-11.3482973182171	-12.1854420689841	29.7429228157322\\
6.9201226129596	9.3481470731835	20.9571820125888\\
-6.50858472568717	-9.00184232185728	21.260372012329\\
-2.85347923095925	10.477957852998	34.351198451234\\
12.9950911223543	11.9918838822538	34.0554224298201\\
9.01454952117888	9.58801842147686	26.9046787848555\\
-7.69928476657793	-0.0403849868151342	33.9534454985171\\
-3.07077646279799	-4.42418194953886	16.7883481179398\\
-2.25001214681163	-2.64101499513737	18.5479229838788\\
-7.50753291743311	-7.50805614412959	25.8224775435019\\
-11.9025875533732	-16.8822861398468	24.9504072540913\\
10.0683711680248	7.64572326414382	31.6847849918854\\
-1.99798680218957	-5.31110659382035	24.8960442758304\\
8.21732461889885	7.93314503916616	27.0599145158902\\
7.88761870293103	7.45707981542375	26.8532090971197\\
-8.36920030862186	-11.5775935255698	21.8114631328726\\
-12.2717850614102	-16.0526492601297	27.3498810702387\\
1.50536091351119	4.02965207413595	23.3495821364288\\
7.41284072475116	10.5951320565102	20.6256664358797\\
3.30426052562957	2.50594890705109	22.662118533061\\
-5.95119580119802	-8.15355465657434	25.0197829630333\\
9.06187774083916	8.75721162533538	28.0974316814148\\
-6.49521543718046	-9.81782533376203	18.1389235666362\\
7.73904612293132	10.6042009793983	21.4233983935326\\
6.60620796641935	6.74307342459081	24.5667685228869\\
-16.0763080526938	-18.9774659885972	34.8306018711885\\
6.82843311838049	4.60716787290358	28.6006555392081\\
8.81051520821618	9.90373599841018	25.9158884696675\\
-15.3776727051497	-16.051472960476	35.9889147631555\\
-5.7778166229428	-8.2869100650669	25.6982343294946\\
-0.675588526096577	-1.47624631512677	18.0108468862725\\
-7.54261675817296	-10.3894585995168	21.1406488828242\\
-12.2353683229429	-14.3095050833518	29.5371674420967\\
5.54048550591641	6.38430871511346	22.0705414660036\\
-11.1079816602991	-12.8451519478713	28.2536074300656\\
9.86051557210219	10.7449248600947	27.6128287996681\\
-8.22162453209813	-8.38529568416682	26.4590958796419\\
-1.90782166395272	-3.53261801853228	9.36761366988148\\
1.50171723398312	3.26259107005938	21.4423042079455\\
7.86682139574773	3.54954781379122	31.1936739640299\\
1.20533406534544	-1.24499487268812	24.0437557484172\\
11.729414675964	5.15250896060814	37.3756480954052\\
12.3455495767317	19.0026365215495	23.0848150517567\\
0.176599218272656	2.74863688377608	23.7156239908518\\
-7.9136090241105	-14.1255060045425	13.6011697798895\\
-7.20537766936811	-11.2918017316042	17.8682248225776\\
-7.15478974387271	-13.6966006296091	9.85876611377953\\
11.0513451834086	18.3776843329505	19.0384094995404\\
2.84042169686346	5.36676510770291	8.65784485440421\\
-11.6656361823691	-11.760633317932	31.0291531522206\\
12.0869615364985	12.002323778258	31.8327576679872\\
-16.8581683138081	-21.3827499084662	34.5853483343995\\
-1.46210154109531	2.0953080078651	25.8070101804636\\
1.76761751501455	2.93691237379576	16.0021377456828\\
15.2318493830981	18.7848719642335	32.7360773431446\\
-7.75620471863177	-7.29003761870013	26.7486923093708\\
-5.87216852774008	-7.90561758881931	23.1366675036529\\
0.704353064564797	1.37475034106227	6.73023218027967\\
6.41439041919208	10.2218111028939	16.8336551306599\\
9.71370739652685	9.13849239572393	29.2314769765017\\
13.555390107115	16.1584420311775	30.9711316613607\\
-9.90582134374854	-7.77990694225785	31.1889441441781\\
-12.2349795196571	-8.35722042329192	35.7678791802932\\
9.44126857325518	14.674446948205	19.8060630449711\\
10.3713456424071	8.37739692310289	31.609621467069\\
7.01373379639754	9.38224149753895	21.239466451354\\
4.55323596675604	7.74862364891754	13.6910558497393\\
0.454332821723006	0.924883840443869	4.41506674948568\\
8.94105111285294	9.96175456621726	26.5519630752457\\
-3.9763156692779	-5.95754991798222	20.2441541836167\\
7.33806199271978	2.43977029232859	31.2596665689165\\
-9.50192871563781	-7.66673740690678	30.3973350436913\\
3.86634314654658	7.37789733133736	8.49230348585206\\
8.69282086407382	7.20507755034938	29.0772343796292\\
-8.39539932242773	-5.86027413937444	29.9122464191606\\
-10.5186061296446	-5.76665007166228	34.41688515468\\
-11.0724434041196	-9.78738205929737	31.7489235251126\\
3.86383776126876	5.80801982180444	16.4804351223947\\
9.79155845660761	14.3022625261407	21.8977215327281\\
4.04090014153045	-0.496499742609222	28.318845357716\\
6.88188380665587	6.45865095051908	25.7010453230619\\
11.3704411562879	8.61778474949145	33.6168230676274\\
-4.77527811566819	-8.25979610665213	13.035431206411\\
-8.11862009001628	-6.82909297496248	28.1958724704292\\
-1.17170890061253	-1.66419570883476	15.4337496653483\\
-4.89963581704333	-9.81644994207013	6.30110114195005\\
-1.2020434002897	-2.38623620613875	5.86668356111888\\
8.42708359189635	10.0392989669955	24.6229625231294\\
2.83699701258814	-1.56584562983939	27.4981953831204\\
5.64672665115024	9.14553682928137	15.9277428719893\\
2.78175518895884	13.2919752248791	32.6373086251405\\
6.26696122816869	11.537456514794	11.2262945014991\\
-12.1051422611806	-21.8548298813862	16.6771280592536\\
1.99206277705597	2.97558392713426	15.4120059414727\\
-0.550633027705011	0.159348991548001	19.8203209066559\\
6.66615195767104	10.2266860023092	18.2091352229653\\
4.59645922333868	6.77601711690699	19.5296954940085\\
1.79890760680615	0.215735817668044	22.9148804756229\\
6.68906685536178	8.58265174554993	21.6811810170728\\
14.1016907312884	21.5691811715983	25.4668851557345\\
7.01031211860873	9.06026014169544	26.9796336250679\\
-6.69177246083476	-7.359177679097	23.8325820927169\\
-9.16390967808985	-8.21502528629591	28.998043534059\\
7.92412542850455	-4.24378981082012	37.3913392179837\\
-10.2305985343115	-10.3916776132517	29.0187441099507\\
2.07221693018558	9.95168795628916	30.1466963791551\\
8.17636234446334	9.82420210919091	24.9909422445037\\
-7.86014381980115	-10.5111220006825	22.5667232898732\\
9.88438395760603	8.09430146912877	30.8200032013144\\
3.82381461156324	5.56381759256988	17.1108491996172\\
7.62913834910635	8.7192326309623	24.3556746818842\\
5.26532340901937	5.09473479683131	23.548619061626\\
1.02613102394575	1.6126276392005	13.8030399921126\\
1.68652792922434	-2.76420899937145	27.0029109220054\\
10.1980303634849	11.4694844694167	27.569403026832\\
-0.221690722764847	1.85414621983144	23.0688781763699\\
-12.1829132798645	-7.67909616477674	36.2598588001168\\
-11.8785436908748	-10.6665401721088	32.7506826076077\\
7.98933067221025	7.46992666969766	27.0880567457659\\
-5.18158195272223	-7.52947067858948	19.7255790401148\\
8.33869333704443	5.76494728763388	29.892156826409\\
-1.68462202784055	-2.44424858919013	15.6451553123205\\
7.65616545900131	7.86423643786516	25.7078128572568\\
-10.7791546567613	-14.5076680651666	24.9899385858674\\
7.15925878634654	-0.32764643698253	33.362042272678\\
-11.0200147572386	-9.54103820912335	31.890020551872\\
-7.96796932673271	-7.54811695148157	26.9357261228862\\
-12.3557292850699	-6.15433067870281	37.8341063803692\\
-3.85382246690763	-2.54918777345623	24.0325476316993\\
5.72638120636625	3.0689907404861	27.4523841624648\\
-3.60406822699958	-1.63024599161865	24.8492976865678\\
-3.35898663136191	-5.19695911351778	18.9322340492452\\
14.0308945433719	15.0949221458483	33.4480562749763\\
-5.8486960066753	6.25849216676677	35.7089210311681\\
-0.674216388184524	1.19841650250281	22.7514644965761\\
-7.06663323101243	-11.5102865108782	16.5111655243623\\
5.92618794452646	4.35310164280735	26.2795310572895\\
7.93966395398973	7.51535151494299	27.0702730032821\\
4.3063803679354	5.268998845839	20.1698602060389\\
7.45727849273343	-0.578112017117437	34.0211736503545\\
-6.41487443672148	-5.3498517980607	26.3881009307251\\
5.44656170403189	-3.55743515689527	33.2762158464987\\
9.29803825061609	14.6006789877882	19.3523679362443\\
15.4228639894679	13.2500458692567	38.8226736670846\\
-7.50913563386728	-7.19193227362688	26.2655860251251\\
-1.32205743639489	-0.104166642685441	22.191472956422\\
-9.18249018311539	-10.4219635512569	26.2114328636951\\
-12.9874017945235	-11.0836290103719	34.9658845617023\\
1.19473171573387	1.54929871629923	16.3378614798202\\
9.08656405387758	4.11000231450088	33.0504581877142\\
6.53720146737696	-1.77440197717975	33.4800554598387\\
3.27245066607438	2.07540471621698	23.3664737720059\\
-2.62521873395869	0.454151536495302	25.7402301740448\\
7.69256114251584	8.98182521446017	24.1242941527276\\
4.98249051529294	5.89670452467289	21.2066438208124\\
8.18710174094222	7.925402174869	26.9841564367757\\
5.61167224440914	7.09632375641564	20.8934741292567\\
-5.6452830261464	-5.27550286560112	24.2904857172428\\
-4.71901006910739	-7.18070823043871	23.9856033425552\\
2.10876830247575	-1.91510952582721	26.6957264925699\\
10.067025917478	7.97318955488017	35.492451723383\\
-7.94606138826127	-8.63564632237217	25.37140927047\\
11.4110687973456	16.1524349184668	24.4644363757213\\
-7.88049249094647	-5.52610252766651	29.1774893984436\\
-5.51455970200261	-7.18743412015872	20.3391545264978\\
8.36506354924647	8.95678690783505	26.0513941631707\\
-8.16271889459859	-2.80820497249341	32.4687198018412\\
6.93159275900656	7.02262492254826	25.0146560052595\\
9.25639428995241	10.3627828413891	26.4971058455603\\
-7.52632196301852	-6.12925198762398	27.6810466497075\\
8.41806701495387	14.0539485156283	16.6007384746256\\
-3.01329068335887	7.44245829249111	32.9075216317385\\
5.81851827858974	9.22863410005114	16.3700072825232\\
9.99342567780146	4.81873076982758	34.1920252395645\\
7.65127195443693	7.92286916881287	25.6116896849517\\
-8.30322125198465	-8.69955462613418	26.2392752194488\\
-11.4705681508574	-12.5959093983678	29.5965142452121\\
8.73599306195819	9.40440727550988	26.414171289707\\
-11.4764785206615	2.75741060596597	42.2561394920875\\
-7.65925728553782	-8.25106533046611	25.1765240894159\\
2.0074735183924	3.52415666715856	11.8489793289182\\
0.0244741921123528	-7.6893265856673	30.0033507804838\\
-8.45671572287548	-8.28696605144922	27.3927348164417\\
-2.21917051471745	-3.12625347087835	16.5102905422555\\
6.76180975920175	8.65993695670908	26.0345257590225\\
-13.1865183450577	-11.3072858929318	35.202794071002\\
-4.44462606428164	-7.97522663029641	11.4529875794094\\
-4.29613101241092	-3.31130852904782	23.8908251635744\\
9.32088610529123	10.5947343007174	26.5443948315804\\
7.72390727035655	10.9790573612229	20.9657841794547\\
8.51657144333698	12.4026589876633	20.7552203462767\\
-1.46429677866117	5.96531782025942	29.9778827544396\\
14.5455137213906	12.1237735909033	37.6246818115098\\
-2.73036025052471	-4.36551246327167	19.3170341704241\\
9.09349235204277	6.9954824550756	30.2160404315846\\
-4.63904592695273	-6.77989883153184	20.0220006456952\\
10.2022725136021	10.7187377058757	28.5474989360454\\
-7.06235907455771	-10.9572191942078	17.8664749650127\\
5.34054784743633	7.46939554137237	22.0752768701714\\
6.2090573255334	5.59775389378337	25.241241927626\\
3.22224394216506	5.43509297427589	22.7194886838845\\
-10.3703923926147	-7.58063067502462	32.4122613075515\\
10.4286711008593	7.5938961462886	32.5303301891581\\
1.88324230996198	6.97303978925909	27.2546639391127\\
-8.44036732166676	-9.54655763375534	25.4037187049742\\
-0.419448185919101	-0.778401350404687	9.37276455394865\\
3.18826259277266	4.49383927276664	17.2463719168233\\
-10.6310771346372	-11.4332309142775	28.7681567134349\\
-4.34647727438961	-6.42668379981823	21.0638278651159\\
-8.6402893325421	-10.9812622408529	31.7268847293544\\
10.249784604124	8.10492913868239	31.6172738611735\\
7.98667420045344	3.03283280039818	31.9237831732922\\
-3.18427305548989	-5.61810115915578	11.8020651050775\\
-0.340189961925248	0.067401178835074	18.158964975371\\
7.13518643913074	10.4273702601468	19.4679950942222\\
-11.3031013871637	-14.5353102402792	26.5325960771881\\
-1.64963090028747	-2.85407661958464	11.8517429340319\\
9.20229977614482	8.63005547989848	28.626518472526\\
14.9532413520634	13.8125999406745	37.1106156852323\\
-7.10086384127541	-8.94378781984036	22.3591624075126\\
3.54727623811471	3.64780170965367	21.1127178051115\\
-14.3288930264569	-12.2722799223645	36.9685881093146\\
-7.78201854211983	-8.11733459587254	25.6784405537061\\
1.46796335590876	2.52040555097718	12.0252945843137\\
-6.03181394496959	-6.23292357126575	23.7994698720578\\
9.14015169740315	7.4431871872553	29.8274013379248\\
-7.14421889221225	-12.6115133793078	13.5374652824513\\
8.87641016330896	8.82643862033553	27.5469391472492\\
-2.92115629716807	-4.92077935779483	21.6353539415551\\
-10.6713683645214	-14.2669949900186	25.0324899696651\\
-12.7307093662555	-16.3680476859858	28.3048328434703\\
2.34735648809858	1.08045370544162	22.7263719441081\\
6.15475045431948	-3.07686218487624	33.8769877109795\\
12.4119726219014	20.1722856927443	21.3062064632789\\
-5.73877382284313	-8.05700424814803	19.2004262663059\\
10.60468292903	12.4584271050997	27.3665001294021\\
-2.28657277879179	-0.72298098449601	23.1919288670295\\
6.23906438931712	9.16339371169066	18.6339874241854\\
-6.70503151190521	-11.6932367085657	13.7558163578145\\
-13.3681800282705	-14.2552329357579	32.6237409286316\\
-9.10429639242163	-12.6833456376423	22.3742010767064\\
-7.73568880246215	-6.08650533156091	28.2114305862803\\
-3.75936684690403	4.96164830884238	31.9795507682952\\
-9.74229914221353	-15.3444286371266	19.7027875611036\\
-20.1033885929191	-20.3262740626252	44.8742449980081\\
-6.30794209917903	1.37109846740331	32.8328592892836\\
9.32598291887361	10.6768644293597	26.4772425947421\\
-2.03411548675128	-3.29472421189974	13.8012015844541\\
-16.4212181690255	-21.0447053593492	33.5078891638898\\
-6.24933320256627	-6.57721809495706	23.8452357804583\\
17.6139174900524	23.4713543903298	34.5559191792796\\
12.9487218310112	18.6357036993736	25.8580463172993\\
-6.13321157846266	-5.95412032681951	24.5080271991848\\
9.1461187847354	12.7798563539876	22.3479171219521\\
-9.12232023624912	-7.28292929536649	29.961432227566\\
-8.88714097875771	-15.1835216874001	16.0930561153259\\
-7.88179008014385	-9.68520836532468	28.7550434833288\\
8.65550398610125	10.3170537715132	25.2963667407547\\
-9.51517486740296	-0.921056543020308	36.4283038785516\\
-8.23926207670174	-7.43185700585198	27.7423723530213\\
6.24703895809572	8.94128271923884	20.2338867392674\\
1.11517038045751	2.05857678378861	17.7427426172241\\
-6.79667502015834	-7.16458482963348	24.5015133856025\\
-12.8786168391304	-11.1364819923804	34.64307500024\\
1.49631342744807	-5.08522763831396	29.2422947916672\\
9.39564476412863	12.4127834482387	23.8164260284908\\
6.64636127268179	10.1378239509441	18.0281752510588\\
3.47725857456849	5.66651379120152	15.1142246660231\\
-5.26487182138153	-3.71003447265813	25.6312267487655\\
6.44613234657475	-2.78409448242388	34.173679301193\\
1.21933908149575	-2.42425771614248	25.8582293443186\\
-13.3456790985352	-20.5772777057319	24.1663993920337\\
-2.32974635554086	-4.1127062813762	20.9767493632341\\
-6.48282046630498	-10.6698376941724	15.8054588587002\\
-2.98085541635934	0.358587078861021	26.2768942193939\\
9.86196138479503	10.6480735233983	27.7419302598929\\
2.59256172965741	-2.10459553408196	27.7242406752539\\
6.55540294600621	8.47761538098288	24.0059002481509\\
10.6787229007519	4.4534088515047	35.8433704973778\\
8.59628771906699	10.5847915977651	32.0012500074775\\
-8.8699223743748	-7.15117922668542	29.5485593985103\\
-3.70512601331066	-5.65860562710549	18.6064372485016\\
15.1789185064399	9.74613175983606	41.0945718995481\\
-6.09148823846317	-3.31438588158872	27.9107467922243\\
-1.24049902092913	-0.862479607672023	19.4662220661058\\
-6.49268173610324	-10.4947826995494	16.4436594746466\\
-7.29052463317428	-7.12539602562451	25.8037785987684\\
-5.66009662883566	-7.64744769343336	19.8574196207862\\
-14.5284542166888	-6.73584930974331	41.9228043854864\\
7.23009212445108	6.87639015442691	25.9967535724046\\
2.05017928936627	3.41012536456453	18.3387937265716\\
3.57273380079504	12.5994485157837	31.1497776808971\\
-7.23062042911517	-14.2545378471559	8.35590415300975\\
-6.96884371601737	-8.18958462353477	23.2696033167671\\
-4.9935788638067	-7.18757959150523	23.122081546197\\
7.10903641893748	8.76151839519177	22.7110118775267\\
6.46069513451473	8.54445346435493	22.9419264558992\\
8.82170281455028	11.4785125037799	23.5133279762108\\
-15.6702246772495	0.228436502388897	48.6271876495112\\
12.6012181926428	0.667057061179501	42.2657811881953\\
-0.351711725142283	-0.64319852186249	9.51432668869372\\
15.2252301666633	13.9730370358876	37.68012557284\\
10.1759926696038	2.8434719102408	36.1583755433729\\
1.06840477953061	1.75804066512028	12.9186223594339\\
8.95682252856149	8.22724714944107	28.4900002122295\\
-2.40060178020019	-4.71231761065092	22.6940095343941\\
11.3401780322145	7.33500917405875	34.7260370497034\\
7.19559243000633	4.37771869154916	29.0023306401197\\
10.3297236244503	8.23758660731913	31.6643011538033\\
6.8506901065726	9.25932840844584	20.8800072200134\\
-1.91900699703255	1.40611398441546	25.7141764891532\\
6.32732024646596	7.36486625780522	22.7918997871349\\
0.470685801071556	4.32228714499222	25.746526921515\\
3.59916371002145	4.04784830492125	20.3848754446015\\
-3.85185388181922	-0.795059527891594	26.4853289389918\\
0.650220343223648	4.33502225209321	25.388047914338\\
10.1063388583803	12.1911582620487	26.4385198902625\\
-9.50494676071074	-2.07112523566833	35.5571409629167\\
-9.29376059020085	-14.1312928953905	20.3870547667276\\
-4.70395700004278	-7.05020547719409	18.4605468214088\\
-8.18813888065588	-6.35877004973978	28.9076683089368\\
-7.34272178663443	-7.38940276530451	25.5622622147108\\
2.88857245281937	1.58602785300736	23.2106855241912\\
-4.91368413247089	-8.99047693392846	10.8278247117984\\
6.14273402938897	7.29666069717148	22.2806315911077\\
-5.39207302481174	-9.45313349941816	27.9279241596099\\
-0.272317884188853	-7.8607918207172	29.8232056094429\\
9.83261524324583	9.64450724360869	28.974654344797\\
3.37076158070874	4.36742768015224	18.6068314229417\\
};
\addplot3[only marks,mark=*,mark options={},mark size=0.5000pt,color=mycolor3] plot table[row sep=crcr]{%
-6.68248402217724	0.418580881245173	32.6636987626523\\
1.42099250326743	2.5550499295224	18.1819406812415\\
8.6112861560789	8.28110537683832	27.5895308974211\\
2.89323554373703	4.61625434752299	16.706458530789\\
9.63695840084795	17.4443854026759	14.3036398046818\\
-5.34723151475163	-9.48533763653296	12.3431412973682\\
7.47745660031004	1.23020256734148	32.6131919264191\\
10.5068781181761	7.74778009842723	32.5481825255953\\
-6.35757686538933	-4.09801222567507	27.5498026719583\\
-11.7950246763179	-7.55761239374758	35.4888965691899\\
-4.78784533043059	-2.4215102521827	26.301632770994\\
6.20302544919518	12.0008736036424	8.80111089418398\\
-9.16383315111279	-17.8285430322565	10.3651164211624\\
1.64932023702286	2.4396733353609	15.2363352863776\\
-9.29204934095615	-9.63737998519655	27.6778321044598\\
-1.77985705268615	-2.50398491924451	16.0884405690503\\
2.7367077541059	4.23012790950267	15.2055628764004\\
0.407237718892129	0.379797744669318	15.9516584133125\\
-12.6160786224861	2.19929771123866	43.8577828299639\\
-8.19742107436969	-8.43241456415042	26.3318366721717\\
-8.8842789302838	-6.24125219546057	30.5617555334424\\
-16.134927832482	-7.80410192574787	44.5501626831132\\
4.52645941768571	3.13895285567241	24.7180495837243\\
-9.64651247541168	-10.8462386495495	27.0468254512813\\
2.03120024086554	3.34698501715046	13.3866750039134\\
9.78300382059857	11.0915581667477	27.0784805753947\\
-7.7054307239529	-4.98043894495649	29.40864909639\\
-5.97767736394026	-5.41499430767789	24.9136395400078\\
6.1807214956476	-3.3223448859776	34.1527417157658\\
7.74999269909002	13.7169589484922	13.8591601280456\\
6.74712089490201	8.58533553118218	24.999653587222\\
-2.63478422486944	9.06005388928746	33.6825735819865\\
10.1218091088985	11.1910509783954	27.819001729998\\
13.7513647732959	10.987393670324	36.7817790636889\\
-6.88129908141511	-7.49482335635912	24.1459024362664\\
8.84297220855303	7.71264971106583	28.8295853298073\\
4.5159136983912	7.59797770241696	13.8621224017095\\
3.97148784478232	-0.816450354878136	28.5630730701096\\
-2.63059013091069	-4.24189977331405	16.2371114328445\\
-11.2012988938697	-11.5074379839841	30.1445490990112\\
11.6363169504495	5.93792057129562	36.5394927319945\\
-7.03741681097189	-1.4920552172506	31.5975214856588\\
-2.37049873735512	-3.5496073039263	15.5946404164874\\
8.52781783355279	7.56595116661127	28.2669875176748\\
-3.73992340687471	-3.06700754009523	22.8222933729769\\
0.629226593129635	9.47586075489457	30.9734296649271\\
13.0676571902285	11.6514249868946	34.5608764195569\\
6.01138297077094	8.19054080249011	22.200342271565\\
-2.25022253469796	-1.82346720288354	20.8081255958648\\
-9.9579734914907	-11.058291074803	27.4689714582631\\
-0.133904750924525	-0.241742096654397	9.43504053485965\\
-5.2688115397131	-7.81751101922614	17.6944346127043\\
11.8838144987846	13.6529601209931	29.3670015925898\\
-7.89513939348228	2.34167747224887	36.06987676872\\
-12.0476356234161	-18.6385264085942	22.5907089675952\\
-1.48388216376242	-2.57253709255734	11.7220640104593\\
-3.30215604481383	-3.02118201801516	21.6196275872143\\
-13.1976143969998	-12.1608728893298	34.3738170482457\\
4.65755994967147	3.92342912895582	23.8228762816388\\
11.4880394964726	14.1384199182365	27.6299785135147\\
-15.5652048072274	-18.1828135903634	34.2590117605357\\
-12.9409154610834	-6.43311344156574	38.8041045190415\\
-6.36635666612655	4.67000097966428	35.3668027738994\\
-1.85874612462863	-4.54655860078553	23.6631451971838\\
11.4571953003734	13.082802066338	28.9072549364018\\
-12.0393253107022	-11.7487951021665	31.9872677407698\\
-0.940017272024001	-1.66397333677956	11.9869897787084\\
9.6338251335548	9.96654788254997	28.0233373499234\\
12.0399416496438	14.6745886314512	28.4998025999431\\
-4.7876058610899	-8.26905181702871	13.0995749647411\\
8.94668832658873	7.04773817859524	29.8358536560483\\
-8.81194984023024	-4.48516400913032	32.1597717939\\
5.49437402197277	9.48137334722071	13.574393122342\\
2.54717853081991	-5.3963565423479	30.8408226600879\\
10.528334766474	0.197592621602683	38.7090997188124\\
7.99997078997832	7.77180612802464	26.91821457275\\
1.31022027767648	1.14528770102472	18.8612717307059\\
10.4741686697755	12.0624218245233	27.5331572178837\\
-11.0639703357248	-9.87793098336509	31.6385146586587\\
-6.04121940823296	-6.08542646130534	24.0513613072865\\
7.50660289253801	10.8778272092195	20.3880679727185\\
-4.88874648617157	-8.05576704736038	15.0501978599821\\
-13.8243805615596	-18.1557569180251	29.2546735869617\\
6.97571104268019	11.071514054407	17.1590432877641\\
-8.06912864326596	-8.22625606866744	26.279028477096\\
-9.17841064151174	-8.13234196710609	29.1424394160407\\
9.10252750913155	5.57514844615029	31.6929233237096\\
10.0923688778247	10.8800320389626	28.0510015271136\\
6.35897984185393	8.08047515516495	21.5011039308209\\
9.84312271862558	2.26245912247768	35.998260373816\\
-9.26172296484361	6.49349339365262	40.7125837127737\\
-17.8229897027902	-17.1163682456531	41.3126863087437\\
-1.22791018189443	-5.03536044174989	25.6366144166553\\
14.2913345708317	10.6729108322559	38.2870400830493\\
5.37534357000266	-1.17336926665191	31.1893058140522\\
12.4464292859031	12.1613670629711	32.5288817131808\\
-4.96288983717622	-7.44331876615182	18.5661539438926\\
16.0318813840065	22.2534031210066	30.9127194510796\\
6.1806829485406	10.1987557310557	15.4271954744705\\
-14.1392420301327	-20.8903467656543	26.5155556949752\\
-7.90547469267008	-11.5833378461403	20.3348737659667\\
6.76189769521274	-2.13431114792646	34.1508293666899\\
-2.90392595847212	-4.82327447314485	13.543787847712\\
-2.45967232411746	-4.48355642760785	10.2924850388247\\
4.92051528778568	-0.0238823563721078	29.3494448600062\\
-13.8588848687048	-17.1157588105326	30.6641519248483\\
-6.04479766012674	-6.08530999062744	24.0565011422056\\
-7.92968855515906	-7.24312154689467	27.2403533343598\\
1.03044746919851	-8.37537078195163	31.5784446291125\\
-2.0875450281132	-5.3066107041693	24.7334389842949\\
8.45576102236985	11.9460143761762	21.4304926748967\\
-13.9182311473954	-15.1402199260641	33.1043201012593\\
9.50944550134126	8.24413585250418	29.7640509945342\\
-13.505275600835	-1.90790486834953	43.059382598231\\
15.6585899349129	16.8999346553597	35.8542039515794\\
-1.99724976053439	-2.625102104318	17.1004832768965\\
6.84383096546994	10.4715029200756	29.8271153240014\\
-10.8093562280757	-14.7513501808155	24.655988917229\\
4.28205277247402	6.4922299077285	16.5696915134467\\
-9.63541806314309	2.41527735295636	38.8991718191919\\
-2.07300463938218	-3.41589875977188	17.3849499110427\\
12.7770990727742	11.566556025916	33.944903346326\\
-2.21405449069068	-0.886617272751983	22.6817749213192\\
-12.1740769198855	-8.51665328160172	35.4578305647353\\
8.04041329869239	10.1837702155055	23.2053527178596\\
12.9495355771784	7.51203615936825	37.9721265254242\\
16.5942680443262	14.2508776247678	40.682866229941\\
-8.62788190255937	-15.7752240605985	13.0568036816493\\
-7.52149104865083	8.23174982333827	39.1347321676076\\
0.58812194989098	0.769646432233088	14.9942283179537\\
-0.935453885844853	-4.61069411433343	25.4413035506872\\
-4.49382419422255	-8.3611776347642	9.89541992339192\\
-8.56663016457757	-8.80678184721733	27.1114204797353\\
8.12380664827124	10.5038951202918	23.4846079574067\\
12.7199853049179	14.1995890612557	30.9645023033112\\
2.6680030292247	4.2985890728855	16.2969571196116\\
11.896047219057	7.95593549385477	35.349888532964\\
-9.00818013828307	-10.8043582843235	25.1643630107055\\
-4.08540632022986	-5.95571029777076	17.2474816210963\\
7.03589551277887	9.46721585637946	28.4222111720835\\
10.8630840138267	2.34471933703951	37.8107781651056\\
-9.6980846411945	-2.89346571863769	35.2344783735933\\
-1.75487153637747	-2.93946034629958	16.9016890742583\\
-0.546574198402382	-0.544756365717659	16.3367039776688\\
-2.15210437968252	-3.38253055313099	14.5124384829992\\
3.8851171401653	4.73002110348435	19.7926660429819\\
-6.25689791336194	-3.20767388715701	28.3912234078568\\
-12.5495142036007	-19.678979005717	22.700533222874\\
-11.9381058865228	-7.38368492668707	35.9422273782959\\
2.2646372142331	3.97861203117945	11.9158530328208\\
-6.73874841396445	-6.28630326991842	25.5793282568196\\
-11.2796906094929	-13.8063312935832	27.4734540129607\\
-10.373938083599	-9.13575371367478	30.8000581616419\\
-7.97460724160793	-9.78127724608907	24.5369303631681\\
-4.31335454971648	-3.2840562625585	23.9770339598514\\
8.58694047443707	8.9194912184542	26.6790603472983\\
9.18836862372706	13.2520396485668	21.64149056921\\
7.68302176688958	7.96336824862146	25.6377223123633\\
-11.6042831513247	-10.8650780429196	31.8748122735451\\
5.60213056922358	9.23079153656338	15.110594920599\\
5.09024126856528	9.14920845271414	11.6863827481692\\
8.64226460799389	9.24857097278234	26.3836565879663\\
-10.1855522647666	-11.174820198427	28.0061699003252\\
8.34226524918011	1.74538883787657	33.7037659364154\\
-9.79680842201433	-11.9096542323664	25.8284522756179\\
-5.16728106474711	2.09823469075322	31.6729578694653\\
-3.63661499457825	-5.87052960702191	15.6497672557121\\
-7.22412828033847	-8.81271048031739	25.6248778677896\\
-4.73716986494977	-9.51102198580459	28.5635867446299\\
2.47251386756941	-8.63732753895019	33.2161104627458\\
-4.82977756117404	-4.64273328511996	23.0940435022978\\
-4.94063906880238	-5.60679789759904	21.6420411566858\\
3.12809106511878	3.79826525581107	19.1072390775973\\
1.93627792953721	3.6896435129087	20.9577526820196\\
-2.96519042707719	-4.67125809876797	18.7050916322513\\
-1.35084218831772	-2.10485366139137	14.1276386229445\\
-12.7292776280339	-14.6906287443035	30.4312038026932\\
-8.45905840519017	-3.3148696753807	32.5521600952726\\
-2.82020076723011	-5.29360440384443	23.1999484430032\\
7.6709357941316	7.82006470906565	25.8134011447554\\
11.0877952402439	12.6772781962924	28.414319586854\\
-9.24917419734075	-8.55938004371549	28.7909726095935\\
-9.76550435573044	-8.80181296099207	29.7455083644755\\
-12.4001328055258	-7.06039074751007	37.1914220879783\\
-9.67567945150268	-13.5924840989056	22.8776733425086\\
1.60821084921953	6.0113926593703	26.511423693586\\
-5.42381885442375	-4.02506181730208	25.562824832069\\
6.7883017394437	4.98168274966717	27.4155811968796\\
-2.82929746812418	-5.08887294660473	10.9368812779037\\
-6.55179844883417	-8.58717550190293	21.1643877777373\\
8.69577244884728	9.44167653944922	26.2579983641095\\
-7.54927120955186	-13.703316879861	12.6225821583206\\
9.08459944881078	14.9828340071075	17.5745434843683\\
6.4756823580134	6.15917352651777	25.0899436029876\\
-3.7727921118046	-4.48259211305363	19.9595349441334\\
-6.12002350010234	-10.6912350960155	13.3689649099754\\
-5.54936365590732	-7.72470990881188	23.8392807609104\\
0.357503256459109	0.628984303977944	10.4213974950543\\
7.95744156665365	9.80598859829001	23.5842939210954\\
-5.4519410475695	0.78014862122983	30.9652188224654\\
2.10794704414124	1.3645640118132	21.3894983406439\\
-1.69297654282871	-3.01062714073995	18.9348975423597\\
2.42814484727037	-5.358282801527	30.6633070611498\\
-12.6639115624605	-11.5180914148732	33.7215454942464\\
7.53235206516722	7.24477688272406	26.2541963197535\\
-6.20806699798049	-10.1563234776072	15.9162347909868\\
-2.84542279868726	-4.68978078240424	14.7577544830017\\
-1.46707318480704	-3.88721052674274	23.0984912958658\\
-10.1360986284951	-10.3416886359566	28.8357858764907\\
-8.53140251073936	-10.2628488759984	24.5853743370464\\
-10.7190760472132	-18.2114092945619	17.9177143780245\\
8.54823852056767	9.20625654987653	26.1892339509779\\
-10.4402172195354	-13.1804586055919	25.8869917824462\\
14.9144928591948	18.7966738849182	31.7140874341609\\
11.5909563349844	13.1135364860592	29.2765185742338\\
8.2116363382428	0.881460044789648	34.1776005484565\\
-2.70330792658981	-3.98277147021685	16.1151504318411\\
7.49993079128414	2.48965383305278	31.512243930259\\
-14.1141833171447	-17.8687067335071	30.4734058163696\\
6.12944564632039	2.07291149127954	29.3701166372773\\
-8.85748233165057	-7.91803457540031	28.6381212568044\\
-10.7072849918435	-15.2873428226992	23.437543699244\\
-3.36859150674849	-7.92930293788143	27.3157762853927\\
1.349865699818	2.46619641525726	18.2530284437775\\
-8.16974503495591	-12.9430422606564	18.2285163582961\\
10.9705303334435	15.8355338464913	23.4415765334297\\
-4.69469685861143	-3.74642449475881	24.2051944776307\\
-6.34578856428651	2.12576595179588	33.4975583368216\\
-5.01076394979246	-3.89284507227085	24.762598514687\\
-4.52628174281907	-5.26342783020692	20.9371422379248\\
0.796801660008453	1.36988984205046	11.6472243569897\\
-8.57456025218562	1.86556638862118	36.8159935881433\\
-6.99750959983414	-1.25832708557226	31.7384900959035\\
7.64536204064553	6.08990109388239	28.0049388130346\\
-7.6911903900253	-7.07691857068751	26.8621860132926\\
9.27863151618967	14.1263223847311	20.2673804755676\\
-2.49573604661794	-2.19616856568037	20.7747919086557\\
4.65358336143011	7.7568573197869	14.2400564110709\\
-6.73119449621201	-2.6897120440068	29.8819573698694\\
5.41089349069728	8.15595556660318	17.4278071482549\\
8.30282584670308	5.24190217766927	30.3844630814593\\
-3.9736946015347	-6.03945329426272	21.532562700985\\
12.5453817820647	10.8896616352595	34.0685644116387\\
0.64103500289147	1.13936340197	12.4287063309643\\
1.97475387146813	3.0817551295023	14.576921860337\\
17.0138919109724	18.4661136915402	37.9397434175999\\
0.769317993110868	-0.133332788485352	20.4800985601689\\
-3.30458184420307	-5.45442784460673	22.1369479168625\\
-12.285336765317	-16.5221935685113	26.7488945307971\\
-1.37355583219878	7.58573199559269	31.3105175592646\\
5.24416018787762	10.8458921613644	4.57269496456354\\
8.74593775647261	9.79958300481469	25.8898203092386\\
-2.38746221294056	-3.94279532433469	14.9270967016608\\
4.55957364747337	8.13883665995867	11.8009647036661\\
4.44877159779638	4.81403551070679	21.6162036278001\\
-9.79777285630554	-13.8382600072599	22.7562752403546\\
-12.3522516949172	-14.1099673206257	30.0816390871187\\
-10.253570046815	-7.77325726844514	31.9627482555301\\
9.48511392004547	9.78232948836384	27.874489869906\\
-10.202728573173	-10.7154093428066	28.5380759647388\\
2.13130626871262	2.14326890869371	19.5012040029486\\
-5.54766898829434	-10.3302378131629	10.3451120847197\\
8.37283167177663	8.46530501105049	26.7341243647296\\
9.01723135876535	-0.097116825308717	36.301733653455\\
17.0502782496007	14.559829441055	41.5446418812735\\
10.3547420553868	15.5541789702665	21.7354219360202\\
-15.5350732517947	-10.4238588947746	41.3066573430766\\
0.491339275536748	0.472480298119954	16.1887002189206\\
-5.74109542668088	-3.73098800764259	26.6797412249969\\
-11.410949142562	-8.8328095684633	33.4924748651082\\
-2.01854020943041	-2.76773412700627	16.6292692916\\
5.97374260822005	7.55829455578713	21.2044935227148\\
7.23483467746615	8.4020523946716	23.7067144422541\\
2.21619263104985	4.55717651594727	4.08381355630283\\
-12.1346717964971	-6.00500315901188	37.4903828736979\\
7.39146610966334	6.38255777005833	27.0494873888046\\
-8.31751621903243	-8.40262214292178	27.4538822232859\\
14.7735669807874	19.8575905585945	29.9776594335721\\
-7.84457574726602	-7.33822530852912	27.2471133496542\\
5.78328833503618	-7.88062474204555	36.6136555364128\\
-13.1463562405035	-12.2428753809567	34.1618237380496\\
6.61886862392926	10.4237577910201	17.143548104975\\
-10.0310682726627	-13.4169568622074	24.3461565909352\\
5.21310075980958	3.27124450700163	26.108373796786\\
-2.44788905502282	4.67338513146181	30.0827660272927\\
-5.04650818957431	-7.81509083482352	17.3558196374283\\
-5.78227872410129	-8.63720074898954	17.9086118622738\\
-14.3601650816993	-6.80825200167777	41.4693321535379\\
-17.4485439193708	-23.5158766061985	33.8603786371329\\
-3.46710933483123	-4.18866580426825	19.4900751348501\\
6.46879021289237	8.67148487693282	22.4173987386758\\
-0.581195481326	-1.29352503439603	17.521915820483\\
-0.258386626144192	-0.481701333981142	11.7723949294262\\
5.44711012740377	3.75079172361811	25.9985518847367\\
4.36925085741249	6.2127588538023	17.9077221249894\\
-7.80884856341371	-9.18628289740955	24.1309759747886\\
-1.56571053468882	-2.1934804993614	15.9192398853504\\
-7.93443336869771	-8.30842241312487	25.8183172437701\\
-1.60555603649985	1.68208271242429	25.5094630176266\\
4.72386895165939	7.57588973219947	16.0197323370258\\
-9.3890081592823	-9.42358913372815	28.0816920751934\\
-0.0638385928115518	-0.134929208781972	11.6429657605381\\
-0.249453846978461	-3.8417390300164	25.3819802462143\\
7.66792351124812	8.51985809444106	24.7782701677663\\
14.4088060413173	14.0928287901215	35.4406444486973\\
-9.27164207800393	-2.90947246780276	34.4328279542391\\
-0.398895665603289	-1.59572274174581	20.209205672597\\
0.414376890953756	0.0840452120024127	17.8564419105937\\
-8.54754976320895	-7.49961677758283	28.3960969693275\\
-12.4159792790397	-17.8498466074695	25.1880282255907\\
-8.02382045791872	4.80419438896315	37.8680036406849\\
-6.50896239336567	-9.12808721650068	19.8691773641536\\
-7.65772872107785	-8.249330142683	25.1504638459236\\
7.21083343356152	9.27922809318012	23.4168893237355\\
-9.08519142246832	-8.8255609211209	28.0752612263161\\
-0.364682212626838	-1.40555392371053	19.6518061529665\\
11.6708550115218	16.1070218460136	25.3576355044038\\
7.75705483961254	2.01471801352049	32.4162933741958\\
-7.74926554375974	-10.2830962970442	22.068491363214\\
-7.68051820960764	-8.43680598153285	24.936875234188\\
1.34360251464209	3.18407773754471	21.6137864946827\\
12.8419410784336	18.1228084847207	26.2072300094954\\
1.97126176001666	3.36793583925581	13.2968715542241\\
5.4037812984172	8.0619288166411	17.6701346601173\\
0.49617493219342	-5.42185756032552	28.3654499563399\\
6.73459347157256	6.78611761630381	24.8386549097677\\
-15.7150588168866	-11.6174423754677	40.7741021549942\\
-10.2552749976398	-11.7196406315123	27.3917735897201\\
5.21739230201574	1.85146679397876	27.8727244996096\\
8.6903926760994	8.61797449709659	27.4183108676438\\
0.487125654041615	8.50484569480585	30.2792275982751\\
0.211463440578538	-2.96993726521472	24.8509540765825\\
4.48654420260062	-0.441585277017022	29.0407444595708\\
2.97955356151624	-3.77474688865912	29.9773498017466\\
5.33277535959821	2.70298235260923	27.093168353958\\
9.59217461064303	16.0676579619237	17.4367125883675\\
-6.80679305037758	-11.0616205413141	16.3347796532727\\
9.45915530492381	1.23981643769465	36.0754860020044\\
-3.76718108087166	-6.36967803102833	13.3563007499149\\
-7.77363693749448	-8.87056312310352	27.1791465285773\\
-0.314542537941641	-0.5777751389661	11.2733008494361\\
3.06888613934174	-3.51453897260553	29.8623756675487\\
-5.22694579459134	-7.26450308214434	18.9903840077376\\
-6.57933564742345	-8.06047771330806	22.276242590143\\
9.27473942848759	16.6350267284358	14.4153793668376\\
-4.99035397994237	-7.12448672973865	21.7993053905516\\
11.2248778014036	11.8023894291133	29.8914046575955\\
-14.5286962809346	-8.36378036966782	40.673246047439\\
8.37298051882834	8.48350046322022	29.2018864279859\\
-9.38642006359568	-10.4555089793535	26.7232611913744\\
4.88982219448221	-2.07487580706025	31.229300898183\\
-1.73507586058634	-3.37702744387741	7.00011713545111\\
0.558495658593	1.01896701261001	10.5239338445829\\
-3.78527288904398	-5.57735052584589	16.845111451267\\
-7.79623055352387	-8.09653603201138	25.7470084447439\\
-8.42127271905666	-10.1522422719762	24.4264812715274\\
-11.7494165048752	-17.3558936537499	23.6510946066101\\
-15.2285074524471	-21.9330134378555	28.73342856931\\
8.91478469447489	7.56960180188642	29.1699427166329\\
-7.5217534431111	-9.70428773730215	22.3621225814393\\
10.1621103942725	13.2776257916224	24.8752628468831\\
3.40762109699943	7.00193240843515	4.3602668132143\\
7.92502116764741	11.2162000941121	21.2603403890585\\
-9.09617297025513	-11.0035584972231	25.116959373688\\
-2.62217816556981	5.68484658652003	31.17322125192\\
10.0609022429684	9.97812585807411	29.130077117754\\
2.22095399754462	4.26883684089265	21.95002034496\\
6.1371141889239	8.89745195392273	27.1573965423698\\
-9.9832302022341	-3.72002013872699	35.0979701027988\\
8.70440627996667	8.86621161482051	27.0577306954155\\
12.0795586874238	6.97734197588233	36.5834579654995\\
-9.16463136298831	-7.49203205667812	29.8343358511866\\
-8.81188659688256	-7.96213428895929	28.4631238631985\\
11.4211794058627	12.3471132040635	29.7597500936139\\
-0.428300080249522	5.84973103603147	28.7287238861051\\
-1.49333427617033	-8.79662843708902	29.7128522373124\\
-0.497997170161944	1.01161588620414	21.816811048082\\
-7.8667567284222	-5.80987715344436	28.8219118980454\\
-1.6350122334683	2.48645369017316	26.6271211149452\\
-8.02617598176377	-6.45732769926704	28.4308373751477\\
-8.72520620185514	-8.74253454153171	27.2728909876861\\
13.2881624364023	11.3695722793743	35.3558207829315\\
-10.4377254384469	-6.24432502202783	33.8211254878896\\
7.33893003484488	6.88796894293471	26.2544185424509\\
-0.92510113981631	0.890095855022987	22.7416812553115\\
3.95379920217571	1.09802975007552	26.3046755773001\\
18.0580217415679	6.03288788975582	49.8924073334174\\
-10.3472812207458	1.36917908536524	39.426886864864\\
16.6234402927713	6.87428765844167	46.2369950606452\\
11.1275833732616	11.5900379447523	29.8624505089943\\
-11.7658794100088	-5.82494257797696	36.8861262163782\\
-7.25110857520357	-3.9692467397803	29.5593621771337\\
5.46830494337343	4.92121276321948	24.357387914593\\
-7.58653388042472	-10.450150668243	21.1763514470463\\
-6.04939245604476	-10.2975085871394	14.2805480360909\\
-0.675630640975735	4.56382993453468	27.6625526945546\\
12.6545013413909	15.8938275852741	28.7040991684742\\
6.83120297786448	8.90139041331719	23.2418903873672\\
15.0297555259372	5.70578371248293	43.6368874309964\\
7.57991433660724	11.6927044613918	18.6434974504953\\
-7.00093706793918	-6.20115620407029	26.6673470942291\\
6.95226231506684	12.4010698229142	13.0067349110051\\
2.38094679708515	4.14818583080638	11.8505204160476\\
8.22641017193197	6.1049336824895	29.2761245549624\\
-7.62554885588428	-11.7308255446349	18.5635398940594\\
10.1219183207451	6.75018618222534	32.6880891175323\\
8.3564598023638	8.47061866138991	26.6925416642696\\
2.58129179913765	2.00921293864775	21.4857017004639\\
10.9863942643434	15.400727551501	24.2537594336313\\
0.810157614740611	1.63530345640063	17.6249790695257\\
10.5196276287761	7.66239983036228	32.6561907588673\\
6.08032799449006	10.2182661735914	14.8905919950226\\
-14.4341293028085	-19.3709159391746	29.554941796843\\
6.13981147640685	6.63815606198358	23.4267580579045\\
-1.56801643741345	-2.77203513843667	11.6260191096927\\
-7.20846525443161	-6.41266623744822	26.5683811017096\\
-3.86486858392885	11.3009359385236	36.3363235756093\\
-8.39203228822348	-9.26019655322552	25.6914562535084\\
8.32630965141229	7.82678221845815	27.4609955833258\\
3.84608311585635	-5.28862175447242	32.3980902068495\\
-4.84653881161129	-8.38410521563011	13.1821563057942\\
7.87653661742924	6.68907946337775	27.8101352030516\\
-8.82932870455038	-8.20475686781945	28.2094140292575\\
7.36009957579954	6.90410366027928	26.2838176794553\\
0.0596834155145695	-0.224111310559262	16.5388688216326\\
-17.1350214130268	-10.9326155030224	44.5818671956402\\
11.3032821869421	6.34673027209117	35.5139121273072\\
-3.49062913199525	-5.67381267752295	14.3994108012738\\
4.30695623325303	2.41152052805501	25.2706852994656\\
6.89277787929386	8.30689494457403	22.8268254951802\\
-6.05887981079782	-6.06961276705626	24.1239984936023\\
4.64320623598591	5.4688001812325	20.905669440695\\
-7.50597881157564	-11.2874047726908	19.3347229642705\\
-2.59571848640226	-4.24902574751605	15.2955003807008\\
0.875575602998511	1.82015899815321	18.3181681370954\\
-15.5464883053572	-9.80583039401703	41.8125433731397\\
-1.90884428013668	-4.58755162574534	23.6444987294622\\
-1.17259811822151	-0.243486635669135	20.9481413889359\\
-5.43951639878218	-4.07875331176577	25.5173789061426\\
-10.5851713558893	-12.5975686897962	27.1264464917193\\
-9.73881820678446	-3.21693513307302	35.0484559793867\\
1.01122251315587	-0.934455907862066	23.0561420982367\\
15.6868862027153	13.4253578908298	39.1899984867071\\
-0.0484334066020775	-0.451339445665152	17.0043232933376\\
3.69760108270428	4.97357954080548	18.3077081375906\\
-10.7772247048192	-17.7979255718306	19.0335888536017\\
-9.56342609178058	-16.2156885417137	16.956115655291\\
14.2572863810121	9.95324008602719	38.8149257058186\\
-3.21776660533187	-5.59677119019154	12.4734657548817\\
8.37970144024952	10.1470663530808	24.8991478660124\\
-1.64208880589785	-3.95415242963002	22.8033824115403\\
-8.25810782657662	-9.25719817564562	26.197135703742\\
-9.5386332660476	-7.9782007746978	30.1376093144138\\
0.819967698933732	3.2560888265968	23.2630593832758\\
-1.24454742269934	-2.50502217578697	19.3993270487215\\
-3.34041960263717	-5.39806050388325	21.6665904684709\\
9.02948421550317	11.8340509934235	23.8392357459962\\
3.35477072148094	0.814020132761288	25.4707992435092\\
-8.43091643113829	9.43374753373037	41.034246520795\\
-2.31623356527241	0.0929645794326834	24.5913124854492\\
-5.44014256513491	-7.99176354727377	18.0217341324173\\
-8.28755775728202	-9.22943055088089	25.4426610555655\\
-0.0246468418090687	-0.0634211137747758	11.3804956437436\\
8.75042635157351	10.0914478247851	25.4774767622071\\
-8.64051020448849	-8.20189238331334	27.7523394751383\\
0.899823930326818	1.49564290402825	12.4491433979014\\
-2.68106351864837	-0.86409352820196	23.8999070178459\\
-9.68965277629521	-11.7619522173634	25.8987091986326\\
2.94903645082248	-1.63466565855605	27.765727582968\\
-7.78813298984868	-9.28809942616513	25.1959652503219\\
14.1751815839818	11.6566092306378	37.154406697859\\
-4.85592296983627	-8.15594788268466	14.3384853836995\\
-9.01270935550989	-5.87761887571788	31.2077802642233\\
-6.92719636878637	2.49406345970263	34.6687642227193\\
-16.3472110940949	-11.9660567429364	41.9675318728061\\
13.2534931086147	11.0956737963613	35.5241696253579\\
-6.62081710368285	-5.21792281874643	26.740196685478\\
15.9341456591858	24.8742213765387	27.0967326416491\\
-1.47684496408905	-1.99079043667079	16.2750993224328\\
-7.4918466812456	-8.09869525990561	24.9161988771571\\
-7.33213790829909	-9.72679241337006	21.6761003330601\\
-7.44753205353605	-13.5416454443753	12.5061322078403\\
-7.38932575486332	-8.50755110530357	24.0023846659009\\
-1.0862809283489	-3.36865185924336	22.8584565966275\\
3.90295881199608	4.89563249636039	19.4535685707708\\
1.78294531227648	-0.383672581592849	23.8904518901622\\
8.00176077655615	7.19423147042521	27.4769544617146\\
12.7613174651007	7.33348911583773	37.7123492130482\\
8.60780855139833	8.50162786436012	27.2865660185839\\
-5.61484769291478	0.0798473965248066	30.5863348913323\\
-5.88671571588234	-8.9389422767079	18.2538865259625\\
-10.3368601214491	-10.6631222354378	28.9537269479937\\
-7.80111416650728	-14.2027186189613	12.6623072916078\\
1.11069328705364	-2.51164463305863	25.7696053383531\\
-6.69186504551174	-5.6963376969838	26.2809278620417\\
1.92108100909032	3.95448437094809	21.9643412895948\\
6.92633719738882	9.96820046517021	19.6524148899888\\
7.7805039236024	-0.340611425754882	34.4027537933696\\
16.7603642248192	20.6337057855962	35.0032941615881\\
-4.92169102897594	-8.09674888814426	15.1218419226313\\
-1.49854989668472	-2.47438307830705	13.0407900707823\\
-2.90376345333417	-4.58696672739591	18.3647707031321\\
11.2085360703164	17.4531954759699	21.4985092899796\\
2.58412174948307	3.46363316968187	17.4465568170969\\
-10.8423280531724	-12.571484978917	27.877296436262\\
-5.25057700646542	-9.41401498789793	11.9086480852359\\
5.34010554327213	-0.88727339785806	30.8814925020228\\
9.20161167651042	10.549018421276	26.0923201601487\\
-10.2646994006062	-9.06006641376903	30.6306409356008\\
-7.15025287635902	-12.3899245868801	14.3046016640158\\
-7.73055030816754	-9.64692001010803	24.1275077934506\\
9.03467272175726	11.2032900629189	24.6221669368337\\
7.47342886615721	10.6858770537053	20.2675253486799\\
6.53403029134883	9.53678758662939	19.6945494812916\\
12.5956509789463	1.36255423987556	41.6785526036763\\
5.96756378139404	-1.03497811349796	32.0232641517459\\
-7.72839842012065	-4.63749659787599	29.8244182171546\\
-4.47564041404256	-7.10093652886943	15.5579038280469\\
-12.1778937271391	-21.1503826883307	18.4482058306846\\
8.30947383577791	10.1935460615037	24.0340482556163\\
-7.96712008994948	-8.91401078980987	25.0100836436298\\
-12.9697700548474	0.668082245057382	43.6165663395063\\
2.37235637934657	3.98916943344811	13.9428586500471\\
10.6852142209282	17.3147111607248	19.6228477759094\\
2.34297691203007	4.45744511554156	22.1303599372844\\
11.520757605652	13.5183189578171	28.5331904954755\\
3.1977798660434	-2.77330325692127	29.3427614770208\\
1.07634225390996	1.82983625790596	12.1114318569484\\
6.63520895188424	5.08349258724576	26.9424414146538\\
-6.50714239058623	-11.824683161335	11.9347352066367\\
2.9958593561016	4.76167968766437	19.8243688545225\\
15.4491946335566	14.7868196397575	37.3817571758005\\
-5.93400697721001	-4.59162178491386	25.9790874651913\\
-13.7056349690202	-15.0076915987786	32.6767467385075\\
-7.18747118164338	-7.51482134783634	24.9686139435401\\
9.7305376441432	9.42564156892335	28.9305543445392\\
10.5880328251998	7.93175665001127	32.5422860852161\\
-8.34229373753613	-10.1187574539303	24.248079432675\\
-7.58793296593545	-11.7360392035777	18.3768595412121\\
2.70852409204875	0.598031431573778	24.3851304668707\\
8.46018428958691	11.0599794738086	23.0461699198868\\
-6.29318254724093	-8.36217876998479	23.1125673438568\\
11.5275804356465	13.1384712139825	29.0404965145574\\
2.05330783559506	3.48628964736531	13.5462408282297\\
13.8522643395274	10.5535908712444	37.3915944663597\\
-3.4379422295204	-2.2251031119016	23.4850764075605\\
-9.43015888277183	-14.9349098399456	19.2592385845124\\
-16.0296080201761	-6.56734188968091	45.1685996710782\\
-3.8854300404279	5.7376365404096	32.7917710758041\\
12.3569296211036	13.4018311532191	30.9320391339204\\
11.4750090583475	14.1709395464605	27.5431472355645\\
-2.70971604824481	-5.29835261525231	6.95295974395964\\
0.0107790618851484	2.02304543220627	22.6561770654726\\
7.80635336270594	7.80231950802822	26.1854578685435\\
-14.8712354732253	-3.86914105978849	44.5044391041426\\
8.72803957698025	8.65954739507943	27.3840456194084\\
-11.6192349827554	-9.57146685633356	33.2267037805055\\
7.38440090832349	9.3298681138924	22.5865167699993\\
10.904705286634	10.5232170653273	30.537069156333\\
9.48685627470365	9.55030603930006	28.1783805107077\\
2.83284106562577	1.49643691498121	23.2004273783111\\
-8.45187781795695	-16.1036494325011	10.929832906892\\
-8.49118634075592	-9.06576342211831	26.2243716403412\\
-7.48211300741964	-3.10070060158642	30.905538554399\\
-1.27042458092502	-2.85095092300553	20.7791237395915\\
9.96660948537703	14.347200465794	22.5271525080526\\
0.982334078488465	-2.65856008553607	25.7650308850821\\
9.24446777384474	13.3398242461226	21.6773378608548\\
-8.69970259647354	-7.73479032448084	28.4794545460602\\
12.8924561726474	10.5699377489225	35.1842240720915\\
3.72727177942244	5.30892329973949	17.4170289221155\\
-14.4642169126118	-18.7107015446878	30.4683181567613\\
-6.77340331560469	-7.94766505842142	23.087146399586\\
-3.46420610332095	-5.45744570183624	16.8481551682889\\
5.20888421733338	7.71494143074578	17.6924804552097\\
6.87663352045609	3.08372343652959	29.7470645141772\\
9.26110391267885	10.9127166039705	25.9820831628384\\
8.52865044836461	10.9173866639852	23.5248854231581\\
-8.49170793984358	-8.6999285834468	26.7340552689618\\
-1.53929478282518	-0.285644017522396	22.0145271236499\\
-1.8976850849173	-3.82542438206083	21.6133324620833\\
12.7722093233468	-1.27751970831063	43.6192065492175\\
-7.74531642190738	-10.237944788813	22.1424628033744\\
2.84041600927316	3.18795954017367	19.5510086422689\\
-5.36982665268399	-4.22035652292736	25.1605293503862\\
-8.24857043431764	-8.68166751635017	26.1158128964562\\
-7.95269330581794	-5.98231188218278	28.8166547094732\\
0.0654822982461984	-3.89737679735163	25.96116051658\\
-6.11754785367069	6.58677714236278	36.2706191415892\\
5.22783888828371	8.54685866785053	15.1509202696015\\
-11.2594139985699	-18.6130186433229	19.4768494233266\\
-13.9036728616933	-16.7135899523871	31.2674508069793\\
6.5880975380258	10.4200900977326	17.24850579246\\
6.92435628203654	12.745341132933	11.6264122334342\\
2.81421513915262	4.52561173500191	19.6607368930781\\
3.70897678540802	-0.764778018340361	28.0683706556915\\
8.07701466481078	8.05226518506958	26.5337076206768\\
-9.03698961412296	-16.5607538561222	13.2552601138446\\
2.54415678886759	-2.84235681606487	28.4457557415177\\
13.9117315133756	17.7725800038966	30.0032449113467\\
-0.568886530181744	-1.04716938347113	14.6376559777032\\
7.98595764560638	13.1769232755493	16.7132016403307\\
9.29213301107139	5.80082527776555	31.863239542277\\
-11.9092304326987	0.32561306183705	41.4880747680263\\
-5.41556002781094	0.723621490055376	30.8535889361732\\
1.5976220918442	1.70944330301781	18.3545334656903\\
1.93958850288561	5.61475809309538	25.4936585424501\\
0.125037984810065	0.0851568164052048	14.316145600131\\
-2.86480163993174	1.30525970816918	27.2488744179917\\
3.39660802472114	5.89050993021577	12.6568867325063\\
1.71020504192904	2.91555506645026	13.4487998872496\\
15.372135383322	19.4523517624576	32.2841441545955\\
-5.60475809570457	-10.3280744665141	10.8424246435532\\
-8.860133932708	-8.98961333614983	27.2963074222458\\
2.48284741212477	-1.08580908055007	26.312160790106\\
-2.76876979060626	-4.72715626836914	12.6627699571293\\
14.4539710628303	11.2652139349252	38.1531322874144\\
-0.713984358846475	-3.15509426690333	23.2966468425984\\
-12.5384618779776	-11.9627351043796	32.9786286991429\\
-4.58307872167381	-14.4416215565192	32.9877115228541\\
8.35405678278897	8.63702545978921	26.4498703084031\\
-3.16297493477989	-5.95668378542787	8.943991493588\\
6.89856312532278	7.03534900593775	24.9071493989239\\
-3.25129024854704	-5.53387778355942	13.3726285593172\\
13.9913081242279	11.7651948460387	36.6369256738376\\
-13.4876299428508	-10.7490902525547	36.3874325690322\\
-1.33092098526768	-2.08090997943213	14.084593729173\\
-8.61067159745087	-8.02578725119747	27.8975846275862\\
-5.03574947458335	-7.55569945448414	18.5234499363084\\
-8.64377797998627	-9.24770987347887	26.3784281010924\\
-1.75371706663988	-3.10554203821583	19.0486477877395\\
-4.38256584455462	0.141864075796748	28.5446173845918\\
-3.98027389035363	-7.71648525232489	7.73194008752226\\
5.0323034428208	4.17087146856157	24.3898369462628\\
-15.4723121127993	-12.6828440868982	39.3215457205198\\
-9.27572910384895	-8.71893511999638	28.659646122427\\
3.1485304549785	4.8195072676462	15.6552328598072\\
6.39985078845191	8.99275843986578	19.7268069260221\\
-11.2647207549644	-8.27193729509429	33.6935956925111\\
-17.1262941191893	-21.1083058835396	35.559771171986\\
-6.93504735179715	-6.37745384131821	25.9445586695689\\
-4.30323217948706	-5.72024810987241	18.998061119296\\
-8.77690287086272	-9.31390510151749	26.6482277194539\\
5.19437374923097	2.33896707835529	27.245248541931\\
-3.36455728967887	1.49446703626526	28.289272243687\\
8.48953819905744	7.84959620541479	27.8235889148477\\
-5.62777600796658	-8.50048626981704	17.5397356626751\\
3.98986293481934	7.50474353450549	9.27799144881814\\
-13.0410887685777	-14.2860112600905	31.7167497958273\\
2.10751632350881	3.45584686378308	17.1744408699261\\
8.75551599466149	8.99023592058173	27.0275388523531\\
-8.20916650065441	-7.83218282979404	27.9138649079747\\
-8.41735762719893	-7.66736424657214	27.8799868239139\\
2.86082240570959	4.7820678122739	13.3390205329673\\
-14.9329304704947	-16.4096061500298	34.4298038201182\\
11.0842683282924	14.1002812067649	26.4867747446994\\
8.23994060815843	-3.74935612217834	37.5397285836469\\
-8.95015305264564	-2.73736540552192	33.9834482544932\\
8.75602728810218	13.1781484110199	20.2380183991416\\
2.11439562245508	4.34234609198406	22.5162102226383\\
-2.7180859814304	-5.30233361789013	23.4710141630699\\
-8.22378000364979	-12.3175162021897	19.7800407454387\\
-5.73486281901671	-5.24249380843723	24.5521117047759\\
9.48579100402523	-1.26653567100788	37.9072612610332\\
1.97367699874025	1.53366002708629	20.5314180792666\\
14.4283089824746	6.62796931506832	41.750086695644\\
1.57997199900012	-4.71312123452968	29.0027710516072\\
4.7632322714552	7.45697595406454	16.9081915290982\\
12.4114693217577	16.4310229974025	27.2438852735106\\
14.4893368451288	11.6291088365748	37.9231972058179\\
1.98544953256082	2.43247824493417	17.8119846830577\\
8.85109028087356	13.8084767154631	19.1820369521566\\
8.68216827459417	1.0948633383243	34.8300008738009\\
-15.8846508643095	-8.42593182413244	43.5724174298193\\
8.59676633391109	9.4439258347626	25.9820113875653\\
-5.61990630796879	-9.5095377290606	14.3963272777556\\
-0.162279345597654	3.59786355361882	25.7036197264797\\
7.88380125927968	-8.56530820423393	39.814941084375\\
-3.68858440575653	-5.71600258981498	21.3598219644829\\
2.11498826421241	0.453567100223967	23.233849562969\\
3.08861866482895	5.07923491996891	21.4117120096081\\
-7.58205083220279	-2.22084052691244	31.9170713546212\\
7.19347036850813	6.37903390680131	26.5732899767128\\
-7.81772625918664	-7.65388131803398	26.4191163956044\\
13.1289469708912	22.7320121955128	19.6071314976812\\
-5.56436325882851	-10.2441133334271	10.9088804708753\\
7.22333628251837	11.8900384168485	16.1872403080154\\
11.4272199076132	11.1339544302081	31.1343852031597\\
-0.68823166422026	-1.2251555769145	13.9604071763701\\
-9.05128143233045	-14.4910175067266	18.5425469452482\\
7.55516626416286	6.76024695897828	26.9572956976488\\
-10.9361072954879	-6.0000924830848	35.0577259006451\\
11.1721306842582	8.07820399333355	33.6753460873635\\
-4.79053628390405	-3.805976878732	24.3487915045643\\
8.09831413043425	8.70444576324095	25.7077721516142\\
-5.70824165082602	-9.24772654274098	16.0095169243187\\
-1.17362711617768	-2.14857337328555	9.70969760247665\\
-0.825262803590305	2.99355767009196	25.9637764041313\\
-3.2511595318471	-2.56386601204716	22.3718756740796\\
10.1033635362616	6.4765749550938	32.9007064367545\\
-11.860111611532	-17.2278986939588	24.2679044557177\\
0.396360658515905	-3.26963803795252	25.6166222489201\\
-8.18146540167152	1.78645181897368	36.1298735656724\\
7.11793694581547	1.66362126239217	31.5814750120778\\
8.50796523814663	12.8955405978045	19.8362248129996\\
-2.15084249005804	-4.1202235166384	21.6970145282516\\
-4.46971252895035	-8.53484162444791	27.3700647582171\\
6.17074335627596	9.85872714392954	16.6870237143848\\
8.7014338386804	10.263752955472	25.0776963696781\\
-11.9606683355108	-14.2428115477125	28.8406220036157\\
7.4753180997835	4.06160295481174	29.9203505934185\\
-0.664404795014955	6.84506723686808	29.924631322568\\
7.06592503829043	7.12941718362642	25.2155603654629\\
8.4238012716746	4.36174320428391	31.5053601488721\\
10.8341964177112	9.12238348953098	31.8842921009239\\
-11.0995459520198	-13.003953923524	28.029966221163\\
8.4472365731234	4.75504793032651	31.1750800411616\\
6.85610568630824	8.75816408728935	24.0304252420694\\
15.2477529963232	13.3655214989322	38.1723246109851\\
6.7420491655808	8.10094442271715	22.722749324218\\
-5.61751771389893	-10.1033287901913	11.9091165707343\\
5.99202437165443	9.9928910401254	15.1064742756901\\
10.5521692164071	8.87341528623946	31.5038690434153\\
17.3712211271593	19.1971784258134	38.1975866335034\\
-16.6496245677285	-12.5431680568846	42.2219662192171\\
-5.15108810481511	-1.27693431057329	28.3922255071922\\
6.97809025343131	8.13844968679408	23.3784339991471\\
-8.26258803087291	-7.88074442761753	27.242330358196\\
14.7500702700281	22.0447621748141	26.9597435397161\\
8.05687452551623	6.32273173542323	28.6643790589852\\
11.7760193873349	11.6565509758971	31.4187459248321\\
13.9772291383352	12.88363565865	35.5452945897444\\
-4.99504417014972	-1.27981758935782	28.1039751144339\\
4.11342083721495	8.00554170380902	7.59558874814133\\
8.56358620835553	7.71178372820352	28.1881542845971\\
-15.7309596937994	-16.4840928705369	36.4659300461272\\
5.17840787555932	7.46977800564561	20.2551614397943\\
0.166258922599647	0.468438432535209	15.6024829929169\\
5.6366645921029	1.52129236722118	29.0332673313893\\
-10.2422172184367	-8.41404678178269	31.2734624206158\\
14.0457123326711	21.7087386498598	24.9631877088238\\
9.71023185959822	11.139247366608	26.8382850060399\\
8.46181234445439	8.98396339447867	26.2782828970706\\
-13.8019125555061	-18.5806448266135	28.6223772538046\\
-1.34373814442246	-0.354209624280255	21.2416187570095\\
8.40799889354121	8.8579147381087	26.2926851447077\\
-12.1466083849231	-11.9582884600696	32.0089932954489\\
-7.99363491537995	-2.30067523647696	32.5951465361367\\
-4.84710575820087	-9.2088760346324	9.08910220442449\\
-5.67173072117015	-2.85784521071623	27.5997281907794\\
-13.0530561678122	-7.64365323693893	38.0797961870825\\
2.22128354260963	3.31018928658249	15.5929526342321\\
6.42682744470452	6.23668669052	24.8504907960936\\
8.68363507995051	7.98819568222651	28.1154419452834\\
-3.36530163224161	-5.95702709183662	11.5937016466136\\
3.53799991466032	6.3059127308837	11.5168369630593\\
-5.12014958351074	-8.60433339125962	14.2226544173948\\
-9.6150241423226	-4.29455579670762	33.9077836272092\\
-5.00138502794128	-7.42393731144319	17.5048598237687\\
2.92884664502237	8.72032684376781	28.5985090630949\\
-1.85796343482881	-1.48114640048218	20.2322236645805\\
-8.49149575662914	-7.7061719717034	27.9993940948461\\
-7.8289601911052	-11.900691255368	19.0385896744062\\
7.85998293075856	9.43833130060866	23.8886012132212\\
-1.51073658533022	-1.56288117055661	18.4071879219337\\
14.7326295949395	14.6186882084849	35.7384871918347\\
-0.262020087345342	-2.870981194283	23.7529910275505\\
11.545143369922	18.3401710955707	21.1729849565003\\
5.54112917366157	7.30116593459507	20.1630890598305\\
5.78337560454668	1.17664544459975	29.6562763635985\\
-0.100599213614873	-0.912056723966437	19.0898335550612\\
7.43091471917466	7.37243876339357	25.8250266279793\\
4.10884843313121	6.15245230535444	19.5566876714405\\
10.2670616252483	7.66817552788711	32.1018070113712\\
1.3127742329764	-1.55461491212198	24.7659866263193\\
15.4615376222438	10.4677949240558	41.1113986114117\\
-3.81404202512711	-6.3280334248737	14.5109750635183\\
-5.07491862252046	-8.68297137643666	13.5715014112528\\
5.46698730770675	9.13279344176644	14.6083006234823\\
3.3777599268466	6.07421160487947	23.9679252964893\\
7.93515505952388	13.4184919920324	15.8002010552158\\
-4.44644134269553	-5.45788201290095	20.2211399510006\\
-4.27319672474454	-3.5861100486162	23.3695901968454\\
-14.3329631292031	-17.8459199886441	31.1460702452522\\
-2.22437379238094	-4.44874289199977	22.490963923569\\
-6.1277128945094	-2.72002672150929	28.6760822012802\\
-2.6874178377969	-2.52103099740461	20.6748121699397\\
4.2227172117197	1.33389034405007	26.5424639052384\\
-9.64533457055612	-10.4454447272134	27.4307682076839\\
2.70627025030094	3.35003114642063	18.4900096667531\\
9.78173729113686	6.97009709666573	31.7438406391597\\
-9.386883265034	-8.97922499432544	28.6319076690389\\
0.183537261497569	-2.1869468241327	23.4636553773603\\
9.31760519969594	15.4348218716733	17.6011516778616\\
16.141662879113	21.0245339030446	32.7323670957117\\
-10.7659144349492	0.524903953283782	39.6017819316196\\
-6.56281286088666	-5.53985293335149	26.1833808444987\\
9.38898811828704	2.61508128476238	34.8885224765934\\
-4.06212701391614	-1.81417731934921	25.576382755175\\
-17.7121964522691	-20.4289552360429	37.9394112796003\\
-10.024888942452	-2.12024556335346	36.4481441855389\\
9.09839195537703	13.341640658942	21.2408314682893\\
4.91521896846699	8.90110037126691	27.6657425728954\\
0.823876289991029	1.6455343793342	5.6211686310153\\
3.41417224646156	-1.26117701795296	28.1130105935794\\
13.0452678347208	11.6895145666479	34.4637572451698\\
2.12975858680845	3.57739703181012	12.9193943608616\\
1.31760718351933	2.19579645766647	12.7700727857876\\
7.32605160148101	3.61674588939322	30.0826803377467\\
6.82745694770867	-4.0393574265752	35.6090185277144\\
-7.85564286704632	-12.0321725611024	19.0345019188375\\
0.656248723462683	1.04020834502041	13.1206684587136\\
-6.37155500777671	-1.74725795161385	30.1552816704002\\
8.76911351073301	12.2249593805283	22.0253709773514\\
-4.22450146087885	-6.43931629667335	16.4006810939008\\
6.75233551354695	8.00158276842252	22.9201918802654\\
-8.61175704457871	-9.85884052165494	25.4309217123033\\
9.42225645580937	1.60982500730811	35.7424292605813\\
-2.82740818712372	-5.35603001155945	8.55320616916876\\
-5.46061572845814	-5.67411470913378	23.1057086537823\\
-12.7763790133324	-16.103779908148	28.7975652881464\\
-8.47734899374364	-8.35070040237631	27.1854167974949\\
9.44174154784041	8.8241505960518	28.9762500495337\\
-8.30557535778609	-7.11642225605502	28.2848399018325\\
0.733984363745711	1.50437273369546	17.4382313297007\\
18.5913935805802	15.1243875406676	44.9020204793335\\
12.673912849192	-1.3497691437187	43.4769007607345\\
2.3325227395067	3.79781386702848	16.0734214293588\\
11.773621109383	14.1586093841831	28.4191497190163\\
-6.00956417467357	-5.84023383468821	24.3513066237035\\
8.67440737852692	6.16978641650747	30.1848996922418\\
-10.0639656960461	-12.1737812965689	26.3477515050904\\
-1.62256624192494	-2.84437370468277	12.1528692672895\\
-13.3407333052348	-15.9326173710129	30.6255671793678\\
12.4800388812597	15.4189846393107	28.8149933591473\\
9.30445662774068	9.21963096548413	28.1195824556233\\
8.77846892874998	9.92619809566159	25.7967658037306\\
-11.2836194729787	-17.2311098890446	22.1677561932801\\
-8.94376367354749	-8.6767914185602	27.9041208237463\\
10.751438128559	5.91595796845465	34.7572348062228\\
4.29524375122809	4.71943047003067	21.3027017468602\\
-4.41170379403774	-6.9195841811067	15.8327185301848\\
13.1909256996126	13.32658811223	33.1527763514943\\
5.34849809482491	1.88180942522371	28.0923754260531\\
8.33255103820975	8.04895583328065	27.18259421122\\
10.2185056193499	15.9667550288194	20.4058361592445\\
1.8128315026703	2.41523512402303	16.769921070363\\
9.33791213068455	7.44469487845847	30.2623172316587\\
5.07926588384358	6.43341178910482	20.3740040811568\\
-6.07667914701722	-7.75481446939606	21.1628836506857\\
9.33517182179806	1.04532154429856	36.0152190671981\\
-0.471952123281648	8.89281150836084	31.4468716733957\\
-8.25929178283674	-8.72806883085686	27.1630944843517\\
-11.5337607622742	-7.66053324152358	34.8440517568133\\
-3.99396617309313	-7.71330648722443	7.95391064918978\\
-1.05209426018116	-0.548117225701804	19.6015982428988\\
8.7126147468076	9.60479543392944	31.4074253774873\\
-0.224898875233646	1.69348201983175	22.5669490055084\\
-11.6886876010211	-16.5296139244218	24.8020037515827\\
3.4659400724323	2.70558797257651	22.7167572499658\\
-9.28797934085622	-15.8280089681068	16.5177045888373\\
-10.9128431622123	-10.9578820033696	30.0545807604317\\
8.31651760984588	8.4158158102606	26.6884514851903\\
-4.9684847423415	-7.62584875128677	25.0975605867211\\
-11.4815894409671	-12.7481812183654	29.3931678068215\\
9.4318218040572	10.9270387290335	26.1957781371329\\
7.5378770430063	1.74924992329592	32.2608599605902\\
8.1458868003577	6.08854781950812	29.113982177892\\
2.04784418651528	3.91246199335936	21.340559606163\\
1.2848208560617	-8.43739042787864	31.8705136475371\\
12.3750505334214	0.883249767605326	41.57890658493\\
4.39154713599882	4.98657916396157	21.0754350187901\\
11.1360044530923	17.1394561620434	21.7740907688374\\
-9.03575213507415	-8.90196810883821	27.8470014188577\\
-5.77817446252953	-8.23397604740685	18.9387041529986\\
8.46009663381408	9.44693741286287	26.2320269348414\\
-6.62187909831674	-10.8260632374763	16.1514218875132\\
-2.50693830206716	-4.05785609683564	14.0368929972224\\
6.76240213092756	10.8485227303388	16.8843571791512\\
-7.92659249338068	-7.49981400596435	27.2216024016299\\
-1.29878569128012	-1.82771355582356	15.5609694131764\\
-5.14686355176772	-7.09206539500057	19.1536685192657\\
-4.52497264290763	-5.3631044213938	20.7256969257439\\
-8.76141441533545	-12.0102550549491	22.3983844187237\\
-12.7194541932469	-9.8294174513582	35.4724538339585\\
7.77562041326293	8.23604428565697	25.4975164393153\\
4.26240856214716	6.99464703517923	14.513784095301\\
7.70312579230369	5.65596927147601	28.6435737626365\\
2.97404579236671	4.15370443782805	17.1792438271635\\
2.96471056148583	5.49499387071344	9.58138779731241\\
-10.8609707888403	-12.055043639217	28.6577220381704\\
-8.49338809310473	-9.10164333374372	26.1850961483162\\
-8.89324776706547	-5.01336198251711	31.8162774631391\\
2.33395041291612	6.59570560708849	26.4777929076708\\
3.27321057269827	2.48024114225858	22.5962599108727\\
-3.28781048529656	-4.58661616486845	17.4400218315676\\
-11.745585251205	-17.1330990797518	24.0011419226716\\
0.962197343539512	1.67363652307185	14.5710580101964\\
-1.93864698245964	1.84374498774789	26.319869616516\\
-2.30666695633157	-4.20204505379907	21.3301526810898\\
6.49325825156841	6.32478127476091	24.8909512893787\\
10.4801952414176	17.1936283481394	18.9999223897208\\
-2.69165647342405	-0.804325490732434	24.0276773233076\\
-7.55602562094263	-2.22191689123674	31.8627591104713\\
14.8540594682408	17.3277304462345	33.226104161456\\
-10.1793633803909	-10.3820503006481	28.9413874465201\\
-7.05413238629074	-7.62418426484842	24.4232704687893\\
-6.93563034245976	-10.9709887786308	17.2358830408375\\
-14.5920794782603	-12.2816549493313	37.5859733430986\\
7.93389338093762	10.3856680978991	23.1170706092096\\
-5.15105644658788	-7.44181100632694	20.1538713017077\\
4.39837668721211	4.06292118587907	22.8914563355382\\
3.11955024061718	4.87926494232182	18.8308140021877\\
-7.99450809534371	-4.16985684742018	30.8455065920734\\
-9.1315727272994	-8.56817693592135	28.498967862064\\
7.40883152200093	-0.815369457789691	34.1616262461801\\
13.4289679376146	16.7674675518231	29.8460885670719\\
1.04222924039199	1.35291431875098	16.0351037731564\\
-5.69596008776174	-8.36255560771784	19.3361405306449\\
3.950991347126	5.94364809276052	19.9005984043919\\
14.6520447416878	4.288051299478	43.7987277319577\\
0.361116832761662	-6.84241113761123	29.6147623740976\\
9.65237186301475	5.96114288206753	32.4599741697381\\
0.0126167440783889	-7.12974665455143	29.5175095586751\\
1.9736470249072	-3.46981825710688	28.270430110777\\
3.26882283212916	4.77475707691935	16.6667352139157\\
-1.98635715715815	-4.2330836551756	22.57375012888\\
5.92417138260063	-0.264844364710874	31.2793922440351\\
-9.46212851763062	-9.16759099206211	28.5891110187559\\
-1.77887751770402	-1.334901206875	20.3130539252417\\
5.02869017201695	2.46180786055527	26.7599103189031\\
-5.13432035924244	-3.01905593323208	26.2645862440701\\
7.70607439566749	11.5079462214623	19.6795975540983\\
-6.67092285212551	-9.09932748593491	20.5426583730111\\
-3.3790557318641	-2.55087486725175	22.7611926578396\\
7.94118092968279	7.79548507802815	26.5367373138067\\
-5.33515994759477	-1.87114964829606	28.0792808232876\\
5.69504462656442	8.69766106862466	17.9739956449235\\
8.47920816332305	8.14155712400799	27.4263077071023\\
8.44198762826757	-4.75460847210341	38.4632359880826\\
-1.67407105888775	-1.85934726945369	18.1920451373679\\
-6.25220782481087	-6.4663442909722	24.0373696633141\\
2.7632450523016	4.48605747254193	14.0497589912707\\
-3.37395498722777	-5.07881218306686	16.1397232431224\\
9.09338650017701	2.70287902456479	34.2702068593056\\
-6.67189976335573	-8.54484289268234	24.8224315992991\\
-12.5063245001444	-6.81979584653621	37.6011075169764\\
-8.400795706798	-9.47483764211234	26.0841550211326\\
12.7043044707272	18.2548626246992	25.5342498777595\\
-1.66657277119422	-2.57138762272241	14.5858689120023\\
6.30065797103573	7.24008384642517	22.8729098247797\\
7.92662189563224	7.64209121995606	26.9278581527527\\
7.87599631067711	8.1809833148484	25.8375936783381\\
13.9773760092625	10.7797664418324	37.4783956157913\\
0.334740685537586	-4.62482784495656	27.273869910664\\
8.3452718426636	9.27202073814938	25.5492947676892\\
0.72317985863324	1.33095557301829	15.5367692378015\\
5.03178164736235	8.87908143231887	12.4100284927492\\
-9.98130143448377	-10.1513245491289	28.678135096787\\
1.240842709164	4.32797864208371	24.4575268560606\\
7.8690636671714	10.5575711363766	22.5220650032514\\
-3.68265424904618	-5.93882649805726	14.7041744219067\\
-10.7907861221035	-11.5655375178703	29.0230652626904\\
6.45089205517788	7.24698455466226	23.3175718656453\\
-11.7792342291048	-12.9447032269167	29.9482288228196\\
8.90783412744936	6.44142428809408	30.4094426106284\\
-7.92580189573631	-13.5641384557685	15.3360821004458\\
-7.25341773051988	-8.85378953971295	25.3968544181274\\
-0.165910196783156	0.511974534695009	18.9581591088185\\
11.2475256474504	15.8519517937176	24.3608608115166\\
-8.48162743525113	-8.64043063506138	26.7855149421397\\
-3.26126602624686	-5.10881088839481	20.0672457938694\\
5.01595096030748	7.29151377643553	23.6418525894035\\
7.70699222184281	10.3604520517593	21.7812740518744\\
-4.88030035126333	-8.52130352759774	12.8194886271958\\
7.76366267470627	11.0544103694762	20.603380077107\\
4.11708682978159	7.96375788124883	7.86130287709368\\
-2.73515307605779	-4.50272571419814	14.8750374045952\\
4.10548819002127	3.86584584867145	22.4059607220924\\
5.94884602674717	7.6124534056759	21.0113447486255\\
-11.6011358302519	-19.8190727070062	18.5443433295449\\
-2.42645836891164	-5.00889259176727	23.4468934290657\\
9.124216164479	9.56559960085737	27.2185365172371\\
-0.648592948238113	-0.699018919947574	16.3344903439308\\
6.5767679705211	7.03395614170531	24.0269790756301\\
9.94236188742389	9.41369490086348	29.4609093399427\\
16.1019871329399	25.4630965204107	26.8845412063714\\
16.3852546441365	12.8219132790146	41.3673856487055\\
8.46294430963663	5.34243496165469	30.612560105062\\
-5.28452796954716	-6.58183989803082	20.8010744713528\\
-7.02828762289472	-4.31969506095919	28.7175684003562\\
-5.53664729103824	-9.40025492768399	14.0891223571888\\
8.22207970032921	8.10742115707484	26.8339770591601\\
9.29138143155005	3.50973967192739	33.9562340595312\\
2.12041835989707	3.99465642247084	8.64226963654911\\
-2.18392465036438	-1.86473724753389	20.4620072891958\\
6.62488955638508	10.1383429630475	17.9290377935681\\
6.58203814001192	5.22039137887353	26.6416842558012\\
1.25066145069926	-2.12612992678193	25.4935800165275\\
-1.95440696820722	-3.22100638163837	13.3192026568808\\
};
\end{axis}
\end{tikzpicture}%}
	\end{subfigure}
	\caption{Empirical distribution of $\phi(Y)$ for $Y$ alternatively the solution of the ODE with stochastic time stepping, the SDE \eqref{eq:modifiedSDE} and the probabilistic solver. Scatter plot of the solution at final time computed with the three methods (blue = ODE, random time stepping; red = SDE \eqref{eq:modifiedSDE} with Euler-Maruyama; yellow = Probabilistic solver \cite{CGS16}).}
	\label{fig:distributionPhi}
\end{figure}


\bibliographystyle{siam}
\bibliography{anmc}

\end{document}