\subsection{Numerical experiment - Posterior distributions}
Let us consider the two-dimensional FitzHug-Nagumo ODE defined in \eqref{eq:FitzNag} and the problem of determining the values of the parameters $\theta = (a, b, c)^T$ in $\R^3$. We consider as the true value of $\theta$ the vector $\bar \theta = (0.2, 0.2, 3)$. We produce a set of synthetic observations $\mathcal{Y}_{10}$ from a numerical solution $\tilde u$ computed using $\bar \theta$ and a small time step at times $t_i = 1, 2, ..., 10$, with an additive independent Gaussian noise, i.e.,
\begin{equation}
	y_i = \tilde u_{\bar\theta}(t_i) + \epl_i, \quad \epl_i \iid \mathcal N (0, 10^{-2}I), \quad i = 1, \ldots, 10,
\end{equation}
where $I$ is the identity matrix in $\R^{2\times2}$. Therefore, we consider a diagonal noise with independent normal components having all variance $10^{-2}$. We approximate the posterior distribution $\pi(\theta|\mathcal Y)$ with both the deterministic and the probabilistic solvers using time step $h = 0.1$. We use the RAM algorithm for the deterministic case and the RAM algorithm applied to MCWM for the probabilistic integrator. In both cases, the proposal distribution $q(x,y)$ is a Gaussian with variance adapted by RAM, and the prior distribution $\mathcal{Q}(\theta)$ is normal with unitary variance and mean $\bar \theta$. We consider $50000$ iterations of MCMC in both the deterministic and the probabilistic case, with the first $10\%$ of guesses considered as a burn-in. Results (Figure \ref{fig:MCMC_FHN}) show that the posterior distribution obtained using the deterministic solver is concentrated and biased with respect to the true value of the parameter. On the other side, the probabilistic solver provides with a posterior distribution having a wider support which contains the true value of the parameter. This confirms the claim that the probabilistic solver allows to identify the source of error given by the numerical integration, while in the deterministic case this uncertainty does not result from the obtained distributions \cite{CGS16}.

\begin{figure}
	\centering
	\includegraphics[width=1\linewidth]{plots/Fitznag}
	\caption{Posterior distribution for the parameter $\theta$ defining the FitzHug-Nagumo model. The posterior distributions given by the probabilistic and the deterministic solvers are displayed in blue and red respectively. The true value of the parameters is displayed in thick green dots.}
	\label{fig:MCMC_FHN}
\end{figure}