\subsection{Stability analysis}

In the previous sections we analyzed the behavior of the probabilistic integrator for ODE's in terms of its convergence with respect to the time step. Moreover, we analyzed the convergence of a Monte Carlo approximation of the probabilistic solution towards the exact solution. Another key feature of numerical methods is their stability. We recall that the one step of the numerical method can be written, under Assumption \ref{assumption_1}, as
\begin{equation}
	U_{k+1} = \Psi(U_{k}, U_{k+1}) + \sqrt{\sigma Q}h^{p+1/2}Z_k,
\end{equation}
with $Z_k$, $k = 0, \ldots, N$ i.i.d. zero-mean normal random variables with unitary variance. We can write therefore the numerical method as
\begin{equation}
	U_{k+1} = \Psi(U_{k}, U_{k+1}) + \sqrt{\sigma Q}h^{p}\Delta W_k,
\end{equation}
where the random variables $\Delta W_k$, $k = 0, \ldots, N$, are standard Wiener increments. This is the stochastic Runge-Kutta method applied to the SDE
\begin{equation}
	\dd U(t) = f(U(t))\dd t + \sqrt{\sigma Q}h^p \dd W(t).
\end{equation}
