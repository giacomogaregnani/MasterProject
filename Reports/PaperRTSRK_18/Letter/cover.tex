\documentclass[10pt]{letter}
%\addtolength{\textwidth}{2.5cm}     %% For longer or shorter text width
\addtolength{\topmargin}{-2.5cm}    %% For more or less top margin
\addtolength{\textheight}{5cm}    %% For longer or shorter textheight
%\addtolength{\oddsidemargin}{-1.25cm} %% For odd side margin (twoside)
%% or margin (oneside) 
\usepackage{amsmath}
\usepackage{amssymb}
\usepackage{color}
%%%%%%%%%%%%%%%%%%%%%%%%%%%%%%%%%%%%%%%%%%%%%%%%%%%%%%   

\address{Prof. Assyr Abdulle\\
 Institute of Mathematics\\ \'Ecole Polytechnique F\'ed\'erale de Lausanne }
%%%%%% The Signature  and Date %%%%%%%%%%%%%%%%%%%%%%%%%%%%%%%%%%%%%%%%%%%%


\begin{document}

\begin{letter}{SIAM Journal on Numerical Analysis\\
               Prof. S. Rech\\
	   Editor\\
\vspace{0,5cm}
}

\date{22 January 2018}

%%%%%% More vertical space can be added here %%%%%%%%%%%%%%%%%%%%%%%%%%%%%%
\vspace{6.0cm}
\opening{Dear Sebastian,}

\vspace{0.5cm}

%%%%%% More vertical space can be added here %%%%%%%%%%%%%%%%%%%%%%%%%%%%%%
Enclosed you will find the revision of our manuscript  ``Random time step probabilistic methods for uncertainty quantification in chaotic and geometric numerical integration" (joint work with Giacomo Garegnani) for publication in the SIAM Journal on Numerical Analysis. 

Below is an answer to the two issues you wanted us to address before proceeding further with our submission. Related changes in the paper  are highlighted in red.

You raised two issues:
\begin{itemize}
 \setcounter{enumi}{1}
	\item {\bf Reformulation of the discussion concerning conservation properties for Hamiltonian systems related to our method.}
We agree that the discussion around symplecticity was misleading. All what was said about conservation of polynomial invariants was of course fine and also the discussion related to polynomial Hamiltonian (H\'enon-Heiles problem). But it is right that the symplecticity without backward error analysis is misleading. We have thus moved section 6.2 of our paper into the numerical experiment section (Sect. 8.8), where we discuss the symplecticity and the lack of backward error analysis. The example of the Harmonic oscillator is discussed showing the drift in the energy. Interestingly, we report there a new finding concerning {\it the conservation of the total energy in the mean} that seems to be {\color{red} verified}. We have no theoretical argument of that though I think it is possible to prove something (work in progress).
	\item {\bf  A discussion of the  added value of randomised time-stepping over standard error estimates for numerical ODE methods both in the context of chaotic systems and Bayesian inference as been added illustrated by new numerics.}
We discuss two set-ups that in our opinion clearly indicate the added value of randomised time-stepping. 

First we built on your remark about the Lorenz-63 example. You mention that ``just taking a single random initial time-step followed by constant and equal time-steps would also produce plots such as Figure 1." While it is true that it is possible to {\it explore} the quantification of the reliability of the numerical solver by introducing a random perturbation on the initial condition, it is a priori unclear what amount of perturbation describes correctly  the sensitivity of the solution to numerical errors of a given order. In order words, while the theoretical results presented in this work guarantee that for the RTS-RK method the random component is well balanced with the deterministic numerical error, the choice of a random perturbation on the initial condition is arbitrary. Hence, it is not possible to quantify the sensitivity of the solution to numerical errors by a simple perturbation of the initial condition, while this result is achievable by integration with the RTS-RK method (this is discussed in the introduction and illustrated by a new numerical example in Section 8.4).

Secondly, probabilistic solvers for ODEs offer a quantitative procedure to correct the behaviour of deterministic solvers for ODEs in statistical inference. In particular, introducing a probability measure over the numerical solution allows to account for the integration error and to obtain posterior measures which reflect the uncertainty given by the numerical approximation. In contrast, the numerical error introduced by deterministic solvers can lead to inappropriate and non-predictive posterior concentrations (this is discussed in the introduction and in Section 8.9).

Finally, we believe that  the last numerical example you suggest (test equation without any chaotic behaviour) is not relevant for probabilistic numerical solvers and there is not benefit in using them for such class of problems (there was also never such a claim in our paper). 

\end{itemize}

We believe that we have responded to the issues that were raised and hope you will be able to proceed further with our manuscript.


\vspace{0.25cm}

%%%%%%% The Closing %%%%%%%%%%%%%%%%%%%%%%%%%%%%%%%%%%%%%%%%%%%%%%%%%%%%%%%
\closing{Best regards,\\ \vspace{1.5cm} Assyr}

\end{letter}

\end{document}

