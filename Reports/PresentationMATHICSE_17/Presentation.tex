\documentclass{beamer}

\usetheme{Madrid}
\usecolortheme{dolphin}

\usepackage{amsthm}
\usepackage{graphicx}
\usepackage[utf8x]{inputenc}
\usepackage{mathtools}
\usepackage{appendixnumberbeamer}
\usepackage{enumitem}
\usepackage{centernot}
\usepackage{tikz}
\usetikzlibrary{positioning}
\setitemize{label=-, leftmargin=*}

\DeclarePairedDelimiter{\abs}{\lvert}{\rvert}
\DeclarePairedDelimiter{\norm}{\|}{\|}
\renewcommand{\Pr}{\mathbb{P}}
\newcommand{\R}{\mathbb{R}}
\newcommand{\Var}{\operatorname{Var}}
\newcommand{\E}{\operatorname{\mathbb{E}}}
\newcommand{\iid}{\ensuremath{\stackrel{\text{i.i.d.}}{\sim}}}
\renewcommand{\phi}{\varphi}

% Theorem blocks
\newenvironment<>{greenblock}[1][]{%
\setbeamercolor{block title}{fg=white,bg=green!40!black}%
\begin{block}#2{#1}}{\end{block}}
\newenvironment<>{redblock}[1][]{%
\setbeamercolor{block title}{fg=white,bg=red!75!black}%
\begin{block}#2{#1}}{\end{block}}

% Add numbers and take out navigation symbols
\setbeamertemplate{footline}[frame number]
\beamertemplatenavigationsymbolsempty

% Text starts always from the top of the frame
\newenvironment{frameT}{\begin{frame}[t]}{\end{frame}}

\title{Probabilistic Runge-Kutta methods for ODEs}
\subtitle{Chaotic problems and geometric properties}
\author{Giacomo Garegnani - ANMC}
\institute{MATHICSE retreat}
\date{June 2017}

\begin{document}

\frame{\titlepage}

\begin{frameT}
	\frametitle{Motivation}
	
	\begin{example} Lorenz system ($\sigma = 10$, $\rho = 28$, $\beta = 8/3$)
		\begin{equation}\label{eq:Lorenz}
			\begin{aligned}
				y_1' &= \sigma(y_2 - y_1), \quad &&y_1(0) = -10,\\
				y_2' &= y_1(\rho - y_3) - y_2, \quad &&y_2(0) = -1,\\
				y_3' &= y_1y_2 - \beta y_3, \quad &&y_3(0) = 40.
			\end{aligned}
		\end{equation}
	\end{example}
	\begin{figure}
		\includegraphics[]{Lorenz}
		\caption{Deterministic (black) and probabilistic (grey) numerical solutions of \eqref{eq:Lorenz}}
	\end{figure}
\end{frameT}

\begin{frameT}
	\frametitle{Problem statement}
	
	Consider $f\colon\R^d\to\R$ and the ODE
	\begin{equation}
		y' = f(y), \quad y(0) = y_0.
	\end{equation}
	Flow of the equation $\phi_t$: for any $y_0\in\R^d$
	\begin{equation}
		y(t) = \phi_t(y_0).
	\end{equation}
	One-step Runge-Kutta method: numerical flow $\Psi_h$ such that
	\begin{equation}\label{eq:NumericalFlow}
		y_{n+1} = \Psi_h(y_n).
	\end{equation}
	\begin{problem}
		Modify \eqref{eq:NumericalFlow} in order to give a probabilistic numerical solution.
	\end{problem}
\end{frameT}

\begin{frameT}
	\frametitle{Existing method}
	
	Additive noise \cite{CGS16}. Build stochastic process $\{Y_n\}_{n=1, 2, \ldots}$
	\begin{equation}
		Y_{n+1} = \underbrace{\Psi_h(Y_n)}_{\text{deterministic}} + \underbrace{\xi_n(h)}_{\text{random}},
	\end{equation}
	where $\{\xi_n\}_{n=1,2,\ldots}$ i.i.d. random variables s.t. for $p > 1$ and $Q \in \R^{d\times d}$
	\begin{equation}
		\E\xi_n(h) = 0, \quad \E \xi_n(h) \xi_n(h)^T = Qh^{2p+1}.
	\end{equation}
	
	\only<2-5>{
	\begin{greenblock}[Properties]
		If $\Psi_h$ is of order $q$ and for $\Phi\colon\R^d\to\R$ smooth
		\begin{itemize}
			\item<3-5> Strong convergence: $\E\norm{y(hn)-Y_n} \leq Ch^{\min\{p, q\}}$,
			\item<4-5> Weak convergence: $\abs{\Phi\big(y(hn)\big) - \E\Phi(Y_n)} \leq Ch^{\min\{2p, q\}}$,
			\item<5> Good qualitative behavior in Bayesian inverse problems.
		\end{itemize}
	\end{greenblock}
	}
	
	\only<6-7>{
	\begin{redblock}[Issues]
		\begin{itemize}
			\item<6-7> Robustness: $\Psi_h(Y_{n-1}) > 0 \centernot\implies \Pr(Y_n > 0) = 0$,
			\item<7> Geometric properties are not conserved. For example if $I(y) = y^TSy$,
			\begin{equation}
				I(Y_1) = I(y_0) + 2\xi_0(h)^T S  \Psi_h(y_0) + \xi_0(h)^T S \xi_0(h).
			\end{equation}
		\end{itemize}
	\end{redblock}
	}
\end{frameT}

\begin{frameT}
	\frametitle{Method proposal}
	
	Intrinsic noise: Random time-stepping Runge-Kutta (RTS-RK)
	\begin{equation}
		Y_{n+1} = \Psi_{H_n}(Y_n),
	\end{equation}
	where $\{H_n\}_{n=1,2,\ldots}$ i.i.d. random variables s.t. for $h, C > 0$ and $p > 1$
	\begin{equation}
		H_n > 0 \text{ a.s.}, \quad \E H_n = h, \quad \Var H_n = Ch^{2p}.
	\end{equation} 
	For example $H_n \iid \mathcal{U}(h-h^p, h+h^p)$ is a suitable choice.
	
	\only<2-5>{
	\begin{greenblock}[Properties]
		\begin{itemize}
			\item<2-5> Strong and weak convergence equivalent to additive noise method,
			\item<3-5> Good qualitative behavior in Bayesian inverse problems,
			\item<4-5> If $\Psi_h$ conserves an invariant $I(y)$, then so do RTS-RK,
			\item<5-5> Simplecticity is guaranteed for $\Psi_h$ symplectic.
		\end{itemize}
	\end{greenblock}
	}
	
\end{frameT}

\begin{frameT}
	\frametitle{Numerical example}
	
	Hénon-Heiles problem \cite{HeH64}. Hamiltonian function ($p, q \in \R^2$)
	
	\begin{equation}
		H(p, q) = \frac{1}{2}\norm{p}^2 + \frac{1}{2}\norm{q}^2 + q_1^2q_2 - \frac{1}{3}q_2^3,
	\end{equation}
	
	Corresponding system of ODEs 
	\begin{equation}\label{eq:HenHei}
	\begin{aligned}
		p' &= -\nabla_q H(p, q), &&p(0) = p_0 \in \R^2,\\
		q' &= \nabla_p H(p, q), &&q(0) = q_0 \in \R^2.
	\end{aligned}
	\end{equation}
	
	\only<2>{
	\begin{greenblock}[Problem properties \cite{HLW06}]
		\begin{itemize}
			\item Chaotic regime for $H(p, q) > 1/8$,
			\item Hamiltonian is conserved $\implies$ use symplectic integrator.
		\end{itemize}
	\end{greenblock}
	}
\end{frameT}

\begin{frame}
	\frametitle{Numerical results}
	\begin{tikzpicture}[remember picture,overlay]
	\node at (4.2,0) (A) {RTS-RK};
	\node at (7.9,0) (B){Additive noise};
	\end{tikzpicture}
	\begin{figure}
		\begin{center}
			\begin{tabular}{c@{\hspace{0.3cm}}c}
				\includegraphics[]{HHStep} & \includegraphics[]{HHAdd} \\
			\end{tabular}
		\end{center}		
	\end{figure}
\end{frame}

\begin{frameT}
	\frametitle{Numerical results}
	\begin{figure}
		\begin{tabular}{l}
			\includegraphics[]{HHStepChaos2} \\
			\includegraphics[]{HHStepChaos}
		\end{tabular}
	\end{figure}
\end{frameT}

\appendix
\begin{frame}[allowframebreaks, t]
	\frametitle{References}
	
	\setbeamertemplate{bibliography item}[text]
	\bibliographystyle{apalike}
	\bibliography{anmc}
\end{frame}

\end{document}
