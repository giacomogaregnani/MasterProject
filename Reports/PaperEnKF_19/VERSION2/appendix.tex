\section*{Appendix}

\subsection*{Proof of Lemma \ref{lemma_difference_inverse}}

Note that
\[ A^{-1} - B^{-1} = A^{-1} (I - A B^{-1}) = A^{-1} (B - A) B^{-1}, \]
therefore we have
\[ \norm{A^{-1} - B^{-1}}_2 \le \norm{A^{-1}}_2 \norm{B - A}_2 \norm{B^{-1}}_2, \]
which is the desired result.

\qed

\subsection*{Proof of Lemma \ref{lemma_inverse_sum}}

Let $n$ be the dimension of the matrices, since $A$ is symmetric positive semidefinite and $B$ is symmetric positive definite, then $A + B$ is symmetric positive definite, and the eigenvalues of $A + B$ and $B$ are real and positive, thus they can be written
\[ 0 < \lambda_1(\cdot) \le \lambda_2(\cdot) \le \dots \le \lambda_n(\cdot), \]
counted with their multiplicity.
First, notice that, using the Rayleigh quotient and the fact that $x^T A x \ge 0$ for all $x$, we have
\[ \lambda_1(A + B) = \min_{x \neq 0} \frac{x^T(A + B)x}{x^T x} = \min_{x \neq 0} \frac{x^T A x + x^T B x}{x^T x} \ge \min_{x \neq 0} \frac{x^T B x}{x^T x} = \lambda_1(B), \]
which implies
\[ \norm{(A + B)^{-1}}_2 = \frac{1}{\lambda_1(A + B)} \le \frac{1}{\lambda_1(B)} = \norm{B^{-1}}_2, \]
which is the desired result.

\qed

\subsection*{Proof of Lemma \ref{G_lipschitz}}

Let $u_1, u_2 \in \mathbb{R}^M$, \corr{and $p_1 = \mathcal S(u_1)$, $p_2 = \mathcal S(u_2)$ Then, writing the weak formulations of \eqref{problem_lemma} we get
	\begin{equation*}
	\int_\Omega \big(A(u_1)\nabla p_1 - A(u_2) \nabla p_2\big) \cdot \nabla v = 0,	
	\end{equation*}
	for all $v \in H^1_0(\Omega)$. Adding and subtracting $A(u_1) \nabla p_2$ to the first factor inside the integral and rearranging terms yields}
\begin{equation*}
\int_{\Omega} A(u_1) (\nabla p_1 - \nabla p_2) \cdot \nabla v = - \int_{\Omega} (A(u_1) - A(u_2)) \nabla p_2 \cdot \nabla v.
\end{equation*}
\corr{Choosing $v = p_1 - p_2$,} using the hypotheses on $A$ and the H\"older inequality, we obtain
\begin{align*}
\alpha \norm{\nabla p_1 - \nabla p_2}_{L^2(\Omega; \R^d)}^2 \le & \;  \int_{\Omega} \left | (A(u_1) - A(u_2)) \nabla p_2 \cdot (\nabla p_1 - \nabla p_2) \right | \\
\le & \; \norm{A(u_1) - A(u_2)}_{L^{\infty}(\Omega; \R^{d \times d})} \norm{\nabla p_2}_{L^2(\Omega; \R^d)} \norm{\nabla p_1 - \nabla p_2}_{L^2(\Omega; \R^d)} \\
\le & \; M \norm{u_1 - u_2}_2 \norm{\nabla p_2}_{L^2(\Omega; \R^d)} \norm{\nabla p_1 - \nabla p_2}_{L^2(\Omega; \R^d)},
\end{align*}
which implies 
\begin{equation} \label{intermediate}
\norm{\nabla p_1 - \nabla p_2}_{L^2(\Omega; \R^d)} \le \frac{M}{\alpha} \norm{\nabla p_2}_{L^2(\Omega; \R^d)} \norm{u_1 - u_2}_2.
\end{equation}
\corr{It remains} to bound $\norm{\nabla p_2}_{L^2(\Omega; \R^N)}$, \corr{which can be achieved with a standard coercivity argument. In particular, we have
	\begin{equation} 
	\norm{\nabla p_2}_{L^2(\Omega; \R^N)} \le \frac{C_p}{\alpha} \norm{f}_{L^2(\Omega)}, 
	\end{equation}
	where $C_p$ is the Poincaré constant associated to the domain $\Omega$. Replacing in \eqref{intermediate}}, we obtain
\begin{equation*}
\norm{\nabla p_1 - \nabla p_2}_{L^2(\Omega; \R^N)} \le \frac{M C_p}{\alpha^2} \norm{f}_{L^2(\Omega)} \norm{u_1 - u_2}_2 = L_{\mathcal{S}} \norm{u_1 - u_2}_2,
\end{equation*}
which shows that $\mathcal{S}$ is Lipschitz with constant
\[ L_{\mathcal{S}} = \frac{M C_p}{\alpha^2} \norm{f}_{L^2(\Omega)}. \]
Finally, $\mathcal{G}$ is the composition of two Lipschitz operators, so it is Lipschitz \corr{of constant $L_{\mathcal G} = L_{\mathcal O} L_{\mathcal S}$}, which concludes the proof.
\qed

%\subsection*{Proof of Lemma \ref{wH1_sL2}}
%
%The sequence $\{ p^{\varepsilon} \}$ is weakly convergent in $H^1_0(\Omega)$, hence it is bounded in $H^1_0(\Omega)$. By Rellich theorem, $H^1_0(\Omega)$ is compactly embedded in $L^2(\Omega)$, so there exists a subsequence $\{ p^{\varepsilon'} \}$ such that
%\[ p^{\varepsilon'} \to \bar{p} \qquad \text{in } L^2(\Omega) \]
%for some $\bar{p}$ in $L^2(\Omega)$. Strong convergence implies weak convergence, therefore we obtain 
%\begin{equation}
%\label{convergence_pbar}
%p^{\varepsilon'} \toweak \bar{p} \qquad \text{in } L^2(\Omega).
%\end{equation}
%Moreover, since $p^{\varepsilon} \toweak p^0$ in $H^1_0(\Omega)$, we also have that this is true for the subsequence $p^{\varepsilon'}$
%\[ p^{\varepsilon'} \toweak p^0 \qquad \text{in } H^1_0(\Omega), \]
%which implies
%\begin{equation}
%\label{convergence_p0}
%p^{\varepsilon'} \toweak p^0 \qquad \text{in } L^2(\Omega).
%\end{equation}
%Thanks to (\ref{convergence_pbar}) and (\ref{convergence_p0}) and by uniqueness of the weak limit we deduce that $\bar{p} = p^0$, thus
%\[ p^{\varepsilon'} \to p^0 \qquad \text{in } L^2(\Omega). \]
%Finally, repeating this argument for subsequences, we deduce that every subsequence admits a sub-subsequence strongly converging to the same limit in $L^2(\Omega)$, hence the whole sequence strongly
%converges to $p^0$, that is
%\[ p^{\varepsilon} \to p^0 \qquad \text{in } L^2(\Omega), \]
%which is the desired result.
%
%\qed

\subsection*{Proof of Lemma \ref{f_goes_to_0}}

Let us consider an ensemble $u \in \mathcal U_{J,M}$ with particles $u^{(j)} \in \R^M$, for $j = 1, \ldots, J$. For each particle we have
\begin{align}
\norm{\mathcal{G}^{\varepsilon}(u^{(j)}) - \mathcal{G}^0(u^{(j)})}_2 = & \; \norm{\mathcal{O}(\mathcal{S}^{\varepsilon}(u^{(j)})) - \mathcal{O}(\mathcal{S}^0(u^{(j)}))}_2 \\
= & \; \norm{\mathcal{O}(p^{\varepsilon}(u^{(j)})) - \mathcal{O}(p^0(u^{(j)}))}_2 \\
\le & \; m \norm{p^{\varepsilon}(u^{(j)}) - p^0(u^{(j)})}_{L^2(\Omega)},
\end{align}
where we write explicitly the dependence of $p^{\varepsilon}$ and $p^0$ on the particle it is generated by.
By homogenization theory, we know that $p^{\varepsilon}(u^{(j)}) \toweak p^0(u^{(j)})$ in $H^1_0(\Omega)$, and therefore we have $p^{\varepsilon}(u^{(j)}) \to p^0(u^{(j)})$ in $L^2(\Omega)$, which implies
\[ e(\varepsilon, u) = \frac{1}{J} \sum_{j=1}^J \norm{\mathcal{G}^{\varepsilon}(u^{(j)}) - \mathcal{G}^0(u^{(j)})}_2 \to 0. \]
Moreover, if the solution of the homogenized problem $p^0$ is sufficiently smooth independently of $u$, namely $p^0 \in H^2(\Omega)$, letting $C > 0$ be a constant, we have for all $j = 1, \dots, J$ the following estimate, which can be found in \cite{Gri05}
\[ \norm{p^{\varepsilon}(u^{(j)}) - p^0(u^{(j)})}_{L^2(\Omega)} \le C \varepsilon, \]
hence we finally obtain
\[ e(\varepsilon, u) = \frac{1}{J} \sum_{j=1}^J \norm{\mathcal{G}^{\varepsilon}(u^{(j)}) - \mathcal{G}^0(u^{(j)})}_2 \le m C \varepsilon, \]
which proves the result for $K = m C$.

\qed

\subsection*{Proof of Lemma \ref{covariance_bound}}

Let $L$ be the Lipschitz constant of $\mathcal{G}$. For all $x \in B_R(u^*)$ we have
\begin{align*}
& \norm{\mathcal{G}(x)}_2 \le \norm{\mathcal{G}(x) - \mathcal{G}(u^*)}_2 + \norm{\mathcal{G}(u^*)}_2 \le L \norm{x - u^*}_2 + \norm{\mathcal{G}(u^*)}_2 \le LR + G, \\
& \norm{x}_2 \le \norm{x - u^*}_2 + \norm{u^*}_2 \le R + g,
\end{align*}
and we define the bounds $M = LR + G$ and $m = R + g$. The same bounds can be deduced for the mean values
\begin{align}
\label{bound_mean_u}
& \norm{\bar{u}}_2 \le \frac{1}{J} \sum_{j=1}^J \norm{u^{(j)}}_2 \le \frac{1}{J} J m = m, \\
\label{bound_mean_G}
& \norm{\bar{\mathcal{G}}}_2 \le \frac{1}{J} \sum_{j=1}^J \norm{\mathcal{G}(u^{(j)})}_2 \le \frac{1}{J} J M = M.
\end{align}
By definition of $2$-norm of a matrix we have
\begin{align*}
\norm{C^{up}(u)}_2 = & \; \sup_{x \in \mathbb{R}^L \colon \norm{x}_2 = 1} \left \lVert \frac{1}{J} \sum_{j=1}^J (u^{(j)} - \bar{u}) (\mathcal{G}(u^{(j)}) - \bar{\mathcal{G}})^T x \right \rVert_2 \\
\le & \; \sup_{x \in \mathbb{R}^L \colon \norm{x}_2 = 1} \frac{1}{J} \sum_{j=1}^J \left | (\mathcal{G}(u^{(j)}) - \bar{\mathcal{G}})^T x \right | \norm{u^{(j)} - \bar{u}}_2 \\
\le & \; \sup_{x \in \mathbb{R}^L \colon \norm{x}_2 = 1} \frac{1}{J} \sum_{j=1}^J \left \lVert \mathcal{G}(u^{(j)}) - \bar{\mathcal{G}} \right \rVert_2 \norm{x}_2 \norm{u^{(j)} - \bar{u}}_2,
\end{align*}
and using (\ref{bound_mean_u}) and (\ref{bound_mean_G}) and the fact that $\norm{x}_2 = 1$ we obtain
\begin{align*}
\norm{C^{up}(u)}_2 \le & \; \frac{1}{J} \sum_{j=1}^J \left ( \norm{\mathcal{G}(u^{(j)})}_2 + \norm{\bar{\mathcal{G}}}_2 \right ) \left ( \norm{u^{(j)}}_2 + \norm{\bar{u}}_2 \right ) \\
\le & \; \frac{1}{J} J (M + M) (m + m) = 4Mm,
\end{align*}
and we define $C_1 = 4Mm$.
The procedure is similar for the matrix $C^{pp}(u)$, where we have
\begin{align*}
\norm{C^{pp}(u)}_2 = & \; \sup_{x \in \mathbb{R}^L \colon \norm{x}_2 = 1} \left \lVert \frac{1}{J} \sum_{j=1}^J (\mathcal{G}(u^{(j)}) - \bar{\mathcal{G}}) (\mathcal{G}(u^{(j)}) - \bar{\mathcal{G}})^T x \right \rVert_2 \\
\le & \; \sup_{x \in \mathbb{R}^L \colon \norm{x}_2 = 1} \frac{1}{J} \sum_{j=1}^J \left | (\mathcal{G}(u^{(j)}) - \bar{\mathcal{G}})^T x \right | \left \lVert \mathcal{G}(u^{(j)}) - \bar{\mathcal{G}} \right \rVert_2 \\
\le & \; \sup_{x \in \mathbb{R}^L \colon \norm{x}_2 = 1} \frac{1}{J} \sum_{j=1}^J \left \lVert \mathcal{G}(u^{(j)}) - \bar{\mathcal{G}} \right \rVert^2_2 \norm{x}_2,
\end{align*}
and using bound (\ref{bound_mean_G}) and the fact that $\norm{x}_2 = 1$ we obtain
\begin{align*}
\norm{C^{up}(u)}_2 \le & \; \frac{1}{J} \sum_{j=1}^J \left ( \norm{\mathcal{G}(u^{(j)})}_2 + \norm{\bar{\mathcal{G}}}_2 \right )^2 \\
\le & \; \frac{1}{J} J (M + M)^2 = 4M^2,
\end{align*}
and we define $C_2 = 4M^2$. \\
Before proving the last two results of the lemma we need the following estimates for the ensemble of particles $u_1$ and $u_2$
\begin{align*}
\norm{\bar{u}_1 - \bar{u}_2}_2 = & \; \left \lVert \frac{1}{J} \sum_{j=1}^J (u_1^{(j)} - u_2^{(j)}) \right \rVert_2 \le \frac{1}{J} \sum_{j=1}^J \norm{u_1^{(j)} - u_2^{(j)}}_2 = \norm{u_1 - u_2}, \\
\norm{\bar{\mathcal{G}}_1 - \bar{\mathcal{G}}_2}_2 = & \; \left \lVert \frac{1}{J} \sum_{j=1}^J (\mathcal{G}(u_1^{(j)}) - \mathcal{G}(u_2^{(j)})) \right \rVert_2 \le \frac{1}{J} \sum_{j=1}^J \norm{\mathcal{G}(u_1^{(j)}) - \mathcal{G}(u_2^{(j)})}_2 \\
& \hspace{137pt} \le L \frac{1}{J} \sum_{j=1}^J \norm{u_1^{(j)} - u_2^{(j)}}_2 \\
& \hspace{137pt} = L \norm{u_1 - u_2}.
\end{align*}
By definition of $2$ norm of a matrix and using the triangle inequality we have
\begin{align*}
& \norm{C^{up}(u_1) - C^{up}(u_2)}_2 \\
& \qquad = \sup_{x \in \mathbb{R}^L \colon \norm{x}_2 = 1} \left \lVert \frac{1}{J} \sum_{j=1}^J \left [ (u_1^{(j)} - \bar{u}_1) (\mathcal{G}(u_1^{(j)}) - \bar{\mathcal{G}}_1)^T x - (u_2^{(j)} - \bar{u}_2) (\mathcal{G}(u_2^{(j)}) - \bar{\mathcal{G}}_2)^T x \right ] \right \rVert_2 \\
& \qquad \le \quad \sup_{x \in \mathbb{R}^L \colon \norm{x}_2 = 1} \frac{1}{J} \sum_{j=1}^J \left \lVert (u_1^{(j)} - \bar{u}_1) (\mathcal{G}(u_1^{(j)}) - \bar{\mathcal{G}}_1)^T x - (u_1^{(j)} - \bar{u}_1) (\mathcal{G}(u_2^{(j)}) - \bar{\mathcal{G}}_2)^T x \right \rVert_2 \\
& \qquad \quad + \sup_{x \in \mathbb{R}^L \colon \norm{x}_2 = 1} \frac{1}{J} \sum_{j=1}^J \left \lVert (u_1^{(j)} - \bar{u}_1) (\mathcal{G}(u_2^{(j)}) - \bar{\mathcal{G}}_2)^T x - (u_2^{(j)} - \bar{u}_2) (\mathcal{G}(u_2^{(j)}) - \bar{\mathcal{G}}_2)^T x \right \rVert_2,
\end{align*}
which implies
\begin{align*}
& \norm{C^{up}(u_1) - C^{up}(u_2)}_2 \\
& \qquad \le \quad \sup_{x \in \mathbb{R}^L \colon \norm{x}_2 = 1} \frac{1}{J} \sum_{j=1}^J \left \lVert (u_1^{(j)} - \bar{u}_1) [ (\mathcal{G}(u_1^{(j)}) - \mathcal{G}(u_2^{(j)})) + (\bar{\mathcal{G}}_2 - \bar{\mathcal{G}}_1)]^T x ] \right \rVert_2 \\
& \qquad \quad + \sup_{x \in \mathbb{R}^L \colon \norm{x}_2 = 1} \frac{1}{J} \sum_{j=1}^J \left \lVert [ (u_1^{(j)} - u_2^{(j)}) + (\bar{u}_2 - \bar{u}_1) ] (\mathcal{G}(u_2^{(j)}) - \bar{\mathcal{G}}_2)^T x \right \rVert_2 \\ 
& \qquad \le \quad \sup_{x \in \mathbb{R}^L \colon \norm{x}_2 = 1} \frac{1}{J} \sum_{j=1}^J \norm{u_1^{(j)} - \bar{u}_1}_2 [ \norm{\mathcal{G}(u_1^{(j)}) - \mathcal{G}(u_2^{(j)})}_2 + \norm{\bar{\mathcal{G}}_2 - \bar{\mathcal{G}}_1}_2 ] \norm{x}_2 \\
& \qquad \quad + \sup_{x \in \mathbb{R}^L \colon \norm{x}_2 = 1} \frac{1}{J} \sum_{j=1}^J [ \norm{u_1^{(j)} - u_2^{(j)}}_2 + \norm{\bar{u}_2 - \bar{u}_1}_2 ] \norm{\mathcal{G}(u_2^{(j)}) - \bar{\mathcal{G}}_2} \norm{x}_2.
\end{align*}
Using bounds (\ref{bound_mean_u}) and (\ref{bound_mean_G}) and the fact that $\mathcal{G}$ is Lipschitz with constant $L$, we obtain
\begin{align*}
& \norm{C^{up}(u_1) - C^{up}(u_2)}_2 \\
& \qquad \le \frac{1}{J} \sum_{j=1}^J \left \{ (\norm{u_1^{(j)}}_2 + \norm{\bar{u}_1}_2) (L \norm{u_1^{(j)} - u_2^{(j)}}_2 + L \norm{u_1 - u_2}) \right \} \\
& \qquad \qquad + \frac{1}{J} \sum_{j=1}^J \left \{ (\norm{u_1^{(j)} - u_2^{(j)}}_2 + \norm{\bar{u}_2 - \bar{u}_1}_2) (\norm{\mathcal{G}(u_2^{(j)})}_2 + \norm{\bar{\mathcal{G}}_2}_2) \right \} \\
& \qquad \le \frac{1}{J} \sum_{j=1}^J \left \{ 2m ( LJ \norm{u_1 - u_2} + L \norm{u_1 - u_2} ) + ( J \norm{u_1 - u_2} + \norm{u_1 - u_2} ) 2M \right \} \\
& \qquad \le 2 (J+1) \max \{ mL, M \} \norm{u_1 - u_2},
\end{align*}
and we define $C_3 = 2 (J+1) \max \{ mL, M \}$. The computation is similar for the last point of the statement, for which we have
\begin{align*}
& \norm{C^{pp}(u_1) - C^{pp}(u_2)}_2 \\
& \qquad = \sup_{x \in \mathbb{R}^L \colon \norm{x}_2 = 1} \left \lVert \frac{1}{J} \sum_{j=1}^J \left [ (\mathcal{G}(u_1^{(j)}) - \bar{\mathcal{G}}_1) (\mathcal{G}(u_1^{(j)}) - \bar{\mathcal{G}}_1)^T x - (\mathcal{G}(u_2^{(j)}) - \bar{\mathcal{G}}_2) (\mathcal{G}(u_2^{(j)}) - \bar{\mathcal{G}}_2)^T x \right ] \right \rVert_2 \\
& \qquad \le \quad \sup_{x \in \mathbb{R}^L \colon \norm{x}_2 = 1} \frac{1}{J} \sum_{j=1}^J \left \lVert (\mathcal{G}(u_1^{(j)}) - \bar{\mathcal{G}}_1) (\mathcal{G}(u_1^{(j)}) - \bar{\mathcal{G}}_1)^T x - (\mathcal{G}(u_1^{(j)}) - \bar{\mathcal{G}}_1) (\mathcal{G}(u_2^{(j)}) - \bar{\mathcal{G}}_2)^T x \right \rVert_2 \\
& \qquad \quad + \sup_{x \in \mathbb{R}^L \colon \norm{x}_2 = 1} \frac{1}{J} \sum_{j=1}^J \left \lVert (\mathcal{G}(u_1^{(j)}) - \bar{\mathcal{G}}_1) (\mathcal{G}(u_2^{(j)}) - \bar{\mathcal{G}}_2)^T x - (\mathcal{G}(u_2^{(j)}) - \bar{\mathcal{G}}_2) (\mathcal{G}(u_2^{(j)}) - \bar{\mathcal{G}}_2)^T x \right \rVert_2, 
\end{align*}
which implies
\begin{align*}
& \norm{C^{pp}(u_1) - C^{pp}(u_2)}_2 \\
& \qquad \le \quad \sup_{x \in \mathbb{R}^L \colon \norm{x}_2 = 1} \frac{1}{J} \sum_{j=1}^J \left \lVert (\mathcal{G}(u_1^{(j)}) - \bar{\mathcal{G}}_1) [ (\mathcal{G}(u_1^{(j)}) - \mathcal{G}(u_2^{(j)})) + (\bar{\mathcal{G}}_2 - \bar{\mathcal{G}}_1)]^T x ] \right \rVert_2 \\
& \qquad \quad + \sup_{x \in \mathbb{R}^L \colon \norm{x}_2 = 1} \frac{1}{J} \sum_{j=1}^J \left \lVert [ (\mathcal{G}(u_1^{(j)}) - \mathcal{G}(u_2^{(j)})) + (\bar{\mathcal{G}}_2 - \bar{\mathcal{G}}_1) ] (\mathcal{G}(u_2^{(j)}) - \bar{\mathcal{G}}_2)^T x \right \rVert_2 , \\
& \qquad \le \quad \sup_{x \in \mathbb{R}^L \colon \norm{x}_2 = 1} \frac{1}{J} \sum_{j=1}^J \norm{\mathcal{G}(u_1^{(j)}) - \bar{\mathcal{G}}_1}_2 [ \norm{\mathcal{G}(u_1^{(j)}) - \mathcal{G}(u_2^{(j)})}_2 + \norm{\bar{\mathcal{G}}_2 - \bar{\mathcal{G}}_1}_2 ] \norm{x}_2 \\
& \qquad \quad + \sup_{x \in \mathbb{R}^L \colon \norm{x}_2 = 1} \frac{1}{J} \sum_{j=1}^J [ \norm{\mathcal{G}(u_1^{(j)}) - \mathcal{G}(u_2^{(j)})}_2 + \norm{\bar{\mathcal{G}}_2 - \bar{\mathcal{G}}_1}_2 ] \norm{\mathcal{G}(u_2^{(j)}) - \bar{\mathcal{G}}_2} \norm{x}_2.
\end{align*}
Using bounds (\ref{bound_mean_u}) and (\ref{bound_mean_G}) and the fact that $\mathcal{G}$ is Lipschitz with constant $L$, we obtain
\begin{align*}
& \norm{C^{pp}(u_1) - C^{pp}(u_2)}_2 \\
& \qquad \le \frac{1}{J} \sum_{j=1}^J \left \{ 2M ( LJ \norm{u_1 - u_2} + L \norm{u_1 - u_2} ) + ( LJ \norm{u_1 - u_2} + L \norm{u_1 - u_2} ) 2M \right \} \\
& \qquad = 4 M L (J+1) \norm{u_1 - u_2},
\end{align*}
and we define $C_4 = 4 M L (J+1)$.

\qed

\subsection*{Proof of Lemma \ref{fh_goes_to_0}}

\corr{Let us consider an ensemble $u \in \mathcal U_{J,M}$ with particles $u^{(j)} \in \R^M$, for $j = 1, \ldots, J$. For each particle we have 
\begin{align}
\norm{\mathcal{G}^0_h(u^{(j)}) - \mathcal{G}^0(u^{(j)})}_2 = & \; \norm{\mathcal{O}(\mathcal{S}^0_h(u^{(j)})) - \mathcal{O}(\mathcal{S}^0(u^{(j)}))}_2 \\
= & \; \norm{\mathcal{O}(p^0_h(u^{(j)})) - \mathcal{O}(p^0(u^{(j)}))}_2 \\
\le & \; m \norm{p^0_h(u^{(j)}) - p^0(u^{(j)})}_{L^2(\Omega)},
\end{align}
where we write explicitly the dependence of $p^0$ and $p^0_h$ on the particle it is generated by. We now consider the standard a priori error estimates of FEM, (see e.g. \cite[Theorem 3.2.5]{Cia02}), which reads
\[ \norm{p^0_h(u^{(j)}) - p^0(u^{(j)})}_{L^2(\Omega)} \le C \abs{p(u^{(j)})}_{H^{s+1}(\Omega)} h^{s+1}. \]
Moreover, higher order boundary regularity results for elliptic partial differential equations (see e.g. \cite[Theorem 6.3.5]{Eva10}), imply for a constant $\tilde{C} > 0$ and for all $j = 1, \ldots, J$
\[ \norm{p^0_h(u^{(j)}) - p^0(u^{(j)})}_{L^2(\Omega)} \leq \norm{f}_{H^{q-1}(\Omega)} h^{s+1}. \]
Finally, we obtain
\begin{align}
\tilde{e}(h, u) &= \; \frac1J \sum_{j=1}^J \norm{\mathcal{G}^0_h(u^{(j)}) - \mathcal{G}^0(u^{(j)})}_2\\
&\le m C \tilde{C} \norm{f}_{H^{q-1}(\Omega)} h^{s+1},
\end{align}
which proves the result for $\tilde{K} = m C \tilde{C} \norm{f}_{H^{q-1}(\Omega)}$.}
\qed

\subsection*{Proof of Lemma \ref{equivalence_convergence_DW1}}

We recall the duality formula (\ref{Wasserstein_dual}) for the Wasserstein distance $W_{1,s}$
\[ W_{1,s}(\mu_n, \mu) = \sup_{\varphi \in \Phi} \left \{ \int_{B_R(u^*)} \varphi d(\mu_n - \mu) \right \}, \]
where $\Phi$ is the set of all globally Lipschitz continuous functions $\varphi \colon B_R(u^*) \to \R$ with Lipschitz constant $L \le 1$. Note that if $\varphi \in \Phi$, then also $- \varphi \in \Phi$.
Therefore we deduce that
\begin{equation}
\label{equality_absolute_value}
W_{1,s}(\mu_n, \mu) = \sup_{\varphi \in \Phi} \left \{ \int_{B_R(u^*)} \varphi d(\mu_n - \mu) \right \} = \sup_{\varphi \in \Phi} \left \{ \left | \int_{B_R(u^*)} \varphi d(\mu_n - \mu) \right | \right \}.
\end{equation}
Indeed we have
\[ \int_{B_R(u^*)} \varphi d(\mu_n - \mu) \le \left | \int_{B_R(u^*)} \varphi d(\mu_n - \mu) \right |, \]
which implies the first inequality
\[ \sup_{\varphi \in \Phi} \left \{ \int_{B_R(u^*)} \varphi d(\mu_n - \mu) \right \} \le \sup_{\varphi \in \Phi} \left \{ \left | \int_{B_R(u^*)} \varphi d(\mu_n - \mu) \right | \right \}. \]
On the other hand, we also have
\begin{equation}
\label{inequality_sets}
A = \left \{ \int_{B_R(u^*)} \varphi d(\mu_n - \mu) \colon \varphi \in \Phi \right \} \supseteq \left \{ \left | \int_{B_R(u^*)} \varphi d(\mu_n - \mu) \right | \colon \varphi \in \Phi \right \} = A',
\end{equation}
because if $c \in A'$, which means that there exists $\varphi \in \Phi$ such that
\[ c = \left | \int_{B_R(u^*)} \varphi d(\mu_n - \mu) \right |, \]
then we can take $\tilde{\varphi} \in \Phi$ defined as
\begin{equation*}
\tilde{\varphi} = 
\begin{cases}
\varphi & \text{ if } \int_{B_R(u^*)} \varphi d(\mu_n - \mu) > 0 \\
- \varphi & \text{ if } \int_{B_R(u^*)} \varphi d(\mu_n - \mu) < 0, \\
\end{cases}
\end{equation*}
and note that that 
\[ c = \int_{B_R(u^*)} \tilde{\varphi} d(\mu_n - \mu), \]
which implies that $c \in A$. Therefore, by (\ref{inequality_sets}), we deduce the opposite inequality
\[ \sup_{\varphi \in \Phi} \left \{ \int_{B_R(u^*)} \varphi d(\mu_n - \mu) \right \} \ge \sup_{\varphi \in \Phi} \left \{ \left | \int_{B_R(u^*)} \varphi d(\mu_n - \mu) \right | \right \}. \]
Then, thanks to (\ref{equality_absolute_value}), we have
\begin{align*}
\sup_{\varphi \in \Phi} \mathbb{E}_{\xi} \left [ \left | \int_{B_R(u^*)} \varphi d(\mu_n - \mu) \right | \right ] \le & \; \mathbb{E}_{\xi} \left [ \sup_{\varphi \in \Phi} \left \{ \left | \int_{B_R(u^*)} \varphi d(\mu_n - \mu) \right | \right \} \right ] \\
= & \; \mathbb{E}_{\xi} [ W_{1,s}(\mu_n, \mu) ],
\end{align*}
and the right hand side vanishes by hypothesis, so we obtain
\[ \sup_{\varphi \in \Phi} \mathbb{E}_{\xi} \left [ \left | \int_{B_R(u^*)} \varphi d(\mu_n - \mu) \right | \right ] \to 0. \]
Hence
\begin{equation}
\label{for_all_phi}
\mathbb{E}_{\xi} \left [ \left | \int_{B_R(u^*)} \varphi d \mu_n - \int_{B_R(u^*)} \varphi d \mu \right | \right ] \to 0 \qquad \text{for all } \varphi \in \Phi.
\end{equation}
It remains to show that (\ref{for_all_phi}) holds true for all functions $f \in C^0(B_R(u^*))$. First, we consider any Lipschitz function $\psi$ with Lipschitz constant $L$. We define $\phi = \psi/L$, then $\phi \in \Phi$, indeed
\begin{equation*}
\abs{\phi(x) - \phi(y)} = \left | \frac{1}{L} \psi(x) - \frac{1}{L} \psi(y) \right | = \frac{1}{L} \abs{\psi(x) - \psi(y)} \le \frac{1}{L} L \norm{x - y}_s = \norm{x - y}_s.
\end{equation*}
Therefore we have
\begin{equation}
\label{for_all_psi}
\mathbb{E}_{\xi} \left [ \left | \int_{B_R(u^*)} \psi d \mu_n - \int_{B_R(u^*)} \psi d \mu \right | \right ] = L \mathbb{E}_{\xi} \left [ \left | \int_{B_R(u^*)} \phi d \mu_n - \int_{B_R(u^*)} \phi d \mu \right | \right ] \to 0.
\end{equation}
By density, any continuous bounded function $f \in C^0(B_R(u^*))$ can be approximated by a sequence of Lipschitz functions $\{ \psi_k \}_{k \in \mathbb{N}}$ such that $\norm{\psi_k}_{L^{\infty}(B_R(u^*))} \le C$ for all $k \in \mathbb{N}$ where $C$ is a constant dependent on $f$ and $\norm{\psi_k - f}_{L^{\infty}(B_R(u^*))} \to 0$ as $k \to \infty$. Thanks to (\ref{for_all_psi}) we have
\[ \mathbb{E}_{\xi} \left [ \left | \int_{B_R(u^*)} \psi_k d \mu_n - \int_{B_R(u^*)} \psi_k d \mu \right | \right ] \to 0, \]
and, applying Lebesgue dominated convergence theorem, we can pass to the limit as $k \to \infty$. We can exchange the limit with the expectation and the integral because the integrand functions are bounded by $C$ and the measures $\mu_n$ and $\mu$ are finite, since they are probability measures. Thus we obtain
\[ \mathbb{E}_{\xi} \left [ \left | \int_{B_R(u^*)} f d \mu_n - \int_{B_R(u^*)} f d \mu \right | \right ] \to 0, \]
for all bounded continuous functions $f \in C^0(B_R(u^*))$ which means
\[ \mu_n \xrightharpoonup{L^1} \mu, \]
which is the desired result.

\qed