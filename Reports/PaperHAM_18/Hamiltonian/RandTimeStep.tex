\documentclass{siamart1116}

% basics
\usepackage[left=3cm,right=3cm,top=2.5cm,bottom=2.5cm]{geometry}
\usepackage[utf8x]{inputenc}
\usepackage[title,titletoc]{appendix}
\usepackage{afterpage}
\usepackage{enumitem}   
\setlist[enumerate]{topsep=3pt,itemsep=3pt,label=(\roman*)}

% maths
\usepackage{mathtools}
\usepackage{amsmath}
\usepackage{amssymb}
\newsiamremark{assumption}{Assumption}
\newsiamremark{remark}{Remark}
\newsiamremark{example}{Example}
\numberwithin{theorem}{section}

% tables
\usepackage{booktabs}

% plots
\usepackage{graphicx}
\usepackage{pgfplots}
\usepackage{tikz}
\usetikzlibrary{arrows,decorations.pathmorphing,backgrounds,positioning,fit,matrix}
\usepackage[labelfont=bf]{caption}
\setlength{\belowcaptionskip}{-5pt}
\usepackage{here}
\usepackage[font=normal]{subcaption}

% title and authors
%\newcommand{\TheTitle}{Probabilistic numerical methods with random time steps for chaotic and geometric integration} 
\newcommand{\TheTitle}{Conservation Hamiltonian RTS-RK} 
\newcommand{\TheAuthors}{A. Abdulle, G. Garegnani}
%\headers{Probabilistic Runge-Kutta method based on random time steps}{\TheAuthors}
\headers{\TheTitle}{\TheAuthors}
\title{{\TheTitle}}
\author{Assyr Abdulle\thanks{Institute of Mathematics, \'Ecole Polytechnique F\'ed\'erale de Lausanne (\email{assyr.abdulle@epfl.ch})}
		\and
		Giacomo Garegnani\thanks{Institute of Mathematics, \'Ecole Polytechnique F\'ed\'erale de Lausanne (\email{giacomo.garegnani@epfl.ch})}}

% my commands 
\DeclarePairedDelimiter{\ceil}{\left\lceil}{\right\rceil}
\DeclarePairedDelimiter{\floor}{\lfloor}{\rfloor}
\DeclarePairedDelimiter{\abs}{\lvert}{\rvert}
\DeclarePairedDelimiter{\norm}{\|}{\|}
\renewcommand{\phi}{\varphi}
\renewcommand{\theta}{\vartheta}
\renewcommand{\Pr}{\mathbb{P}}
\newcommand{\eqtext}[1]{\ensuremath{\stackrel{#1}{=}}}
\newcommand{\leqtext}[1]{\ensuremath{\stackrel{#1}{\leq}}}
\newcommand{\iid}{\ensuremath{\stackrel{\text{i.i.d.}}{\sim}}}
\newcommand{\totext}[1]{\ensuremath{\stackrel{#1}{\to}}}
\newcommand{\rightarrowtext}[1]{\ensuremath{\stackrel{#1}{\longrightarrow}}}
\newcommand{\leftrightarrowtext}[1]{\ensuremath{\stackrel{#1}{\longleftrightarrow}}}
\newcommand{\pdv}[2]{\ensuremath\partial_{#2}#1}
\newcommand{\N}{\mathbb{N}}
\newcommand{\R}{\mathbb{R}}
\newcommand{\C}{\mathbb{C}}
\newcommand{\OO}{\mathcal{O}}
\newcommand{\epl}{\varepsilon}
\newcommand{\diffL}{\mathcal{L}}
\newcommand{\prior}{\mathcal{Q}}
\newcommand{\defeq}{\coloneqq}
\newcommand{\eqdef}{\eqqcolon}
\newcommand{\Var}{\operatorname{Var}}
\newcommand{\E}{\operatorname{\mathbb{E}}}
\newcommand{\MSE}{\operatorname{MSE}}
\newcommand{\trace}{\operatorname{tr}}
\newcommand{\MH}{\mathrm{MH}}
\newcommand{\ttt}{\texttt}
\newcommand{\Hell}{d_{\mathrm{Hell}}}
\newcommand{\sksum}{{\textstyle\sum}}
\newcommand{\dd}{\mathrm{d}}
\definecolor{shade}{RGB}{100, 100, 100}
\definecolor{bordeaux}{RGB}{128, 0, 50}
\newcommand{\corr}[1]{{\color{red}#1}}

\ifpdf
\hypersetup{
	pdftitle={\TheTitle},
	pdfauthor={\TheAuthors}
}
\fi

\begin{document}
\maketitle

\section{Mean Hamiltonian} Consider the Hamiltonian $E\colon \R^d \to \R$ and the ODE
\begin{equation}
	y' = J^{-1}\nabla Q(y), \quad y(0) = y_0.
\end{equation}
Applying a symplectic Runge Kutta method identified by its numerical flow $\Psi$, we have that the modified equation is still Hamiltonian and there exist functions $E_j$, $j = 2, \ldots$, such that
\begin{equation}\label{eq:ModifiedHamiltonian}
	\tilde Q(y) = Q(y) + h Q_2(y) + h^2 Q_3(y) + \ldots,
\end{equation}
where $h$ is the time step. The series in \eqref{eq:ModifiedHamiltonian} does not converge, hence we consider the truncation after $N$ terms 
\begin{equation}
	\tilde Q(y) = Q(y) + h Q_2(y) + \ldots + h^{N-1} Q_N(y).
\end{equation}
Moreover, if $q$ is the order of convergence for $\Psi$, we have that $E_i \equiv 0$ for $i = 2, \ldots, q$, hence
\begin{equation}\label{eq:ModifiedHamiltonianTrunc}
	\tilde Q(y) = Q(y) + h^q Q_{q+1}(y) + \ldots + h^{N-1} Q_N(y).
\end{equation}
Let us assume that $E$ is analytic in a neighbourhood of $y_0$ and denoting $f = J^{-1}\nabla E$ that there exist positive constants $R$ and $M$ such that $\norm{f(y)} \leq M$ for all $y \in B_{2R}(y_0) \subset \R^d$. Let us moreover introduce the constants $\mu$ and $\kappa$ given by
\begin{equation}
	\mu = \sum_{i=1}^{s} \abs{b_i}, \quad \kappa = \max_{i=1, \ldots, s} \sum_{j=1}^{s} \abs{a_{ij}},
\end{equation}
where $\{b_i\}_{i=1}^s$ and $\{a_{ij}\}_{i,j=1}^s$ are the coefficients of the Runge-Kutta method. Finally, let us introduce the constant $\eta = \max\{\kappa, \mu/(2\log 2 - 1)\}$. Denoting by $\tilde \phi_{N,t}(y)$ the flow of the equation corresponding to $\tilde E$, we have that the local error satisfies \cite[Theorem IX.7.6]{HLW06}
\begin{equation}\label{eq:LocalErrorModified}
	\norm{\Psi_h(y_0) - \tilde \phi_{N,h}(y_0)} \leq h \gamma M e^{-I_0 / h},
\end{equation}
for all $h \leq I_0 / 4$, where $I_0 = R / (eM\eta)$ and $\gamma = e(2 + 1.65\eta + \mu)$.

$[\ldots]$

\begin{lemma} Assume there exists a function $g\colon \R_+ \to (1, +\infty)$ such that $H_i \leq g(h) h$ almost surely. Then under the assumption \ldots, 
	\begin{align}
		\E \tilde Q(Y_n) - \tilde Q(y_0) &= \OO\big(e^{-I_0/(2g(h)h)}\big), \\
		\E Q(Y_n) - Q(y_0) &= \OO\big(h^q\big),
	\end{align}
	over exponentially long time intervals $nh \leq e^{I_0 / (2g(h)h)}$.
\end{lemma}
\begin{proof} We exploit the conservation of $\tilde E$ along the trajectories of its corresponding dynamical system, i.e., $\tilde E (\tilde \phi_{N,z} (y)) = \tilde Q(y)$ for $y \in \R^d$ and $z > 0$ and employ a telescopic sum to obtain
\begin{equation}
\begin{aligned}
	\E \tilde Q(Y_n) - \tilde Q(y_0) &= \sum_{j=1}^{n} \E\big(\tilde Q(Y_{j-1}) - \tilde Q(Y_{j-1})\big) \\
	&= \sum_{j=1}^{n} \E\big(\tilde Q(Y_{j-1}) - \tilde Q(\tilde\phi_{N, H_{j-1}}(Y_{j-1}))\big) \\
	&= \sum_{j=1}^{n} \E\E\big(\tilde Q(Y_{j-1}) - \tilde Q\tilde\phi_{N, H_{j-1}}(Y_{j-1})) \mid H_{j-1}\big),
\end{aligned}
\end{equation}
where we applied the total expectation with respect to $H_{j-1}$ for the last equality. Then, as $E$ is Lipschitz with constant independent of $h$ and under the assumptions on $\{H_i\}_{i\geq 0}$ and \eqref{eq:LocalErrorModified} we have
\begin{equation}
\begin{aligned}
\E \tilde Q(Y_n) - \tilde Q(y_0) &\leq C \sum_{j=0}^{n-1} \E\big(H_{j}e^{-I_0/H_j}\big)\\
&= C n \E\big(H_0e^{-I_0/H_0}\big),
\end{aligned}
\end{equation}
where the equality is given by the assumption of the random time steps being \textit{i.i.d.} Thanks to the assumption $H_0 \leq g(h)h$ a.s., we have
\begin{equation}
	\E \tilde Q(Y_n) - \tilde Q(y_0) \leq nh e^{-I_0/(g(h)h)},
\end{equation}
which implies the first result. The second result derives then immediately as $Q_{q+1} + h Q_{q+2} + \ldots + h^{N-q-1}Q_N$ is uniformly bounded independently of $h$.
\end{proof}

\section{Distribution} Guess by numerics, Figure \ref{fig:StdDev}, Kepler system, two-stage Gauss ($q = 4$). Over exponentially long times
\begin{equation}
	\sigma_n = \Var(Q(Y_n))^{1/2} = \OO(n^{1/2}h^{q+p}),
\end{equation}

\begin{figure}
	\centering
	\includegraphics[]{ConvSigma}
	\caption{Convergence of the standard deviation, $h = \{0.2, 0.1, 0.05, 0.025, 0.0125\}$. Reference slopes correspond to $n^{1/2}h^{q+p}$.}
	\label{fig:StdDev}
\end{figure}

\bibliographystyle{siam}
\bibliography{anmc}

\end{document}
