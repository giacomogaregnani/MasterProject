\documentclass[10pt]{article}

% Packages and macros go here
\usepackage[T1]{fontenc}
\usepackage{lmodern}
\usepackage[utf8x]{inputenx}
\usepackage{microtype}
\usepackage{framed}
\usepackage{listings}
\usepackage{vmargin}
\usepackage{setspace}
\usepackage{mathrsfs, mathenv}
\usepackage{amsmath, amsthm, amssymb, amsfonts, amscd}
\usepackage{graphicx}
\usepackage{epstopdf}
\usepackage[svgnames]{xcolor}
\usepackage{hyperref}
\usepackage[capitalise]{cleveref}
\hypersetup{citecolor=blue, colorlinks=true, linkcolor=black}
\setlength{\parskip}{6pt}
\setlength\parindent{0pt}
\usepackage{subcaption}
\usepackage{bbm}
\usepackage{cite}
\usepackage{verbatim}
\usepackage{pgfplots}
\usepackage{tikz}
\usepackage{etoolbox}
\usepackage{color}
\usepackage{lipsum}
\usepackage{ifthen}
\usepackage[linesnumbered, ruled, vlined]{algorithm2e}
\crefname{algocf}{Algorithm}{Algorithms}
\usepackage{autonum}

\theoremstyle{plain}
\newtheorem{theorem}{Theorem}[section]
\newtheorem{corollary}[theorem]{Corollary}
\newtheorem{lemma}[theorem]{Lemma}
\newtheorem{proposition}[theorem]{Proposition}
\numberwithin{equation}{section}

\theoremstyle{definition}
\newtheorem{definition}[theorem]{Definition}

\theoremstyle{remark}
\newtheorem{remark}[theorem]{Remark}
\newtheorem{assumption}[theorem]{Assumption}
\newtheorem{example}[theorem]{Example}


\ifpdf
  \DeclareGraphicsExtensions{.eps,.pdf,.png,.jpg}
\else
  \DeclareGraphicsExtensions{.eps}
\fi

\usepackage{mathtools}
% basics

% tables
\usepackage{booktabs}

% plots
\usepackage{pgfplots}
\usepackage{tikz}
\usetikzlibrary{patterns,arrows,decorations.pathmorphing,backgrounds,positioning,fit,matrix}
\usepackage[labelfont=bf]{caption}
\setlength{\belowcaptionskip}{-5pt}
\usepackage{here}
\usepackage[font=normal]{subcaption}

% Prevent itemized lists from running into the left margin inside theorems and proofs
\usepackage{enumitem}
\setlist[itemize]{leftmargin=.5in}
\setlist[enumerate]{leftmargin=.5in,topsep=3pt,itemsep=3pt,label=(\roman*)}

% Add a serial/Oxford comma by default.
\newcommand{\creflastconjunction}{, and~}

% Sets running headers as well as PDF title and authors
% title and authors

\newcommand{\email}[1]{\href{#1}{#1}}
\newcommand{\TheTitle}{Probabilistic methods for elliptic partial differential equations} 
\newcommand{\TheAuthors}{A. Abdulle, G. Garegnani}
%\headers{Random time steps for quantifying chaotic numerical integration}{\TheAuthors}
\title{{\TheTitle}}
\newcommand*\samethanks[1][\value{footnote}]{\footnotemark[#1]}
\author{Assyr Abdulle\thanks{Institute of Mathematics, \'Ecole Polytechnique F\'ed\'erale de Lausanne (\email{\{assyr.abdulle, giacomo.garegnani\}@epfl.ch})}
	\and
	Giacomo Garegnani\samethanks}
\date{}

\usepackage{amsopn}
\DeclareMathOperator{\diag}{diag}
\DeclarePairedDelimiter{\ceil}{\left\lceil}{\right\rceil}
\DeclarePairedDelimiter{\floor}{\lfloor}{\rfloor}
\DeclarePairedDelimiter{\abs}{\lvert}{\rvert}
\DeclarePairedDelimiter{\norm}{\lVert}{\rVert}
\renewcommand{\phi}{\varphi}
\renewcommand{\theta}{\vartheta}
\renewcommand{\Pr}{\mathbb{P}}
\newcommand{\btilde}{\widetilde}
\newcommand{\bhat}{\widehat}
\newcommand{\eqtext}[1]{\ensuremath{\stackrel{#1}{=}}}
\newcommand{\leqtext}[1]{\ensuremath{\stackrel{#1}{\leq}}}
\newcommand{\iid}{\ensuremath{\stackrel{\text{i.i.d.}}{\sim}}}
\newcommand{\totext}[1]{\ensuremath{\stackrel{#1}{\to}}}
\newcommand{\rightarrowtext}[1]{\ensuremath{\stackrel{#1}{\longrightarrow}}}
\newcommand{\leftrightarrowtext}[1]{\ensuremath{\stackrel{#1}{\longleftrightarrow}}}
\newcommand{\pdv}[2]{\ensuremath\partial_{#2}#1}
\newcommand{\N}{\mathbb{N}}
\newcommand{\R}{\mathbb{R}}
\newcommand{\C}{\mathbb{C}}
\newcommand{\OO}{\mathcal{O}}
\newcommand{\epl}{\varepsilon}
\newcommand{\diffL}{\mathcal{L}}
\newcommand{\prior}{\mathcal{Q}}
\newcommand{\defeq}{\coloneqq}
\newcommand{\eqdef}{\eqqcolon}
\newcommand{\Var}{\operatorname{Var}}
\newcommand{\E}{\operatorname{\mathbb{E}}}
\newcommand{\MSE}{\operatorname{MSE}}
\newcommand{\trace}{\operatorname{tr}}
\newcommand{\MH}{\mathrm{MH}}
\newcommand{\ttt}{\texttt}
\newcommand{\Hell}{d_{\mathrm{H}}}
\newcommand{\sksum}{{\textstyle\sum}}
\newcommand{\dd}{\, \mathrm{d}}
\renewcommand{\d}{\mathrm{d}}
\definecolor{shade}{RGB}{100, 100, 100}
\definecolor{bordeaux}{RGB}{128, 0, 50}
\newcommand{\corr}[1]{{\color{red}#1}}
\newcommand{\Tau}{\tau}
\newcommand{\LL}{L}
\newcommand{\HH}{H}
\newcommand{\WW}{W}
\newcommand{\mbf}{\mathbf}
\newcommand{\bfs}{\boldsymbol}
\newcommand{\todo}{{\color{red} TO DO}}
\newcommand{\X}{\mathbb{X}}
\newcommand{\nablar}{\nabla_{\hat x}}
\newcommand{\eval}[1]{\bigr\rvert_{#1}}
\newcommand{\normm}[1]{\norm{#1}_a}
%\newcommand{\normm}[1]{{\left\vert\kern-0.25ex\left\vert\kern-0.25ex\left\vert #1 
%		\right\vert\kern-0.25ex\right\vert\kern-0.25ex\right\vert}}

\usepackage[usestackEOL]{stackengine}
\newcommand\fop[3][9pt]{\mathop{\ensurestackMath{\stackengine{#1}%
			{\displaystyle#2}{\scriptstyle#3}{U}{c}{F}{F}{L}}}\limits}
\newcommand\finf[2][9pt]{\fop[#1]{\inf}{#2}}
\newcommand\fsum[2][14pt]{\fop[#1]{\sum}{#2}}

\definecolor{leg1}{RGB}{0,114,189}
\definecolor{leg2}{RGB}{217,83,25}
\definecolor{leg3}{RGB}{237,177,32}
\definecolor{leg4}{RGB}{126,47,142}
\definecolor{leg5}{RGB}{119,172,48}

\definecolor{leg21}{RGB}{62,38,169}
\definecolor{leg22}{RGB}{46,135,247}
\definecolor{leg23}{RGB}{55,200,151}
\definecolor{leg24}{RGB}{254,195,56}


\ifpdf
\hypersetup{
	pdftitle={\TheTitle},
	pdfauthor={\TheAuthors}
}
\fi


\begin{document}
\maketitle	

\section{Introduction}
Let $(\Omega, \mathcal A, P)$ be a probability space and consider the scalar Ornstein--Uhlenbeck process
\begin{equation}\label{eq:OU}
\begin{aligned}
	\dd X_t(\omega, a) &= -a X_t(\omega, a) \; \dd t + \sqrt{2\Sigma} \; \dd W_t(\omega),\\
	X_0(\omega, a) &= x \in \R,
\end{aligned}
\end{equation}
where $a$ is an unknown parameter with true value $a^\dagger \in \R^+$ and $\Sigma \in \R$ is fixed. Moreover, we consider the initial condition $x\in\R$ to be deterministic and independent of $a$. Let us define the forward mapping $\mathcal G\colon \Omega \times \R \to \R^N$ as
\begin{equation}\label{eq:ForwardMap}
	\mathcal G(\omega, a) \mapsto \mathcal X_N = \{X_{t_j}(\omega, a)\}_{j=1}^N,
\end{equation}
for a sequence of time instants $t_j$ defined as $t_j = jh$, for $h > 0$ and for $j = 1, \ldots, N$, $t_N = T$. We are given a set of observations $\mathcal Y_N \in \R^n$ of the process $X_t$ as
\begin{equation}\label{eq:Observations}
	\mathcal Y_N = \mathcal G(\omega^\dagger, a^\dagger) + \eta,
\end{equation}
where $\eta$ is an additive Gaussian source of noise distributed as $\eta \sim \mathcal N(0, \sigma^2_\eta I_N)$, where $I_d$ is the identity in $d$ dimensions, and $\omega^\dagger \in \Omega$ is a fixed event. As in \eqref{eq:ForwardMap}, we consider $\mathcal Y_N = \{Y_{t_j}(\omega, a)\}_{j=1}^N$. In the following, we denote by $\mathcal N(x; \mu, \sigma^2)$ the density of a Gaussian random variable of mean $\mu$ and variance $\sigma^2$ evaluated at a point $x$, i.e.,
\begin{equation}
	\mathcal N(x; \mu, \sigma^2) = \frac{1}{\sqrt{2\pi\sigma^2}} \exp\left\{ \frac{(x - \mu)^2}{2\sigma^2} \right\}.
\end{equation}
Fixed a Gaussian prior $p(a) = \mathcal N(a; 0, \sigma^2_a)$ on the drift coefficient, our goal is to compute the posterior distribution $p(a \mid \mathcal Y_N)$ of the parameter given the observations in a Bayesian framework. First, we apply Bayes' rule to obtain
\begin{equation}
\begin{aligned}
	p(a, \mathcal X_N \mid \mathcal Y_N) &\propto p(a, \mathcal X_N) \, p(\mathcal Y_N \mid a, \mathcal X_N)\\
	&\propto p(a) \, p(\mathcal X_N \mid a) \, p(\mathcal Y_N \mid a, \mathcal X_N),
\end{aligned}
\end{equation}
which, thanks to the observation model and to the Markov property of $X_t$, can be written as
\begin{equation}\label{eq:JointPosterior}
\begin{aligned}
	p(a, \mathcal X_N \mid \mathcal Y_N) &\propto p(a) \, p(\mathcal X_N \mid a) \prod_{j=1}^{N} p(Y_{t_j} \mid X_{t_j}, a)\\
	&= p(a) \, \prod_{j=1}^{N} p(X_{t_j} \mid X_{t_{j-1}}, a) \prod_{j=1}^{N} p(Y_{t_j} \mid X_{t_j}, a)
\end{aligned}
\end{equation}
Then, in order to obtain the marginal posterior of the parameter, one has to integrate out $\mathcal X_N$ and get the posterior distribution of the parameter
\begin{equation}\label{eq:MarginalPosterior}
	p(a \mid \mathcal Y_N) \propto \int p(a, \mathcal X_N \mid \mathcal Y_N) \, \dd \mathcal X_N
\end{equation}
Let us compute the terms in \eqref{eq:JointPosterior} separately. First, the prior is given by $p(a) = \mathcal N(a; 0, \sigma^2_a)$. Then, for the likelihood term we have $p(Y_{t_j} \mid X_{t_j}, a) = \mathcal N(X_{t_j}; Y_{t_j}, \sigma_\eta^2)$. Finally, due to the distribution of the OU process we get
\begin{equation}
	p(X_{t_j} \mid X_{t_{j-1}}, a) = \mathcal N\big(X_{t_j}; e^{-ah}X_{t_{j-1}}, a^{-1}\Sigma(1 - e^{-2ah})\big).
\end{equation}
For $h$ small enough, we can consider the transition probability of the Euler--Maruyama method applied to \eqref{eq:OU}, which reads
\begin{equation}
	p(X_{t_j} \mid X_{t_{j-1}}, a) = \mathcal N\big(X_{t_j}; (1-ah)X_{t_{j-1}}, 2\Sigma h\big),
\end{equation}
which is a good approximation of the true transition probability and allows for a simpler treatment. 

\subsection*{Analytical posterior for the drift coefficient}

\subsubsection*{Noiseless observations}

We consider
\begin{equation}\label{eq:Langevin}
\begin{aligned}
\dd X_t &= -a \nabla V(X_t) \; \dd t + \sqrt{2\beta^{-1}} \; \dd W_t(\omega),\\
X_0 &= x \in \R,
\end{aligned}
\end{equation}
and an EM sequence $\mathcal X = \{X_n\}_{n=0}^N$ with time step $h$. The approximated MLE $\hat a_{N, h}$ for $a$ is given by
\begin{equation}
\hat a_{N, h} = - \frac{\sum_{n=0}^{N-1} \langle \nabla V(X_n), X_{n+1} - X_n\rangle}{\sum_{n=0}^{N-1} \abs{\nabla V(X_n)}^2 h}.
\end{equation}
In the Bayesian framework, if no noise is added to the observations $\mathcal X$ and for a prior $a \sim \mu_0$ with density $P(a)$ we have
\begin{equation}
P(a \mid \mathcal X) \propto P(\mathcal X \mid a) \, P(a).
\end{equation}
Let us consider $\mu_0 = \mathcal U(A)$, where $A = \{\text{admissible values for } a \}$. Then, maximising $P(a \mid \mathcal X)$ is equivalent to maximising $P(\mathcal X \mid a)$ and therefore the MAP estimator $\tilde a_{N, h}$ is equal to the MLE estimator $\hat a_{N, h}$. For a Gaussian prior $\mu_0 = \mathcal N(\mu_a, \sigma^2_a)$, we get
\begin{equation}
\tilde a_{N, h} = - \frac{ \sigma_a^2 \beta\sum_{n=0}^{N-1} \langle \nabla V(X_n), X_{n+1} - X_n\rangle + \mu_a}{\sigma_a^2 \beta \sum_{n=0}^{N-1} \abs{\nabla V(X_n)}^2 h - 1}.
\end{equation}
Hence, in the limit $N \to \infty$, we have, independently of $\mu_a$ and $\sigma_a$,
\begin{equation}
\lim_{N\to \infty} \tilde a_{N, h} = \lim_{N\to \infty} \hat a_{N, h}.
\end{equation}

\subsubsection*{Noisy observations}

We first state a known result on Gaussian densities we will employ extensively in the computations below.
\begin{lemma}\label{lem:GaussianDensities} Let $p(x) = \mathcal N(x; \mu_p, \sigma^2_p)$ and $q(x) = \mathcal N(x; \mu_q, \sigma^2_q)$. Then
	\begin{equation}
		p(x) q(x) = Z \, \mathcal N(x; \mu_{pq}, \sigma^2_{pq}),
	\end{equation}
	where
	\begin{equation}
			\mu_{pq} = \frac{\mu_p \sigma^2_q + \mu_q \sigma^2}{\sigma^2_p + \sigma^2_q}, \quad \sigma^2_{pq} = \frac{\sigma^2_p \sigma^2_q}{\sigma^2_p + \sigma^2_q},
	\end{equation}
	and where
	\begin{equation}
		Z = \mathcal N(\mu_p; \mu_q, \sigma^2_p + \sigma^2_q).
	\end{equation}
\end{lemma}
\begin{proof} Find a reliable reference.
\end{proof}
Moreover, we will use the relation valid for any $c \in \R$ 
\begin{equation}\label{eq:GaussianConstant}
	\mathcal N(cx; \mu, \sigma^2) = \frac1c \mathcal N(x; c^{-1}\mu, c^{-2}\sigma^2). 
\end{equation}
In the following, we omit all constants which are independent of $\mathcal X_N$ and $a$. Let us compute the posterior for one single observation $Y_{t_1}$, where $t_1 = h$. In this case, the transition probability is given by $p(X_{t_1} \mid X_{t_0}, a) = \mathcal N(X_{t_1}; (1-ah)X_{t_0}, 2\Sigma h)$, and the observation likelihood is $p(Y_{t_1} \mid X_{t_1}, a) = \mathcal N(Y_{t_1}; X_{t_1}, 2\sigma_\eta^2)$, which by symmetry can be rewritten as
$p(Y_{t_1} \mid X_{t_1}, a) = \mathcal N(X_{t_1}; Y_{t_1}, 2\sigma_\eta^2)$. Thanks to Lemma \ref{lem:GaussianDensities} we have
\begin{equation}
	\mathcal N(X_{t_1}; (1-ah)X_{t_0}, 2\Sigma h) \; \mathcal N(X_{t_1}; Y_{t_1}, \sigma_\eta^2) = Z_0(a) \, \mathcal N(X_{t_1}; \mu_1, \sigma_1^2), 
\end{equation}
where $\mu_1$ and $\sigma_1^2$ are given by
\begin{equation}
\begin{aligned}
	\mu_1 &= \frac{(1-ah)X_{t_0} \sigma_\eta^2 + 2 Y_{t_1}\Sigma h}{2\Sigma h + \sigma_\eta^2},\\
	\sigma_1^2 &= \frac{2\Sigma h \sigma_\eta^2}{2\Sigma h + \sigma_\eta^2}, 
\end{aligned}
\end{equation}
and where $Z_0(a)$ reads
\begin{equation}
	Z_0(a) = \mathcal N((1-ah)X_{t_0}; Y_{t_1}, \sigma_\eta^2 + 2\Sigma h),
\end{equation}
which can be rewritten using \eqref{eq:GaussianConstant} as
\begin{equation}\label{eq:ZXT0}
	Z_0(a) = \frac{1}{1 - ah} \mathcal N\big(X_{t_0}; (1-ah)^{-1} Y_{t_1}, (1-ah)^{-2} (\sigma_\eta^2 + 2\Sigma h)\big)
\end{equation}
In general, in the following we will write
\begin{equation}
\begin{aligned}
	\mu_j &= \frac{(1-ah)X_{t_{j-1}} \sigma_\eta^2 + 2 Y_{t_j}\Sigma h}{2\Sigma h + \sigma_\eta^2},\\
	\sigma_j^2 &= \frac{2\Sigma h \sigma_\eta^2}{2\Sigma h + \sigma_\eta^2}, 
\end{aligned}
\end{equation}
and
\begin{equation}
\begin{aligned}
	Z_j(a) &= \mathcal N((1-ah)X_{t_j}; Y_{t_{j+1}}, \sigma_\eta^2 + 2\Sigma h)\\
	&= \frac{1}{1 - ah} \mathcal N\big(X_{t_j}; (1-ah)^{-1} Y_{t_{j+1}}, (1-ah)^{-2} (\sigma_\eta^2 + 2\Sigma h)\big).
\end{aligned}
\end{equation}
Finally, we get
\begin{equation}
\begin{aligned}
	p(a \mid \mathcal Y_1) &= \int p(a, \mathcal X_N \mid \mathcal Y_N)\, \dd \mathcal X_N\\
	&\propto \int p(a) \, p(X_{t_1} \mid X_{t_0}, a) \, p(Y_{t_1} \mid X_{t_0}, a) \, \dd X_{t_1} \\
	&= p(a) \, Z_0(a) \int \mathcal N(X_{t_1}; \mu_1, \sigma_1^2)\, \dd X_{t_1} = p(a) \, Z_0(a),
\end{aligned}
\end{equation}
and the marginal posterior on the drift coefficient is therefore given in closed form. Let us now consider a second observation $Y_{t_2}$ at time $t_2 = 2h$. In this case and thanks to the computations above, we have
\begin{equation}\label{eq:TwoObs}
	p(a \mid \mathcal Y_2) \propto p(a) \, Z_0(a) \int \mathcal N(X_{t_1}; \mu_1, \sigma_1^2) \, Z_1(a) \, \mathcal N(X_{t_2}; \mu_2, \sigma_2^2) \, \dd X_{t_2} \dd X_{t_1}.
\end{equation}
Let us compute the product $\mathcal N(X_{t_1}; \mu_1, \sigma_1^2) \, Z_1(a)$. Thanks to the relation \eqref{eq:ZXT0}, we have
\begin{equation}
	\mathcal N(X_{t_1}; \mu_1, \sigma_1^2) \, Z_1(a) \propto \frac{1}{1 - ah} \, \mathcal N(X_{t_1}; \mu_1, \sigma_1^2) \, \mathcal N\big(X_{t_1}; (1-ah)^{-1} Y_{t_2}, (1-ah)^{-2} (\sigma_\eta^2 + 2\Sigma h)\big),
\end{equation}
which gives thanks to Lemma \ref{lem:GaussianDensities}
\begin{equation}
	\mathcal N(X_{t_1}; \mu_1, \sigma_1^2) \, Z_1(a) \propto \btilde Z_1(a) \, \mathcal N (X_{t_1}; \tilde \mu_1, \tilde \sigma_1^2),
\end{equation}
where 
\begin{equation}
\begin{aligned}
	\tilde \mu_1 &= \frac{\mu_1 (1-ah)^{-2} (\sigma_\eta^2 + 2\Sigma h) + (1-ah)^{-1} Y_{t_2} \sigma_1^2}{\sigma_1^2 + (1-ah)^{-2} (\sigma_\eta^2 + 2\Sigma h)},\\
	\tilde \sigma_1^2 &= \frac{\sigma_1^2(1-ah)^{-2} (\sigma_\eta^2 + 2\Sigma h)}{\sigma_1^2 + (1-ah)^{-2} (\sigma_\eta^2 + 2\Sigma h)},
\end{aligned}
\end{equation}
and where
\begin{equation}
%\begin{aligned}
	\btilde Z_1(a) =  \frac{1}{1-ah} \mathcal N(\mu_1; (1-ah)^{-1} Y_{t_2}, \sigma_1^2 + (1-ah)^{-2} (\sigma_\eta^2 + 2\Sigma h))\\
%	&= \mathcal N\big((1-ah)\mu_1; Y_{t_2}, (1-ah)^2 \sigma_1^2 + (\sigma_\eta^2 + 2\Sigma h)\big).
%\end{aligned}
\end{equation}
%We introduce for the following the notation
%\begin{equation}
%	\btilde Z_j(a) = \mathcal N\big((1-ah)\mu_j; Y_{t_{j+1}}, (1-ah)^2 \sigma_j^2 + (\sigma_\eta^2 + 2\Sigma h)\big).
%\end{equation}
Replacing back into \eqref{eq:TwoObs}, one gets
\begin{equation}
\begin{aligned}
	p(a \mid \mathcal Y_2) &\propto p(a) \, Z_0(a) \, \btilde Z_1(a) \int \mathcal N(X_{t_1}; \tilde \mu_1, \tilde \sigma_1^2)  \int \mathcal N(X_{t_2}; \mu_2, \sigma_2^2) \, \dd X_{t_2} \, \dd X_{t_1}\\
	&= p(a) \, Z_0(a) \, \btilde Z_1(a).
\end{aligned}
\end{equation}
A recursive argument gives now the following result.
\begin{theorem} {\color{red} Maybe} The posterior distribution on the drift coefficient corresponding to $N$ observations is given by
	\begin{equation}
		p(a \mid \mathcal Y_N) \propto p(a) \, Z_0(a) \prod_{j=1}^{N-1} \btilde Z_j(a). 
	\end{equation}
	{\color{red} NO!}
\end{theorem}
\begin{proof} The result is obtained through an inductive argument. In particular, we first show
	\begin{equation}\label{eq:FactorizationInduction}
		p(a, \mathcal X_N \mid \mathcal Y_N) \propto p(a) \, Z_0(a) \Big(\prod_{j=1}^{N-1} \btilde Z_j(a)\Big) \, \mathcal N(X_{t_N}; \mu_N, \sigma_N^2) \prod_{j=1}^{N-1} \mathcal N(X_{t_j}; \tilde \mu_j, \tilde \sigma_j^2).
	\end{equation}
	The base cases $N = 1$ and $N = 2$ are treated above. Let us suppose the equality above to be true for $\mathcal X_{N-1}$ and $\mathcal Y_{N-1}$. Then
	\begin{equation}
	\begin{aligned}
		p(a, \mathcal X_N \mid \mathcal Y_N) &\propto p(a) \, \prod_{j=1}^{N} p(X_{t_j} \mid X_{t_{j-1}}, a) \prod_{j=1}^{N} p(Y_{t_j} \mid X_{t_j}, a) \\
		&= p(a, \mathcal X_{N-1} \mid \mathcal Y_{N-1}) \, p(X_{t_N} \mid X_{t_{N-1}}) \, p(Y_{t_N} \mid X_{t_N}, a)\\
		&\propto p(a) \, Z_0(a) \Big(\prod_{j=1}^{N-2} \btilde Z_j(a)\Big) \, \mathcal N(X_{t_{N-1}}; \mu_{N-1}, \sigma_{N-1}^2) \prod_{j=1}^{N-2} \mathcal N(X_{t_j}; \tilde \mu_j, \tilde \sigma_j^2)\\
		&\quad \times Z_{N-1}(a) \, \mathcal N(X_{t_N}; \mu_N, \sigma_N^2).
	\end{aligned}
	\end{equation}
	Computations similar to the case $N = 2$ lead to 
	\begin{equation}
		Z_{N-1}(a) \, \mathcal N(X_{t_{N-1}}; \mu_{N-1}, \sigma_{N-1}^2) = \btilde Z_{N-1}(a) \, \mathcal N(X_{t_{N-1}}; \tilde \mu_{N-1}, \tilde \sigma^2_{N-1}),
	\end{equation}
	which shows \eqref{eq:FactorizationInduction}. Let us now remark that for each $j$, $\tilde Z_j$ depends only on $\mu_j$, which in turn depends only on $X_{t_{j-1}}$.
	Hence, 
	\begin{equation}
	\begin{aligned}
		\frac{p(a \mid \mathcal Y_N)}{p(a) \, Z_0(a)} &\propto  \int \Big(\prod_{j=1}^{N-1} \btilde Z_j(a)\Big) \, \mathcal N(X_{t_N}; \mu_N, \sigma_N^2) \prod_{j=1}^{N-1} \mathcal N(X_{t_1}; \tilde \mu_j, \tilde \sigma_j^2) \, \dd \mathcal X_N\\
		&=  \int \cdots \int \Big(\prod_{j=1}^{N-1} \btilde Z_j(a)\Big) \,  \prod_{j=1}^{N-1} \mathcal N(X_{t_j}; \tilde \mu_j, \tilde \sigma_j^2) \int \mathcal N(X_{t_N}; \mu_N, \sigma_N^2) \, \dd X_{t_N} \, \dd \mathcal X_{N-1} \\
		&=  \int \cdots \int \Big(\prod_{j=1}^{N-1} \btilde Z_j(a)\Big) \,  \prod_{j=1}^{N-2} \mathcal N(X_{t_j}; \tilde \mu_j, \tilde \sigma_j^2) \int \mathcal N(X_{t_{N-1}}; \tilde \mu_{N-1}, \tilde \sigma_{N-1}^2) \, \dd X_{t_{N-1}} \, \dd \mathcal X_{N-2} \\
		&=  \int \cdots \int \Big(\prod_{j=1}^{N-2} \btilde Z_j(a)\Big) \,  \prod_{j=1}^{N-3} \mathcal N(X_{t_j}; \tilde \mu_j, \tilde \sigma_j^2) \int \btilde Z_{N-1}(a) \, \mathcal N(X_{t_{N-2}}; \tilde \mu_{N-2}, \tilde \sigma_{N-1}^2) \, \dd X_{t_{N-2}} \, \dd \mathcal X_{N-3} \\
	\end{aligned}
	\end{equation}
	{\color{red} The recursion looks worse than I thought.}
\end{proof}

\section{Particle Markov Chain Monte Carlo}

The Particle Markov Chain Monte Carlo methods (PMCMC) \cite{ADH10} offer a 


\bibliographystyle{siam}
\bibliography{anmc}
\end{document}