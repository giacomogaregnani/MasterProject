\documentclass{article}
\usepackage[utf8]{inputenc}
\usepackage[T1]{fontenc}
\usepackage{lmodern}
\usepackage{amsmath,amsfonts,amssymb}
%\usepackage{geometry}
\usepackage{vmargin}
\usepackage[hidelinks]{hyperref}
\newcommand*\samethanks[1][\value{footnote}]{\footnotemark[#1]}
\renewcommand{\arraystretch}{1.20}


\title{Model Misspecification And Uncertainty Quantification For Drift Estimation In Multiscale Diffusion Processes}
\author{\begin{tabular}{ccc}
		Assyr Abdulle \thanks{Institute of Mathematics, École Polytechnique Fédérale de
		Lausanne, \href{mailto:assyr.abdulle@epfl.ch}{assyr.abdulle@epfl.ch}}
		& Giacomo Garegnani \thanks{Institute of Mathematics, École Polytechnique Fédérale de Lausanne, \href{mailto:giacomo.garegnani@epfl.ch}{giacomo.garegnani@epfl.ch} (presenter)}
		& Grigorios A. Pavliotis \thanks{Department of Mathematics, Imperial College London, \href{mailto:g.pavliotis@imperial.ac.uk}{g.pavliotis@imperial.ac.uk}}
		\\ & Andrew M. Stuart \thanks{Department of Computing and Mathematical Sciences, Caltech, Pasadena, \href{mailto:astuart@caltech.edu}{astuart@caltech.edu}} &
		\end{tabular}}
\date{}

\begin{document}

\maketitle

\noindent We present a novel technique for estimating the drift function of a diffusion process possessing two separated time scales. Our aim is fitting a homogenized diffusion model to a continuous sample path coming from the full multiscale process, thus dealing with an issue of model misspecification. We consider a Bayesian framework and study the asymptotic limit of posterior distributions over the drift function. In this setting, we show on the one hand that if the continuous multiscale data are not pre-processed, then the posterior distribution concentrates asymptotically on the wrong value of the drift function. On the other hand, we show that data can be treated ahead of the inference procedure in order to obtain the desired posterior. In particular, we prove that there exists a family of transformations which are linear on the space of continuous sample paths and which, when applied to multiscale data, allow the posterior distribution to be asymptotically correct. We present a series of numerical examples on test cases which corroborate our theoretical findings.

\end{document}
