\documentclass{scrartcl}

% BASICS
%\usepackage{vmargin}
\usepackage[margin=3cm,footskip=0.25in]{geometry}
%\setlength{\footskip}{20pt}
\usepackage[utf8x]{inputenc}
\usepackage[title,titletoc]{appendix}
% MATHS
\usepackage{physics}
\usepackage{amssymb}
\usepackage{amsthm}
\usepackage{mathtools}
\usepackage{bbm}
% PLOTS
\usepackage{pgfplots} 
\usepackage{graphicx}
\usepackage{tikz}
\usepackage{subcaption}
\usepackage{here}
\usepackage[labelfont=bf]{caption}
% TABS AND ALGOS
\usepackage[linesnumbered, ruled]{algorithm2e}
\usepackage{lipsum}
\usepackage{array,booktabs}
% REFERENCES
\usepackage[hidelinks]{hyperref}
\mathtoolsset{showonlyrefs}
% MISC
\DeclarePairedDelimiter{\ceil}{\lceil}{\rceil}
\DeclarePairedDelimiter{\floor}{\lfloor}{\rfloor}
\renewcommand{\phi}{\varphi}
\newcommand{\eqtext}[1]{\ensuremath{\stackrel{#1}{=}}}
\newcommand{\leqtext}[1]{\ensuremath{\stackrel{#1}{\leq}}}
\newcommand{\iid}{\ensuremath{\stackrel{\text{i.i.d.}}{\sim}}}
\newcommand{\totext}[1]{\ensuremath{\stackrel{#1}{\to}}}
\newtheorem{theorem}{Proposition}[section]
\newtheorem{lemma}{Lemma}[section]
\newtheorem{assumption}{Assumption}[section]
\newtheorem{definition}{Definition}[section]
\theoremstyle{remark}
\newtheorem{remark}{Remark}[section]
\newcommand{\N}{\mathbb{N}}
\newcommand{\R}{\mathbb{R}}
\newcommand{\E}{\mathbb{E}}
\newcommand{\OO}{\mathcal{O}}
\newcommand{\epl}{\varepsilon}
\newcommand{\diffL}{\mathcal{L}}
\newcommand{\prior}{\mathcal{Q}}
\newcommand{\defeq}{\coloneqq}
\newcommand{\Var}{\operatorname{Var}}
\newcommand{\MSE}{\operatorname{MSE}}
\newcommand{\MH}{\mathrm{MH}}
\newcommand{\ttt}{\texttt}
\newcommand{\Hell}{d_{\mathrm{Hell}}}
\newcommand{\sksum}{\textstyle\sum}


\begin{document}
	
\section*{Idea}

Let us consider $f\colon \R^d \to \R^d$ and the differential equation
\begin{equation}\label{eq:ODE}
	u' = f(u), \quad u(0) = u_0 \in \R^d.
\end{equation}
Moreover, let us consider the Runge-Kutta approximation $U_k(h)$ at time $t_k = kh$ depending on the time step $h > 0$, i.e.
\begin{equation}\label{eq:Deterministic}
	U_k(h) = \Psi_h(U_{k-1}), \quad U_0(h) = u_0,
\end{equation} 
where $\Psi_t(\cdot)$ is the numerical flow of the RK method applied to \eqref{eq:ODE}. Finally, we consider $U_k^\xi(h)$ the solution given by the probabilistic solver with time step $h$ and r.v. $\xi$, i.e.
\begin{equation}\label{eq:Probabilistic}
	U_k^\xi(h) = \Psi_h(U_{k-1}) + \xi_k(h),
\end{equation}
where we use the same RK method as deterministic part. Now consider \eqref{eq:ODE} to be chaotic (e.g. Lorenz). In this case, since the system is chaotic, integrating with two different time steps $h_1$ and $h_2$ we get two different numerical solutions, hence the punctual solution with obtained with any time steps is not reliable (motivation of Probabilistic method).

Given a limit time step $h_{\max}$ (e.g., the maximum time step for which $\Psi$ is stable applied to \eqref{eq:ODE}), we could think of the numerical solution as depending continuously on the time step $h$, and then we could consider the long time behavior of the numerical solution w.r.t. $h$. For example, we could investigate if there exists a measure $\mu^{h_{\max}}$ such that for a function $\Phi\colon\R^d\to\R$ we have 
\begin{equation}\label{eq:detEquality}
	\lim_{N\to\infty} \frac{1}{h_{\max}} \int_{0}^{h_{\max}} \Phi(U_N(s))\dd s =
	\int_{\R^d} \Phi(y)\dd \mu^{h_{\max}}(y),
\end{equation}
where $U_N$ is the \textbf{deterministic} numerical solution \eqref{eq:Deterministic}. It is clear that if the ODE is not chaotic (e.g., test equation with $\lambda < 0$), at long term all the numerical solutions with different time steps will converge to the same stable value (e.g., zero), therefore $\mu = \delta$ for some Dirac delta. Now considering the probabilistic solver with fixed time step $h$, we could have a measure $\nu^{h,M}$ such that
\begin{equation}\label{eq:probEquality}
	\lim_{N\to\infty} \frac{1}{M} \sum_{i=1}^M \Phi(U_N^{\xi^{(i)}}(h)) =  
	\int_{\R^d} \Phi(y)\dd \nu^{h,M}(y),
\end{equation}
where $\xi^{(i)}$ are realizations of $\xi$ for $i = 1, \ldots, M$. Now, are $\mu^{h_{\max}}$ and $\nu^{h,M}$ approximations of the same probability measure?

\subsection*{Numerical example}

Consider the Lorenz system 
\begin{equation}\label{eq:Lorenz}
\begin{aligned}
x' &= \sigma(y - x), \quad &&x(0) = -10,\\
y' &= x(\rho - z) - y, \quad &&y(0) = -1,\\
z' &= xy - \beta z, \quad &&z(0) = 40.
\end{aligned}
\end{equation}
with $\sigma = 10$, $\rho = 28$, $\beta = 8/3$, so that the system has chaotic behavior. We consider final time $T = 200$ and the explicit midpoint as RK. Moreover, we consider
\begin{itemize}
	\item deterministic solver with variable $h$ (4000 different values of $h$ in the range $[1.68 \cdot 10^{-4}, 2.5 \cdot 10^{-2}]$)
	\item probabilistic solver with fixed $h = 1 \cdot 10^{-3}$, and $M = 10^4$ trajectories
	\item the function $\Phi\colon\R^3\to\R$ defined as $$\Phi(x) = x^Tx.$$ 
\end{itemize}
Then we consider the value of the functional $\Phi$ applied to the solution at final time obtained with the deterministic solver and probabilistic solver (Figure \ref{fig:densities}). Numerical results of the integrals in \eqref{eq:detEquality} and \eqref{eq:probEquality} displayed in Table \ref{tab:numResults} seem to confirm the idea that $\mu^{h_{\max}}$ and $\nu^{h,M}$ tend to the same measure. The integrals with respect to the measure are computed assuming the two measures admit a density function and approximating the density function from data (blue lines in plots).

\begin{figure}[t]
\begin{subfigure}{0.49\linewidth}
	\centering
	\resizebox{1.0\linewidth}{!}{% This file was created by matlab2tikz.
%
%The latest EFupdates can be retrieved from
%  http://www.mathworks.com/matlabcentral/fileexchange/22022-matlab2tikz-matlab2tikz
%where you can also make suggestions and rate matlab2tikz.
%
\definecolor{mycolor1}{rgb}{0.00000,0.44700,0.74100}%
\definecolor{mycolor2}{rgb}{0.85000,0.32500,0.09800}%
%
\begin{tikzpicture}

\begin{axis}[%
width=4.717in,
height=3.721in,
at={(0.791in,0.502in)},
scale only axis,
xmin=0,
xmax=2800,
xlabel={$\phi\text{(Y)}$},
xlabel style = {font = \LARGE},
xtick = {0, 400, 800, 1200, 1600, 2000, 2400, 2800},
ymin=0,
ymax=0.0012,
axis background/.style={fill=white},
title style={font=\bfseries},
title={deterministic solver, variable $h$},
ticklabel style={font=\LARGE},legend style={font=\LARGE},title style={font=\LARGE}
]
\addplot [color=mycolor1,solid,line width=2.0pt,forget plot]
  table[row sep=crcr]{%
3.41212810139583	0.000168704578842816\\
6.08839594988198	0.000174724546309478\\
8.76466379836814	0.000180876204875828\\
11.4409316468543	0.000187156104310055\\
14.1171994953404	0.000193565548360766\\
16.7934673438266	0.000200103181195958\\
19.4697351923128	0.000206768796671621\\
22.1460030407989	0.000213560400318306\\
24.8222708892851	0.00022047920332143\\
27.4985387377712	0.000227519740096496\\
30.1748065862574	0.000234681200455501\\
32.8510744347435	0.000241964885580722\\
35.5273422832297	0.000249368615286464\\
38.2036101317158	0.000256890108774356\\
40.879877980202	0.000264527220243847\\
43.5561458286882	0.000272279371629062\\
46.2324136771743	0.000280141077875544\\
48.9086815256605	0.000288112108148936\\
51.5849493741466	0.000296189595802146\\
54.2612172226328	0.000304372180885026\\
56.9374850711189	0.000312654350630897\\
59.6137529196051	0.00032103607925272\\
62.2900207680912	0.000329514589015864\\
64.9662886165774	0.000338085595276192\\
67.6425564650635	0.000346745709472986\\
70.3188243135497	0.000355493302516107\\
72.9950921620358	0.00036432347406221\\
75.671360010522	0.000373232436283429\\
78.3476278590081	0.000382217803363439\\
81.0238957074943	0.000391277143625662\\
83.7001635559805	0.000400406697453809\\
86.3764314044666	0.000409599402718463\\
89.0526992529528	0.000418857074869377\\
91.7289671014389	0.000428169322736617\\
94.4052349499251	0.000437538796553203\\
97.0815027984112	0.00044695611318823\\
99.7577706468974	0.000456418652781125\\
102.434038495384	0.000465922528123518\\
105.11030634387	0.000475463975287737\\
107.786574192356	0.000485036483497685\\
110.462842040842	0.000494639351829162\\
113.139109889328	0.000504265287615486\\
115.815377737814	0.000513911525687203\\
118.4916455863	0.000523575844553346\\
121.167913434787	0.000533246428786868\\
123.844181283273	0.000542924001565444\\
126.520449131759	0.000552602449950226\\
129.196716980245	0.000562276971753411\\
131.872984828731	0.000571943978851072\\
134.549252677217	0.000581596860318233\\
137.225520525704	0.000591232533432922\\
139.90178837419	0.000600849669255716\\
142.578056222676	0.000610438972873141\\
145.254324071162	0.000619996828406906\\
147.930591919648	0.000629516825831048\\
150.606859768134	0.000638996805965844\\
153.28312761662	0.000648434429995951\\
155.959395465107	0.000657820120215028\\
158.635663313593	0.000667152119789123\\
161.311931162079	0.000676426686606402\\
163.988199010565	0.000685637749513966\\
166.664466859051	0.000694780856717957\\
169.340734707537	0.000703854187342637\\
172.017002556024	0.000712851954218421\\
174.69327040451	0.000721770055843391\\
177.369538252996	0.000730606302648444\\
180.045806101482	0.000739353979145696\\
182.722073949968	0.00074801225831769\\
185.398341798454	0.000756573981621472\\
188.074609646941	0.000765038675936387\\
190.750877495427	0.00077340073779383\\
193.427145343913	0.000781658228372098\\
196.103413192399	0.000789808297523012\\
198.779681040885	0.000797847018215178\\
201.455948889371	0.000805771770034639\\
204.132216737857	0.000813580736444433\\
206.808484586344	0.00082126865346266\\
209.48475243483	0.000828837403043207\\
212.161020283316	0.000836282480421481\\
214.837288131802	0.000843597873868083\\
217.513555980288	0.000850784213129447\\
220.189823828774	0.000857841950085097\\
222.86609167726	0.000864768975012718\\
225.542359525747	0.00087156140679251\\
228.218627374233	0.000878218053633692\\
230.894895222719	0.000884740010538823\\
233.571163071205	0.000891123256439118\\
236.247430919691	0.000897369141268502\\
238.923698768177	0.000903476849771337\\
241.599966616664	0.000909444809635508\\
244.27623446515	0.000915271447861028\\
246.952502313636	0.000920959057546842\\
249.628770162122	0.000926504179403406\\
252.305038010608	0.000931908460875115\\
254.981305859094	0.000937172154119849\\
257.657573707581	0.000942295186390428\\
260.333841556067	0.000947277851655011\\
263.010109404553	0.0009521218231668\\
265.686377253039	0.000956824063697629\\
268.362645101525	0.000961390060247251\\
271.038912950011	0.000965816606994448\\
273.715180798497	0.000970107386033803\\
276.391448646984	0.000974259816479319\\
279.06771649547	0.000978275567587376\\
281.743984343956	0.000982157645868904\\
284.420252192442	0.000985907585964833\\
287.096520040928	0.000989524220663264\\
289.772787889414	0.000993011081043835\\
292.4490557379	0.000996366876904698\\
295.125323586387	0.000999595119183685\\
297.801591434873	0.001002696993474\\
300.477859283359	0.00100567324989493\\
303.154127131845	0.00100852413805349\\
305.830394980331	0.00101125360787416\\
308.506662828817	0.00101386318946721\\
311.182930677304	0.00101635021878777\\
313.85919852579	0.00101871977904661\\
316.535466374276	0.00102097126501604\\
319.211734222762	0.00102310746829425\\
321.888002071248	0.00102513047314659\\
324.564269919734	0.0010270388707151\\
327.24053776822	0.0010288354519693\\
329.916805616707	0.0010305236090231\\
332.593073465193	0.0010320993087823\\
335.269341313679	0.00103356764312962\\
337.945609162165	0.00103492921338805\\
340.621877010651	0.00103618370334296\\
343.298144859137	0.00103733509246878\\
345.974412707624	0.00103838152653934\\
348.65068055611	0.00103932540783622\\
351.326948404596	0.00104016701025008\\
354.003216253082	0.00104090693101532\\
356.679484101568	0.0010415465144325\\
359.355751950054	0.00104208899095499\\
362.032019798541	0.00104253254975705\\
364.708287647027	0.00104287753107067\\
367.384555495513	0.00104312462989489\\
370.060823343999	0.00104327430950432\\
372.737091192485	0.00104332948801698\\
375.413359040971	0.00104328821218405\\
378.089626889457	0.00104315199308615\\
380.765894737944	0.00104291961935768\\
383.44216258643	0.001042593419003\\
386.118430434916	0.00104217358283176\\
388.794698283402	0.00104165865736043\\
391.470966131888	0.00104105160564212\\
394.147233980374	0.00104035222616282\\
396.823501828861	0.00103955956381258\\
399.499769677347	0.00103867451828721\\
402.176037525833	0.00103769780792294\\
404.852305374319	0.00103662902106238\\
407.528573222805	0.00103547062355243\\
410.204841071291	0.00103421977549496\\
412.881108919777	0.00103287661684072\\
415.557376768264	0.00103144286479931\\
418.23364461675	0.00102991827307455\\
420.909912465236	0.00102830628804451\\
423.586180313722	0.00102660390177804\\
426.262448162208	0.00102481331358892\\
428.938716010694	0.00102293409482182\\
431.614983859181	0.00102096435692861\\
434.291251707667	0.0010189063936851\\
436.967519556153	0.00101675994979391\\
439.643787404639	0.0010145294453486\\
442.320055253125	0.00101221277426178\\
444.996323101611	0.00100980887984788\\
447.672590950097	0.0010073206103648\\
450.348858798584	0.00100474877965785\\
453.02512664707	0.00100209356163335\\
455.701394495556	0.000999353376404717\\
458.377662344042	0.000996532330303439\\
461.053930192528	0.00099362937349334\\
463.730198041014	0.000990646099223909\\
466.406465889501	0.000987584356576108\\
469.082733737987	0.000984447213054844\\
471.759001586473	0.000981230831862867\\
474.435269434959	0.000977939150326881\\
477.111537283445	0.000974573505213534\\
479.787805131931	0.000971133243128655\\
482.464072980417	0.0009676232831968\\
485.140340828904	0.000964041484080393\\
487.81660867739	0.000960390555616914\\
490.492876525876	0.000956671190354371\\
493.169144374362	0.000952887782913449\\
495.845412222848	0.000949038804879001\\
498.521680071334	0.000945125289724557\\
501.197947919821	0.000941152009494272\\
503.874215768307	0.000937119628721247\\
506.550483616793	0.000933029412630908\\
509.226751465279	0.000928882436292795\\
511.903019313765	0.000924680594257792\\
514.579287162251	0.000920428324459502\\
517.255555010737	0.000916125348457101\\
519.931822859224	0.000911776491773238\\
522.60809070771	0.000907380274046548\\
525.284358556196	0.00090294059453762\\
527.960626404682	0.000898458033404456\\
530.636894253168	0.000893938862396628\\
533.313162101654	0.000889381754154079\\
535.989429950141	0.000884786619613116\\
538.665697798627	0.000880160843964614\\
541.341965647113	0.000875505461785643\\
544.018233495599	0.000870822863682969\\
546.694501344085	0.000866111669576895\\
549.370769192571	0.000861377134950753\\
552.047037041057	0.000856622409139825\\
554.723304889544	0.00085184536148116\\
557.39957273803	0.000847053548295311\\
560.075840586516	0.000842247215684023\\
562.752108435002	0.000837428440721887\\
565.428376283488	0.000832601882774916\\
568.104644131975	0.000827768482684186\\
570.780911980461	0.000822929537797089\\
573.457179828947	0.000818085287068958\\
576.133447677433	0.00081324313261884\\
578.809715525919	0.0008084052907428\\
581.485983374405	0.00080357068257607\\
584.162251222891	0.000798742519012609\\
586.838519071378	0.000793923420152536\\
589.514786919864	0.0007891143517963\\
592.19105476835	0.000784314054692365\\
594.867322616836	0.000779532118013287\\
597.543590465322	0.000774768239245435\\
600.219858313808	0.000770023440722154\\
602.896126162294	0.000765296935258902\\
605.572394010781	0.000760597489755369\\
608.248661859267	0.000755923727485394\\
610.924929707753	0.00075127671256804\\
613.601197556239	0.000746653632083654\\
616.277465404725	0.000742061883767147\\
618.953733253211	0.000737503506449574\\
621.630001101698	0.000732980162266932\\
624.306268950184	0.000728489144981681\\
626.98253679867	0.000724035084580242\\
629.658804647156	0.000719617156898488\\
632.335072495642	0.000715237750609013\\
635.011340344128	0.000710895597850165\\
637.687608192614	0.000706594191557634\\
640.363876041101	0.000702334909788083\\
643.040143889587	0.000698117785776235\\
645.716411738073	0.000693942327245738\\
648.392679586559	0.000689815098665797\\
651.068947435045	0.000685730201069662\\
653.745215283531	0.000681687623444572\\
656.421483132018	0.000677691309763325\\
659.097750980504	0.000673739204829089\\
661.77401882899	0.000669832792030303\\
664.450286677476	0.000665971856731822\\
667.126554525962	0.000662156552541684\\
669.802822374448	0.000658387378552082\\
672.479090222934	0.000654665817241514\\
675.155358071421	0.000650989447144257\\
677.831625919907	0.000647360033776639\\
680.507893768393	0.000643772781733171\\
683.184161616879	0.000640234159850094\\
685.860429465365	0.000636738505846009\\
688.536697313851	0.000633284960060753\\
691.212965162337	0.000629876407660673\\
693.889233010824	0.000626512206354852\\
696.56550085931	0.000623190340159024\\
699.241768707796	0.000619909066330408\\
701.918036556282	0.000616669935222631\\
704.594304404768	0.000613469863952843\\
707.270572253254	0.00061030816430399\\
709.946840101741	0.00060718640710879\\
712.623107950227	0.000604101602525685\\
715.299375798713	0.000601054635253189\\
717.975643647199	0.00059804447609261\\
720.651911495685	0.000595068118862683\\
723.328179344171	0.000592128125448293\\
726.004447192658	0.000589221447399304\\
728.680715041144	0.000586345976348252\\
731.35698288963	0.000583504674753822\\
734.033250738116	0.000580693301263451\\
736.709518586602	0.000577913245105832\\
739.385786435088	0.000575161975149293\\
742.062054283575	0.000572440237214673\\
744.738322132061	0.000569747612669242\\
747.414589980547	0.000567081550083931\\
750.090857829033	0.000564440195342124\\
752.767125677519	0.000561824342184734\\
755.443393526005	0.000559235073660114\\
758.119661374491	0.000556670195587306\\
760.795929222978	0.000554127432535864\\
763.472197071464	0.000551608904928492\\
766.14846491995	0.000549111990243933\\
768.824732768436	0.000546638005735746\\
771.501000616922	0.000544186096773174\\
774.177268465408	0.000541757883036577\\
776.853536313894	0.000539350424941044\\
779.529804162381	0.000536961225780067\\
782.206072010867	0.000534592301704577\\
784.882339859353	0.000532245164193145\\
787.558607707839	0.000529916294101178\\
790.234875556325	0.000527608159760468\\
792.911143404811	0.000525320010094453\\
795.587411253298	0.000523052532824439\\
798.263679101784	0.000520804225314649\\
800.93994695027	0.000518578866742024\\
803.616214798756	0.000516371525549877\\
806.292482647242	0.000514182623381053\\
808.968750495728	0.000512013326555684\\
811.645018344214	0.00050986397467358\\
814.321286192701	0.000507735367055335\\
816.997554041187	0.000505628208406777\\
819.673821889673	0.00050354110651638\\
822.350089738159	0.000501475360989167\\
825.026357586645	0.000499429523584055\\
827.702625435131	0.000497403105804891\\
830.378893283618	0.000495400207047965\\
833.055161132104	0.000493416104135841\\
835.73142898059	0.000491457851450662\\
838.407696829076	0.000489517118738933\\
841.083964677562	0.000487602416702541\\
843.760232526048	0.000485708842203493\\
846.436500374534	0.000483838991592597\\
849.112768223021	0.000481991885206447\\
851.789036071507	0.000480168505320009\\
854.465303919993	0.000478367523501112\\
857.141571768479	0.00047659152246336\\
859.817839616965	0.000474839033616718\\
862.494107465451	0.000473109469171155\\
865.170375313938	0.000471406844152939\\
867.846643162424	0.000469730042341482\\
870.52291101091	0.000468076592157523\\
873.199178859396	0.000466448046468173\\
875.875446707882	0.000464844412746507\\
878.551714556368	0.00046326731297851\\
881.227982404855	0.000461715288505725\\
883.904250253341	0.000460184731453139\\
886.580518101827	0.000458680120747863\\
889.256785950313	0.000457201177950733\\
891.933053798799	0.000455749860177743\\
894.609321647285	0.000454323222085098\\
897.285589495771	0.000452919004380962\\
899.961857344258	0.000451542910849919\\
902.638125192744	0.00045019008043075\\
905.31439304123	0.00044886121607514\\
907.990660889716	0.000447556307252502\\
910.666928738202	0.000446276706085333\\
913.343196586688	0.000445019446108587\\
916.019464435175	0.00044378816076206\\
918.695732283661	0.000442579783509185\\
921.372000132147	0.000441393646977954\\
924.048267980633	0.000440230931179061\\
926.724535829119	0.00043908811007402\\
929.400803677605	0.000437970816486267\\
932.077071526091	0.000436874339002007\\
934.753339374578	0.000435799587266329\\
937.429607223064	0.000434745726094099\\
940.10587507155	0.000433712774093722\\
942.782142920036	0.000432698029158176\\
945.458410768522	0.000431704665991868\\
948.134678617008	0.000430728710766202\\
950.810946465495	0.000429773357645511\\
953.487214313981	0.000428834768801517\\
956.163482162467	0.000427915671040761\\
958.839750010953	0.000427015811491543\\
961.516017859439	0.000426134504933916\\
964.192285707925	0.000425267920422568\\
966.868553556411	0.000424415233328601\\
969.544821404898	0.000423579972036609\\
972.221089253384	0.00042276195825752\\
974.89735710187	0.000421958079573871\\
977.573624950356	0.000421171101819622\\
980.249892798842	0.000420398739681501\\
982.926160647328	0.000419638539715027\\
985.602428495815	0.000418891793902511\\
988.278696344301	0.000418159645675392\\
990.954964192787	0.000417439477646685\\
993.631232041273	0.000416733522883091\\
996.307499889759	0.000416040633826395\\
998.983767738245	0.0004153590116948\\
1001.66003558673	0.000414689893550726\\
1004.33630343522	0.000414034001383981\\
1007.0125712837	0.000413387261056875\\
1009.68883913219	0.000412755042865308\\
1012.36510698068	0.000412129873765799\\
1015.04137482916	0.000411517677893596\\
1017.71764267765	0.000410914475947307\\
1020.39391052613	0.000410322368343424\\
1023.07017837462	0.000409743252094182\\
1025.74644622311	0.000409172739808337\\
1028.42271407159	0.000408610876391351\\
1031.09898192008	0.000408061480467244\\
1033.77524976857	0.000407519984821438\\
1036.45151761705	0.000406989404769785\\
1039.12778546554	0.000406468386634336\\
1041.80405331402	0.000405955389348218\\
1044.48032116251	0.0004054519053899\\
1047.156589011	0.000404959329376347\\
1049.83285685948	0.000404474413631641\\
1052.50912470797	0.000403997180252652\\
1055.18539255645	0.00040353245755014\\
1057.86166040494	0.000403076143354443\\
1060.53792825343	0.000402630258142023\\
1063.21419610191	0.000402192205324205\\
1065.8904639504	0.000401762932878113\\
1068.56673179889	0.00040134425923782\\
1071.24299964737	0.000400934795623217\\
1073.91926749586	0.000400532624784948\\
1076.59553534434	0.000400143765094018\\
1079.27180319283	0.000399762886091133\\
1081.94807104132	0.000399389366640806\\
1084.6243388898	0.000399027244509826\\
1087.30060673829	0.000398672633726073\\
1089.97687458677	0.000398327721198424\\
1092.65314243526	0.000397990756442623\\
1095.32941028375	0.000397663921668274\\
1098.00567813223	0.000397347623502007\\
1100.68194598072	0.000397038618133974\\
1103.35821382921	0.000396738450923028\\
1106.03448167769	0.000396446075673549\\
1108.71074952618	0.000396163601024313\\
1111.38701737466	0.000395890497101763\\
1114.06328522315	0.000395626854736287\\
1116.73955307164	0.000395371347068424\\
1119.41582092012	0.000395126149166038\\
1122.09208876861	0.00039488872770245\\
1124.76835661709	0.000394662217490598\\
1127.44462446558	0.000394443096951045\\
1130.12089231407	0.000394232797701903\\
1132.79716016255	0.000394032357799443\\
1135.47342801104	0.000393838555916644\\
1138.14969585953	0.000393654282403602\\
1140.82596370801	0.000393480343881681\\
1143.5022315565	0.000393312958685223\\
1146.17849940498	0.000393154064967488\\
1148.85476725347	0.000393003851773313\\
1151.53103510196	0.000392860949882326\\
1154.20730295044	0.000392727217675776\\
1156.88357079893	0.000392601543877505\\
1159.55983864741	0.000392484890453678\\
1162.2361064959	0.000392376581105668\\
1164.91237434439	0.000392274927563146\\
1167.58864219287	0.00039218074002955\\
1170.26491004136	0.000392093544516296\\
1172.94117788985	0.000392013486920625\\
1175.61744573833	0.000391942710953314\\
1178.29371358682	0.000391877593949788\\
1180.9699814353	0.000391820153385388\\
1183.64624928379	0.000391767019754524\\
1186.32251713228	0.000391720769078546\\
1188.99878498076	0.000391680939439279\\
1191.67505282925	0.000391646435171972\\
1194.35132067773	0.000391617118468528\\
1197.02758852622	0.000391593205260723\\
1199.70385637471	0.00039157670953882\\
1202.38012422319	0.00039156428301335\\
1205.05639207168	0.000391555567058178\\
1207.73265992017	0.000391549647221491\\
1210.40892776865	0.000391545979732377\\
1213.08519561714	0.000391545133554968\\
1215.76146346562	0.000391548435144421\\
1218.43773131411	0.000391552283599442\\
1221.1139991626	0.000391558083254861\\
1223.79026701108	0.000391564738064835\\
1226.46653485957	0.00039157098533521\\
1229.14280270805	0.000391577057142918\\
1231.81907055654	0.000391581840165575\\
1234.49533840503	0.000391585341642529\\
1237.17160625351	0.000391585516102856\\
1239.847874102	0.000391584927628418\\
1242.52414195049	0.000391581055348416\\
1245.20040979897	0.000391571361550825\\
1247.87667764746	0.000391556617136828\\
1250.55294549594	0.000391536734705225\\
1253.22921334443	0.000391509217435061\\
1255.90548119292	0.000391475203113969\\
1258.5817490414	0.000391431829381798\\
1261.25801688989	0.000391380666172161\\
1263.93428473837	0.000391320041582806\\
1266.61055258686	0.000391249230603218\\
1269.28682043535	0.000391165939585904\\
1271.96308828383	0.0003910705031037\\
1274.63935613232	0.000390960827068403\\
1277.31562398081	0.000390838611831314\\
1279.99189182929	0.000390703961367356\\
1282.66815967778	0.000390552547487127\\
1285.34442752626	0.000390382617491455\\
1288.02069537475	0.000390196699314281\\
1290.69696322324	0.000389993184702724\\
1293.37323107172	0.00038977185106682\\
1296.04949892021	0.000389531538685462\\
1298.72576676869	0.000389272093776462\\
1301.40203461718	0.000388989333434689\\
1304.07830246567	0.00038868585234371\\
1306.75457031415	0.000388359876659125\\
1309.43083816264	0.000388011165834848\\
1312.10710601113	0.000387638352613928\\
1314.78337385961	0.000387242040915607\\
1317.4596417081	0.000386824213716648\\
1320.13590955658	0.000386379713337533\\
1322.81217740507	0.000385908578140695\\
1325.48844525356	0.0003854132750371\\
1328.16471310204	0.000384891638145464\\
1330.84098095053	0.000384344013154696\\
1333.51724879901	0.000383769288066606\\
1336.1935166475	0.000383168941672887\\
1338.86978449599	0.000382541661857803\\
1341.54605234447	0.000381885755791557\\
1344.22232019296	0.000381203213407347\\
1346.89858804145	0.000380495203715858\\
1349.57485588993	0.000379760178086349\\
1352.25112373842	0.000378997153725839\\
1354.9273915869	0.000378207781525163\\
1357.60365943539	0.000377393093239998\\
1360.27992728388	0.000376550993652253\\
1362.95619513236	0.000375682837997365\\
1365.63246298085	0.000374788089490543\\
1368.30873082933	0.000373869677269908\\
1370.98499867782	0.00037292574934343\\
1373.66126652631	0.000371958594988793\\
1376.33753437479	0.000370967197464087\\
1379.01380222328	0.00036995334789716\\
1381.69007007177	0.000368917347652384\\
1384.36633792025	0.000367859309525903\\
1387.04260576874	0.000366780802080623\\
1389.71887361722	0.000365681961224313\\
1392.39514146571	0.000364561364302534\\
1395.0714093142	0.000363423262758925\\
1397.74767716268	0.000362266922620573\\
1400.42394501117	0.000361093687558269\\
1403.10021285965	0.000359904391729053\\
1405.77648070814	0.000358700518859325\\
1408.45274855663	0.00035748232734626\\
1411.12901640511	0.000356249980657006\\
1413.8052842536	0.000355005988170757\\
1416.48155210209	0.00035374772272852\\
1419.15781995057	0.000352480608235541\\
1421.83408779906	0.000351201668550073\\
1424.51035564754	0.000349914890049387\\
1427.18662349603	0.000348619687091351\\
1429.86289134452	0.000347317254899317\\
1432.539159193	0.000346008816106273\\
1435.21542704149	0.000344695186311863\\
1437.89169488997	0.000343376721479332\\
1440.56796273846	0.000342054505014424\\
1443.24423058695	0.000340728346221079\\
1445.92049843543	0.000339401742749152\\
1448.59676628392	0.000338073773684319\\
1451.27303413241	0.000336744870192542\\
1453.94930198089	0.000335415709085044\\
1456.62556982938	0.000334086795254274\\
1459.30183767786	0.00033275985965464\\
1461.97810552635	0.000331433866177141\\
1464.65437337484	0.000330113618120673\\
1467.33064122332	0.000328795042538881\\
1470.00690907181	0.000327479922074681\\
1472.68317692029	0.000326169267969391\\
1475.35944476878	0.000324862104269883\\
1478.03571261727	0.000323560891764295\\
1480.71198046575	0.00032226508518763\\
1483.38824831424	0.000320973884555275\\
1486.06451616273	0.000319688640707718\\
1488.74078401121	0.000318407185406137\\
1491.4170518597	0.000317132711722667\\
1494.09331970818	0.000315866483856824\\
1496.76958755667	0.000314605354834789\\
1499.44585540516	0.000313349899869586\\
1502.12212325364	0.000312099792539549\\
1504.79839110213	0.000310857416008753\\
1507.47465895061	0.000309619038449839\\
1510.1509267991	0.000308389059203578\\
1512.82719464759	0.000307162766810304\\
1515.50346249607	0.000305943447917801\\
1518.17973034456	0.000304728474883563\\
1520.85599819305	0.000303517090476531\\
1523.53226604153	0.000302312286375554\\
1526.20853389002	0.000301111651817366\\
1528.8848017385	0.000299913705238494\\
1531.56106958699	0.000298718570381384\\
1534.23733743548	0.00029752788315769\\
1536.91360528396	0.000296339810950584\\
1539.58987313245	0.000295152741888982\\
1542.26614098093	0.000293967152792001\\
1544.94240882942	0.000292781991462078\\
1547.61867667791	0.000291598053259261\\
1550.29494452639	0.000290414969782631\\
1552.97121237488	0.000289231222926891\\
1555.64748022337	0.000288047199556516\\
1558.32374807185	0.000286859596901121\\
1561.00001592034	0.000285669898753208\\
1563.67628376882	0.000284478122630058\\
1566.35255161731	0.000283281653198314\\
1569.0288194658	0.000282081917282368\\
1571.70508731428	0.000280877137417333\\
1574.38135516277	0.000279667359649699\\
1577.05762301125	0.000278451389911749\\
1579.73389085974	0.000277228448879711\\
1582.41015870823	0.000276000017721617\\
1585.08642655671	0.000274761971842111\\
1587.7626944052	0.000273518765933526\\
1590.43896225369	0.000272265137936176\\
1593.11523010217	0.00027100315290575\\
1595.79149795066	0.000269732362596404\\
1598.46776579914	0.000268451395198181\\
1601.14403364763	0.000267160184833119\\
1603.82030149612	0.000265857370949633\\
1606.4965693446	0.000264542862917393\\
1609.17283719309	0.000263218752604561\\
1611.84910504157	0.000261881219457937\\
1614.52537289006	0.000260533631495964\\
1617.20164073855	0.000259174257459718\\
1619.87790858703	0.000257803220810002\\
1622.55417643552	0.000256418912029265\\
1625.23044428401	0.000255021668112624\\
1627.90671213249	0.000253612646212709\\
1630.58297998098	0.000252191757466884\\
1633.25924782946	0.000250757552820053\\
1635.93551567795	0.000249309688220328\\
1638.61178352644	0.000247849155231835\\
1641.28805137492	0.000246376706161578\\
1643.96431922341	0.000244891332991987\\
1646.64058707189	0.000243394084139365\\
1649.31685492038	0.000241885205716173\\
1651.99312276887	0.000240365617716958\\
1654.66939061735	0.000238833544544945\\
1657.34565846584	0.000237290417903781\\
1660.02192631433	0.000235734828777251\\
1662.69819416281	0.000234169894714774\\
1665.3744620113	0.000232594788669767\\
1668.05072985978	0.000231011434303365\\
1670.72699770827	0.000229418997270007\\
1673.40326555676	0.000227817325138822\\
1676.07953340524	0.000226206913747291\\
1678.75580125373	0.000224590341542213\\
1681.43206910221	0.000222966708199209\\
1684.1083369507	0.000221336695907317\\
1686.78460479919	0.00021970304981199\\
1689.46087264767	0.000218063135983824\\
1692.13714049616	0.00021642035281701\\
1694.81340834465	0.000214774172503498\\
1697.48967619313	0.000213124279009883\\
1700.16594404162	0.000211472320795547\\
1702.8422118901	0.000209819437579576\\
1705.51847973859	0.000208166535022698\\
1708.19474758708	0.000206514480913915\\
1710.87101543556	0.00020486476111712\\
1713.54728328405	0.000203216811946389\\
1716.22355113253	0.000201571895015028\\
1718.89981898102	0.00019992878265647\\
1721.57608682951	0.000198290344280798\\
1724.25235467799	0.000196656275462823\\
1726.92862252648	0.000195029711404207\\
1729.60489037497	0.000193410248524153\\
1732.28115822345	0.000191798589452112\\
1734.95742607194	0.000190194867157805\\
1737.63369392042	0.000188600626988046\\
1740.30996176891	0.000187015564017312\\
1742.9862296174	0.00018544132073569\\
1745.66249746588	0.000183877340711337\\
1748.33876531437	0.00018232337722799\\
1751.01503316285	0.000180783745426047\\
1753.69130101134	0.000179256229480178\\
1756.36756885983	0.000177743076795571\\
1759.04383670831	0.000176243368810316\\
1761.7201045568	0.000174758850995515\\
1764.39637240529	0.000173289569685191\\
1767.07264025377	0.000171833659608767\\
1769.74890810226	0.000170394157227891\\
1772.42517595074	0.00016897078789971\\
1775.10144379923	0.000167562945082213\\
1777.77771164772	0.000166171887187542\\
1780.4539794962	0.000164797182646666\\
1783.13024734469	0.000163439028351746\\
1785.80651519317	0.000162098262885416\\
1788.48278304166	0.000160774684656871\\
1791.15905089015	0.000159467751272253\\
1793.83531873863	0.000158178062581406\\
1796.51158658712	0.000156905487572897\\
1799.18785443561	0.000155649161585695\\
1801.86412228409	0.000154409496796249\\
1804.54039013258	0.000153186434020093\\
1807.21665798106	0.000151979670281692\\
1809.89292582955	0.000150788811239528\\
1812.56919367804	0.000149613759916959\\
1815.24546152652	0.000148454727805487\\
1817.92172937501	0.000147310073316973\\
1820.59799722349	0.000146179903556421\\
1823.27426507198	0.000145064545655139\\
1825.95053292047	0.000143963963491438\\
1828.62680076895	0.000142876161886296\\
1831.30306861744	0.000141801199458802\\
1833.97933646593	0.000140739997222393\\
1836.65560431441	0.000139690533458339\\
1839.3318721629	0.000138651761294293\\
1842.00814001138	0.000137623256757356\\
1844.68440785987	0.000136606396383787\\
1847.36067570836	0.000135598315574972\\
1850.03694355684	0.000134599955957615\\
1852.71321140533	0.000133610186234789\\
1855.38947925381	0.000132628070923013\\
1858.0657471023	0.00013165279272288\\
1860.74201495079	0.000130683557485164\\
1863.41828279927	0.000129720074424902\\
1866.09455064776	0.000128763024542094\\
1868.77081849625	0.000127809936590543\\
1871.44708634473	0.000126861864698388\\
1874.12335419322	0.000125916581396308\\
1876.7996220417	0.000124974474223769\\
1879.47588989019	0.00012403554333669\\
1882.15215773868	0.000123097563816202\\
1884.82842558716	0.000122159815429706\\
1887.50469343565	0.00012122487335255\\
1890.18096128413	0.000120289296127472\\
1892.85722913262	0.000119353826975253\\
1895.53349698111	0.000118417431275292\\
1898.20976482959	0.000117479823894579\\
1900.88603267808	0.000116541725284909\\
1903.56230052657	0.000115600834409374\\
1906.23856837505	0.000114657956934403\\
1908.91483622354	0.000113711128277621\\
1911.59110407202	0.000112762629156944\\
1914.26737192051	0.000111810671978245\\
1916.943639769	0.000110855190472067\\
1919.61990761748	0.000109896563976582\\
1922.29617546597	0.000108934628115599\\
1924.97244331445	0.000107968908046648\\
1927.64871116294	0.000106998685036078\\
1930.32497901143	0.000106023352054872\\
1933.00124685991	0.000105044772014496\\
1935.6775147084	0.000104061387795174\\
1938.35378255689	0.000103073364632077\\
1941.03005040537	0.000102081259566059\\
1943.70631825386	0.000101083784891346\\
1946.38258610234	0.000100082428719697\\
1949.05885395083	9.90770963455889e-05\\
1951.73512179932	9.8069053135372e-05\\
1954.4113896478	9.70572984618953e-05\\
1957.08765749629	9.60406813712541e-05\\
1959.76392534477	9.5019865323321e-05\\
1962.44019319326	9.39973023907303e-05\\
1965.11646104175	9.29706491079374e-05\\
1967.79272889023	9.19424650893945e-05\\
1970.46899673872	9.09121054453473e-05\\
1973.14526458721	8.98799153206255e-05\\
1975.82153243569	8.88450044219644e-05\\
1978.49780028418	8.78086906987595e-05\\
1981.17406813266	8.67715927954309e-05\\
1983.85033598115	8.57327526117358e-05\\
1986.52660382964	8.46945869504981e-05\\
1989.20287167812	8.36571082943649e-05\\
1991.87913952661	8.26195892830705e-05\\
1994.55540737509	8.15828001228202e-05\\
1997.23167522358	8.05486079552551e-05\\
1999.90794307207	7.95156925391988e-05\\
2002.58421092055	7.84859022250962e-05\\
2005.26047876904	7.74578477334484e-05\\
2007.93674661753	7.64330010199978e-05\\
2010.61301446601	7.54116541676087e-05\\
2013.2892823145	7.43948555323868e-05\\
2015.96555016298	7.33827799935392e-05\\
2018.64181801147	7.23766712485029e-05\\
2021.31808585996	7.13760820375651e-05\\
2023.99435370844	7.03802075704835e-05\\
2026.67062155693	6.93917211865619e-05\\
2029.34688940541	6.84107851983758e-05\\
2032.0231572539	6.74367445376425e-05\\
2034.69942510239	6.6470786327113e-05\\
2037.37569295087	6.55120156084489e-05\\
2040.05196079936	6.45617987551506e-05\\
2042.72822864785	6.36205352174484e-05\\
2045.40449649633	6.26908349105073e-05\\
2048.08076434482	6.17709405206398e-05\\
2050.7570321933	6.08609506800637e-05\\
2053.43330004179	5.99615506886168e-05\\
2056.10956789028	5.9072230842743e-05\\
2058.78583573876	5.81952907015901e-05\\
2061.46210358725	5.73279049921972e-05\\
2064.13837143573	5.6473302965798e-05\\
2066.81463928422	5.56313890144674e-05\\
2069.49090713271	5.48020459224953e-05\\
2072.16717498119	5.39854583092914e-05\\
2074.84344282968	5.31810057068472e-05\\
2077.51971067817	5.23894440097447e-05\\
2080.19597852665	5.1611215457107e-05\\
2082.87224637514	5.08458505736736e-05\\
2085.54851422362	5.00937988915537e-05\\
2088.22478207211	4.93548615070612e-05\\
2090.9010499206	4.86299822338293e-05\\
2093.57731776908	4.79183050969869e-05\\
2096.25358561757	4.72204172342615e-05\\
2098.92985346605	4.65357210198556e-05\\
2101.60612131454	4.58645472009151e-05\\
2104.28238916303	4.52070740022597e-05\\
2106.95865701151	4.45624305240466e-05\\
2109.63492486	4.39318764035461e-05\\
2112.31119270849	4.33141692630921e-05\\
2114.98746055697	4.27094963016053e-05\\
2117.66372840546	4.21179642548841e-05\\
2120.33999625394	4.15386020631369e-05\\
2123.01626410243	4.09723112722758e-05\\
2125.69253195092	4.04180663915468e-05\\
2128.3687997994	3.9875383744198e-05\\
2131.04506764789	3.93451837537005e-05\\
2133.72133549638	3.88263076616294e-05\\
2136.39760334486	3.83188309643424e-05\\
2139.07387119335	3.78229915326479e-05\\
2141.75013904183	3.73381743638758e-05\\
2144.42640689032	3.68637766622881e-05\\
2147.10267473881	3.6399220945798e-05\\
2149.77894258729	3.59442235617878e-05\\
2152.45521043578	3.54984753415927e-05\\
2155.13147828426	3.50616612314476e-05\\
2157.80774613275	3.46335442436264e-05\\
2160.48401398124	3.42153345060992e-05\\
2163.16028182972	3.38043750253307e-05\\
2165.83654967821	3.34017929914362e-05\\
2168.5128175267	3.30066126291823e-05\\
2171.18908537518	3.26187551223469e-05\\
2173.86535322367	3.22373071578208e-05\\
2176.54162107215	3.18622008258427e-05\\
2179.21788892064	3.1492845217401e-05\\
2181.89415676913	3.1129452180473e-05\\
2184.57042461761	3.07711322885265e-05\\
2187.2466924661	3.04183745867255e-05\\
2189.92296031458	3.00697209479328e-05\\
2192.59922816307	2.97254599000934e-05\\
2195.27549601156	2.93855129347346e-05\\
2197.95176386004	2.90493237395528e-05\\
2200.62803170853	2.87165877221764e-05\\
2203.30429955701	2.83870367583742e-05\\
2205.9805674055	2.80603796629229e-05\\
2208.65683525399	2.7735286786513e-05\\
2211.33310310247	2.74126941828528e-05\\
2214.00937095096	2.70937548567114e-05\\
2216.68563879945	2.67767143756656e-05\\
2219.36190664793	2.64610846216852e-05\\
2222.03817449642	2.61467536901132e-05\\
2224.7144423449	2.58332768511857e-05\\
2227.39071019339	2.55213397487326e-05\\
2230.06697804188	2.52110460731406e-05\\
2232.74324589036	2.49016474625123e-05\\
2235.41951373885	2.45927440197347e-05\\
2238.09578158733	2.42840064845609e-05\\
2240.77204943582	2.39756511738346e-05\\
2243.44831728431	2.36687070133208e-05\\
2246.12458513279	2.33627208963622e-05\\
2248.80085298128	2.30567688511261e-05\\
2251.47712082977	2.27514024448513e-05\\
2254.15338867825	2.24457543957372e-05\\
2256.82965652674	2.21401269117975e-05\\
2259.50592437522	2.18345302478339e-05\\
2262.18219222371	2.15295088691489e-05\\
2264.8584600722	2.12256292716711e-05\\
2267.53472792068	2.092143168158e-05\\
2270.21099576917	2.0617567419079e-05\\
2272.88726361766	2.03146353054928e-05\\
2275.56353146614	2.00117900133058e-05\\
2278.23979931463	1.97096627776739e-05\\
2280.91606716311	1.94071656371329e-05\\
2283.5923350116	1.91061611764331e-05\\
2286.26860286009	1.88057738162022e-05\\
2288.94487070857	1.85055288089333e-05\\
2291.62113855706	1.82058925795541e-05\\
2294.29740640554	1.79071956794783e-05\\
2296.97367425403	1.76103399279028e-05\\
2299.64994210252	1.7313982977111e-05\\
2302.326209951	1.70193496642983e-05\\
2305.00247779949	1.67251324943911e-05\\
2307.67874564798	1.64325913993427e-05\\
2310.35501349646	1.61420583359604e-05\\
2313.03128134495	1.58524927412627e-05\\
2315.70754919343	1.55645691773948e-05\\
2318.38381704192	1.52786717673457e-05\\
2321.06008489041	1.49946053142986e-05\\
2323.73635273889	1.47124365380264e-05\\
2326.41262058738	1.4431689874874e-05\\
2329.08888843586	1.4153644329874e-05\\
2331.76515628435	1.38772239726844e-05\\
2334.44142413284	1.36036317643601e-05\\
2337.11769198132	1.33320952871925e-05\\
2339.79395982981	1.3063256388609e-05\\
2342.4702276783	1.27965774642725e-05\\
2345.14649552678	1.25324610768376e-05\\
2347.82276337527	1.22715221481214e-05\\
2350.49903122375	1.20132000455607e-05\\
2353.17529907224	1.17570199455693e-05\\
2355.85156692073	1.15041693614e-05\\
2358.52783476921	1.1253814981065e-05\\
2361.2041026177	1.10066076120892e-05\\
2363.88037046618	1.07612060293102e-05\\
2366.55663831467	1.05193580244504e-05\\
2369.23290616316	1.02810521433134e-05\\
2371.90917401164	1.00451614366967e-05\\
2374.58544186013	9.81262471739603e-06\\
2377.26170970861	9.58339455825901e-06\\
2379.9379775571	9.35664846704246e-06\\
2382.61424540559	9.13358584803969e-06\\
2385.29051325407	8.91331601362632e-06\\
2387.96678110256	8.69618363577154e-06\\
2390.64304895105	8.48217641121098e-06\\
2393.31931679953	8.27158347558902e-06\\
2395.99558464802	8.06410462926501e-06\\
2398.6718524965	7.85921877952907e-06\\
2401.34812034499	7.65804114713621e-06\\
2404.02438819348	7.45998237213924e-06\\
2406.70065604196	7.26448300561529e-06\\
2409.37692389045	7.07219466892968e-06\\
2412.05319173894	6.88331531999781e-06\\
2414.72945958742	6.69756466218099e-06\\
2417.40572743591	6.51521302949975e-06\\
2420.08199528439	6.3359394490052e-06\\
2422.75826313288	6.15973641972942e-06\\
2425.43453098137	5.9865952708386e-06\\
2428.11079882985	5.81623402145549e-06\\
2430.78706667834	5.64893222118766e-06\\
2433.46333452682	5.48471027144639e-06\\
2436.13960237531	5.3235140224593e-06\\
2438.8158702238	5.16534275232816e-06\\
2441.49213807228	5.01046101506597e-06\\
2444.16840592077	4.8582673235686e-06\\
2446.84467376926	4.70933088092786e-06\\
2449.52094161774	4.5630464390241e-06\\
2452.19720946623	4.41999400404353e-06\\
2454.87347731471	4.2795584141531e-06\\
2457.5497451632	4.14232509640623e-06\\
2460.22601301169	4.00740608222761e-06\\
2462.90228086017	3.87595300186012e-06\\
2465.57854870866	3.74704257467671e-06\\
2468.25481655714	3.62126327238784e-06\\
2470.93108440563	3.49827694955471e-06\\
2473.60735225412	3.37806639658088e-06\\
2476.2836201026	3.26061374498838e-06\\
2478.95988795109	3.14590045525344e-06\\
2481.63615579957	3.03364901333274e-06\\
2484.31242364806	2.92438442008309e-06\\
2486.98869149655	2.81752144456975e-06\\
2489.66495934503	2.71361903888123e-06\\
2492.34122719352	2.61234712226126e-06\\
2495.01749504201	2.51340234197978e-06\\
2497.69376289049	2.41735401156317e-06\\
2500.37003073898	2.32358231627978e-06\\
2503.04629858746	2.23265499248562e-06\\
2505.72256643595	2.14423167640806e-06\\
2508.39883428444	2.05828611234156e-06\\
2511.07510213292	1.97479108251153e-06\\
2513.75136998141	1.89371840647798e-06\\
2516.42763782989	1.8150389433805e-06\\
2519.10390567838	1.73843881678321e-06\\
2521.78017352687	1.66396258546158e-06\\
2524.45644137535	1.59236356313017e-06\\
2527.13270922384	1.52276055084442e-06\\
2529.80897707233	1.4556740910296e-06\\
2532.48524492081	1.39077618044469e-06\\
2535.1615127693	1.32803229521654e-06\\
2537.83778061778	1.26740700024076e-06\\
2540.51404846627	1.20886398986068e-06\\
2543.19031631476	1.15236613141274e-06\\
2545.86658416324	1.0975903110357e-06\\
2548.54285201173	1.04509941244587e-06\\
2551.21911986021	9.9427632221416e-07\\
2553.8953877087	9.45626245324243e-07\\
2556.57165555719	8.98819397715757e-07\\
2559.24792340567	8.53529858458002e-07\\
2561.92419125416	8.10318630687607e-07\\
2564.60045910265	7.68557692983626e-07\\
2567.27672695113	7.28768364482766e-07\\
2569.95299479962	6.90609751153112e-07\\
2572.6292626481	6.53764451644711e-07\\
2575.30553049659	6.18771741577288e-07\\
2577.98179834508	5.84712050209959e-07\\
2580.65806619356	5.52742524199852e-07\\
2583.33433404205	5.22184136911288e-07\\
2586.01060189053	4.9299549896992e-07\\
2588.68686973902	4.64875087436328e-07\\
2591.36313758751	4.37769187687041e-07\\
2594.03940543599	4.12259183099493e-07\\
2596.71567328448	3.87705985580341e-07\\
2599.39194113297	3.64904161082461e-07\\
2602.06820898145	3.43210644633955e-07\\
2604.74447682994	3.22587259409518e-07\\
2607.42074467842	3.02996157806938e-07\\
2610.09701252691	2.84142498487572e-07\\
2612.7732803754	2.66532307277898e-07\\
2615.44954822388	2.49840503207768e-07\\
2618.12581607237	2.33751370732448e-07\\
2620.80208392085	2.18820536284407e-07\\
2623.47835176934	2.04699725305394e-07\\
2626.15461961783	1.91355444632531e-07\\
2628.83088746631	1.7875487438784e-07\\
2631.5071553148	1.66865903278109e-07\\
2634.18342316329	1.55657159969498e-07\\
2636.85969101177	1.45098040633982e-07\\
2639.53595886026	1.34880193644413e-07\\
2642.21222670874	1.25295117055127e-07\\
2644.88849455723	1.16565812500262e-07\\
2647.56476240572	1.08365943535248e-07\\
2650.2410302542	1.00669600813925e-07\\
2652.91729810269	9.34516832817788e-08\\
2655.59356595117	8.66879064338464e-08\\
2658.26983379966	8.03548075530583e-08\\
2660.94610164815	7.41707074747859e-08\\
2663.62236949663	6.86603046139394e-08\\
2666.29863734512	6.35121822325581e-08\\
2668.97490519361	5.87064489192619e-08\\
2671.65117304209	5.4223996448203e-08\\
2674.32744089058	5.00464925145525e-08\\
2677.00370873906	4.58864964457944e-08\\
};
\addplot[fill=mycolor2,fill opacity=0.5,draw=black,ybar interval,area legend] plot table[row sep=crcr] 
\end{subfigure}
\begin{subfigure}{0.49\linewidth}
	\centering
	\resizebox{1.0\linewidth}{!}{% This file was created by matlab2tikz.
%
%The latest EFupdates can be retrieved from
%  http://www.mathworks.com/matlabcentral/fileexchange/22022-matlab2tikz-matlab2tikz
%where you can also make suggestions and rate matlab2tikz.
%
\definecolor{mycolor1}{rgb}{0.00000,0.44700,0.74100}%
\definecolor{mycolor2}{rgb}{0.85000,0.32500,0.09800}%
%
\begin{tikzpicture}

\begin{axis}[%
width=4.717in,
height=3.721in,
at={(0.791in,0.502in)},
scale only axis,
xmin=0,
xmax=2800,
xlabel={$\phi\text{(Y)}$},
xlabel style = {font = \LARGE},
xtick = {0, 400, 800, 1200, 1600, 2000, 2400, 2800},
ymin=0,
ymax=0.0012,
axis background/.style={fill=white},
title style={font=\bfseries},
title={probabilistic solver, $M$ realizations},
ticklabel style={font=\LARGE},legend style={font=\LARGE},title style={font=\LARGE}
]
\addplot [color=mycolor1,solid,line width=2.0pt,forget plot]
  table[row sep=crcr]{%
4.41530241405264	0.000147560316192683\\
4.69004947971748	0.000148182670089277\\
4.96479654538232	0.000148806993158385\\
5.23954361104716	0.000149432956428872\\
5.514290676712	0.000150060392857437\\
5.78903774237684	0.000150689135545758\\
6.06378480804168	0.000151319346615526\\
6.33853187370652	0.000151951354099978\\
6.61327893937137	0.000152585322454082\\
6.88802600503621	0.000153220761081926\\
7.16277307070105	0.000153857666569393\\
7.43752013636589	0.000154496201025145\\
7.71226720203073	0.000155136529725375\\
7.98701426769557	0.000155778822898456\\
8.26176133336041	0.000156422581375431\\
8.53650839902526	0.000157067475986743\\
8.8112554646901	0.000157714488849926\\
9.08600253035494	0.000158362472215786\\
9.36074959601978	0.000159012407246197\\
9.63549666168462	0.000159664297061535\\
9.91024372734946	0.000160317481710433\\
10.1849907930143	0.000160973116300871\\
10.4597378586791	0.000161629720384047\\
10.734484924344	0.000162287782467107\\
11.0092319900088	0.000162947629858437\\
11.2839790556737	0.000163609097462677\\
11.5587261213385	0.000164272684297025\\
11.8334731870033	0.000164937395548012\\
12.1082202526682	0.000165603726661259\\
12.382967318333	0.000166271509226534\\
12.6577143839979	0.000166941402634662\\
12.9324614496627	0.000167612748167546\\
13.2072085153276	0.000168285377758619\\
13.4819555809924	0.000168959619289107\\
13.7567026466572	0.00016963563665922\\
14.0314497123221	0.00017031326279918\\
14.3061967779869	0.000170992499856432\\
14.5809438436518	0.000171674008662564\\
14.8556909093166	0.000172356136793996\\
15.1304379749814	0.000173040199139888\\
15.4051850406463	0.000173726027451087\\
15.6799321063111	0.000174412965389386\\
15.954679171976	0.000175101995729686\\
16.2294262376408	0.00017579196926536\\
16.5041733033056	0.000176483866943161\\
16.7789203689705	0.000177177524764285\\
17.0536674346353	0.000177872946049364\\
17.3284145003002	0.000178570130353687\\
17.603161565965	0.000179268247437496\\
17.8779086316299	0.000179968618645152\\
18.1526556972947	0.00018067041450889\\
18.4274027629595	0.000181373637074154\\
18.7021498286244	0.000182078281550417\\
18.9768968942892	0.000182784507639609\\
19.2516439599541	0.000183492151286666\\
19.5263910256189	0.00018420170211955\\
19.8011380912837	0.000184913000930699\\
20.0758851569486	0.000185625551645528\\
20.3506322226134	0.000186339844469464\\
20.6253792882783	0.000187055876892931\\
20.9001263539431	0.000187772989665841\\
21.1748734196079	0.000188492001764842\\
21.4496204852728	0.000189212749339773\\
21.7243675509376	0.000189934739661264\\
21.9991146166025	0.000190658293285989\\
22.2738616822673	0.00019138308058447\\
22.5486087479321	0.000192109920736486\\
22.823355813597	0.000192838324081089\\
23.0981028792618	0.000193567954806133\\
23.3728499449267	0.000194299803088863\\
23.6475970105915	0.000195033043909706\\
23.9223440762564	0.000195767345675921\\
24.1970911419212	0.000196503194837455\\
24.471838207586	0.000197240589814816\\
24.7465852732509	0.000197979694414536\\
25.0213323389157	0.00019872067469649\\
25.2960794045806	0.000199463367190812\\
25.5708264702454	0.000200207108868037\\
25.8455735359102	0.000200952887271004\\
26.1203206015751	0.00020170004077112\\
26.3950676672399	0.000202448567077948\\
26.6698147329048	0.00020319846119565\\
26.9445617985696	0.000203950050875901\\
27.2193088642344	0.000204703501388504\\
27.4940559298993	0.000205458649393125\\
27.7688029955641	0.00020621483122937\\
28.043550061229	0.000206972700989013\\
28.3182971268938	0.000207731762194228\\
28.5930441925587	0.000208492672311214\\
28.8677912582235	0.00020925477015091\\
29.1425383238883	0.000210018379013905\\
29.4172853895532	0.00021078349645666\\
29.692032455218	0.000211550451951558\\
29.9667795208829	0.000212318916367145\\
30.2415265865477	0.000213088721888558\\
30.5162736522125	0.000213859863112106\\
30.7910207178774	0.00021463283401194\\
31.0657677835422	0.000215407137455425\\
31.3405148492071	0.000216182935549452\\
31.6152619148719	0.000216960389960767\\
31.8900089805368	0.000217739505028003\\
32.1647560462016	0.000218519944480338\\
32.4395031118664	0.000219301537287772\\
32.7142501775313	0.000220084609576386\\
32.9889972431961	0.000220869159294921\\
33.263744308861	0.000221655184503775\\
33.5384913745258	0.000222442844582445\\
33.8132384401906	0.000223231646769264\\
34.0879855058555	0.000224022245587448\\
34.3627325715203	0.000224814308068428\\
34.6374796371852	0.000225607999024125\\
34.91222670285	0.000226403319935854\\
35.1869737685148	0.000227199770863616\\
35.4617208341797	0.000227998008956935\\
35.7364678998445	0.000228797868022448\\
36.0112149655094	0.00022959885271337\\
36.2859620311742	0.000230400789238388\\
36.5607090968391	0.000231204329766063\\
36.8354561625039	0.000232009310103658\\
37.1102032281687	0.000232815723030827\\
37.3849502938336	0.000233623240813482\\
37.6596973594984	0.000234432680092792\\
37.9344444251633	0.000235243382871411\\
38.2091914908281	0.000236055342998738\\
38.4839385564929	0.000236868887674468\\
38.7586856221578	0.000237685016288676\\
39.0334326878226	0.000238501409392386\\
39.3081797534875	0.000239319213177474\\
39.5829268191523	0.000240138590688264\\
39.8576738848171	0.000240959045754406\\
40.132420950482	0.000241780901957115\\
40.4071680161468	0.000242604487680586\\
40.6819150818117	0.000243429304750291\\
40.9566621474765	0.000244255349639184\\
41.2314092131414	0.000245083112838652\\
41.5061562788062	0.000245911769092566\\
41.780903344471	0.000246741969117757\\
42.0556504101359	0.000247574212925546\\
42.3303974758007	0.000248407505710745\\
42.6051445414656	0.000249242007824992\\
42.8798916071304	0.000250078047478327\\
43.1546386727952	0.000250915620811345\\
43.4293857384601	0.000251754560472556\\
43.7041328041249	0.000252594370168444\\
43.9788798697898	0.000253435698925521\\
44.2536269354546	0.000254278382155683\\
44.5283740011194	0.000255122916485085\\
44.8031210667843	0.00025596863550381\\
45.0778681324491	0.000256816036824055\\
45.352615198114	0.000257664120810127\\
45.6273622637788	0.000258513874509827\\
45.9021093294437	0.000259364799025532\\
46.1768563951085	0.000260216888442675\\
46.4516034607733	0.000261070304016837\\
46.7263505264382	0.000261925209763994\\
47.001097592103	0.00026278176724285\\
47.2758446577679	0.000263638983744216\\
47.5505917234327	0.00026449767676758\\
47.8253387890975	0.000265358009985934\\
48.1000858547624	0.000266219488466831\\
48.3748329204272	0.000267082434665636\\
48.6495799860921	0.000267946515888764\\
48.9243270517569	0.000268811561712204\\
49.1990741174217	0.00026967806207229\\
49.4738211830866	0.000270545849527645\\
49.7485682487514	0.000271414919424049\\
50.0233153144163	0.000272285766178896\\
50.2980623800811	0.000273157727408453\\
50.5728094457459	0.000274030633100372\\
50.8475565114108	0.000274905140036993\\
51.1223035770756	0.000275780914566104\\
51.3970506427405	0.000276657956380439\\
51.6717977084053	0.000277535925242932\\
51.9465447740702	0.000278415146488205\\
52.221291839735	0.000279296115840455\\
52.4960389053998	0.000280177837900645\\
52.7707859710647	0.000281061132112436\\
53.0455330367295	0.000281945337165195\\
53.3202801023944	0.000282830941040361\\
53.5950271680592	0.000283717776868611\\
53.869774233724	0.000284605840895608\\
54.1445212993889	0.000285495627557138\\
54.4192683650537	0.000286386638734333\\
54.6940154307186	0.000287278209393251\\
54.9687624963834	0.000288171320293482\\
55.2435095620482	0.000289065311536876\\
55.5182566277131	0.000289960670942282\\
55.7930036933779	0.000290857395697197\\
56.0677507590428	0.000291754821289054\\
56.3424978247076	0.000292654097446026\\
56.6172448903725	0.000293553903056406\\
56.8919919560373	0.000294455054281413\\
57.1667390217021	0.000295357549482155\\
57.441486087367	0.000296261552111621\\
57.7162331530318	0.000297167223383533\\
57.9909802186967	0.000298073078783618\\
58.2657272843615	0.000298979767442398\\
58.5404743500263	0.000299888269010833\\
58.8152214156912	0.000300797930897804\\
59.089968481356	0.000301708745785226\\
59.3647155470209	0.000302620540394252\\
59.6394626126857	0.00030353331225872\\
59.9142096783505	0.000304447385620112\\
60.1889567440154	0.000305362261933056\\
60.4637038096802	0.0003062787620393\\
60.7384508753451	0.000307196222316828\\
61.0131979410099	0.000308114804653537\\
61.2879450066748	0.000309034505237427\\
61.5626920723396	0.000309955315408844\\
61.8374391380044	0.000310877068503598\\
62.1121862036693	0.000311800918552\\
62.3869332693341	0.000312725048025015\\
62.661680334999	0.000313649943036626\\
62.9364274006638	0.000314576589115739\\
63.2111744663286	0.000315503992203018\\
63.4859215319935	0.0003164321461287\\
63.7606685976583	0.000317361540419539\\
64.0354156633232	0.000318292667982469\\
64.310162728988	0.000319224540048223\\
64.5849097946528	0.000320156980434688\\
64.8596568603177	0.000321090642281026\\
65.1344039259825	0.000322026186879492\\
65.4091509916474	0.000322962294629136\\
65.6838980573122	0.000323899118789473\\
65.958645122977	0.000324837312212221\\
66.2333921886419	0.000325776381038526\\
66.5081392543067	0.000326716316324749\\
66.7828863199716	0.000327657941090993\\
67.0576333856364	0.000328600598285238\\
67.3323804513013	0.000329544448381599\\
67.6071275169661	0.000330488491574702\\
67.8818745826309	0.000331434206573029\\
68.1566216482958	0.000332380604663059\\
68.4313687139606	0.000333328004932483\\
68.7061157796255	0.000334276238763966\\
68.9808628452903	0.000335225132729353\\
69.2556099109551	0.000336175338817608\\
69.53035697662	0.000337126034503221\\
69.8051040422848	0.000338078529632173\\
70.0798511079497	0.00033903150591987\\
70.3545981736145	0.00033998627646182\\
70.6293452392793	0.000340941358672099\\
70.9040923049442	0.000341897566321031\\
71.178839370609	0.000342855064204695\\
71.4535864362739	0.000343813352927966\\
71.7283335019387	0.000344772094316484\\
72.0030805676036	0.000345731937627684\\
72.2778276332684	0.000346692223745991\\
72.5525746989332	0.000347653765070648\\
72.8273217645981	0.000348616397906321\\
73.1020688302629	0.000349580284073511\\
73.3768158959277	0.000350544594921481\\
73.6515629615926	0.000351509815833164\\
73.9263100272574	0.000352476274810115\\
74.2010570929223	0.000353443636184831\\
74.4758041585871	0.000354411900026153\\
74.750551224252	0.000355381391161808\\
75.0252982899168	0.000356351114855693\\
75.3000453555816	0.000357322552164107\\
75.5747924212465	0.000358293878526207\\
75.8495394869113	0.000359266241610577\\
76.1242865525762	0.000360239804339684\\
76.399033618241	0.000361214400955355\\
76.6737806839058	0.000362189694342955\\
76.9485277495707	0.000363165843717857\\
77.2232748152355	0.000364142348812787\\
77.4980218809004	0.000365119695302029\\
77.7727689465652	0.000366098208999042\\
78.0475160122301	0.000367077724365518\\
78.3222630778949	0.000368057742002145\\
78.5970101435597	0.000369038914579821\\
78.8717572092246	0.000370020910291768\\
79.1465042748894	0.000371003888792771\\
79.4212513405543	0.00037198834962201\\
79.6959984062191	0.000372972632307791\\
79.9707454718839	0.000373957715979923\\
80.2454925375488	0.000374943762505277\\
80.5202396032136	0.00037593027132809\\
80.7949866688785	0.000376918229084599\\
81.0697337345433	0.000377906974018711\\
81.3444808002082	0.000378896006970101\\
81.619227865873	0.000379886473896983\\
81.8939749315378	0.000380876885818941\\
82.1687219972027	0.000381868224204657\\
82.4434690628675	0.000382860484883912\\
82.7182161285323	0.000383853332892893\\
82.9929631941972	0.000384846928987878\\
83.267710259862	0.000385840934938498\\
83.5424573255269	0.000386836334896456\\
83.8172043911917	0.00038783180989797\\
84.0919514568566	0.000388828007860874\\
84.3666985225214	0.000389825422250347\\
84.6414455881862	0.000390823222734732\\
84.9161926538511	0.000391821566389421\\
85.1909397195159	0.000392821276562362\\
85.4656867851808	0.000393821028072461\\
85.7404338508456	0.00039482163830003\\
86.0151809165104	0.000395822937262232\\
86.2899279821753	0.000396824920063402\\
86.5646750478401	0.000397827912813053\\
86.839422113505	0.000398831418526174\\
87.1141691791698	0.000399835429445387\\
87.3889162448347	0.000400840105901858\\
87.6636633104995	0.000401845605512211\\
87.9384103761643	0.000402851759870454\\
88.2131574418292	0.000403858069697457\\
88.487904507494	0.000404865679761896\\
88.7626515731589	0.000405873435554977\\
89.0373986388237	0.000406882485726953\\
89.3121457044885	0.000407891673660559\\
89.5868927701534	0.000408901816644201\\
89.8616398358182	0.000409912581477838\\
90.1363869014831	0.000410924294605325\\
90.4111339671479	0.00041193629055186\\
90.6858810328127	0.000412948727565698\\
90.9606280984776	0.000413962263205121\\
91.2353751641424	0.000414975570453828\\
91.5101222298073	0.000415989794512437\\
91.7848692954721	0.000417005097342396\\
92.059616361137	0.000418019995533128\\
92.3343634268018	0.000419035460969224\\
92.6091104924666	0.000420052317066986\\
92.8838575581315	0.000421069575809653\\
93.1586046237963	0.000422087227528426\\
93.4333516894612	0.000423105101902168\\
93.708098755126	0.000424124016076755\\
93.9828458207908	0.0004251431440823\\
94.2575928864557	0.000426162642903173\\
94.5323399521205	0.000427182668870296\\
94.8070870177854	0.000428203550617446\\
95.0818340834502	0.000429224953250208\\
95.356581149115	0.000430246541995562\\
95.6313282147799	0.00043126913609782\\
95.9060752804447	0.000432292238380161\\
96.1808223461096	0.000433315676378574\\
96.4555694117744	0.000434339280764082\\
96.7303164774392	0.000435363537932024\\
97.0050635431041	0.000436388444133866\\
97.2798106087689	0.000437413995868945\\
97.5545576744338	0.000438439528696665\\
97.8293047400986	0.000439465693038964\\
98.1040518057635	0.000440492318892164\\
98.3787988714283	0.000441519565920292\\
98.6535459370931	0.000442546774431614\\
98.928293002758	0.000443574754762943\\
99.2030400684228	0.00044460317742577\\
99.4777871340877	0.000445632367936735\\
99.7525341997525	0.000446661496920655\\
100.027281265417	0.000447691216717707\\
100.302028331082	0.000448721689089264\\
100.576775396747	0.000449752251427438\\
100.851522462412	0.000450783721743114\\
101.126269528077	0.000451815271270714\\
101.401016593742	0.000452847553755284\\
101.675763659406	0.000453879574288148\\
101.950510725071	0.000454912314025346\\
102.225257790736	0.000455945276255995\\
102.500004856401	0.000456978783845301\\
102.774751922066	0.000458012831971797\\
103.049498987731	0.000459047419637828\\
103.324246053395	0.000460081880361373\\
103.59899311906	0.00046111669983912\\
103.873740184725	0.000462152037241013\\
104.14848725039	0.000463188389877933\\
104.423234316055	0.000464224597872576\\
104.69798138172	0.000465260817107204\\
104.972728447384	0.000466297699274112\\
105.247475513049	0.00046733524664668\\
105.522222578714	0.000468372627832206\\
105.796969644379	0.000469410493220173\\
106.071716710044	0.000470448508881673\\
106.346463775709	0.000471486998844871\\
106.621210841374	0.000472525461187329\\
106.895957907038	0.000473563891707415\\
107.170704972703	0.000474602940196421\\
107.445452038368	0.000475642273557703\\
107.720199104033	0.000476682384502123\\
107.994946169698	0.000477722280532565\\
108.269693235363	0.000478762446511688\\
108.544440301027	0.000479803540143963\\
108.819187366692	0.000480844239549647\\
109.093934432357	0.000481885526670836\\
109.368681498022	0.000482927233019623\\
109.643428563687	0.000483969024624749\\
109.918175629352	0.000485010561873202\\
110.192922695016	0.000486052991840061\\
110.467669760681	0.00048709549219951\\
110.742416826346	0.000488138056532028\\
111.017163892011	0.000489180840403339\\
111.291910957676	0.000490223507787256\\
111.566658023341	0.000491266546709986\\
111.841405089006	0.00049230962314585\\
112.11615215467	0.000493353224711112\\
112.390899220335	0.000494396527263301\\
112.665646286	0.000495440841599796\\
112.940393351665	0.000496485012495091\\
113.21514041733	0.00049752969166225\\
113.489887482995	0.000498574384288795\\
113.764634548659	0.000499618917574059\\
114.039381614324	0.000500663778645534\\
114.314128679989	0.000501708800440435\\
114.588875745654	0.000502753641562689\\
114.863622811319	0.000503798955395452\\
115.138369876984	0.000504843918942714\\
115.413116942648	0.000505889508997181\\
115.687864008313	0.000506935066522415\\
115.962611073978	0.000507980750608134\\
116.237358139643	0.000509026552902424\\
116.512105205308	0.000510072968349567\\
116.786852270973	0.000511119168532256\\
117.061599336637	0.000512165476735866\\
117.336346402302	0.000513211889029529\\
117.611093467967	0.00051425822998487\\
117.885840533632	0.000515305324279321\\
118.160587599297	0.000516351680970417\\
118.435334664962	0.000517398278974053\\
118.710081730627	0.000518444623700859\\
118.984828796291	0.000519491197688203\\
119.259575861956	0.000520538330706183\\
119.534322927621	0.000521585693440948\\
119.809069993286	0.000522632949311501\\
120.083817058951	0.000523679926518488\\
120.358564124616	0.000524726782556591\\
120.63331119028	0.000525774007014152\\
120.908058255945	0.00052682143226717\\
121.18280532161	0.000527868558570622\\
121.457552387275	0.000528915708305835\\
121.73229945294	0.000529963040428246\\
122.007046518605	0.00053101022272993\\
122.281793584269	0.000532057578591286\\
122.556540649934	0.000533104940415003\\
122.831287715599	0.000534152137026063\\
123.106034781264	0.000535199987765544\\
123.380781846929	0.000536247006551438\\
123.655528912594	0.00053729400612773\\
123.930275978259	0.000538340982298351\\
124.205023043923	0.00053938842967719\\
124.479770109588	0.000540435847188153\\
124.754517175253	0.000541483063845507\\
125.029264240918	0.000542530738027792\\
125.304011306583	0.000543577710953876\\
125.578758372248	0.000544624465264194\\
125.853505437912	0.000545671325358613\\
126.128252503577	0.000546718456746445\\
126.402999569242	0.000547765026335319\\
126.677746634907	0.00054881168605923\\
126.952493700572	0.000549858268439881\\
127.227240766237	0.000550904934826315\\
127.501987831901	0.000551951848049759\\
127.776734897566	0.000552998012216218\\
128.051481963231	0.000554044905824888\\
128.326229028896	0.000555090877773231\\
128.600976094561	0.000556137072739424\\
128.875723160226	0.000557183322700918\\
129.150470225891	0.000558229293303545\\
129.425217291555	0.000559274979695127\\
129.69996435722	0.000560320704021366\\
129.974711422885	0.000561366134719126\\
130.24945848855	0.000562411426054845\\
130.524205554215	0.000563456410750476\\
130.79895261988	0.000564501411667866\\
131.073699685544	0.000565546262030199\\
131.348446751209	0.000566591454037142\\
131.623193816874	0.000567635993685163\\
131.897940882539	0.000568680533717605\\
132.172687948204	0.000569724572286886\\
132.447435013869	0.000570768598836053\\
132.722182079533	0.000571813275613457\\
132.996929145198	0.000572857777641763\\
133.271676210863	0.000573901103922429\\
133.546423276528	0.00057494472959444\\
133.821170342193	0.00057598799092298\\
134.095917407858	0.000577030556437667\\
134.370664473522	0.000578073071136293\\
134.645411539187	0.000579115535510795\\
134.920158604852	0.000580158112219723\\
135.194905670517	0.000581200302419172\\
135.469652736182	0.00058224243182162\\
135.744399801847	0.000583284167454182\\
136.019146867512	0.000584326331190143\\
136.293893933176	0.000585367433072012\\
136.568640998841	0.000586408447024529\\
136.843388064506	0.000587448717931211\\
137.118135130171	0.000588489057101251\\
137.392882195836	0.000589529133327437\\
137.667629261501	0.000590569272763024\\
137.942376327165	0.000591608648884909\\
138.21712339283	0.00059264857831122\\
138.491870458495	0.000593687566650504\\
138.76661752416	0.000594726268768084\\
139.041364589825	0.000595764515874502\\
139.31611165549	0.000596803460404683\\
139.590858721154	0.000597841451977619\\
139.865605786819	0.000598879640995842\\
140.140352852484	0.000599917028301946\\
140.415099918149	0.000600954434418148\\
140.689846983814	0.000601990865914771\\
140.964594049479	0.00060302730288829\\
141.239341115144	0.000604063247985624\\
141.514088180808	0.000605099193987515\\
141.788835246473	0.000606134638384802\\
142.063582312138	0.000607169744594512\\
142.338329377803	0.000608204669321219\\
142.613076443468	0.000609238916613196\\
142.887823509133	0.000610273138359135\\
143.162570574797	0.000611307006290958\\
143.437317640462	0.000612340181163657\\
143.712064706127	0.000613372986385192\\
143.986811771792	0.00061440558298659\\
144.261558837457	0.000615437636336595\\
144.536305903122	0.000616469472681002\\
144.811052968786	0.000617501087654177\\
145.085800034451	0.000618531981244791\\
145.360547100116	0.00061956231340662\\
145.635294165781	0.000620592573414743\\
145.910041231446	0.000621622593635201\\
146.184788297111	0.000622651710059214\\
146.459535362775	0.000623680574463114\\
146.73428242844	0.000624709680692931\\
147.009029494105	0.00062573754358569\\
147.28377655977	0.000626765305864655\\
147.558523625435	0.000627792637401661\\
147.8332706911	0.000628819533487984\\
148.108017756765	0.000629845985289881\\
148.382764822429	0.000630871499946227\\
148.657511888094	0.000631896888498242\\
148.932258953759	0.000632921821413843\\
149.207006019424	0.000633946460353767\\
149.481753085089	0.000634970469776633\\
149.756500150754	0.000635994342676127\\
150.031247216418	0.000637017580511199\\
150.305994282083	0.000638040175246576\\
150.580741347748	0.000639062452297506\\
150.855488413413	0.000640084243437898\\
151.130235479078	0.000641105380907623\\
151.404982544743	0.00064212602097575\\
151.679729610407	0.000643145997475259\\
151.954476676072	0.000644166962871663\\
152.229223741737	0.000645185941467635\\
152.503970807402	0.000646204569704164\\
152.778717873067	0.000647222515197958\\
153.053464938732	0.000648240436116524\\
153.328212004397	0.000649257666550981\\
153.602959070061	0.000650274037376993\\
153.877706135726	0.000651290367322971\\
154.152453201391	0.000652305830421238\\
154.427200267056	0.000653321580720991\\
154.701947332721	0.000654336461871367\\
154.976694398386	0.000655349970460771\\
155.25144146405	0.000656363415739813\\
155.526188529715	0.000657376303016205\\
155.80093559538	0.00065838862491833\\
156.075682661045	0.000659400380763416\\
156.35042972671	0.000660411401489223\\
156.625176792375	0.000661421843494801\\
156.899923858039	0.000662431704566956\\
157.174670923704	0.000663440811429344\\
157.449417989369	0.00066444932661495\\
157.724165055034	0.000665457576712499\\
157.998912120699	0.000666464899247565\\
158.273659186364	0.000667472113540629\\
158.548406252029	0.000668478227538959\\
158.823153317693	0.00066948389440865\\
159.097900383358	0.000670489440509668\\
159.372647449023	0.000671494374341605\\
159.647394514688	0.000672498693608971\\
159.922141580353	0.000673502227696299\\
160.196888646018	0.000674504634750653\\
160.471635711682	0.000675507233261904\\
160.746382777347	0.00067650870284973\\
161.021129843012	0.000677509864232713\\
161.295876908677	0.000678510218101474\\
161.570623974342	0.000679509758111657\\
161.845371040007	0.000680508311064653\\
162.120118105671	0.000681506532364139\\
162.394865171336	0.000682503760854823\\
162.669612237001	0.000683501147140292\\
162.944359302666	0.000684497706156258\\
163.219106368331	0.000685493097203009\\
163.493853433996	0.000686487807878827\\
163.76860049966	0.000687482166607725\\
164.043347565325	0.000688475504835158\\
164.31809463099	0.000689467823849719\\
164.592841696655	0.000690459771905763\\
164.86758876232	0.000691450854477205\\
165.142335827985	0.000692441396290621\\
165.41708289365	0.000693431063269537\\
165.691829959314	0.000694420512967948\\
165.966577024979	0.000695409084697189\\
166.241324090644	0.000696396605813521\\
166.516071156309	0.000697383402217176\\
166.790818221974	0.000698369631582062\\
167.065565287639	0.000699354800826689\\
167.340312353303	0.000700339725517204\\
167.615059418968	0.000701324410131323\\
167.889806484633	0.000702307696464518\\
168.164553550298	0.000703290234838019\\
168.439300615963	0.000704272187834991\\
168.714047681628	0.000705253222169867\\
168.988794747292	0.000706233663318049\\
169.263541812957	0.000707213343461501\\
169.538288878622	0.000708192426624953\\
169.813035944287	0.000709170245140411\\
170.087783009952	0.000710147452922155\\
170.362530075617	0.000711124544797662\\
170.637277141282	0.000712100364160204\\
170.912024206946	0.000713075070441989\\
171.186771272611	0.000714048985415805\\
171.461518338276	0.00071502243547978\\
171.736265403941	0.000715994926656225\\
172.011012469606	0.000716966949301611\\
172.285759535271	0.000717938004824669\\
172.560506600935	0.000718908254723877\\
172.8352536666	0.000719877860681359\\
173.110000732265	0.000720847153277693\\
173.38474779793	0.000721814641219746\\
173.659494863595	0.000722781301378289\\
173.93424192926	0.000723746971542138\\
174.208988994924	0.000724712135745208\\
174.483736060589	0.000725676467289876\\
174.758483126254	0.000726639961880133\\
175.033230191919	0.000727602944561423\\
175.307977257584	0.000728564590474002\\
175.582724323249	0.000729526047630163\\
175.857471388913	0.000730486661462582\\
176.132218454578	0.000731446258598631\\
176.406965520243	0.000732404669827088\\
176.681712585908	0.0007333622174398\\
176.956459651573	0.000734318732646484\\
177.231206717238	0.000735274373672416\\
177.505953782903	0.000736228977427675\\
177.780700848567	0.000737182863199181\\
178.055447914232	0.000738136198082543\\
178.330194979897	0.000739088485712963\\
178.604942045562	0.000740040380384757\\
178.879689111227	0.000740990895040284\\
179.154436176892	0.000741940844695266\\
179.429183242556	0.00074288956924619\\
179.703930308221	0.000743837392699294\\
179.978677373886	0.000744784641475308\\
180.253424439551	0.000745730821627182\\
180.528171505216	0.000746675925131543\\
180.802918570881	0.000747619947991446\\
181.077665636545	0.00074856354633571\\
181.35241270221	0.000749505564627486\\
181.627159767875	0.000750446983396945\\
181.90190683354	0.000751387470269806\\
182.176653899205	0.000752327190546003\\
182.45140096487	0.000753266143205967\\
182.726148030535	0.0007542039951106\\
183.000895096199	0.000755141071210674\\
183.275642161864	0.000756077206370904\\
183.550389227529	0.000757011899513578\\
183.825136293194	0.000757945801471586\\
184.099883358859	0.000758878747455437\\
184.374630424524	0.00075981090149548\\
184.649377490188	0.000760741762366375\\
184.924124555853	0.000761671821392152\\
185.198871621518	0.000762601076630254\\
185.473618687183	0.000763529192364201\\
185.748365752848	0.000764456003167751\\
186.023112818513	0.000765381994861821\\
186.297859884177	0.00076630749899253\\
186.572606949842	0.000767231854520593\\
186.847354015507	0.000768155222477595\\
187.122101081172	0.000769077434439424\\
187.396848146837	0.000769999313647827\\
187.671595212502	0.000770919872075789\\
187.946342278167	0.000771838936075899\\
188.221089343831	0.000772757324812744\\
188.495836409496	0.000773674704911088\\
188.770583475161	0.000774591076054245\\
189.045330540826	0.000775506601199671\\
189.320077606491	0.000776420779608636\\
189.594824672156	0.000777333772308366\\
189.86957173782	0.000778245904160591\\
190.144318803485	0.000779157174908703\\
190.41906586915	0.000780067417733967\\
190.693812934815	0.000780976296250617\\
190.96856000048	0.000781884137003133\\
191.243307066145	0.00078279110083665\\
191.518054131809	0.00078369702225859\\
191.792801197474	0.000784601897985255\\
192.067548263139	0.000785506386804631\\
192.342295328804	0.000786409166298191\\
192.617042394469	0.000787310563100946\\
192.891789460134	0.000788211227407876\\
193.166536525798	0.000789110663063558\\
193.441283591463	0.000790009033991558\\
193.716030657128	0.000790906499449263\\
193.990777722793	0.000791803060647035\\
194.265524788458	0.000792698547555014\\
194.540271854123	0.000793592467867235\\
194.815018919788	0.000794485467895686\\
195.089765985452	0.000795377384325758\\
195.364513051117	0.000796268878023915\\
195.639260116782	0.000797158957139096\\
195.914007182447	0.000798047947132364\\
196.188754248112	0.00079893567920126\\
196.463501313777	0.000799822146867174\\
196.738248379441	0.000800707512267471\\
197.012995445106	0.000801592270283234\\
197.287742510771	0.000802475596342964\\
197.562489576436	0.000803357811965854\\
197.837236642101	0.00080423891753392\\
198.111983707766	0.000805118742404822\\
198.38673077343	0.000805997613923071\\
198.661477839095	0.000806875528917724\\
198.93622490476	0.000807752159046159\\
199.210971970425	0.000808627827892499\\
199.48571903609	0.000809502038668586\\
199.760466101755	0.000810375116295703\\
200.03521316742	0.000811247222811355\\
200.309960233084	0.000812118027214735\\
200.584707298749	0.000812988352624632\\
200.859454364414	0.000813856878689753\\
201.134201430079	0.000814724256772217\\
201.408948495744	0.000815590649247986\\
201.683695561409	0.000816456057107168\\
201.958442627073	0.000817320314294127\\
202.233189692738	0.000818183250752764\\
202.507936758403	0.00081904502810457\\
202.782683824068	0.000819905644227262\\
203.057430889733	0.000820765426378677\\
203.332177955398	0.000821624209805522\\
203.606925021062	0.000822481497839004\\
203.881672086727	0.000823337286502165\\
204.156419152392	0.000824192557973935\\
204.431166218057	0.000825045997537714\\
204.705913283722	0.000825898253218429\\
204.980660349387	0.000826749324298274\\
205.255407415051	0.000827599536893772\\
205.530154480716	0.000828448234044055\\
205.804901546381	0.000829296401872197\\
206.079648612046	0.000830143216692896\\
206.354395677711	0.000830988340743077\\
206.629142743376	0.000831832428588495\\
206.903889809041	0.000832674986766086\\
207.178636874705	0.00083351699708563\\
207.45338394037	0.000834357144813839\\
207.728131006035	0.000835196737809586\\
208.0028780717	0.000836034464863217\\
208.277625137365	0.0008368714693401\\
208.55237220303	0.000837707259446037\\
208.827119268694	0.000838542162364828\\
209.101866334359	0.000839375024600894\\
209.376613400024	0.000840207154734325\\
209.651360465689	0.00084103789463481\\
209.926107531354	0.000841867569962253\\
210.200854597019	0.000842695686041554\\
210.475601662683	0.00084352339740931\\
210.750348728348	0.000844349213901934\\
211.025095794013	0.000845173627306827\\
211.299842859678	0.000845997129065197\\
211.574589925343	0.000846819056413293\\
211.849336991008	0.000847639736434\\
212.124084056673	0.000848459665055981\\
212.398831122337	0.000849278015504454\\
212.673578188002	0.000850095444658786\\
212.948325253667	0.000850911289269475\\
213.223072319332	0.000851726375755582\\
213.497819384997	0.000852539709551098\\
213.772566450662	0.000853351945931923\\
214.047313516326	0.000854162753408149\\
214.322060581991	0.000854972294676039\\
214.596807647656	0.000855781063668592\\
214.871554713321	0.000856588234715453\\
215.146301778986	0.000857393969748568\\
215.421048844651	0.000858198594593333\\
215.695795910315	0.000859001943612916\\
215.97054297598	0.000859803847801213\\
216.245290041645	0.000860604635541173\\
216.52003710731	0.000861403807677007\\
216.794784172975	0.000862201855540415\\
217.06953123864	0.000862998448508284\\
217.344278304305	0.00086379391398053\\
217.619025369969	0.000864588746869721\\
217.893772435634	0.000865381629778535\\
218.168519501299	0.000866173712830804\\
218.443266566964	0.000866964668352482\\
218.718013632629	0.000867753664865816\\
218.992760698294	0.000868541190771418\\
219.267507763958	0.000869327573562712\\
219.542254829623	0.000870112812549513\\
219.817001895288	0.000870896909602479\\
220.091748960953	0.000871679198844275\\
220.366496026618	0.000872460170646871\\
220.641243092283	0.000873240152767764\\
220.915990157947	0.000874018485556375\\
221.190737223612	0.000874795331709699\\
221.465484289277	0.000875571013647624\\
221.740231354942	0.000876345533227996\\
222.014978420607	0.000877118724633921\\
222.289725486272	0.000877890750267032\\
222.564472551936	0.000878660951125574\\
222.839219617601	0.000879430641291711\\
223.113966683266	0.000880198173703718\\
223.388713748931	0.000880964526291878\\
223.663460814596	0.000881729538155534\\
223.938207880261	0.000882493534394099\\
224.212954945926	0.000883255694261266\\
224.48770201159	0.000884016832388671\\
224.762449077255	0.000884776621973161\\
225.03719614292	0.000885534895848268\\
225.311943208585	0.000886292313016312\\
225.58669027425	0.000887048215086527\\
225.861437339915	0.000887802431326635\\
226.136184405579	0.000888555454124202\\
226.410931471244	0.000889307115477831\\
226.685678536909	0.000890057414868583\\
226.960425602574	0.000890806682545268\\
227.235172668239	0.00089155425614894\\
227.509919733904	0.000892300961198892\\
227.784666799568	0.000893046136286957\\
228.059413865233	0.000893789444198333\\
228.334160930898	0.000894531214207793\\
228.608907996563	0.000895271935260992\\
228.883655062228	0.000896011445180041\\
229.158402127893	0.000896749411489165\\
229.433149193558	0.000897486331115547\\
229.707896259222	0.000898221539040284\\
229.982643324887	0.000898955362948859\\
230.257390390552	0.000899687800913132\\
230.532137456217	0.000900418849401683\\
230.806884521882	0.000901149003368977\\
231.081631587547	0.000901876948747502\\
231.356378653211	0.000902604488785407\\
231.631125718876	0.000903329813679646\\
231.905872784541	0.000904053740879027\\
232.180619850206	0.000904776432073229\\
232.455366915871	0.000905497888601724\\
232.730113981536	0.000906217780650581\\
233.0048610472	0.000906936435293979\\
233.279608112865	0.000907653522432579\\
233.55435517853	0.000908369366337025\\
233.829102244195	0.000909083473745526\\
234.10384930986	0.000909796666706973\\
234.378596375525	0.000910508453677816\\
234.653343441189	0.000911218663551289\\
234.928090506854	0.000911927292651238\\
235.202837572519	0.000912634176608551\\
235.477584638184	0.000913339968297183\\
235.752331703849	0.000914044503444142\\
236.027078769514	0.000914747288759182\\
236.301825835179	0.000915448647315341\\
236.576572900843	0.000916148415827875\\
236.851319966508	0.000916847083113759\\
237.126067032173	0.000917544489802924\\
237.400814097838	0.000918240467740785\\
237.675561163503	0.000918935346358655\\
237.950308229168	0.000919627972728919\\
238.225055294832	0.000920319986519823\\
238.499802360497	0.000921010074595833\\
238.774549426162	0.000921698889738404\\
239.049296491827	0.000922385773873346\\
239.324043557492	0.000923071873480733\\
239.598790623157	0.000923756201148794\\
239.873537688821	0.00092443875930689\\
240.148284754486	0.000925120197524967\\
240.423031820151	0.000925800191118031\\
240.697778885816	0.000926478569461389\\
240.972525951481	0.000927156159249935\\
241.247273017146	0.000927832472283964\\
241.522020082811	0.00092850651079316\\
241.796767148475	0.000929179092676876\\
242.07151421414	0.000929850054498567\\
242.346261279805	0.000930520052631828\\
242.62100834547	0.000931188261715356\\
242.895755411135	0.000931855341847173\\
243.1705024768	0.000932520797620342\\
243.445249542464	0.000933184958400557\\
243.719996608129	0.000933847656886282\\
243.994743673794	0.00093450889205931\\
244.269490739459	0.000935168498143379\\
244.544237805124	0.000935827137017713\\
244.818984870789	0.000936484312028389\\
245.093731936453	0.000937139688588549\\
245.368479002118	0.000937793265627563\\
245.643226067783	0.000938445531808294\\
245.917973133448	0.00093909632495326\\
246.192720199113	0.000939745975287159\\
246.467467264778	0.000940393822239088\\
246.742214330442	0.000941040354956647\\
247.016961396107	0.000941685080526718\\
247.291708461772	0.000942328488527724\\
247.566455527437	0.000942970580158679\\
247.841202593102	0.000943611187517006\\
248.115949658767	0.00094424982071577\\
248.390696724432	0.000944887293951845\\
248.665443790096	0.000945523447478371\\
248.940190855761	0.000946157782788161\\
249.214937921426	0.000946790792943429\\
249.489684987091	0.000947422808107148\\
249.764432052756	0.000948052512116034\\
250.039179118421	0.000948681213878904\\
250.313926184085	0.000949308259298235\\
250.58867324975	0.000949933975381076\\
250.863420315415	0.000950558030569665\\
251.13816738108	0.000951180423894047\\
251.412914446745	0.000951801481814316\\
251.68766151241	0.000952421701507856\\
251.962408578075	0.000953040098046476\\
252.237155643739	0.000953656831738323\\
252.511902709404	0.00095427189834333\\
252.786649775069	0.000954885297132221\\
253.061396840734	0.000955497187929184\\
253.336143906399	0.000956108400368985\\
253.610890972064	0.000956717284404265\\
253.885638037728	0.000957325156493112\\
254.160385103393	0.000957931523886177\\
254.435132169058	0.000958535880274076\\
254.709879234723	0.000959138564454568\\
254.984626300388	0.00095973973158404\\
255.259373366053	0.000960339380565656\\
255.534120431717	0.000960937840280702\\
255.808867497382	0.000961535281196816\\
256.083614563047	0.000962130549887499\\
256.358361628712	0.000962724463927847\\
256.633108694377	0.000963317025594076\\
256.907855760042	0.000963907737635041\\
257.182602825707	0.000964497091688697\\
257.457349891371	0.000965084758717669\\
257.732096957036	0.000965671232337485\\
258.006844022701	0.000966256018764042\\
258.281591088366	0.000966839281481829\\
258.556338154031	0.000967421018537105\\
258.831085219696	0.000968001394666303\\
259.10583228536	0.000968580079763862\\
259.380579351025	0.000969157403393307\\
259.65532641669	0.000969733032003768\\
259.930073482355	0.000970306965709629\\
260.20482054802	0.000970879363966815\\
260.479567613685	0.000971450065379511\\
260.754314679349	0.000972019892707399\\
261.029061745014	0.000972588356926013\\
261.303808810679	0.000973155124766287\\
261.578555876344	0.000973719700778138\\
261.853302942009	0.000974283063483323\\
262.128050007674	0.000974844561630814\\
262.402797073338	0.000975404682298889\\
262.677544139003	0.00097596392598449\\
262.952291204668	0.000976521139307419\\
263.227038270333	0.000977076976795082\\
263.501785335998	0.000977630943857928\\
263.776532401663	0.000978183696265709\\
264.051279467328	0.000978734413691472\\
264.326026532992	0.000979284077896314\\
264.600773598657	0.000979832035168995\\
264.875520664322	0.000980378447114378\\
265.150267729987	0.000980923147370456\\
265.425014795652	0.000981466629877602\\
265.699761861317	0.000982008234741516\\
265.974508926981	0.000982548287449965\\
266.249255992646	0.00098308662750229\\
266.524003058311	0.000983623577465045\\
266.798750123976	0.000984159304794293\\
267.073497189641	0.000984693153770853\\
267.348244255306	0.000985226277609503\\
267.62299132097	0.000985757027104863\\
267.897738386635	0.000986286221288659\\
268.1724854523	0.0009868140257695\\
268.447232517965	0.000987340443011137\\
268.72197958363	0.000987865303319119\\
268.996726649295	0.000988388116357426\\
269.271473714959	0.000988910030458319\\
269.546220780624	0.000989430226359999\\
269.820967846289	0.00098994953016839\\
270.095714911954	0.000990466290460281\\
270.370461977619	0.000990981490127269\\
270.645209043284	0.000991495125343196\\
270.919956108949	0.000992007035982847\\
271.194703174613	0.000992517546254557\\
271.469450240278	0.00099302665842749\\
271.744197305943	0.000993534208967748\\
272.018944371608	0.000994040195961505\\
272.293691437273	0.000994544951600071\\
272.568438502938	0.000995047648742618\\
272.843185568602	0.000995548780609068\\
273.117932634267	0.000996048676438436\\
273.392679699932	0.000996546674169529\\
273.667426765597	0.00099704294216313\\
273.942173831262	0.000997537802892459\\
274.216920896927	0.000998031757618119\\
274.491667962591	0.000998523653957645\\
274.766415028256	0.000999014479380209\\
275.041162093921	0.000999503077442101\\
275.315909159586	0.000999989941969848\\
275.590656225251	0.0010004757270397\\
275.865403290916	0.00100095977993949\\
276.140150356581	0.00100144242951184\\
276.414897422245	0.00100192334258221\\
276.68964448791	0.00100240301507051\\
276.964391553575	0.00100288062166785\\
277.23913861924	0.00100335665528145\\
277.513885684905	0.00100383128046209\\
277.78863275057	0.00100430433014601\\
278.063379816234	0.00100477630212467\\
278.338126881899	0.00100524604561878\\
278.612873947564	0.00100571437728484\\
278.887621013229	0.00100618096574738\\
279.162368078894	0.0010066458148341\\
279.437115144559	0.00100710924807117\\
279.711862210223	0.00100757110132206\\
279.986609275888	0.00100803154217601\\
280.261356341553	0.00100849024082945\\
280.536103407218	0.0010089473590075\\
280.810850472883	0.00100940306277992\\
281.085597538548	0.00100985752013709\\
281.360344604213	0.00101031040482455\\
281.635091669877	0.00101076171271082\\
281.909838735542	0.00101121128108849\\
282.184585801207	0.00101165927140677\\
282.459332866872	0.00101210617933892\\
282.734079932537	0.0010125505254204\\
283.008826998202	0.00101299394648908\\
283.283574063866	0.00101343529580901\\
283.558321129531	0.00101387506135664\\
283.833068195196	0.00101431340978175\\
284.107815260861	0.0010147501779527\\
284.382562326526	0.00101518536349884\\
284.657309392191	0.00101561896481011\\
284.932056457855	0.00101605098531915\\
285.20680352352	0.00101648142227781\\
285.481550589185	0.00101691011153478\\
285.75629765485	0.00101733771513358\\
286.031044720515	0.00101776356912097\\
286.30579178618	0.00101818734795549\\
286.580538851844	0.00101861036742268\\
286.855285917509	0.00101903131034573\\
287.130032983174	0.00101945083681693\\
287.404780048839	0.00101986927811271\\
287.679527114504	0.00102028547865149\\
287.954274180169	0.00102069992857699\\
288.229021245834	0.00102111279247053\\
288.503768311498	0.00102152423496899\\
288.778515377163	0.00102193475803868\\
289.053262442828	0.0010223430380303\\
289.328009508493	0.00102274973255564\\
289.602756574158	0.0010231551752824\\
289.877503639823	0.00102355870210824\\
290.152250705487	0.00102396080972684\\
290.426997771152	0.00102436149825476\\
290.701744836817	0.00102476043887577\\
290.976491902482	0.00102515746544969\\
291.251238968147	0.00102555306886728\\
291.525986033812	0.00102594708596066\\
291.800733099476	0.00102633951528021\\
292.075480165141	0.00102673019331425\\
292.350227230806	0.00102711961302215\\
292.624974296471	0.00102750728245586\\
292.899721362136	0.00102789319952638\\
293.174468427801	0.001028277856807\\
293.449215493465	0.0010286607645683\\
293.72396255913	0.00102904208520278\\
293.998709624795	0.00102942165070442\\
294.27345669046	0.00102979946501422\\
294.548203756125	0.00103017585078319\\
294.82295082179	0.00103055097880181\\
295.097697887455	0.00103092435309508\\
295.372444953119	0.00103129581282435\\
295.647192018784	0.00103166584514537\\
295.921939084449	0.00103203428606784\\
296.196686150114	0.00103240113571235\\
296.471433215779	0.00103276623340793\\
296.746180281444	0.00103312990308308\\
297.020927347108	0.00103349231488251\\
297.295674412773	0.00103385280962523\\
297.570421478438	0.00103421187968361\\
297.845168544103	0.00103456919474421\\
298.119915609768	0.00103492492107227\\
298.394662675433	0.00103527938572896\\
298.669409741097	0.00103563177191133\\
298.944156806762	0.00103598273048353\\
299.218903872427	0.00103633226587459\\
299.493650938092	0.00103668037892932\\
299.768398003757	0.0010370269054061\\
300.043145069422	0.00103737168124125\\
300.317892135087	0.00103771520193023\\
300.592639200751	0.00103805664192208\\
300.867386266416	0.0010383968216959\\
301.142133332081	0.00103873508651269\\
301.416880397746	0.0010390720941435\\
301.691627463411	0.00103940735121889\\
301.966374529076	0.00103974118825374\\
302.24112159474	0.00104007343981337\\
302.515868660405	0.00104040443725771\\
302.79061572607	0.00104073319377886\\
303.065362791735	0.00104106102480066\\
303.3401098574	0.00104138677869918\\
303.614856923065	0.00104171177823237\\
303.88960398873	0.00104203420675622\\
304.164351054394	0.00104235521264331\\
304.439098120059	0.00104267446705627\\
304.713845185724	0.00104299230072173\\
304.988592251389	0.00104330888199174\\
305.263339317054	0.00104362371500691\\
305.538086382719	0.00104393663450207\\
305.812833448383	0.00104424796784023\\
306.087580514048	0.00104455838026566\\
306.362327579713	0.0010448668821921\\
306.637074645378	0.00104517363615616\\
306.911821711043	0.00104547913528429\\
307.186568776708	0.00104578239661709\\
307.461315842372	0.00104608440099967\\
307.736062908037	0.00104638465686467\\
308.010809973702	0.00104668332754335\\
308.285557039367	0.00104698041405458\\
308.560304105032	0.00104727658153363\\
308.835051170697	0.0010475706724053\\
309.109798236361	0.00104786318209734\\
309.384545302026	0.00104815410920215\\
309.659292367691	0.00104844312696371\\
309.934039433356	0.00104873121998171\\
310.208786499021	0.00104901724109743\\
310.483533564686	0.0010493018440149\\
310.75828063035	0.00104958470159317\\
311.033027696015	0.00104986630943752\\
311.30777476168	0.00105014650647397\\
311.582521827345	0.00105042479777529\\
311.85726889301	0.00105070134361788\\
312.132015958675	0.00105097647531021\\
312.40676302434	0.0010512500284753\\
312.681510090004	0.00105152183781712\\
312.956257155669	0.00105179256497674\\
313.231004221334	0.00105206121958038\\
313.505751286999	0.00105232813284313\\
313.780498352664	0.00105259379262605\\
314.055245418329	0.00105285754873808\\
314.329992483993	0.00105312022127456\\
314.604739549658	0.00105338165211555\\
314.879486615323	0.00105364101727124\\
315.154233680988	0.0010538991400589\\
315.428980746653	0.00105415552512967\\
315.703727812318	0.00105441000840875\\
315.978474877982	0.00105466357826972\\
316.253221943647	0.00105491491952823\\
316.527969009312	0.00105516468600816\\
316.802716074977	0.0010554130431564\\
317.077463140642	0.00105565982880041\\
317.352210206307	0.00105590520729599\\
317.626957271971	0.00105614901815037\\
317.901704337636	0.00105639093023329\\
318.176451403301	0.00105663143434287\\
318.451198468966	0.00105687003960873\\
318.725945534631	0.0010571077331941\\
319.000692600296	0.00105734369658797\\
319.275439665961	0.00105757859274899\\
319.550186731625	0.00105781126566832\\
319.82493379729	0.00105804220578003\\
320.099680862955	0.00105827174254209\\
320.37442792862	0.00105849987916938\\
320.649174994285	0.00105872645034623\\
320.92392205995	0.0010589509640452\\
321.198669125614	0.00105917440526241\\
321.473416191279	0.00105939628464879\\
321.748163256944	0.00105961626912451\\
322.022910322609	0.00105983468751789\\
322.297657388274	0.00106005154230583\\
322.572404453939	0.0010602671646093\\
322.847151519604	0.00106048139359169\\
323.121898585268	0.00106069373276635\\
323.396645650933	0.00106090418194756\\
323.671392716598	0.00106111356326189\\
323.946139782263	0.00106132155052271\\
324.220886847928	0.00106152781923709\\
324.495633913593	0.0010617321996802\\
324.770380979257	0.0010619350206674\\
325.045128044922	0.00106213677937849\\
325.319875110587	0.00106233665403811\\
325.594622176252	0.00106253497282629\\
325.869369241917	0.00106273223363356\\
326.144116307582	0.00106292761223049\\
326.418863373246	0.00106312094570835\\
326.693610438911	0.00106331304920328\\
326.968357504576	0.00106350327159038\\
327.243104570241	0.00106369243381886\\
327.517851635906	0.00106387971423052\\
327.792598701571	0.00106406593715545\\
328.067345767236	0.00106425044636466\\
328.3420928329	0.00106443291511564\\
328.616839898565	0.00106461399477972\\
328.89158696423	0.00106479402024579\\
329.166334029895	0.001064972005058\\
329.44108109556	0.00106514843766048\\
329.715828161225	0.00106532315978378\\
329.990575226889	0.00106549665969639\\
330.265322292554	0.00106566844949674\\
330.540069358219	0.00106583869246254\\
330.814816423884	0.00106600739029964\\
331.089563489549	0.00106617454228294\\
331.364310555214	0.00106634014795711\\
331.639057620878	0.00106650420999462\\
331.913804686543	0.00106666689395757\\
332.188551752208	0.00106682820280032\\
332.463298817873	0.00106698780854224\\
332.738045883538	0.00106714587309014\\
333.012792949203	0.0010673020707172\\
333.287540014867	0.0010674568899897\\
333.562287080532	0.00106761001029536\\
333.837034146197	0.00106776175169012\\
334.111781211862	0.00106791179298066\\
334.386528277527	0.00106806062482533\\
334.661275343192	0.00106820792179302\\
334.936022408857	0.00106835368606906\\
335.210769474521	0.00106849775048917\\
335.485516540186	0.00106864028136572\\
335.760263605851	0.00106878111884205\\
336.035010671516	0.00106892091664069\\
336.309757737181	0.00106905885635768\\
336.584504802846	0.00106919510158745\\
336.85925186851	0.00106933014835594\\
337.133998934175	0.00106946400113653\\
337.40874599984	0.00106959583107313\\
337.683493065505	0.00106972613364331\\
337.95824013117	0.00106985540571291\\
338.232987196835	0.00106998249623816\\
338.507734262499	0.00107010821871455\\
338.782481328164	0.00107023225647489\\
339.057228393829	0.00107035460484806\\
339.331975459494	0.0010704757543999\\
339.606722525159	0.00107059522006963\\
339.881469590824	0.00107071349149718\\
340.156216656488	0.00107083024152503\\
340.430963722153	0.00107094514691559\\
340.705710787818	0.00107105886253277\\
340.980457853483	0.0010711715645134\\
341.255204919148	0.00107128275824948\\
341.529951984813	0.00107139194238508\\
341.804699050478	0.00107149928584303\\
342.079446116142	0.00107160527230726\\
342.354193181807	0.00107170942148378\\
342.628940247472	0.00107181222145306\\
342.903687313137	0.00107191367074282\\
343.178434378802	0.00107201360879296\\
343.453181444467	0.00107211187751299\\
343.727928510131	0.00107220896467356\\
344.002675575796	0.00107230438215173\\
344.277422641461	0.00107239846103848\\
344.552169707126	0.00107249120093988\\
344.826916772791	0.0010725826033595\\
345.101663838456	0.00107267184779157\\
345.37641090412	0.00107275975257632\\
345.651157969785	0.0010728463238829\\
345.92590503545	0.00107293122681394\\
346.200652101115	0.00107301446669324\\
346.47539916678	0.00107309653447847\\
346.750146232445	0.0010731774395919\\
347.02489329811	0.00107325603070871\\
347.299640363774	0.00107333361884591\\
347.574387429439	0.00107341004389593\\
347.849134495104	0.00107348481478026\\
348.123881560769	0.0010735579318149\\
348.398628626434	0.00107362939430407\\
348.673375692099	0.00107369936707841\\
348.948122757763	0.00107376785161849\\
349.222869823428	0.00107383501097693\\
349.497616889093	0.00107390035698029\\
349.772363954758	0.00107396454319861\\
350.047111020423	0.00107402724590734\\
350.321858086088	0.0010740884605764\\
350.596605151752	0.00107414786666672\\
350.871352217417	0.00107420562216477\\
351.146099283082	0.0010742622261067\\
351.420846348747	0.00107431734634876\\
351.695593414412	0.00107437082698267\\
351.970340480077	0.00107442282610875\\
352.245087545742	0.00107447350573429\\
352.519834611406	0.00107452254719673\\
352.794581677071	0.00107457027265076\\
353.069328742736	0.00107461636076209\\
353.344075808401	0.00107466146990982\\
353.618822874066	0.00107470478020887\\
353.893569939731	0.00107474662586336\\
354.168317005395	0.00107478716643095\\
354.44306407106	0.00107482624070615\\
354.717811136725	0.00107486368123482\\
354.99255820239	0.00107489964968027\\
355.267305268055	0.00107493399376394\\
355.54205233372	0.00107496687078493\\
355.816799399384	0.00107499828186603\\
356.091546465049	0.00107502822897559\\
356.366293530714	0.00107505688066607\\
356.641040596379	0.00107508456926298\\
356.915787662044	0.00107511014485024\\
357.190534727709	0.00107513442594114\\
357.465281793373	0.00107515708795421\\
357.740028859038	0.0010751786209481\\
358.014775924703	0.00107519837244043\\
358.289522990368	0.00107521700284449\\
358.564270056033	0.0010752341822505\\
358.839017121698	0.00107524958538598\\
359.113764187363	0.00107526403430281\\
359.388511253027	0.00107527687166614\\
359.663258318692	0.00107528810077169\\
359.938005384357	0.00107529805272328\\
360.212752450022	0.00107530590937788\\
360.487499515687	0.00107531298225818\\
360.762246581352	0.00107531844906808\\
361.036993647016	0.00107532247694385\\
361.311740712681	0.00107532490518637\\
361.586487778346	0.00107532639251724\\
361.861234844011	0.00107532644752133\\
362.135981909676	0.0010753245759465\\
362.410728975341	0.00107532143416578\\
362.685476041005	0.00107531702735668\\
362.96022310667	0.00107531119239401\\
363.234970172335	0.00107530426066692\\
363.509717238	0.00107529557571011\\
363.784464303665	0.00107528513285223\\
364.05921136933	0.00107527326659992\\
364.333958434994	0.00107525997885519\\
364.608705500659	0.0010752460924125\\
364.883452566324	0.00107523045950428\\
365.158199631989	0.00107521275300791\\
365.432946697654	0.00107519412023316\\
365.707693763319	0.00107517407520872\\
365.982440828984	0.00107515261790674\\
366.257187894648	0.00107512974829891\\
366.531934960313	0.00107510563181224\\
366.806682025978	0.00107507994185428\\
367.081429091643	0.00107505235367854\\
367.356176157308	0.00107502401510226\\
367.630923222973	0.00107499393976328\\
367.905670288637	0.00107496230055629\\
368.180417354302	0.00107492958485491\\
368.455164419967	0.00107489580271837\\
368.729911485632	0.00107486045894297\\
369.004658551297	0.00107482371901932\\
369.279405616962	0.00107478541951132\\
369.554152682627	0.00107474539427232\\
369.828899748291	0.00107470447002959\\
370.103646813956	0.00107466198918687\\
370.378393879621	0.00107461795289428\\
370.653140945286	0.00107457268734801\\
370.927888010951	0.00107452570713572\\
371.202635076616	0.00107447733603092\\
371.47738214228	0.00107442807470357\\
371.752129207945	0.00107437694036329\\
372.02687627361	0.00107432458811194\\
372.301623339275	0.00107427036384598\\
372.57637040494	0.00107421541725084\\
372.851117470605	0.00107415876820889\\
373.125864536269	0.00107410090354954\\
373.400611601934	0.00107404149598229\\
373.675358667599	0.00107398055454914\\
373.950105733264	0.0010739184022754\\
374.224852798929	0.00107385520833605\\
374.499599864594	0.00107379015432047\\
374.774346930259	0.00107372406038071\\
375.049093995923	0.00107365693078537\\
375.323841061588	0.00107358860698037\\
375.598588127253	0.00107351842414815\\
375.873335192918	0.00107344671812091\\
376.148082258583	0.00107337364756801\\
376.422829324248	0.00107329921937882\\
376.697576389912	0.00107322327431511\\
376.972323455577	0.00107314646690421\\
377.247070521242	0.00107306830994537\\
377.521817586907	0.00107298897499606\\
377.796564652572	0.00107290829307812\\
378.071311718237	0.00107282577281291\\
378.346058783901	0.00107274223393342\\
378.620805849566	0.0010726568607891\\
378.895552915231	0.00107257063488265\\
379.170299980896	0.00107248257673597\\
379.445047046561	0.00107239350666938\\
379.719794112226	0.0010723030974177\\
379.99454117789	0.00107221135253208\\
380.269288243555	0.0010721182761513\\
380.54403530922	0.00107202353718789\\
380.818782374885	0.00107192829897343\\
381.09352944055	0.00107183140658968\\
381.368276506215	0.00107173302956744\\
381.643023571879	0.001071633488048\\
381.917770637544	0.00107153213605618\\
382.192517703209	0.00107142962471071\\
382.467264768874	0.00107132595237929\\
382.742011834539	0.00107122063359281\\
383.016758900204	0.00107111432330154\\
383.291505965869	0.00107100653459029\\
383.566253031533	0.00107089792539005\\
383.841000097198	0.00107078734866584\\
384.115747162863	0.00107067579628078\\
384.390494228528	0.00107056244474602\\
384.665241294193	0.00107044844564671\\
384.939988359858	0.00107033331491416\\
385.214735425522	0.00107021672318834\\
385.489482491187	0.00107009883386314\\
385.764229556852	0.00106997932090409\\
386.038976622517	0.00106985868257568\\
386.313723688182	0.00106973659161641\\
386.588470753847	0.00106961370345278\\
386.863217819511	0.0010694887100255\\
387.137964885176	0.00106936275496561\\
387.412711950841	0.00106923551857841\\
387.687459016506	0.00106910748958767\\
387.962206082171	0.00106897768448653\\
388.236953147836	0.00106884675950875\\
388.511700213501	0.00106871455930967\\
388.786447279165	0.00106858141508376\\
389.06119434483	0.00106844650743049\\
389.335941410495	0.001068310163024\\
389.61068847616	0.00106817255384242\\
389.885435541825	0.0010680338382701\\
390.16018260749	0.00106789418268991\\
390.434929673154	0.0010677532646748\\
390.709676738819	0.00106761091848495\\
390.984423804484	0.00106746747407954\\
391.259170870149	0.00106732260622902\\
391.533917935814	0.00106717648247185\\
391.808665001479	0.00106702894492008\\
392.083412067143	0.00106688048719388\\
392.358159132808	0.00106673078082428\\
392.632906198473	0.00106657966354323\\
392.907653264138	0.00106642697391054\\
393.182400329803	0.00106627320234077\\
393.457147395468	0.00106611901678459\\
393.731894461133	0.00106596293217383\\
394.006641526797	0.00106580560688315\\
394.281388592462	0.00106564704348352\\
394.556135658127	0.00106548708583795\\
394.830882723792	0.00106532589426374\\
395.105629789457	0.00106516314874705\\
395.380376855122	0.00106499967153756\\
395.655123920786	0.00106483463789694\\
395.929870986451	0.00106466822205176\\
396.204618052116	0.00106450091181704\\
396.479365117781	0.00106433271246978\\
396.754112183446	0.00106416329262988\\
397.028859249111	0.00106399200176727\\
397.303606314775	0.0010638201586164\\
397.57835338044	0.00106364677740088\\
397.853100446105	0.00106347218476491\\
398.12784751177	0.00106329605865935\\
398.402594577435	0.00106311938992287\\
398.6773416431	0.00106294135418348\\
398.952088708765	0.00106276211687085\\
399.226835774429	0.00106258118844353\\
399.501582840094	0.00106239939037282\\
399.776329905759	0.00106221624043883\\
400.051076971424	0.00106203189693142\\
400.325824037089	0.00106184652423436\\
400.600571102754	0.00106165979788115\\
400.875318168418	0.00106147172865274\\
401.150065234083	0.00106128264128445\\
401.424812299748	0.00106109269983363\\
401.699559365413	0.00106090174734756\\
401.974306431078	0.00106070945339141\\
402.249053496743	0.00106051615422911\\
402.523800562407	0.00106032135581039\\
402.798547628072	0.00106012523076437\\
403.073294693737	0.0010599281009453\\
403.348041759402	0.00105972980225511\\
403.622788825067	0.00105953066676148\\
403.897535890732	0.00105933004259073\\
404.172282956396	0.00105912842823318\\
404.447030022061	0.00105892566219449\\
404.721777087726	0.00105872141247651\\
404.996524153391	0.00105851634397455\\
405.271271219056	0.00105831046459329\\
405.546018284721	0.00105810278360098\\
405.820765350386	0.00105789380372482\\
406.09551241605	0.00105768401098851\\
406.370259481715	0.00105747324183925\\
406.64500654738	0.00105726133735945\\
406.919753613045	0.00105704813539396\\
407.19450067871	0.00105683412921982\\
407.469247744375	0.00105661883235959\\
407.743994810039	0.00105640257594765\\
408.018741875704	0.00105618519490072\\
408.293488941369	0.00105596652958688\\
408.568236007034	0.00105574690930475\\
408.842983072699	0.00105552601169239\\
409.117730138364	0.0010553038356355\\
409.392477204028	0.00105508021888048\\
409.667224269693	0.00105485648350185\\
409.941971335358	0.00105463147198952\\
410.216718401023	0.00105440486930191\\
410.491465466688	0.00105417715995366\\
410.766212532353	0.00105394801766067\\
411.040959598017	0.00105371843611459\\
411.315706663682	0.00105348743215555\\
411.590453729347	0.00105325550355619\\
411.865200795012	0.00105302298126417\\
412.139947860677	0.00105278887764474\\
412.414694926342	0.00105255352659436\\
412.689441992007	0.0010523170888609\\
412.964189057671	0.00105207973295904\\
413.238936123336	0.00105184113930967\\
413.513683189001	0.00105160180051888\\
413.788430254666	0.00105136106210425\\
414.063177320331	0.00105111942163809\\
414.337924385996	0.00105087621632456\\
414.61267145166	0.00105063211897944\\
414.887418517325	0.00105038728374303\\
415.16216558299	0.0010501410639121\\
415.436912648655	0.00104989329404919\\
415.71165971432	0.00104964496426442\\
415.986406779985	0.00104939575025661\\
416.261153845649	0.00104914564706936\\
416.535900911314	0.00104889499178171\\
416.810647976979	0.00104864263533269\\
417.085395042644	0.00104838907399971\\
417.360142108309	0.00104813496598731\\
417.634889173974	0.00104787933079978\\
417.909636239639	0.00104762265881616\\
418.184383305303	0.00104736545165303\\
418.459130370968	0.00104710705183017\\
418.733877436633	0.00104684779042181\\
419.008624502298	0.00104658766677349\\
419.283371567963	0.00104632554255302\\
419.558118633628	0.00104606256736666\\
419.832865699292	0.00104579890174749\\
420.107612764957	0.00104553422520615\\
420.382359830622	0.00104526853509372\\
420.657106896287	0.00104500233655756\\
420.931853961952	0.00104473464501968\\
421.206601027617	0.00104446660888457\\
421.481348093281	0.00104419692091399\\
421.756095158946	0.00104392639843271\\
422.030842224611	0.00104365488317721\\
422.305589290276	0.00104338221021124\\
422.580336355941	0.00104310921720561\\
422.855083421606	0.00104283474731881\\
423.129830487271	0.00104255946147635\\
423.404577552935	0.00104228221239693\\
423.6793246186	0.00104200464616269\\
423.954071684265	0.00104172676808657\\
424.22881874993	0.00104144693437733\\
424.503565815595	0.00104116679056321\\
424.77831288126	0.0010408851878003\\
425.053059946924	0.00104060247300558\\
425.327807012589	0.00104031863071833\\
425.602554078254	0.00104003432612248\\
425.877301143919	0.00103974906608026\\
426.152048209584	0.00103946301728026\\
426.426795275249	0.00103917634752604\\
426.701542340913	0.00103888823670945\\
426.976289406578	0.00103859918150118\\
427.251036472243	0.00103830951500533\\
427.525783537908	0.00103801858697416\\
427.800530603573	0.00103772722246488\\
428.075277669238	0.00103743509228223\\
428.350024734903	0.00103714153889032\\
428.624771800567	0.00103684705605375\\
428.899518866232	0.00103655181323071\\
429.174265931897	0.00103625565204738\\
429.449012997562	0.00103595873864495\\
429.723760063227	0.00103566075016226\\
429.998507128892	0.00103536168703223\\
430.273254194556	0.00103506187624303\\
430.548001260221	0.00103476033480189\\
430.822748325886	0.00103445822469495\\
431.097495391551	0.00103415537530779\\
431.372242457216	0.00103385163149964\\
431.646989522881	0.00103354682364423\\
431.921736588545	0.00103324128453195\\
432.19648365421	0.00103293486779736\\
432.471230719875	0.00103262706804173\\
432.74597778554	0.00103231903906252\\
433.020724851205	0.00103200996825495\\
433.29547191687	0.00103169985033183\\
433.570218982534	0.00103138902188542\\
433.844966048199	0.00103107748415077\\
434.119713113864	0.00103076474570159\\
434.394460179529	0.00103045081101102\\
434.669207245194	0.00103013667595072\\
434.943954310859	0.00102982102305845\\
435.218701376524	0.00102950516791625\\
435.493448442188	0.00102918813117599\\
435.768195507853	0.00102887073560577\\
436.042942573518	0.00102855199732655\\
436.317689639183	0.00102823273937113\\
436.592436704848	0.00102791280325306\\
436.867183770513	0.00102759153299266\\
437.141930836177	0.00102726942122464\\
437.416677901842	0.00102694647562669\\
437.691424967507	0.00102662269864061\\
437.966172033172	0.00102629842346257\\
438.240919098837	0.00102597315723216\\
438.515666164502	0.0010256475655485\\
438.790413230166	0.00102532082836723\\
439.065160295831	0.00102499344455692\\
439.339907361496	0.00102466459097995\\
439.614654427161	0.0010243349325613\\
439.889401492826	0.00102400462893469\\
440.164148558491	0.00102367351861229\\
440.438895624155	0.00102334161248218\\
440.71364268982	0.00102300924353675\\
440.988389755485	0.00102267624361014\\
441.26313682115	0.00102234212017085\\
441.537883886815	0.00102200720622551\\
441.81263095248	0.00102167134193367\\
442.087378018145	0.00102133535096018\\
442.362125083809	0.00102099792383532\\
442.636872149474	0.00102066021501476\\
442.911619215139	0.00102032139740064\\
443.186366280804	0.00101998131385081\\
443.461113346469	0.00101964112105019\\
443.735860412134	0.00101929983336286\\
444.010607477798	0.00101895811436812\\
444.285354543463	0.00101861497406117\\
444.560101609128	0.00101827174085932\\
444.834848674793	0.00101792791606382\\
445.109595740458	0.00101758317171073\\
445.384342806123	0.00101723702367852\\
445.659089871788	0.00101689078745941\\
445.933836937452	0.00101654397605166\\
446.208584003117	0.00101619642616522\\
446.483331068782	0.00101584830542644\\
446.758078134447	0.00101549879709893\\
447.032825200112	0.00101514855282375\\
447.307572265777	0.00101479774485035\\
447.582319331441	0.00101444604755926\\
447.857066397106	0.00101409362539557\\
448.131813462771	0.0010137406481819\\
448.406560528436	0.00101338679115184\\
448.681307594101	0.00101303238632986\\
448.956054659766	0.00101267727113394\\
449.23080172543	0.00101232160733883\\
449.505548791095	0.00101196507390828\\
449.78029585676	0.00101160751166938\\
450.055042922425	0.00101124941336232\\
450.32978998809	0.00101089078735394\\
450.604537053755	0.00101053179826407\\
450.87928411942	0.00101017162923002\\
451.154031185084	0.00100981093307341\\
451.428778250749	0.00100944972005533\\
451.703525316414	0.00100908782289646\\
451.978272382079	0.00100872508459722\\
452.253019447744	0.00100836150227343\\
452.527766513409	0.00100799740933007\\
452.802513579073	0.00100763215198915\\
453.077260644738	0.0010072664015516\\
453.352007710403	0.00100690081303795\\
453.626754776068	0.00100653390499584\\
453.901501841733	0.0010061668348154\\
454.176248907398	0.00100579894826305\\
454.450995973062	0.00100543024497867\\
454.725743038727	0.00100506089368851\\
455.000490104392	0.00100469105907889\\
455.275237170057	0.00100432025708907\\
455.549984235722	0.00100394931164593\\
455.824731301387	0.00100357723357249\\
456.099478367051	0.00100320452143745\\
456.374225432716	0.00100283085244193\\
456.648972498381	0.00100245721803093\\
456.923719564046	0.00100208279257983\\
457.198466629711	0.00100170807460062\\
457.473213695376	0.00100133241353145\\
457.747960761041	0.00100095580262089\\
458.022707826705	0.00100057940838613\\
458.29745489237	0.00100020191224296\\
458.572201958035	0.000999823317257447\\
458.8469490237	0.000999444614368557\\
459.121696089365	0.00099906498572942\\
459.39644315503	0.000998684921995397\\
459.671190220694	0.000998304426007573\\
459.945937286359	0.000997923668837503\\
460.220684352024	0.000997541500636423\\
460.495431417689	0.000997158916939041\\
460.770178483354	0.000996776076172787\\
461.044925549019	0.00099639265359069\\
461.319672614683	0.000996008653227705\\
461.594419680348	0.000995624569414019\\
461.869166746013	0.000995238765212269\\
462.143913811678	0.000994853051260451\\
462.418660877343	0.000994466929947975\\
462.693407943008	0.000994079590698507\\
462.968155008672	0.000993692354316541\\
463.242902074337	0.00099330423087666\\
463.517649140002	0.000992915717854842\\
463.792396205667	0.000992526156808792\\
464.067143271332	0.000992136384191677\\
464.341890336997	0.000991746389269832\\
464.616637402661	0.000991355027763643\\
464.891384468326	0.000990963627756456\\
465.166131533991	0.000990571527940721\\
465.440878599656	0.000990179226366531\\
465.715625665321	0.000989786549689226\\
465.990372730986	0.000989393013387103\\
466.265119796651	0.000988998946078913\\
466.539866862315	0.000988604193074305\\
466.81461392798	0.000988209252690267\\
467.089360993645	0.000987813789107626\\
467.36410805931	0.000987417479167971\\
467.638855124975	0.000987020827077001\\
467.91360219064	0.000986623991467667\\
468.188349256304	0.000986226320980669\\
468.463096321969	0.000985828310131615\\
468.737843387634	0.000985429799620136\\
469.012590453299	0.000985030950266987\\
469.287337518964	0.000984631603631456\\
469.562084584629	0.000984231434986974\\
469.836831650294	0.000983831108128317\\
470.111578715958	0.000983430456448509\\
470.386325781623	0.000983029322823616\\
470.661072847288	0.000982627381678442\\
470.935819912953	0.000982224962602179\\
471.210566978618	0.000981822226293704\\
471.485314044283	0.00098141852675032\\
471.760061109947	0.000981014683978541\\
472.034808175612	0.000980610049812984\\
472.309555241277	0.000980205280223811\\
472.584302306942	0.000979799385053434\\
472.859049372607	0.000979393371207329\\
473.133796438272	0.000978986574278351\\
473.408543503936	0.000978579981343708\\
473.683290569601	0.000978172931903055\\
473.958037635266	0.000977764942028861\\
474.232784700931	0.000977356677322565\\
474.507531766596	0.000976948303153654\\
474.782278832261	0.000976539488751132\\
475.057025897926	0.000976130403412566\\
475.33177296359	0.00097572023690939\\
475.606520029255	0.000975309640782756\\
475.88126709492	0.000974899271528941\\
476.156014160585	0.000974488149130927\\
476.43076122625	0.000974076774467542\\
476.705508291915	0.000973665311649548\\
476.980255357579	0.000973252938402716\\
477.255002423244	0.000972840311811003\\
477.529749488909	0.000972427108604681\\
477.804496554574	0.000972013660902652\\
478.079243620239	0.000971599476891568\\
478.353990685904	0.000971185049854002\\
478.628737751568	0.000970769736579439\\
478.903484817233	0.00097035418894608\\
479.178231882898	0.000969938245797516\\
479.452978948563	0.000969522402234124\\
479.727726014228	0.000969106009711955\\
480.002473079893	0.000968689396747754\\
480.277220145557	0.000968272727658472\\
480.551967211222	0.000967855842446654\\
480.826714276887	0.000967438246129472\\
481.101461342552	0.000967020106796093\\
481.376208408217	0.000966601594398382\\
481.650955473882	0.000966182712736912\\
481.925702539547	0.000965762972291188\\
482.200449605211	0.000965343352759298\\
482.475196670876	0.000964923204042086\\
482.749943736541	0.000964502687255699\\
483.024690802206	0.000964081649959097\\
483.299437867871	0.00096366041638554\\
483.574184933536	0.000963238828196563\\
483.8489319992	0.000962817215516206\\
484.123679064865	0.0009623944321352\\
484.39842613053	0.000961971961904512\\
484.673173196195	0.000961549144126226\\
484.94792026186	0.000961125818636685\\
485.222667327525	0.000960701659829598\\
485.497414393189	0.000960277829076674\\
485.772161458854	0.000959853499673125\\
486.046908524519	0.00095942916991153\\
486.321655590184	0.000959004346470413\\
486.596402655849	0.000958579030535944\\
486.871149721514	0.000958153718226984\\
487.145896787178	0.000957727753269979\\
487.420643852843	0.000957301965539681\\
487.695390918508	0.000956875855537739\\
487.970137984173	0.000956448607194117\\
488.244885049838	0.000956021538226487\\
488.519632115503	0.000955594322623106\\
488.794379181168	0.000955166466605434\\
489.069126246832	0.000954738302123156\\
489.343873312497	0.000954309670036865\\
489.618620378162	0.000953881232269352\\
489.893367443827	0.000953452167573774\\
490.168114509492	0.000953022807719509\\
490.442861575157	0.000952593319868431\\
490.717608640821	0.000952163538812161\\
490.992355706486	0.000951733794969462\\
491.267102772151	0.00095130376277806\\
491.541849837816	0.000950872953337411\\
491.816596903481	0.00095044185575387\\
492.091343969146	0.000950010144707706\\
492.36609103481	0.00094957914428139\\
492.640838100475	0.000949147372629482\\
492.91558516614	0.000948714834884417\\
493.190332231805	0.00094828252078383\\
493.46507929747	0.000947849275271413\\
493.739826363135	0.000947416101531079\\
494.0145734288	0.000946982989031291\\
494.289320494464	0.000946549778008528\\
494.564067560129	0.000946115806423212\\
494.838814625794	0.000945681415383521\\
495.113561691459	0.000945246772013832\\
495.388308757124	0.000944812372160595\\
495.663055822789	0.000944377224493166\\
495.937802888453	0.000943942161463228\\
496.212549954118	0.000943507341613898\\
496.487297019783	0.000943071291477012\\
496.762044085448	0.000942635330126665\\
497.036791151113	0.000942199784338433\\
497.311538216778	0.000941763342045779\\
497.586285282443	0.000941326831138853\\
497.861032348107	0.000940890414328239\\
498.135779413772	0.000940453270586366\\
498.410526479437	0.000940016220382778\\
498.685273545102	0.000939578773628033\\
498.960020610767	0.000939141592476227\\
499.234767676432	0.000938704019948729\\
499.509514742096	0.000938265569285761\\
499.784261807761	0.00093782722537689\\
500.059008873426	0.000937388498513668\\
500.333755939091	0.000936950046455043\\
500.608503004756	0.000936510722966925\\
500.883250070421	0.000936071846810184\\
501.157997136085	0.000935631940094308\\
501.43274420175	0.000935192651985269\\
501.707491267415	0.000934753323092813\\
501.98223833308	0.000934313788359821\\
502.256985398745	0.000933873389676651\\
502.53173246441	0.000933432625934953\\
502.806479530074	0.000932992159915005\\
503.081226595739	0.000932551661649365\\
503.355973661404	0.000932110796802144\\
503.630720727069	0.000931669570597137\\
503.905467792734	0.000931227984885986\\
504.180214858399	0.000930786377579388\\
504.454961924063	0.00093034441727251\\
504.729708989728	0.000929902437162484\\
505.004456055393	0.000929460109570529\\
505.279203121058	0.000929017437762192\\
505.553950186723	0.000928574910434026\\
505.828697252388	0.000928132533115288\\
506.103444318053	0.000927689640414243\\
506.378191383717	0.000927246245503265\\
506.652938449382	0.000926803496201126\\
506.927685515047	0.000926360077654003\\
507.202432580712	0.000925916648292121\\
507.477179646377	0.000925472720377167\\
507.751926712042	0.000925028782950997\\
508.026673777706	0.000924584690158603\\
508.301420843371	0.000924140425959819\\
508.576167909036	0.000923696157859509\\
508.850914974701	0.000923251720926354\\
509.125662040366	0.000922806625512933\\
509.400409106031	0.000922362028962606\\
509.675156171695	0.000921916616497288\\
509.94990323736	0.000921471218790199\\
510.224650303025	0.000921025826934924\\
510.49939736869	0.00092058028388123\\
510.774144434355	0.000920135087624598\\
511.04889150002	0.000919689079922187\\
511.323638565684	0.000919243583896718\\
511.598385631349	0.000918797617048858\\
511.873132697014	0.000918351169946857\\
512.147879762679	0.000917904746558897\\
512.422626828344	0.000917458842931594\\
512.697373894009	0.000917011815640494\\
512.972120959674	0.000916565142339261\\
513.246868025338	0.000916117835844867\\
513.521615091003	0.000915670566669086\\
513.796362156668	0.000915224157152379\\
514.071109222333	0.000914777283518318\\
514.345856287998	0.000914329619301364\\
514.620603353663	0.000913882156901747\\
514.895350419328	0.00091343489380338\\
515.170097484992	0.000912986518061772\\
515.444844550657	0.000912538513935606\\
515.719591616322	0.000912090379214739\\
515.994338681987	0.000911641632655954\\
516.269085747652	0.000911193258589118\\
516.543832813317	0.000910745096520943\\
516.818579878981	0.000910296655165626\\
517.093326944646	0.00090984792898405\\
517.368074010311	0.000909399087973168\\
517.642821075976	0.00090895078860709\\
517.917568141641	0.000908501556900613\\
518.192315207306	0.000908052213702876\\
518.46706227297	0.000907603251535469\\
518.741809338635	0.00090715352905384\\
519.0165564043	0.000906703871541233\\
519.291303469965	0.000906254601251186\\
519.56605053563	0.000905804734520204\\
519.840797601295	0.000905355094004509\\
520.115544666959	0.000904905025954053\\
520.390291732624	0.000904455017742604\\
520.665038798289	0.000904004587032981\\
520.939785863954	0.00090355504293498\\
521.214532929619	0.000903104087011883\\
521.489279995284	0.000902653537816317\\
521.764027060949	0.000902203220905634\\
522.038774126613	0.000901751333171068\\
522.313521192278	0.000901301173149784\\
522.588268257943	0.000900850265676947\\
522.863015323608	0.000900399756532922\\
523.137762389273	0.000899948994516709\\
523.412509454938	0.000899498467739342\\
523.687256520602	0.000899047192692348\\
523.962003586267	0.000898595993507858\\
524.236750651932	0.000898144388169955\\
524.511497717597	0.000897693026448615\\
524.786244783262	0.000897241415503225\\
525.060991848927	0.000896790220112059\\
525.335738914592	0.000896338290834906\\
525.610485980256	0.000895885791030657\\
525.885233045921	0.000895434046179656\\
526.159980111586	0.000894982553300954\\
526.434727177251	0.000894530982612428\\
526.709474242916	0.000894079500097232\\
526.98422130858	0.000893627775143796\\
527.258968374245	0.000893175151912071\\
527.53371543991	0.000892723121422582\\
527.808462505575	0.000892270850723411\\
528.08320957124	0.000891818185286936\\
528.357956636905	0.000891366440089106\\
528.63270370257	0.000890913971067944\\
528.907450768234	0.000890461761534609\\
529.182197833899	0.000890008664176806\\
529.456944899564	0.000889555826887218\\
529.731691965229	0.000889102438316933\\
530.006439030894	0.000888649647644455\\
530.281186096559	0.000888196792258791\\
530.555933162223	0.000887743224552351\\
530.830680227888	0.000887289929310592\\
531.105427293553	0.000886836572145406\\
531.380174359218	0.000886383317724683\\
531.654921424883	0.000885930172525236\\
531.929668490548	0.00088547729755578\\
532.204415556213	0.000885024365261234\\
532.479162621877	0.000884570886544749\\
532.753909687542	0.000884118013646523\\
533.028656753207	0.000883664430715473\\
533.303403818872	0.000883210958903581\\
533.578150884537	0.000882757930049858\\
533.852897950202	0.000882304353948997\\
534.127645015866	0.000881850399996108\\
534.402392081531	0.000881396556535197\\
534.677139147196	0.000880942833815307\\
534.951886212861	0.000880489220545229\\
535.226633278526	0.000880035231485754\\
535.501380344191	0.00087958168816232\\
535.776127409855	0.000879127772131886\\
536.05087447552	0.000878673974119189\\
536.325621541185	0.000878219965912468\\
536.60036860685	0.000877765744399311\\
536.875115672515	0.000877310990996546\\
537.14986273818	0.000876856527417706\\
537.424609803845	0.000876402025006285\\
537.699356869509	0.000875948147765673\\
537.974103935174	0.000875494234869536\\
538.248851000839	0.00087503978459197\\
538.523598066504	0.000874585626884695\\
538.798345132169	0.000874131922324054\\
539.073092197834	0.000873677689028297\\
539.347839263498	0.000873223260757607\\
539.622586329163	0.000872768310798323\\
539.897333394828	0.000872313819935968\\
540.172080460493	0.000871859462553322\\
540.446827526158	0.000871405236309031\\
540.721574591823	0.000870950157114325\\
540.996321657487	0.000870495703436476\\
541.271068723152	0.000870040884505622\\
541.545815788817	0.000869585712688788\\
541.820562854482	0.000869130832765842\\
542.095309920147	0.000868675761651262\\
542.370056985812	0.000868221316554549\\
542.644804051476	0.000867766352357221\\
542.919551117141	0.000867311694756596\\
543.194298182806	0.000866856510612741\\
543.469045248471	0.000866401794368601\\
543.743792314136	0.000865947053531741\\
544.018539379801	0.000865492127794768\\
544.293286445465	0.000865037504517277\\
544.56803351113	0.000864582364131719\\
544.842780576795	0.000864127522177069\\
545.11752764246	0.000863672162995219\\
545.392274708125	0.000863217103855344\\
545.66702177379	0.000862761697684529\\
545.941768839455	0.000862306758787802\\
546.216515905119	0.000861852288256195\\
546.491262970784	0.000861396648752364\\
546.766010036449	0.000860940824095947\\
547.040757102114	0.000860485633703893\\
547.315504167779	0.000860030418588039\\
547.590251233444	0.000859574690923093\\
547.864998299108	0.000859119600724987\\
548.139745364773	0.000858664161142847\\
548.414492430438	0.00085820969130396\\
548.689239496103	0.00085775437380659\\
548.963986561768	0.000857299037907199\\
549.238733627433	0.000856842691912313\\
549.513480693097	0.000856387814008805\\
549.788227758762	0.00085593242032513\\
550.062974824427	0.000855477010252804\\
550.337721890092	0.000855021740722168\\
550.612468955757	0.000854566286726717\\
550.887216021422	0.000854110816777237\\
551.161963087087	0.000853655333026272\\
551.436710152751	0.000853199339877196\\
551.711457218416	0.000852743334950835\\
551.986204284081	0.000852287977840715\\
552.260951349746	0.000851833424740753\\
552.535698415411	0.000851377876026121\\
552.810445481076	0.000850921980993065\\
553.08519254674	0.000850465575418713\\
553.359939612405	0.00085001014171568\\
553.63468667807	0.000849554367316504\\
553.909433743735	0.000849098574543578\\
554.1841808094	0.000848642762082536\\
554.458927875065	0.000848186939097453\\
554.73367494073	0.000847731594153588\\
555.008422006394	0.000847275743851472\\
555.283169072059	0.000846819720077358\\
555.557916137724	0.000846364842191706\\
555.832663203389	0.000845908474040205\\
556.107410269054	0.000845452427619843\\
556.382157334718	0.0008449962081728\\
556.656904400383	0.000844540306710297\\
556.931651466048	0.000844084398505499\\
557.206398531713	0.00084362864393787\\
557.481145597378	0.000843172878959055\\
557.755892663043	0.00084271759364161\\
558.030639728708	0.000842262302690603\\
558.305386794372	0.000841806342202973\\
558.580133860037	0.000841350706955531\\
558.854880925702	0.000840894570047518\\
559.129627991367	0.000840438915290962\\
559.404375057032	0.000839983085527914\\
559.679122122697	0.000839526913771467\\
559.953869188361	0.000839071391710489\\
560.228616254026	0.000838615365742509\\
560.503363319691	0.000838159660715956\\
560.778110385356	0.000837703784437818\\
561.052857451021	0.00083724757501817\\
561.327604516686	0.000836791692283333\\
561.602351582351	0.000836335638397616\\
561.877098648015	0.000835879737690832\\
562.15184571368	0.00083542399229244\\
562.426592779345	0.00083496741750296\\
562.70133984501	0.000834511495346841\\
562.976086910675	0.000834055568346397\\
563.25083397634	0.000833600125730919\\
563.525581042004	0.0008331435265783\\
563.800328107669	0.000832687746999997\\
564.075075173334	0.000832231138798062\\
564.349822238999	0.000831775030050697\\
564.624569304664	0.000831319402107256\\
564.899316370329	0.00083086344008523\\
565.174063435993	0.000830406484795701\\
565.448810501658	0.000829950519045626\\
565.723557567323	0.000829493731826328\\
565.998304632988	0.000829037272746257\\
566.273051698653	0.000828581470000419\\
566.547798764318	0.000828125332290671\\
566.822545829982	0.000827669357980868\\
567.097292895647	0.000827212882432911\\
567.372039961312	0.000826756579576615\\
567.646787026977	0.000826300273402308\\
567.921534092642	0.000825843794161326\\
568.196281158307	0.000825387809023451\\
568.471028223972	0.000824931330220557\\
568.745775289636	0.000824474849790196\\
569.020522355301	0.000824018700700631\\
569.295269420966	0.000823562382466504\\
569.570016486631	0.00082310671578404\\
569.844763552296	0.000822650066013252\\
570.119510617961	0.000822194233880162\\
570.394257683625	0.000821738725700318\\
570.66900474929	0.000821283048798502\\
570.943751814955	0.000820825902778527\\
571.21849888062	0.000820370238778074\\
571.493245946285	0.000819914567647106\\
571.76799301195	0.000819458562958152\\
572.042740077614	0.000819002392164785\\
572.317487143279	0.000818546548365312\\
572.592234208944	0.000818090863684665\\
572.866981274609	0.000817634845303521\\
573.141728340274	0.000817178667060061\\
573.416475405939	0.00081672280747026\\
573.691222471604	0.000816266455783164\\
573.965969537268	0.00081581026331375\\
574.240716602933	0.000815354395496514\\
574.515463668598	0.000814898523106301\\
574.790210734263	0.000814442158787311\\
575.064957799928	0.000813986286012895\\
575.339704865593	0.000813530248567359\\
575.614451931257	0.000813074047421933\\
575.889198996922	0.00081261833615019\\
576.163946062587	0.000812161810887394\\
576.438693128252	0.000811705450296444\\
576.713440193917	0.000811248759303802\\
576.988187259582	0.000810792732070613\\
577.262934325247	0.000810337197524171\\
577.537681390911	0.00080988100416149\\
577.812428456576	0.000809425134035958\\
578.087175522241	0.000808968939271157\\
578.361922587906	0.000808512743763782\\
578.636669653571	0.000808056054220651\\
578.911416719235	0.000807599377480098\\
579.1861637849	0.000807143693397671\\
579.460910850565	0.000806687682438026\\
579.73565791623	0.000806231676017714\\
580.010404981895	0.000805775994689857\\
580.28515204756	0.00080531950032157\\
580.559899113225	0.000804863497377468\\
580.834646178889	0.000804406352698857\\
581.109393244554	0.000803950365713175\\
581.384140310219	0.000803494542283488\\
581.658887375884	0.000803038885963029\\
581.933634441549	0.000802583065282901\\
582.208381507214	0.000802127575770439\\
582.483128572878	0.000801671270270208\\
582.757875638543	0.000801215300434426\\
583.032622704208	0.000800759174285186\\
583.307369769873	0.000800303383763344\\
583.582116835538	0.000799847761611327\\
583.856863901203	0.000799391972624553\\
584.131610966867	0.000798936189422281\\
584.406358032532	0.000798480414110688\\
584.681105098197	0.000798024641207199\\
584.955852163862	0.000797568211269348\\
585.230599229527	0.000797112121483852\\
585.505346295192	0.000796655715382605\\
585.780093360857	0.000796199981484035\\
586.054840426521	0.000795744083930422\\
586.329587492186	0.000795288355415261\\
586.604334557851	0.000794831813115458\\
586.879081623516	0.000794375771969491\\
587.153828689181	0.000793919902999053\\
587.428575754846	0.000793463876127666\\
587.70332282051	0.000793007694716853\\
587.978069886175	0.000792551683522368\\
588.25281695184	0.000792095688202216\\
588.527564017505	0.00079163953611573\\
588.80231108317	0.000791183059632277\\
589.077058148835	0.000790727249842976\\
589.351805214499	0.000790270957328092\\
589.626552280164	0.000789814676573798\\
589.901299345829	0.000789359231634449\\
590.176046411494	0.000788903627965401\\
590.450793477159	0.000788447712590863\\
590.725540542824	0.000787991637945868\\
591.000287608489	0.000787535410730768\\
591.275034674153	0.000787079521597651\\
591.549781739818	0.000786622987101944\\
591.824528805483	0.000786166970027442\\
592.099275871148	0.000785711128403653\\
592.374022936813	0.00078525513867454\\
592.648770002478	0.000784799495380127\\
592.923517068142	0.000784343375212435\\
593.198264133807	0.000783887429639483\\
593.473011199472	0.000783431827000358\\
593.747758265137	0.000782975909364251\\
594.022505330802	0.000782520663338914\\
594.297252396467	0.000782064607198414\\
594.571999462131	0.000781609064074246\\
594.846746527796	0.00078115304189789\\
595.121493593461	0.000780697694389334\\
595.396240659126	0.000780242193376712\\
595.670987724791	0.000779786538316405\\
595.945734790456	0.000779330574216218\\
596.220481856121	0.00077887495619132\\
596.495228921785	0.000778418867409173\\
596.76997598745	0.000777962963024265\\
597.044723053115	0.000777507244400332\\
597.31947011878	0.000777051377344234\\
597.594217184445	0.000776596351014465\\
597.868964250109	0.00077614084658562\\
598.143711315774	0.000775685023363524\\
598.418458381439	0.000775229228217885\\
598.693205447104	0.00077477411063276\\
598.967952512769	0.000774318512919207\\
599.242699578434	0.000773863270134622\\
599.517446644099	0.00077340804574637\\
599.792193709763	0.000772952349615922\\
600.066940775428	0.000772497177929749\\
600.341687841093	0.000772042022070774\\
600.616434906758	0.000771586233218436\\
600.891181972423	0.000771130805396462\\
601.165929038088	0.000770675559582377\\
601.440676103752	0.000770220180566174\\
601.715423169417	0.000769765314022168\\
601.990170235082	0.000769310138234523\\
602.264917300747	0.000768855644983985\\
602.539664366412	0.000768400518551813\\
602.814411432077	0.000767945745169655\\
603.089158497742	0.000767491002100244\\
603.363905563406	0.000767036120915041\\
603.638652629071	0.000766581099304135\\
603.913399694736	0.000766125786376117\\
604.188146760401	0.000765670820838125\\
604.462893826066	0.000765215565010031\\
604.737640891731	0.000764760994857216\\
605.012387957395	0.000764306446392616\\
605.28713502306	0.000763851919240949\\
605.561882088725	0.00076339676357642\\
605.83662915439	0.000762941809002789\\
606.111376220055	0.000762487708100954\\
606.38612328572	0.000762032981848151\\
606.660870351384	0.000761578609141806\\
606.935617417049	0.000761123454524142\\
607.210364482714	0.000760668829455256\\
607.485111548379	0.000760214235889239\\
607.759858614044	0.000759759844438864\\
608.034605679709	0.000759304999398965\\
608.309352745374	0.000758851005831577\\
608.584099811038	0.000758396555396645\\
608.858846876703	0.000757941810924356\\
609.133593942368	0.000757487265848043\\
609.408341008033	0.000757032932763129\\
609.683088073698	0.000756578470167897\\
609.957835139363	0.000756123883905005\\
610.232582205027	0.000755670162853548\\
610.507329270692	0.000755216479987815\\
610.782076336357	0.000754762508472194\\
611.056823402022	0.000754308257350599\\
611.331570467687	0.00075385404764318\\
611.606317533352	0.000753400539849669\\
611.881064599016	0.000752946578111928\\
612.155811664681	0.000752492997392835\\
612.430558730346	0.000752039121148181\\
612.705305796011	0.00075158562619468\\
612.980052861676	0.00075113151527612\\
613.254799927341	0.000750678270779298\\
613.529546993006	0.00075022441479519\\
613.80429405867	0.000749770606940367\\
614.079041124335	0.000749316857452996\\
614.35378819	0.000748863327627001\\
614.628535255665	0.000748410181823685\\
614.90328232133	0.000747957080312112\\
615.178029386995	0.000747504193702789\\
615.452776452659	0.000747051194880322\\
615.727523518324	0.000746597597916679\\
616.002270583989	0.000746144712184038\\
616.277017649654	0.000745691710417704\\
616.551764715319	0.000745237948556198\\
616.826511780984	0.000744784414517706\\
617.101258846648	0.000744331603204768\\
617.376005912313	0.000743879015570581\\
617.650752977978	0.00074342615808873\\
617.925500043643	0.000742973199680722\\
618.200247109308	0.000742520475460222\\
618.474994174973	0.000742068304568016\\
618.749741240637	0.000741616526873407\\
619.024488306302	0.00074116383265757\\
619.299235371967	0.000740711526437877\\
619.573982437632	0.000740258794837061\\
619.848729503297	0.000739806784492051\\
620.123476568962	0.000739354680710813\\
620.398223634626	0.000738902313531552\\
620.672970700291	0.000738450184003312\\
620.947717765956	0.000737997970859353\\
621.222464831621	0.000737545662584393\\
621.497211897286	0.000737093601023875\\
621.771958962951	0.000736641609522303\\
622.046706028616	0.000736189862182341\\
622.32145309428	0.000735737867713884\\
622.596200159945	0.000735285949980411\\
622.87094722561	0.00073483476906376\\
623.145694291275	0.000734383827704578\\
623.42044135694	0.000733932965635075\\
623.695188422605	0.000733481686362818\\
623.969935488269	0.000733030322385454\\
624.244682553934	0.000732578881076084\\
624.519429619599	0.000732127690346873\\
624.794176685264	0.000731676092847354\\
625.068923750929	0.000731225410009048\\
625.343670816594	0.00073077497913379\\
625.618417882259	0.000730324466670075\\
625.893164947923	0.000729873216936892\\
626.167912013588	0.000729423046117727\\
626.442659079253	0.000728972309446127\\
626.717406144918	0.000728521333725908\\
626.992153210583	0.000728070949199559\\
627.266900276247	0.000727620986430873\\
627.541647341912	0.00072717127992126\\
627.816394407577	0.000726721171626994\\
628.091141473242	0.000726271159251051\\
628.365888538907	0.00072582141129133\\
628.640635604572	0.000725371597510353\\
628.915382670237	0.000724921560838973\\
629.190129735901	0.000724471955348902\\
629.464876801566	0.000724022131738809\\
629.739623867231	0.000723573403185554\\
630.014370932896	0.000723124614320233\\
630.289117998561	0.000722675928170924\\
630.563865064226	0.000722227508571046\\
630.83861212989	0.000721778699290114\\
631.113359195555	0.000721329505008464\\
631.38810626122	0.000720881248927743\\
631.662853326885	0.000720432774323793\\
631.93760039255	0.000719984252076412\\
632.212347458215	0.000719536004892013\\
632.48709452388	0.000719088205130678\\
632.761841589544	0.000718640348118259\\
633.036588655209	0.000718191461916456\\
633.311335720874	0.000717744010427918\\
633.586082786539	0.000717296831339445\\
633.860829852204	0.000716849434245844\\
634.135576917869	0.000716401831116806\\
634.410323983533	0.000715954837156061\\
634.685071049198	0.000715507470623407\\
634.959818114863	0.000715060718277697\\
635.234565180528	0.000714614080771786\\
635.509312246193	0.000714167072847239\\
635.784059311858	0.000713720524400158\\
636.058806377523	0.000713273936262619\\
636.333553443187	0.00071282666193548\\
636.608300508852	0.000712379689773431\\
636.883047574517	0.00071193350686438\\
637.157794640182	0.000711487462258427\\
637.432541705847	0.000711041881847686\\
637.707288771511	0.000710596104257875\\
637.982035837176	0.000710150627145638\\
638.256782902841	0.000709705781101821\\
638.531529968506	0.000709260255571167\\
638.806277034171	0.000708814541128767\\
639.081024099836	0.000708369627325859\\
639.355771165501	0.000707924850327754\\
639.630518231165	0.000707479893038388\\
639.90526529683	0.000707035403891696\\
640.180012362495	0.000706590237640207\\
640.45475942816	0.000706145560649063\\
640.729506493825	0.000705700378802908\\
641.00425355949	0.000705256338065592\\
641.279000625154	0.000704812607990297\\
641.553747690819	0.000704368541512197\\
641.828494756484	0.00070392529029724\\
642.103241822149	0.000703481865149816\\
642.377988887814	0.000703038761549398\\
642.652735953479	0.000702595658177441\\
642.927483019143	0.000702152380975739\\
643.202230084808	0.000701709426115755\\
643.476977150473	0.000701267131725913\\
643.751724216138	0.000700824993939042\\
644.026471281803	0.000700382857150304\\
644.301218347468	0.000699940550760389\\
644.575965413133	0.000699498899985389\\
644.850712478797	0.000699057079562345\\
645.125459544462	0.000698614937251835\\
645.400206610127	0.000698172978421797\\
645.674953675792	0.000697731517857114\\
645.949700741457	0.000697290064962034\\
646.224447807122	0.000696849284437174\\
646.499194872786	0.000696408515717289\\
646.773941938451	0.000695967764790874\\
647.048689004116	0.000695527518198588\\
647.323436069781	0.000695087117265443\\
647.598183135446	0.000694647393900688\\
647.872930201111	0.000694207843009722\\
648.147677266776	0.000693767000303865\\
648.42242433244	0.000693327828084181\\
648.697171398105	0.000692888679651574\\
648.97191846377	0.000692449544336476\\
649.246665529435	0.000692010437036467\\
649.5214125951	0.000691571683916303\\
649.796159660764	0.000691132634210989\\
650.070906726429	0.000690694113078072\\
650.345653792094	0.000690256284156411\\
650.620400857759	0.00068981864443292\\
650.895147923424	0.000689380873383223\\
651.169894989089	0.000688943629033136\\
651.444642054754	0.000688505600930292\\
651.719389120418	0.000688068271153405\\
651.994136186083	0.000687630985898602\\
652.268883251748	0.00068719439799838\\
652.543630317413	0.000686757851606897\\
652.818377383078	0.00068632150927818\\
653.093124448743	0.000685885376307489\\
653.367871514407	0.000685449610061245\\
653.642618580072	0.000685013563452161\\
653.917365645737	0.000684577734124745\\
654.192112711402	0.000684142615468018\\
654.466859777067	0.000683707220116447\\
654.741606842732	0.000683272693383778\\
655.016353908397	0.000682838055099379\\
655.291100974061	0.000682403631347542\\
655.565848039726	0.000681969429962718\\
655.840595105391	0.000681535617417064\\
656.115342171056	0.000681102024531173\\
656.390089236721	0.000680668654201876\\
656.664836302386	0.000680234849075701\\
656.93958336805	0.000679801445440073\\
657.214330433715	0.000679368927184268\\
657.48907749938	0.000678935974603426\\
657.763824565045	0.000678504242614081\\
658.03857163071	0.00067807208419981\\
658.313318696375	0.000677640156225294\\
658.588065762039	0.000677208467119166\\
658.862812827704	0.000676776847795149\\
659.137559893369	0.000676345313182418\\
659.412306959034	0.000675914510474598\\
659.687054024699	0.000675483621507993\\
659.961801090364	0.000675053134696917\\
660.236548156028	0.000674622567814081\\
660.511295221693	0.000674192581688479\\
660.786042287358	0.000673763001338109\\
661.060789353023	0.000673333515069854\\
661.335536418688	0.000672903781042246\\
661.610283484353	0.000672474629848344\\
661.885030550018	0.00067204605838401\\
662.159777615682	0.000671617740329745\\
662.434524681347	0.000671189515495239\\
662.709271747012	0.000670761877041043\\
662.984018812677	0.000670334168686908\\
663.258765878342	0.000669905902856787\\
663.533512944007	0.000669479053100303\\
663.808260009671	0.000669052298130588\\
664.083007075336	0.000668625314314234\\
664.357754141001	0.000668199253345606\\
664.632501206666	0.00066777329169116\\
664.907248272331	0.0006673477661876\\
665.181995337996	0.000666922342543361\\
665.45674240366	0.000666497022452779\\
665.731489469325	0.000666071808090724\\
666.00623653499	0.000665646545278545\\
666.280983600655	0.000665222216310354\\
666.55573066632	0.000664798334490394\\
666.830477731985	0.000664374737023998\\
667.105224797649	0.000663951256454434\\
667.379971863314	0.000663527897806589\\
667.654718928979	0.000663105145174442\\
667.929465994644	0.000662682510398152\\
668.204213060309	0.000662260155821317\\
668.478960125974	0.000661837921993799\\
668.753707191638	0.000661416313691674\\
669.028454257303	0.000660994502883025\\
669.303201322968	0.000660572981279069\\
669.577948388633	0.000660152244141926\\
669.852695454298	0.000659731467499911\\
670.127442519963	0.000659310986777749\\
670.402189585628	0.000658890804025353\\
670.676936651292	0.000658471082827012\\
670.951683716957	0.000658051823126767\\
671.226430782622	0.000657632856107581\\
671.501177848287	0.000657213703275807\\
671.775924913952	0.000656795343932799\\
672.050671979617	0.000656376628578847\\
672.325419045281	0.000655958550857483\\
672.600166110946	0.000655540782251554\\
672.874913176611	0.000655123977239506\\
673.149660242276	0.000654707317585663\\
673.424407307941	0.000654290804079149\\
673.699154373606	0.000653874107319913\\
673.973901439271	0.000653457728710209\\
674.248648504935	0.000653042165796909\\
674.5233955706	0.000652626264266157\\
674.798142636265	0.000652211179164361\\
675.07288970193	0.000651796087170924\\
675.347636767595	0.000651381646159606\\
675.62238383326	0.000650967359878517\\
675.897130898924	0.00065055339714384\\
676.171877964589	0.000650139110103913\\
676.446625030254	0.000649725809435056\\
676.721372095919	0.000649312675350291\\
676.996119161584	0.000648900365735869\\
677.270866227249	0.000648488055079156\\
677.545613292913	0.000648075917374176\\
677.820360358578	0.000647664436024205\\
678.095107424243	0.000647253122409962\\
678.369854489908	0.000646841976330189\\
678.644601555573	0.000646431335106798\\
678.919348621238	0.000646021363476521\\
679.194095686903	0.000645611405915272\\
679.468842752567	0.000645201785992759\\
679.743589818232	0.000644792667775363\\
680.018336883897	0.000644383739100695\\
680.293083949562	0.000643974657183644\\
680.567831015227	0.000643565924868091\\
680.842578080892	0.000643157704387949\\
681.117325146556	0.000642750489257592\\
681.392072212221	0.000642343457595324\\
681.666819277886	0.000641936939720146\\
681.941566343551	0.00064153077386479\\
682.216313409216	0.000641124793152679\\
682.491060474881	0.00064071900384851\\
682.765807540545	0.000640313566020001\\
683.04055460621	0.000639908485046831\\
683.315301671875	0.000639503926176902\\
683.59004873754	0.00063909924035146\\
683.864795803205	0.000638695412328613\\
684.13954286887	0.000638291786446327\\
684.414289934535	0.000637888199203719\\
684.689037000199	0.000637485314895954\\
684.963784065864	0.000637083127589437\\
685.238531131529	0.00063668113798127\\
685.513278197194	0.000636279512752733\\
685.788025262859	0.00063587809179496\\
686.062772328524	0.000635477528638579\\
686.337519394188	0.00063507683634272\\
686.612266459853	0.000634676845196067\\
686.887013525518	0.00063427706414813\\
687.161760591183	0.000633877646000717\\
687.436507656848	0.000633477296827631\\
687.711254722513	0.000633078480558949\\
687.986001788177	0.000632680697705168\\
688.260748853842	0.000632282636360356\\
688.535495919507	0.000631885443123491\\
688.810242985172	0.000631488306859717\\
689.084990050837	0.0006310912194573\\
689.359737116502	0.000630694677120511\\
689.634484182166	0.000630298842988376\\
689.909231247831	0.000629903397222682\\
690.183978313496	0.000629508665073247\\
690.458725379161	0.000629113986234718\\
690.733472444826	0.000628719037775435\\
691.008219510491	0.000628325142739659\\
691.282966576155	0.000627931639993542\\
691.55771364182	0.000627538533575472\\
691.832460707485	0.000627145324665156\\
692.10720777315	0.000626752350154633\\
692.381954838815	0.000626360593755265\\
692.65670190448	0.000625969069738575\\
692.931448970145	0.000625578603079891\\
693.206196035809	0.000625187061713496\\
693.480943101474	0.000624796244720394\\
693.755690167139	0.000624405345729266\\
694.030437232804	0.000624015506252151\\
694.305184298469	0.000623625902972059\\
694.579931364134	0.000623237032632209\\
694.854678429798	0.000622848568082667\\
695.129425495463	0.000622460831216181\\
695.404172561128	0.000622072682566523\\
695.678919626793	0.000621684941645005\\
695.953666692458	0.000621298430918104\\
696.228413758123	0.000620911999581875\\
696.503160823788	0.000620525639047198\\
696.777907889452	0.000620139703845297\\
697.052654955117	0.000619754012394973\\
697.327402020782	0.000619368898357457\\
697.602149086447	0.000618984522200337\\
697.876896152112	0.000618600390405753\\
698.151643217776	0.000618217002625041\\
698.426390283441	0.000617833860802056\\
698.701137349106	0.000617450479860159\\
698.975884414771	0.000617068178272928\\
699.250631480436	0.000616686460250084\\
699.525378546101	0.000616304508264885\\
699.800125611766	0.000615923139714341\\
700.07487267743	0.000615542197631648\\
700.349619743095	0.000615162165867488\\
700.62436680876	0.00061478223306981\\
700.899113874425	0.000614402228604751\\
701.17386094009	0.000614023142756379\\
701.448608005755	0.000613644473841163\\
701.723355071419	0.000613266072079551\\
701.998102137084	0.000612888093315435\\
702.272849202749	0.000612511034743545\\
702.547596268414	0.000612134409612859\\
702.822343334079	0.00061175756079326\\
703.097090399744	0.000611381312855557\\
703.371837465409	0.00061100565978005\\
703.646584531073	0.00061062994798295\\
703.921331596738	0.000610255160973768\\
704.196078662403	0.000609880802958814\\
704.470825728068	0.000609506879835662\\
704.745572793733	0.000609133553741522\\
705.020319859398	0.00060876033051587\\
705.295066925062	0.000608387710753497\\
705.569813990727	0.000608015688180782\\
705.844561056392	0.000607644266503916\\
706.119308122057	0.000607272794169378\\
706.394055187722	0.000606901268163596\\
706.668802253387	0.000606530844266257\\
706.943549319051	0.000606161192204897\\
707.218296384716	0.000605791478274677\\
707.493043450381	0.000605422042653328\\
707.767790516046	0.00060505338199227\\
708.042537581711	0.000604685493062216\\
708.317284647376	0.000604318046101341\\
708.59203171304	0.000603950385339234\\
708.866778778705	0.000603583820077837\\
709.14152584437	0.000603217041714397\\
709.416272910035	0.000602851037192812\\
709.6910199757	0.00060248547450669\\
709.965767041365	0.000602120197341232\\
710.24051410703	0.000601755855038948\\
710.515261172694	0.000601391635688559\\
710.790008238359	0.000601027199361414\\
711.064755304024	0.000600663717404839\\
711.339502369689	0.000600301011825795\\
711.614249435354	0.000599938424530294\\
711.888996501019	0.000599576449276926\\
712.163743566683	0.000599215092035576\\
712.438490632348	0.000598854344537747\\
712.713237698013	0.000598493389356641\\
712.987984763678	0.000598132891678357\\
713.262731829343	0.000597773671513706\\
713.537478895008	0.000597414736705707\\
713.812225960672	0.000597055431833625\\
714.086973026337	0.000596697080560212\\
714.361720092002	0.000596339188915277\\
714.636467157667	0.000595981579992659\\
714.911214223332	0.000595624757359747\\
715.185961288997	0.000595268225448145\\
715.460708354662	0.000594911974838742\\
715.735455420326	0.000594555701762684\\
716.010202485991	0.000594199725727395\\
716.284949551656	0.000593844865350662\\
716.559696617321	0.000593490294291592\\
716.834443682986	0.000593136520439525\\
717.109190748651	0.000592783204126664\\
717.383937814315	0.00059243018111662\\
717.65868487998	0.000592078108786794\\
717.933431945645	0.000591726156692028\\
718.20817901131	0.000591374993977322\\
718.482926076975	0.000591023622299523\\
718.75767314264	0.000590673373674237\\
719.032420208305	0.000590322590836829\\
719.307167273969	0.000589972773913383\\
719.581914339634	0.000589623410540506\\
719.856661405299	0.00058927418477317\\
720.131408470964	0.000588925751414488\\
720.406155536629	0.000588577948844034\\
720.680902602293	0.000588230768896132\\
720.955649667958	0.000587884538501251\\
721.230396733623	0.000587538108875581\\
721.505143799288	0.000587191972865301\\
721.779890864953	0.000586846294076793\\
722.054637930618	0.000586501406863718\\
722.329384996283	0.000586156818278579\\
722.604132061947	0.000585812358888595\\
722.878879127612	0.000585468852980621\\
723.153626193277	0.000585125480143829\\
723.428373258942	0.000584782084084545\\
723.703120324607	0.000584439974245107\\
723.977867390272	0.000584097505923637\\
724.252614455936	0.00058375682355067\\
724.527361521601	0.00058341626834763\\
724.802108587266	0.000583075515113163\\
725.076855652931	0.000582735065516264\\
725.351602718596	0.000582395738641369\\
725.626349784261	0.000582056705637595\\
725.901096849926	0.00058171814422063\\
726.17584391559	0.000581380202098968\\
726.450590981255	0.000581042550146571\\
726.72533804692	0.000580705197453112\\
727.000085112585	0.000580368804489552\\
727.27483217825	0.000580032214000212\\
727.549579243915	0.000579696412958273\\
727.824326309579	0.000579360583088009\\
728.099073375244	0.000579024400250299\\
728.373820440909	0.000578690004071491\\
728.648567506574	0.000578356066242911\\
728.923314572239	0.000578022426812222\\
729.198061637904	0.000577689580937595\\
729.472808703568	0.000577356703304204\\
729.747555769233	0.000577023957497569\\
730.022302834898	0.000576692181570815\\
730.297049900563	0.000576360206758218\\
730.571796966228	0.000576030012016936\\
730.846544031893	0.000575699786822426\\
731.121291097557	0.000575369856384814\\
731.396038163222	0.00057504038999681\\
731.670785228887	0.000574711383364085\\
731.945532294552	0.000574383824585413\\
732.220279360217	0.000574055903938358\\
732.495026425882	0.000573728774629062\\
732.769773491547	0.000573401942633407\\
733.044520557211	0.00057307523496046\\
733.319267622876	0.000572749154470345\\
733.594014688541	0.000572423855942628\\
733.868761754206	0.000572098358891438\\
734.143508819871	0.000571773977777905\\
734.418255885536	0.000571449882334718\\
734.6930029512	0.000571126087077443\\
734.967750016865	0.000570802745674355\\
735.24249708253	0.000570479530134269\\
735.517244148195	0.000570156935968838\\
735.79199121386	0.000569835126767227\\
736.066738279525	0.000569514105613078\\
736.341485345189	0.000569193046499661\\
736.616232410854	0.000568872282293472\\
736.890979476519	0.000568552791708009\\
737.165726542184	0.000568232769997042\\
737.440473607849	0.000567913856477204\\
737.715220673514	0.000567594577974477\\
737.989967739178	0.000567276251332719\\
738.264714804843	0.000566958378372503\\
738.539461870508	0.00056664096175515\\
738.814208936173	0.000566323348256635\\
739.088956001838	0.000566006849668378\\
739.363703067503	0.000565690968109128\\
739.638450133168	0.000565374559350957\\
739.913197198832	0.000565059594228109\\
740.187944264497	0.000564744910328387\\
740.462691330162	0.000564430188789405\\
740.737438395827	0.000564117075350644\\
741.012185461492	0.000563804238040049\\
741.286932527157	0.000563490865428496\\
741.561679592821	0.000563178111243586\\
741.836426658486	0.000562866463231474\\
742.111173724151	0.000562554938710127\\
742.385920789816	0.000562243692476486\\
742.660667855481	0.000561933223308189\\
742.935414921146	0.000561623203810241\\
743.21016198681	0.000561313792135295\\
743.484909052475	0.000561004825319985\\
743.75965611814	0.000560695980490366\\
744.034403183805	0.00056038807506445\\
744.30915024947	0.000560079953590358\\
744.583897315135	0.000559772118586023\\
744.8586443808	0.000559464893050048\\
745.133391446464	0.000559158767557034\\
745.408138512129	0.000558852921898701\\
745.682885577794	0.000558547189640383\\
745.957632643459	0.000558241735393342\\
746.232379709124	0.0005579370565056\\
746.507126774789	0.00055763314370749\\
746.781873840453	0.000557329510990033\\
747.056620906118	0.000557026652670385\\
747.331367971783	0.000556724404595933\\
747.606115037448	0.000556421772027524\\
747.880862103113	0.000556119736986637\\
748.155609168778	0.000555817809484295\\
748.430356234442	0.000555516655038159\\
748.705103300107	0.000555215768040552\\
748.979850365772	0.000554915484060449\\
749.254597431437	0.000554615639089704\\
749.529344497102	0.000554315573936468\\
749.804091562767	0.000554016600520076\\
750.078838628432	0.000553717409748633\\
750.353585694096	0.000553419637607484\\
750.628332759761	0.000553121807091293\\
750.903079825426	0.000552824243804741\\
751.177826891091	0.000552527271432093\\
751.452573956756	0.000552230404848769\\
751.727321022421	0.00055193396967846\\
752.002068088085	0.000551638627445369\\
752.27681515375	0.000551343550193994\\
752.551562219415	0.00055104889995677\\
752.82630928508	0.000550753693338424\\
753.101056350745	0.000550459578899387\\
753.37580341641	0.000550166059646511\\
753.650550482074	0.00054987279698951\\
753.925297547739	0.000549579634326273\\
754.200044613404	0.000549287397450161\\
754.474791679069	0.000548995916614097\\
754.749538744734	0.000548704203731492\\
755.024285810399	0.000548413404900502\\
755.299032876064	0.000548122704183136\\
755.573779941728	0.000547833078149718\\
755.848527007393	0.000547543701037596\\
756.123274073058	0.000547254416671261\\
756.398021138723	0.000546965213491662\\
756.672768204388	0.00054667676643045\\
756.947515270053	0.000546388733360801\\
757.222262335717	0.000546100614982944\\
757.497009401382	0.000545813417053621\\
757.771756467047	0.000545526466671078\\
758.046503532712	0.000545239933808799\\
758.321250598377	0.000544953821133433\\
758.595997664042	0.000544668948026555\\
758.870744729706	0.00054438382629799\\
759.145491795371	0.000544098955145581\\
759.420238861036	0.000543814665125514\\
759.694985926701	0.000543530786768552\\
759.969732992366	0.000543247808275389\\
760.244480058031	0.000542964738360436\\
760.519227123695	0.000542682404356325\\
760.79397418936	0.000542400309310787\\
761.068721255025	0.00054211911676973\\
761.34346832069	0.000541837835434169\\
761.618215386355	0.000541556629497874\\
761.89296245202	0.000541275500329165\\
762.167709517685	0.00054099510626449\\
762.442456583349	0.000540715277432024\\
762.717203649014	0.000540436003702526\\
762.991950714679	0.00054015681502344\\
763.266697780344	0.000539878349645781\\
763.541444846009	0.000539599946949054\\
763.816191911674	0.000539322431414251\\
764.090938977338	0.000539045135810053\\
764.365686043003	0.000538768241954262\\
764.640433108668	0.000538491410868242\\
764.915180174333	0.000538215462001591\\
765.189927239998	0.00053793957883869\\
765.464674305663	0.000537664424094221\\
765.739421371327	0.000537389336594034\\
766.014168436992	0.000537114473694539\\
766.288915502657	0.00053684032762977\\
766.563662568322	0.00053656607657364\\
766.838409633987	0.000536292712866288\\
767.113156699652	0.000536019727872893\\
767.387903765317	0.000535747457077373\\
767.662650830981	0.000535475398687585\\
767.937397896646	0.000535203064707922\\
768.212144962311	0.000534931445451575\\
768.486892027976	0.000534660047851123\\
768.761639093641	0.000534389200480508\\
769.036386159305	0.000534119388096241\\
769.31113322497	0.00053384929715436\\
769.585880290635	0.00053357942033286\\
769.8606273563	0.000533310249569139\\
770.135374421965	0.000533041292444094\\
770.41012148763	0.000532773035317107\\
770.684868553295	0.000532505323868812\\
770.959615618959	0.000532237989331735\\
771.234362684624	0.000531970702691562\\
771.509109750289	0.000531703622931898\\
771.783856815954	0.000531436408915565\\
772.058603881619	0.000531170238801968\\
772.333350947284	0.000530904431741461\\
772.608098012949	0.000530638986696167\\
772.882845078613	0.000530373414022781\\
773.157592144278	0.000530108205618195\\
773.432339209943	0.000529843689375054\\
773.707086275608	0.000529579207780903\\
773.981833341273	0.000529315427892094\\
774.256580406938	0.000529052335617885\\
774.531327472602	0.000528789600482765\\
774.806074538267	0.000528526896814189\\
775.080821603932	0.000528264721829316\\
775.355568669597	0.000528002896226065\\
775.630315735262	0.000527741263320158\\
775.905062800927	0.000527480141986306\\
776.179809866591	0.000527218727706655\\
776.454556932256	0.000526958327288734\\
776.729303997921	0.000526697452445296\\
777.004051063586	0.000526437920291406\\
777.278798129251	0.000526178737856664\\
777.553545194916	0.000525919581489404\\
777.82829226058	0.000525661105772609\\
778.103039326245	0.00052540265081379\\
778.37778639191	0.000525144379044793\\
778.652533457575	0.000524886947126146\\
778.92728052324	0.000524629857332653\\
779.202027588905	0.000524372782684283\\
779.476774654569	0.000524116374122597\\
779.751521720234	0.000523860142622737\\
780.026268785899	0.000523603927265236\\
780.301015851564	0.000523348385612223\\
780.575762917229	0.000523093018108973\\
780.850509982894	0.000522838154242254\\
781.125257048559	0.000522584451263586\\
781.400004114223	0.000522329610497003\\
781.674751179888	0.00052207494736154\\
781.949498245553	0.000521821453384091\\
782.224245311218	0.000521567792964043\\
782.498992376883	0.000521314795333113\\
782.773739442548	0.000521062619513605\\
783.048486508212	0.000520810441064481\\
783.323233573877	0.000520558749199371\\
783.597980639542	0.000520306565385361\\
783.872727705207	0.000520055364306493\\
784.147474770872	0.000519803999855149\\
784.422221836537	0.000519553125871227\\
784.696968902202	0.000519302250719803\\
784.971715967866	0.000519052367588685\\
785.246463033531	0.000518802638976794\\
785.521210099196	0.000518554055155504\\
785.795957164861	0.000518305132810773\\
786.070704230526	0.00051805669412206\\
786.345451296191	0.000517808077296847\\
786.620198361855	0.000517560112519951\\
786.89494542752	0.000517311971994179\\
787.169692493185	0.000517064806297605\\
787.44443955885	0.000516817627822587\\
787.719186624515	0.000516571586573262\\
787.99393369018	0.000516324541838674\\
788.268680755844	0.000516078315379134\\
788.543427821509	0.000515831741003377\\
788.818174887174	0.000515585975608022\\
789.092921952839	0.000515340352226932\\
789.367669018504	0.000515095700424369\\
789.642416084169	0.000514851184048754\\
789.917163149834	0.000514606980505599\\
790.191910215498	0.000514362580139592\\
790.466657281163	0.000514118819005456\\
790.741404346828	0.00051387503191449\\
791.016151412493	0.000513631709632358\\
791.290898478158	0.000513389175680895\\
791.565645543822	0.000513146290353066\\
791.840392609487	0.000512903540500214\\
792.115139675152	0.00051266207964166\\
792.389886740817	0.000512420584041173\\
792.664633806482	0.000512179057162596\\
792.939380872147	0.0005119379882232\\
793.214127937812	0.000511697211514534\\
793.488875003476	0.00051145656340561\\
793.763622069141	0.000511217037585328\\
794.038369134806	0.000510977637256392\\
794.313116200471	0.00051073803065214\\
794.587863266136	0.000510498389520346\\
794.862610331801	0.000510259529923699\\
795.137357397465	0.000510020952473562\\
795.41210446313	0.00050978233220734\\
795.686851528795	0.000509544335389099\\
795.96159859446	0.000509306455624714\\
796.236345660125	0.000509069023389262\\
796.51109272579	0.00050883155175327\\
796.785839791455	0.000508594523536313\\
797.060586857119	0.000508358271280243\\
797.335333922784	0.000508121961224961\\
797.610080988449	0.000507885768847921\\
797.884828054114	0.000507649852545985\\
798.159575119779	0.000507414542316139\\
798.434322185444	0.000507179016242156\\
798.709069251108	0.000506944266691392\\
798.983816316773	0.000506709796677029\\
799.258563382438	0.000506475768681649\\
799.533310448103	0.000506241521130702\\
799.808057513768	0.000506007710656896\\
800.082804579433	0.000505774173455198\\
800.357551645097	0.000505541064135707\\
800.632298710762	0.00050530658955866\\
800.907045776427	0.000505074035415754\\
801.181792842092	0.000504841417008867\\
801.456539907757	0.000504609554609004\\
801.731286973422	0.00050437730230962\\
802.006034039086	0.000504145319287039\\
802.280781104751	0.000503914090743633\\
802.555528170416	0.00050368279573634\\
802.830275236081	0.000503452095347285\\
803.105022301746	0.000503220506168696\\
803.379769367411	0.000502990183756549\\
803.654516433076	0.000502760453720908\\
803.92926349874	0.000502530815832857\\
804.204010564405	0.000502300948842622\\
804.47875763007	0.000502071832416998\\
804.753504695735	0.000501842812423828\\
805.0282517614	0.000501614052918544\\
805.302998827065	0.000501385546319966\\
805.577745892729	0.000501157130832963\\
805.852492958394	0.000500929959856468\\
806.127240024059	0.000500703042319134\\
806.401987089724	0.000500475229202867\\
806.676734155389	0.000500248161395782\\
806.951481221054	0.000500020849424181\\
807.226228286719	0.000499794445099544\\
807.500975352383	0.000499567798369373\\
807.775722418048	0.000499341066377325\\
808.050469483713	0.000499114587683105\\
808.325216549378	0.000498888532388267\\
808.599963615043	0.000498663054863577\\
808.874710680707	0.000498437833837916\\
809.149457746372	0.00049821203055683\\
809.424204812037	0.000497986970315942\\
809.698951877702	0.000497761991506767\\
809.973698943367	0.000497537422757513\\
810.248446009032	0.000497313264630317\\
810.523193074697	0.000497090173890917\\
810.797940140361	0.000496866663568266\\
811.072687206026	0.000496643397890008\\
811.347434271691	0.000496419874163947\\
811.622181337356	0.000496196599111369\\
811.896928403021	0.000495974061507929\\
812.171675468686	0.000495751758485543\\
812.446422534351	0.000495529363053418\\
812.721169600015	0.000495307535230926\\
812.99591666568	0.000495086602062231\\
813.270663731345	0.000494865407112236\\
813.54541079701	0.000494644613171037\\
813.820157862675	0.000494423721326794\\
814.094904928339	0.000494203217864987\\
814.369651994004	0.000493982293426609\\
814.644399059669	0.000493761767831811\\
814.919146125334	0.000493541798837983\\
815.193893190999	0.000493322054550236\\
815.468640256664	0.000493102702288303\\
815.743387322329	0.000492883905700079\\
816.018134387993	0.000492664679070426\\
816.292881453658	0.000492445352438943\\
816.567628519323	0.000492226586092136\\
816.842375584988	0.000492008049438569\\
817.117122650653	0.000491789732818566\\
817.391869716318	0.000491571647829194\\
817.666616781982	0.000491353944156963\\
817.941363847647	0.000491136143762459\\
818.216110913312	0.000490918404261963\\
818.490857978977	0.000490701542470935\\
818.765605044642	0.000490484575153869\\
819.040352110307	0.000490267331913605\\
819.315099175971	0.000490050810920527\\
819.589846241636	0.000489835005961785\\
819.864593307301	0.000489619085117631\\
820.139340372966	0.000489403222201924\\
820.414087438631	0.000489187413410463\\
820.688834504296	0.00048897198714447\\
820.963581569961	0.000488756619005491\\
821.238328635625	0.000488541631856537\\
821.51307570129	0.000488326697012602\\
821.787822766955	0.000488111328639892\\
822.06256983262	0.00048789733357056\\
822.337316898285	0.000487683549007333\\
822.61206396395	0.000487469979380312\\
822.886811029614	0.000487256451428627\\
823.161558095279	0.000487043136799493\\
823.436305160944	0.000486829874834181\\
823.711052226609	0.000486616991077705\\
823.985799292274	0.0004864044886741\\
824.260546357939	0.000486191699837386\\
824.535293423603	0.00048597880543881\\
824.810040489268	0.000485766617412768\\
825.084787554933	0.000485554649276725\\
825.359534620598	0.000485343384004234\\
825.634281686263	0.000485132168437086\\
825.909028751928	0.000484921323585783\\
826.183775817593	0.000484710687662613\\
826.458522883257	0.000484500090456743\\
826.733269948922	0.00048428920955518\\
827.008017014587	0.000484079030905002\\
827.282764080252	0.000483868240120555\\
827.557511145917	0.000483658485277555\\
827.832258211582	0.000483448611860115\\
828.107005277246	0.000483239279691338\\
828.381752342911	0.000483029990485056\\
828.656499408576	0.000482820911136424\\
828.931246474241	0.000482612046713753\\
829.205993539906	0.000482403389633962\\
829.480740605571	0.000482195100336066\\
829.755487671236	0.000481986362352671\\
830.0302347369	0.000481778815827133\\
830.304981802565	0.000481571143200922\\
830.57972886823	0.000481364008871762\\
830.854475933895	0.000481157248374371\\
831.12922299956	0.000480949374005038\\
831.403970065224	0.000480742046190607\\
831.678717130889	0.000480535086566418\\
831.953464196554	0.000480327674154255\\
832.228211262219	0.000480121450247497\\
832.502958327884	0.0004799151022357\\
832.777705393549	0.000479709280859398\\
833.052452459214	0.00047950317076097\\
833.327199524878	0.000479297755031729\\
833.601946590543	0.000479092214424627\\
833.876693656208	0.000478886871122954\\
834.151440721873	0.000478681733004465\\
834.426187787538	0.00047847728470292\\
834.700934853203	0.000478272865248595\\
834.975681918867	0.000478067993891597\\
835.250428984532	0.000477863488136847\\
835.525176050197	0.0004776595085782\\
835.799923115862	0.000477455571525866\\
836.074670181527	0.000477251341291301\\
836.349417247192	0.00047704747783476\\
836.624164312856	0.000476844316375454\\
836.898911378521	0.000476641023132826\\
837.173658444186	0.000476438422938925\\
837.448405509851	0.000476235532860333\\
837.723152575516	0.000476032845191581\\
837.997899641181	0.000475830680571333\\
838.272646706846	0.000475628722068958\\
838.54739377251	0.000475426794218805\\
838.822140838175	0.000475224899639401\\
839.09688790384	0.000475023366021866\\
839.371634969505	0.000474822199748495\\
839.64638203517	0.000474621391949081\\
839.921129100834	0.000474420779396968\\
840.195876166499	0.000474220041756055\\
840.470623232164	0.000474019664615799\\
840.745370297829	0.000473818826462339\\
841.020117363494	0.000473619015969895\\
841.294864429159	0.000473419409083811\\
841.569611494824	0.000473219665643773\\
841.844358560488	0.000473020612452936\\
842.119105626153	0.000472821591346937\\
842.393852691818	0.00047262277017543\\
842.668599757483	0.000472423983198182\\
842.943346823148	0.000472225235352494\\
843.218093888813	0.000472027181281518\\
843.492840954477	0.000471829320550812\\
843.767588020142	0.000471631331355726\\
844.042335085807	0.000471433211965033\\
844.317082151472	0.000471235459530335\\
844.591829217137	0.000471037747182393\\
844.866576282802	0.000470840726122366\\
845.141323348466	0.000470643900366115\\
845.416070414131	0.00047044711220264\\
845.690817479796	0.000470250025062506\\
845.965564545461	0.000470053471080084\\
846.240311611126	0.000469857277281169\\
846.515058676791	0.000469661282359276\\
846.789805742456	0.000469465647466643\\
847.06455280812	0.000469269887468748\\
847.339299873785	0.000469074323251434\\
847.61404693945	0.000468878957133778\\
847.888794005115	0.000468683629351126\\
848.16354107078	0.000468489325792268\\
848.438288136445	0.000468295058152075\\
848.713035202109	0.000468100662041751\\
848.987782267774	0.000467906796956854\\
849.262529333439	0.000467712477812639\\
849.537276399104	0.000467519019931038\\
849.812023464769	0.000467325438088465\\
850.086770530434	0.000467131888238057\\
850.361517596098	0.000466938869042516\\
850.636264661763	0.000466746059620344\\
850.911011727428	0.000466553290837584\\
851.185758793093	0.000466360724601908\\
851.460505858758	0.000466168521931012\\
851.735252924423	0.000465975870143904\\
852.009999990088	0.000465783423923053\\
852.284747055753	0.000465591182538913\\
852.559494121417	0.000465399803456408\\
852.834241187082	0.000465208129266431\\
853.108988252747	0.000465016999245525\\
853.383735318412	0.00046482607007152\\
853.658482384077	0.000464635499841993\\
853.933229449741	0.000464444807280625\\
854.207976515406	0.000464254318549779\\
854.482723581071	0.000464064036225792\\
854.757470646736	0.000463873798941765\\
855.032217712401	0.000463684103030451\\
855.306964778066	0.000463494607691612\\
855.581711843731	0.000463305644864072\\
855.856458909395	0.000463116065425341\\
856.13120597506	0.000462927511582189\\
856.405953040725	0.000462738342682738\\
856.68070010639	0.000462550204706588\\
856.955447172055	0.000462361776499598\\
857.23019423772	0.00046217372490168\\
857.504941303384	0.000461985391382377\\
857.779688369049	0.000461798258588135\\
858.054435434714	0.000461611007096945\\
858.329182500379	0.000461423804460722\\
858.603929566044	0.000461236810576249\\
858.878676631709	0.000461050358423136\\
859.153423697373	0.000460862970521053\\
859.428170763038	0.000460676615022607\\
859.702917828703	0.000460490472761942\\
859.977664894368	0.000460304374298383\\
860.252411960033	0.000460118983439913\\
860.527159025698	0.000459933473742298\\
860.801906091363	0.000459747843518939\\
861.076653157027	0.000459562597156303\\
861.351400222692	0.00045937739769084\\
861.626147288357	0.000459192905685135\\
861.900894354022	0.00045900797477648\\
862.175641419687	0.00045882391740808\\
862.450388485351	0.00045863975210397\\
862.725135551016	0.000458455636977387\\
862.999882616681	0.000458272062729324\\
863.274629682346	0.000458088204085747\\
863.549376748011	0.000457904740472641\\
863.824123813676	0.000457721656736449\\
864.098870879341	0.000457538625008323\\
864.373617945006	0.000457355975971038\\
864.64836501067	0.000457173540039117\\
864.923112076335	0.000456991162796904\\
865.197859142	0.00045680966029027\\
865.472606207665	0.000456628206967233\\
865.74735327333	0.000456446815705221\\
866.022100338995	0.000456265477123401\\
866.296847404659	0.000456083698932703\\
866.571594470324	0.000455903299099497\\
866.846341535989	0.000455722302242455\\
867.121088601654	0.000455541372162839\\
867.395835667319	0.000455361326799518\\
867.670582732984	0.000455181834916975\\
867.945329798648	0.000455002234370092\\
868.220076864313	0.000454822695111915\\
868.494823929978	0.000454643385189859\\
868.769570995643	0.000454464631951465\\
869.044318061308	0.000454286099944089\\
869.319065126973	0.000454107462656541\\
869.593812192637	0.000453929383760147\\
869.868559258302	0.000453751853649504\\
870.143306323967	0.00045357406067763\\
870.418053389632	0.000453396325539908\\
870.692800455297	0.000453219312356528\\
870.967547520962	0.000453042036560639\\
871.242294586627	0.00045286532106365\\
871.517041652291	0.000452688839961715\\
871.791788717956	0.000452512923717792\\
872.066535783621	0.000452336907902221\\
872.341282849286	0.000452161451346238\\
872.616029914951	0.000451986232788648\\
872.890776980616	0.000451810923505184\\
873.16552404628	0.000451635674084169\\
873.440271111945	0.000451460497353249\\
873.71501817761	0.00045128620975105\\
873.989765243275	0.000451111173508164\\
874.26451230894	0.000450936706043284\\
874.539259374605	0.000450762804833664\\
874.814006440269	0.000450588973979369\\
875.088753505934	0.000450415378288488\\
875.363500571599	0.000450241858100424\\
875.638247637264	0.000450068409716668\\
875.912994702929	0.000449895532504966\\
876.187741768594	0.000449722403776508\\
876.462488834258	0.000449549848315077\\
876.737235899923	0.000449377705892155\\
877.011982965588	0.000449206123257327\\
877.286730031253	0.00044903429092509\\
877.561477096918	0.000448862534833449\\
877.836224162583	0.000448691348581532\\
878.110971228247	0.000448520244042938\\
878.385718293912	0.000448349387935134\\
878.660465359577	0.000448179103323212\\
878.935212425242	0.000448008738906131\\
879.209959490907	0.000447839107328358\\
879.484706556572	0.000447668905277313\\
879.759453622237	0.000447499448748424\\
880.034200687901	0.000447329906149417\\
880.308947753566	0.000447160949409535\\
880.583694819231	0.00044699240395357\\
880.858441884896	0.000446824270960149\\
881.133188950561	0.000446655567184522\\
881.407936016226	0.000446487771956203\\
881.68268308189	0.000446319735648697\\
881.957430147555	0.000446152282348927\\
882.23217721322	0.000445985078873115\\
882.506924278885	0.000445818135086418\\
882.78167134455	0.000445651111829544\\
883.056418410215	0.000445484350454804\\
883.331165475879	0.00044531833926665\\
883.605912541544	0.000445152260258167\\
883.880659607209	0.000444986927259169\\
884.155406672874	0.000444821518159719\\
884.430153738539	0.000444656199646337\\
884.704900804204	0.000444491307157336\\
884.979647869868	0.000444327165236028\\
885.254394935533	0.000444163283251046\\
885.529142001198	0.000443999164914738\\
885.803889066863	0.000443835144392951\\
886.078636132528	0.000443671880986471\\
886.353383198193	0.000443507896569009\\
886.628130263858	0.000443345329195638\\
886.902877329522	0.000443182040430615\\
887.177624395187	0.000443019181243977\\
887.452371460852	0.000442857081684533\\
887.727118526517	0.000442695579520699\\
888.001865592182	0.000442533681805406\\
888.276612657847	0.000442372215481413\\
888.551359723511	0.000442211018890145\\
888.826106789176	0.000442050083234978\\
889.100853854841	0.000441889581419102\\
889.375600920506	0.00044172951240566\\
889.650347986171	0.000441568726990327\\
889.925095051836	0.000441408382038347\\
890.1998421175	0.000441248805199521\\
890.474589183165	0.000441089499357118\\
890.74933624883	0.000440930789452595\\
891.024083314495	0.000440772017839171\\
891.29883038016	0.000440613524421901\\
891.573577445825	0.000440454971387399\\
891.84832451149	0.000440297355068778\\
892.123071577154	0.000440139517550227\\
892.397818642819	0.000439981796619893\\
892.672565708484	0.00043982434993275\\
892.947312774149	0.000439667840953078\\
893.222059839814	0.000439510627070486\\
893.496806905479	0.000439354025753746\\
893.771553971143	0.000439198036483812\\
894.046301036808	0.000439042658790824\\
894.321048102473	0.00043888722674608\\
894.595795168138	0.000438731911655068\\
894.870542233803	0.000438576712941576\\
895.145289299468	0.00043842229491414\\
895.420036365132	0.000438268159902197\\
895.694783430797	0.000438114143442274\\
895.969530496462	0.000437960408432992\\
896.244277562127	0.000437806792379028\\
896.519024627792	0.000437653460862562\\
896.793771693457	0.000437500747042862\\
897.068518759122	0.000437348143520337\\
897.343265824786	0.000437195502732724\\
897.618012890451	0.000437043637881457\\
897.892759956116	0.000436891732183503\\
898.167507021781	0.000436739783124671\\
898.442254087446	0.000436589113406097\\
898.71700115311	0.000436438238324275\\
898.991748218775	0.000436287321523204\\
899.26649528444	0.000436137027376045\\
899.541242350105	0.000435987191991718\\
899.81598941577	0.000435837979415462\\
900.090736481435	0.000435688565399681\\
900.3654835471	0.000435538785559062\\
900.640230612765	0.000435389799747132\\
900.914977678429	0.000435240946731196\\
901.189724744094	0.000435093212855607\\
901.464471809759	0.000434945283467966\\
901.739218875424	0.000434797977095775\\
902.013965941089	0.000434649816274491\\
902.288713006753	0.000434502941514704\\
902.563460072418	0.000434355707320911\\
902.838207138083	0.000434208932394619\\
903.112954203748	0.000434062791760067\\
903.387701269413	0.000433917107796944\\
903.662448335078	0.00043377204597061\\
903.937195400743	0.000433627114928473\\
904.211942466407	0.000433482482768809\\
904.486689532072	0.000433337661210081\\
904.761436597737	0.000433193475731625\\
905.036183663402	0.000433049264301894\\
905.310930729067	0.000432905517838558\\
905.585677794732	0.000432762077333355\\
905.860424860397	0.000432619426108204\\
906.135171926061	0.000432476593942744\\
906.409918991726	0.000432333733629733\\
906.684666057391	0.000432191506464037\\
906.959413123056	0.000432049419274946\\
907.234160188721	0.000431908133535729\\
907.508907254385	0.000431766662421655\\
907.78365432005	0.000431625827639383\\
908.058401385715	0.000431485140351973\\
908.33314845138	0.000431344925726525\\
908.607895517045	0.000431204691390231\\
908.88264258271	0.000431065087357939\\
909.157389648375	0.000430925792718657\\
909.432136714039	0.000430786310919263\\
909.706883779704	0.000430647136572021\\
909.981630845369	0.000430508123741849\\
910.256377911034	0.000430370078964868\\
910.531124976699	0.000430232010166143\\
910.805872042364	0.000430094579589034\\
911.080619108028	0.000429957621723425\\
911.355366173693	0.000429820320168773\\
911.630113239358	0.000429683667992187\\
911.904860305023	0.000429547327114796\\
912.179607370688	0.000429411138241943\\
912.454354436353	0.000429275260515144\\
912.729101502018	0.000429140521115036\\
913.003848567682	0.000429005275365619\\
913.278595633347	0.000428870343718074\\
913.553342699012	0.000428736059198993\\
913.828089764677	0.000428602257318045\\
914.102836830342	0.000428468277128055\\
914.377583896007	0.000428334782178436\\
914.652330961671	0.000428201761240186\\
914.927078027336	0.000428069059956969\\
915.201825093001	0.000427936669051327\\
915.476572158666	0.000427804103297438\\
915.751319224331	0.000427672513988453\\
916.026066289996	0.000427540095777175\\
916.30081335566	0.000427409156316521\\
916.575560421325	0.000427278701406048\\
916.85030748699	0.00042714807305463\\
917.125054552655	0.000427017599355189\\
917.39980161832	0.000426887289657599\\
917.674548683985	0.000426757794081368\\
917.949295749649	0.000426627956851523\\
918.224042815314	0.00042649910759529\\
918.498789880979	0.000426369765668658\\
918.773536946644	0.000426241738388775\\
919.048284012309	0.00042611370899285\\
919.323031077974	0.000425985999640921\\
919.597778143639	0.000425859265382055\\
919.872525209303	0.000425732199975549\\
920.147272274968	0.000425605453837478\\
920.422019340633	0.000425478871370151\\
920.696766406298	0.000425352769291903\\
920.971513471963	0.000425226506927169\\
921.246260537628	0.000425101064220097\\
921.521007603292	0.000424975942778941\\
921.795754668957	0.000424851144999024\\
922.070501734622	0.000424726676864116\\
922.345248800287	0.000424602859696875\\
922.619995865952	0.000424479198323771\\
922.894742931617	0.000424355208598931\\
923.169489997281	0.000424231879944024\\
923.444237062946	0.000424108714050986\\
923.718984128611	0.000423986048300731\\
923.993731194276	0.000423864197544031\\
924.268478259941	0.000423742015915578\\
924.543225325606	0.000423620657181991\\
924.81797239127	0.000423499468954032\\
925.092719456935	0.0004233786106174\\
925.3674665226	0.000423257912683127\\
925.642213588265	0.000423137720041876\\
925.91696065393	0.000423017852473783\\
926.191707719595	0.000422898148878848\\
926.46645478526	0.00042277909817173\\
926.741201850924	0.000422659393150349\\
927.015948916589	0.000422541008338043\\
927.290695982254	0.000422422788880518\\
927.565443047919	0.000422305065440566\\
927.840190113584	0.000422187667162625\\
928.114937179249	0.000422070596818301\\
928.389684244913	0.000421953856718638\\
928.664431310578	0.000421837278227084\\
928.939178376243	0.000421721035702369\\
929.213925441908	0.000421605278696907\\
929.488672507573	0.000421489529096837\\
929.763419573238	0.000421373944668534\\
930.038166638902	0.000421258367602947\\
930.312913704567	0.00042114378164619\\
930.587660770232	0.000421029204226606\\
930.862407835897	0.000420915456895822\\
931.137154901562	0.000420801057033881\\
931.411901967227	0.000420687987010527\\
931.686649032891	0.000420574915001745\\
931.961396098556	0.000420462506823486\\
932.236143164221	0.000420350269166384\\
932.510890229886	0.000420238358997775\\
932.785637295551	0.000420126780792097\\
933.060384361216	0.000420015374582922\\
933.335131426881	0.000419904456578556\\
933.609878492545	0.000419793711301676\\
933.88462555821	0.000419683298738441\\
934.159372623875	0.000419573545958354\\
934.43411968954	0.000419463799661372\\
934.708866755205	0.000419354385882299\\
934.98361382087	0.000419245638316749\\
935.258360886534	0.000419136560163993\\
935.533107952199	0.000419028150944069\\
935.807855017864	0.000418920072305243\\
936.082602083529	0.000418811838511508\\
936.357349149194	0.000418704768298398\\
936.632096214859	0.000418598031283686\\
936.906843280524	0.00041849195340088\\
937.181590346188	0.000418385553810855\\
937.456337411853	0.000418279154847208\\
937.731084477518	0.000418173430456125\\
938.005831543183	0.00041806819981046\\
938.280578608848	0.000417962644528344\\
938.555325674512	0.000417858743383363\\
938.830072740177	0.000417754517605853\\
939.104819805842	0.000417650137144051\\
939.379566871507	0.000417546256699938\\
939.654313937172	0.000417443362397623\\
939.929061002837	0.000417340311035674\\
940.203808068502	0.000417237590709951\\
940.478555134166	0.000417135361262599\\
940.753302199831	0.000417033625794362\\
941.028049265496	0.000416931730796196\\
941.302796331161	0.00041683033373747\\
941.577543396826	0.000416729593163391\\
941.852290462491	0.000416629018109994\\
942.127037528155	0.000416529601663708\\
942.40178459382	0.000416429205564187\\
942.676531659485	0.000416329960120772\\
942.95127872515	0.000416230882591106\\
943.226025790815	0.000416132130156859\\
943.50077285648	0.00041603354071355\\
943.775519922144	0.000415935284521429\\
944.050266987809	0.000415837525516141\\
944.325014053474	0.00041574025020051\\
944.599761119139	0.000415642981760492\\
944.874508184804	0.00041554603745085\\
945.149255250469	0.000415449745165569\\
945.424002316134	0.000415353452447939\\
945.698749381799	0.000415257487822413\\
945.973496447463	0.00041516201825875\\
946.248243513128	0.000415066711935608\\
946.522990578793	0.000414971899135559\\
946.797737644458	0.000414877574873687\\
947.072484710123	0.000414782923662743\\
947.347231775787	0.000414689257372436\\
947.621978841452	0.000414595915974413\\
947.896725907117	0.000414502741067949\\
948.171472972782	0.00041440956243014\\
948.446220038447	0.000414316549244124\\
948.720967104112	0.000414224520053417\\
948.995714169777	0.000414132003596486\\
949.270461235441	0.000414040306941801\\
949.545208301106	0.00041394892831497\\
949.819955366771	0.000413857883350387\\
950.094702432436	0.00041376683086076\\
950.369449498101	0.000413676433961756\\
950.644196563765	0.000413586197756254\\
950.91894362943	0.000413496450681734\\
951.193690695095	0.000413406539877674\\
951.46843776076	0.000413317286559274\\
951.743184826425	0.000413228356778662\\
952.01793189209	0.000413139918417343\\
952.292678957755	0.000413051151358297\\
952.567426023419	0.000412963871920402\\
952.842173089084	0.00041287642160941\\
953.116920154749	0.000412789454505265\\
953.391667220414	0.000412702642815319\\
953.666414286079	0.00041261582660976\\
953.941161351744	0.000412529992264137\\
954.215908417409	0.000412444157111984\\
954.490655483073	0.000412358802427106\\
954.765402548738	0.000412273606466469\\
955.040149614403	0.00041218874076769\\
955.314896680068	0.000412104360318021\\
955.589643745733	0.000412019804570578\\
955.864390811398	0.000411936066673901\\
956.139137877062	0.000411852475695836\\
956.413884942727	0.000411769533814129\\
956.688632008392	0.000411686737794991\\
956.963379074057	0.000411604096576131\\
957.238126139722	0.000411521113095039\\
957.512873205387	0.000411439103270507\\
957.787620271051	0.000411357901182065\\
958.062367336716	0.000411276030422458\\
958.337114402381	0.000411195294330549\\
958.611861468046	0.000411114371703252\\
958.886608533711	0.00041103377013413\\
959.161355599376	0.00041095364029402\\
959.43610266504	0.00041087415254371\\
959.710849730705	0.000410794483964337\\
959.98559679637	0.000410714962557833\\
960.260343862035	0.000410635925088826\\
960.5350909277	0.000410557360513848\\
960.809837993365	0.000410479268858698\\
961.08458505903	0.000410401322192475\\
961.359332124694	0.000410323848127376\\
961.634079190359	0.000410246188630386\\
961.908826256024	0.000410169331658102\\
962.183573321689	0.000410092782171628\\
962.458320387354	0.000410016369870371\\
962.733067453019	0.000409940592461002\\
963.007814518683	0.000409864789860789\\
963.282561584348	0.00040978896943164\\
963.557308650013	0.000409713788737432\\
963.832055715678	0.000409638917143333\\
964.106802781343	0.00040956419075651\\
964.381549847008	0.00040949026385074\\
964.656296912672	0.00040941581377228\\
964.931043978337	0.000409342169766923\\
965.205791044002	0.000409268824493554\\
965.480538109667	0.000409195775435637\\
965.755285175332	0.000409123024035774\\
966.030032240997	0.000409050412081362\\
966.304779306662	0.000408978262737625\\
966.579526372326	0.000408906417376193\\
966.854273437991	0.000408834870957785\\
967.129020503656	0.000408764113839687\\
967.403767569321	0.000408692666250861\\
967.678514634986	0.000408621845342896\\
967.953261700651	0.000408550831414678\\
968.228008766315	0.000408480775873477\\
968.50275583198	0.000408411017038528\\
968.777502897645	0.000408341384436757\\
969.05224996331	0.000408272370774521\\
969.326997028975	0.00040820316496267\\
969.60174409464	0.00040813474361791\\
969.876491160304	0.000408066115615518\\
970.151238225969	0.000407997944017862\\
970.425985291634	0.000407930389678099\\
970.700732357299	0.000407862960355551\\
970.975479422964	0.000407795980223406\\
971.250226488629	0.000407728959210249\\
971.524973554293	0.000407662229019452\\
971.799720619958	0.000407595792757383\\
972.074467685623	0.000407530297646694\\
972.349214751288	0.000407464758862812\\
972.623961816953	0.00040739885029061\\
972.898708882618	0.000407333726589866\\
973.173455948283	0.000407269220763414\\
973.448203013947	0.000407204996260811\\
973.722950079612	0.00040714039657288\\
973.997697145277	0.000407076576987397\\
974.272444210942	0.000407013205012752\\
974.547191276607	0.000406949293218681\\
974.821938342272	0.000406885832586697\\
975.096685407936	0.000406823474569373\\
975.371432473601	0.000406760899817061\\
975.646179539266	0.000406698931308501\\
975.920926604931	0.000406636740872303\\
976.195673670596	0.000406574996577353\\
976.470420736261	0.000406513195292058\\
976.745167801925	0.00040645199265323\\
977.01991486759	0.000406391241802325\\
977.294661933255	0.000406330594459592\\
977.56940899892	0.000406269722266362\\
977.844156064585	0.000406210115762022\\
978.11890313025	0.000406150770786362\\
978.393650195914	0.00040609137690873\\
978.668397261579	0.000406031917431581\\
978.943144327244	0.00040597288944653\\
979.217891392909	0.000405914126406612\\
979.492638458574	0.000405855792139809\\
979.767385524239	0.000405797887791707\\
980.042132589904	0.00040574008350734\\
980.316879655568	0.000405682379273694\\
980.591626721233	0.000405625101777154\\
980.866373786898	0.000405567927037936\\
981.141120852563	0.000405510851365166\\
981.415867918228	0.00040545420521241\\
981.690614983893	0.000405398145899205\\
981.965362049557	0.00040534185291666\\
982.240109115222	0.00040528647550205\\
982.514856180887	0.000405230863727923\\
982.789603246552	0.000405175999260301\\
983.064350312217	0.000405120733403275\\
983.339097377882	0.000405066203760099\\
983.613844443546	0.000405011454759691\\
983.888591509211	0.000404957124557159\\
984.163338574876	0.000404902879036002\\
984.438085640541	0.000404849215426353\\
984.712832706206	0.000404795305896465\\
984.987579771871	0.000404741978925107\\
985.262326837536	0.000404688899993959\\
985.5370739032	0.000404636564276634\\
985.811820968865	0.000404583499215297\\
986.08656803453	0.000404531671112064\\
986.361315100195	0.000404479928153038\\
986.63606216586	0.000404427778373223\\
986.910809231525	0.000404376532011284\\
987.185556297189	0.000404325202413473\\
987.460303362854	0.00040427493665926\\
987.735050428519	0.000404224259239714\\
988.009797494184	0.000404173984347624\\
988.284544559849	0.000404123951932043\\
988.559291625514	0.000404073827605223\\
988.834038691178	0.000404023783584466\\
989.108785756843	0.000403974635419166\\
989.383532822508	0.000403925067724184\\
989.658279888173	0.000403875249042858\\
989.933026953838	0.000403826000842103\\
990.207774019503	0.000403777166232415\\
990.482521085167	0.000403728893956101\\
990.757268150832	0.000403680848340211\\
991.032015216497	0.00040363303847913\\
991.306762282162	0.000403585621131381\\
991.581509347827	0.000403538269420409\\
991.856256413492	0.000403491481982106\\
992.131003479156	0.000403444760521473\\
992.405750544821	0.000403397940361177\\
992.680497610486	0.000403351182506306\\
992.955244676151	0.000403304656393674\\
993.229991741816	0.000403258523743285\\
993.504738807481	0.000403212292204773\\
993.779485873146	0.00040316711271797\\
994.054232938811	0.000403121501299616\\
994.328980004475	0.000403076276364219\\
994.60372707014	0.000403031435228652\\
994.878474135805	0.000402986819598824\\
995.15322120147	0.000402942256518716\\
995.427968267135	0.000402897257309\\
995.702715332799	0.00040285363352458\\
995.977462398464	0.000402809733735358\\
996.252209464129	0.00040276589097517\\
996.526956529794	0.000402722103594497\\
996.801703595459	0.000402678703304741\\
997.076450661124	0.000402635682724059\\
997.351197726789	0.000402593371176787\\
997.625944792453	0.000402550945596461\\
997.900691858118	0.000402508409817632\\
998.175438923783	0.000402466909457265\\
998.450185989448	0.000402425952894163\\
998.724933055113	0.000402383894247172\\
998.999680120777	0.000402342541691912\\
999.274427186442	0.000402301400040366\\
999.549174252107	0.000402260623722068\\
999.823921317772	0.000402219723870107\\
1000.09866838344	0.000402179196914303\\
1000.3734154491	0.000402139032750789\\
1000.64816251477	0.000402098910434762\\
1000.92290958043	0.000402058821967661\\
1001.1976566461	0.000402019425752313\\
1001.47240371176	0.000401979737437388\\
1001.74715077743	0.000401940576437811\\
1002.02189784309	0.000401901119457812\\
1002.29664490876	0.000401861865312576\\
1002.57139197442	0.000401823141930166\\
1002.84613904009	0.000401784776091079\\
1003.12088610575	0.000401745957315731\\
1003.39563317142	0.000401707661905844\\
1003.67038023708	0.000401669720987343\\
1003.94512730274	0.000401632292327398\\
1004.21987436841	0.000401594737254547\\
1004.49462143407	0.000401557369169994\\
1004.76936849974	0.000401520024131608\\
1005.0441155654	0.000401482866100076\\
1005.31886263107	0.000401445402006108\\
1005.59360969673	0.000401408781939456\\
1005.8683567624	0.000401372189268378\\
1006.14310382806	0.000401335944403567\\
1006.41785089373	0.000401299882316469\\
1006.69259795939	0.000401264003297917\\
1006.96734502506	0.00040122798330014\\
1007.24209209072	0.000401192309027221\\
1007.51683915639	0.000401157145842587\\
1007.79158622205	0.000401121508235435\\
1008.06633328772	0.000401086380405663\\
1008.34108035338	0.000401051598472439\\
1008.61582741905	0.000401016668831326\\
1008.89057448471	0.00040098224940664\\
1009.16532155038	0.000400948003627854\\
1009.44006861604	0.000400914098882435\\
1009.71481568171	0.00040088004337879\\
1009.98956274737	0.000400846322262284\\
1010.26430981304	0.000400812285800915\\
1010.5390568787	0.000400778427631406\\
1010.81380394437	0.000400745389413801\\
1011.08855101003	0.000400712034734296\\
1011.3632980757	0.000400678848168356\\
1011.63804514136	0.000400645830596736\\
1011.91279220703	0.000400612813299454\\
1012.18753927269	0.000400580137022901\\
1012.46228633835	0.00040054795201402\\
1012.73703340402	0.000400515767776571\\
1013.01178046968	0.00040048424097953\\
1013.28652753535	0.000400451887294445\\
1013.56127460101	0.000400420033640713\\
1013.83602166668	0.000400388839520895\\
1014.11076873234	0.000400357639854967\\
1014.38551579801	0.000400326109555656\\
1014.66026286367	0.000400295233656745\\
1014.93500992934	0.000400264192504836\\
1015.209756995	0.000400233300367282\\
1015.48450406067	0.000400202575176118\\
1015.75925112633	0.000400172168768035\\
1016.033998192	0.000400142079251212\\
1016.30874525766	0.000400111978815678\\
1016.58349232333	0.000400082202924713\\
1016.85823938899	0.000400052243658497\\
1017.13298645466	0.000400022277763528\\
1017.40773352032	0.000399992629893933\\
1017.68248058599	0.000399963134266888\\
1017.95722765165	0.000399933957586251\\
1018.23197471732	0.000399905091535943\\
1018.50672178298	0.000399875879333818\\
1018.78146884865	0.000399847134924586\\
1019.05621591431	0.000399818709027218\\
1019.33096297998	0.000399789931016673\\
1019.60571004564	0.000399761788081399\\
1019.88045711131	0.000399733781953882\\
1020.15520417697	0.000399705588782759\\
1020.42995124264	0.000399677869075119\\
1020.7046983083	0.000399650456687688\\
1020.97944537396	0.000399623018107372\\
1021.25419243963	0.000399595878500619\\
1021.52893950529	0.000399568552331045\\
1021.80368657096	0.000399541523087896\\
1022.07843363662	0.000399514628475876\\
1022.35318070229	0.000399488360168969\\
1022.62792776795	0.000399461568666514\\
1022.90267483362	0.000399435071931804\\
1023.17742189928	0.000399408709897716\\
1023.45216896495	0.000399382470702156\\
1023.72691603061	0.000399355871524538\\
1024.00166309628	0.000399329402312897\\
1024.27641016194	0.000399303561833776\\
1024.55115722761	0.000399278176854185\\
1024.82590429327	0.000399252583983236\\
1025.10065135894	0.00039922695484071\\
1025.3753984246	0.000399201777047086\\
1025.65014549027	0.000399176393000877\\
1025.92489255593	0.000399151131081997\\
1026.1996396216	0.000399125671933636\\
1026.47438668726	0.000399100664791182\\
1026.74913375293	0.000399075786072744\\
1027.02388081859	0.000399051022114994\\
1027.29862788426	0.000399026544779327\\
1027.57337494992	0.000399002180891049\\
1027.84812201559	0.000398978102181328\\
1028.12286908125	0.000398953969783662\\
1028.39761614692	0.000398929790092377\\
1028.67236321258	0.000398905884625461\\
1028.94711027825	0.000398881934349007\\
1029.22185734391	0.000398858422261783\\
1029.49660440957	0.000398834205277077\\
1029.77135147524	0.000398810598560245\\
1030.0460985409	0.000398787104085632\\
1030.32084560657	0.000398763724313051\\
1030.59559267223	0.00039874061470915\\
1030.8703397379	0.000398718265802333\\
1031.14508680356	0.000398695208906101\\
1031.41983386923	0.000398672420123472\\
1031.69458093489	0.000398649568536304\\
1031.96932800056	0.000398627311890773\\
1032.24407506622	0.000398604494645405\\
1032.51882213189	0.000398581788514221\\
1032.79356919755	0.000398559349425913\\
1033.06831626322	0.000398537172083286\\
1033.34306332888	0.000398515257298828\\
1033.61781039455	0.000398492624086036\\
1033.89255746021	0.000398470425379653\\
1034.16730452588	0.000398448654143672\\
1034.44205159154	0.000398426655605716\\
1034.71679865721	0.000398404589192753\\
1034.99154572287	0.000398383113436392\\
1035.26629278854	0.000398361239000001\\
1035.5410398542	0.000398340122334372\\
1035.81578691987	0.000398318767988909\\
1036.09053398553	0.000398298161917657\\
1036.3652810512	0.000398276666376058\\
1036.64002811686	0.000398255752934829\\
1036.91477518253	0.000398234759083748\\
1037.18952224819	0.000398213690751029\\
1037.46426931386	0.000398192709798308\\
1037.73901637952	0.000398171822565167\\
1038.01376344518	0.000398151349442963\\
1038.28851051085	0.000398130963423653\\
1038.56325757651	0.000398110820426083\\
1038.83800464218	0.000398090596989079\\
1039.11275170784	0.000398070129926076\\
1039.38749877351	0.000398049913059993\\
1039.66224583917	0.00039802977068277\\
1039.93699290484	0.000398009713563417\\
1040.2117399705	0.000397989569310211\\
1040.48648703617	0.0003979700029191\\
1040.76123410183	0.000397950356308403\\
1041.0359811675	0.000397931124552126\\
1041.31072823316	0.000397911317078131\\
1041.58547529883	0.000397892245264752\\
1041.86022236449	0.000397872913985535\\
1042.13496943016	0.000397852841145418\\
1042.40971649582	0.000397833501807386\\
1042.68446356149	0.000397814560463461\\
1042.95921062715	0.000397795364193713\\
1043.23395769282	0.000397776568903875\\
1043.50870475848	0.000397757515532402\\
1043.78345182415	0.000397738699526981\\
1044.05819888981	0.000397720115638289\\
1044.33294595548	0.000397701272794786\\
1044.60769302114	0.000397683155975652\\
1044.88244008681	0.000397664609675895\\
1045.15718715247	0.000397645966446616\\
1045.43193421814	0.000397627552502345\\
1045.7066812838	0.000397609206914321\\
1045.98142834947	0.000397590922896024\\
1046.25617541513	0.000397572542812417\\
1046.53092248079	0.000397554548491802\\
1046.80566954646	0.000397536454529241\\
1047.08041661212	0.000397518585629139\\
1047.35516367779	0.000397500780203442\\
1047.62991074345	0.000397482541519376\\
1047.90465780912	0.000397464855394146\\
1048.17940487478	0.000397447225863342\\
1048.45415194045	0.000397429651497683\\
1048.72889900611	0.000397411968433807\\
1049.00364607178	0.000397394328991968\\
1049.27839313744	0.000397376586381875\\
1049.55314020311	0.000397359058872811\\
1049.82788726877	0.000397341579532025\\
1050.10263433444	0.0003973239799919\\
1050.3773814001	0.000397306274162992\\
1050.65212846577	0.000397288934035397\\
1050.92687553143	0.000397270834063073\\
1051.2016225971	0.000397253446866304\\
1051.47636966276	0.000397236269407244\\
1051.75111672843	0.000397219134770499\\
1052.02586379409	0.000397202042234374\\
1052.30061085976	0.000397184828372944\\
1052.57535792542	0.000397167658917229\\
1052.85010499109	0.000397150538258578\\
1053.12485205675	0.000397133784046717\\
1053.39959912242	0.000397116754638693\\
1053.67434618808	0.000397099599448239\\
1053.94909325375	0.000397081996059924\\
1054.22384031941	0.000397065086757343\\
1054.49858738508	0.000397048390963285\\
1054.77333445074	0.000397031729815032\\
1055.04808151641	0.000397014772968681\\
1055.32282858207	0.000396998184905476\\
1055.59757564773	0.000396981636744148\\
1055.8723227134	0.000396965124018739\\
1056.14706977906	0.00039694848916443\\
1056.42181684473	0.000396932059430055\\
1056.69656391039	0.000396915502071596\\
1056.97131097606	0.000396899145482159\\
1057.24605804172	0.00039688265870556\\
1057.52080510739	0.000396866210815528\\
1057.79555217305	0.000396850124174431\\
1058.07029923872	0.000396834070469409\\
1058.34504630438	0.000396817387650141\\
1058.61979337005	0.000396801068912331\\
1058.89454043571	0.000396785108752309\\
1059.16928750138	0.000396769014568035\\
1059.44403456704	0.000396752943073102\\
1059.71878163271	0.000396736894175559\\
1059.99352869837	0.000396720869158327\\
1060.26827576404	0.000396704714619059\\
1060.5430228297	0.000396687926823961\\
1060.81776989537	0.000396671825150143\\
1061.09251696103	0.000396656071922656\\
1061.3672640267	0.000396640336910702\\
1061.64201109236	0.000396624132213673\\
1061.91675815803	0.000396608112341392\\
1062.19150522369	0.000396592279314609\\
1062.46625228936	0.000396576300005901\\
1062.74099935502	0.00039656034735303\\
1063.01574642069	0.000396544575715748\\
1063.29049348635	0.000396529317672265\\
1063.56524055202	0.000396513416212851\\
1063.83998761768	0.00039649802574946\\
1064.11473468334	0.000396482324171081\\
1064.38948174901	0.000396466799919409\\
1064.66422881467	0.000396451290261239\\
1064.93897588034	0.000396436290954674\\
1065.213722946	0.000396420324941109\\
1065.48847001167	0.000396404870547197\\
1065.76321707733	0.000396389592134213\\
1066.037964143	0.000396373675679323\\
1066.31271120866	0.000396358097456456\\
1066.58745827433	0.000396342687655244\\
1066.86220533999	0.000396326794819762\\
1067.13695240566	0.000396311238243393\\
1067.41169947132	0.00039629536594581\\
1067.68644653699	0.000396279999139838\\
1067.96119360265	0.000396264811933996\\
1068.23594066832	0.000396249629608916\\
1068.51068773398	0.000396234124701252\\
1068.78543479965	0.000396218952786523\\
1069.06018186531	0.000396203292307789\\
1069.33492893098	0.000396187307922681\\
1069.60967599664	0.000396172153884195\\
1069.88442306231	0.000396156021399793\\
1070.15917012797	0.000396140387313098\\
1070.43391719364	0.000396124917543932\\
1070.7086642593	0.00039610945114437\\
1070.98341132497	0.000396093821396561\\
1071.25815839063	0.000396078031126154\\
1071.5329054563	0.000396062250279454\\
1071.80765252196	0.00039604679422851\\
1072.08239958763	0.000396031341935885\\
1072.35714665329	0.000396015883238671\\
1072.63189371895	0.000395999928814089\\
1072.90664078462	0.000395984631828515\\
1073.18138785028	0.000395968840462328\\
1073.45613491595	0.000395953045491433\\
1073.73088198161	0.000395938073890616\\
1074.00562904728	0.0003959226037282\\
1074.28037611294	0.00039590762074045\\
1074.55512317861	0.000395892302396403\\
1074.82987024427	0.000395876649520735\\
1075.10461730994	0.000395860812936577\\
1075.3793643756	0.00039584514414594\\
1075.65411144127	0.000395829463342175\\
1075.92885850693	0.000395813440070577\\
1076.2036055726	0.000395797572755477\\
1076.47835263826	0.000395781695021527\\
1076.75309970393	0.000395765642559223\\
1077.02784676959	0.000395749915158274\\
1077.30259383526	0.000395734501200508\\
1077.57734090092	0.000395718911740438\\
1077.85208796659	0.000395703148599677\\
1078.12683503225	0.000395687206268381\\
1078.40158209792	0.000395671094630435\\
1078.67632916358	0.000395655463984712\\
1078.95107622925	0.000395640146873409\\
1079.22582329491	0.000395624489193608\\
1079.50057036058	0.00039560881042654\\
1079.77531742624	0.000395593116252546\\
1080.05006449191	0.000395577728834508\\
1080.32481155757	0.000395561995837521\\
1080.59955862324	0.00039554590991215\\
1080.8743056889	0.000395530136894661\\
1081.14905275456	0.000395514501706257\\
1081.42379982023	0.00039549835446794\\
1081.69854688589	0.000395482513388518\\
1081.97329395156	0.000395466644842116\\
1082.24804101722	0.000395450753194719\\
1082.52278808289	0.00039543500085441\\
1082.79753514855	0.000395418899140569\\
1083.07228221422	0.00039540293295028\\
1083.34702927988	0.000395386942691165\\
1083.62177634555	0.00039537109176904\\
1083.89652341121	0.000395354882001027\\
1084.17127047688	0.000395338813354639\\
1084.44601754254	0.000395322554038564\\
1084.72076460821	0.000395306928565702\\
1084.99551167387	0.000395290947531802\\
1085.27025873954	0.000395275279342\\
1085.5450058052	0.000395259251908195\\
1085.81975287087	0.000395242869390911\\
1086.09449993653	0.000395226453880796\\
1086.3692470022	0.000395210169195299\\
1086.64399406786	0.000395193851339547\\
1086.91874113353	0.000395177177629646\\
1087.19348819919	0.000395160307083067\\
1087.46823526486	0.000395143903821018\\
1087.74298233052	0.000395126977812628\\
1088.01772939619	0.000395110190475897\\
1088.29247646185	0.000395094026040958\\
1088.56722352752	0.000395077660667415\\
1088.84197059318	0.000395061091250026\\
1089.11671765885	0.000395043998797124\\
1089.39146472451	0.000395027036352534\\
1089.66621179018	0.000395010199670086\\
1089.94095885584	0.000394993333258974\\
1090.2157059215	0.000394976264819535\\
1090.49045298717	0.000394959649393208\\
1090.76520005283	0.000394942501788673\\
1091.0399471185	0.000394925315959917\\
1091.31469418416	0.000394908261387988\\
1091.58944124983	0.000394891003546098\\
1091.86418831549	0.000394874194530576\\
1092.13893538116	0.000394857016777181\\
1092.41368244682	0.000394839800045487\\
1092.68842951249	0.000394823041188586\\
1092.96317657815	0.000394806242902099\\
1093.23792364382	0.000394788910390687\\
1093.51267070948	0.000394771537690074\\
1093.78741777515	0.000394754125889165\\
1094.06216484081	0.000394737163311635\\
1094.33691190648	0.000394719824983705\\
1094.61165897214	0.000394701956310055\\
1094.88640603781	0.000394684045392835\\
1095.16115310347	0.000394666253873964\\
1095.43590016914	0.00039464858804165\\
1095.7106472348	0.00039463103542073\\
1095.98539430047	0.00039461343551978\\
1096.26014136613	0.000394595790235492\\
1096.5348884318	0.00039457825771506\\
1096.80963549746	0.000394560351717535\\
1097.08438256313	0.000394542559623986\\
1097.35912962879	0.000394524559062576\\
1097.63387669446	0.000394507001860714\\
1097.90862376012	0.000394489224077546\\
1098.18337082579	0.00039447091168991\\
1098.45811789145	0.000394452875995065\\
1098.73286495711	0.000394434621214709\\
1099.00761202278	0.000394416640536555\\
1099.28235908844	0.000394397960826585\\
1099.55710615411	0.000394379881988902\\
1099.83185321977	0.000394361426961323\\
1100.10660028544	0.000394342589780522\\
1100.3813473511	0.000394324357469683\\
1100.65609441677	0.000394306066112953\\
1100.93084148243	0.00039428756207822\\
1101.2055885481	0.000394268835899813\\
1101.48033561376	0.000394250053992213\\
1101.75508267943	0.000394231542143249\\
1102.02982974509	0.000394212317928147\\
1102.30457681076	0.000394193372868139\\
1102.57932387642	0.000394174537148467\\
1102.85407094209	0.00039415581687496\\
1103.12881800775	0.000394137211392442\\
1103.40356507342	0.000394118219073999\\
};
\addplot [color=mycolor1,solid,line width=2.0pt,forget plot]
  table[row sep=crcr]{%
1103.40356507342	0.000394118219073999\\
1103.67831213908	0.000394098841480589\\
1103.95305920475	0.000394079739068018\\
1104.22780627041	0.000394060582179599\\
1104.50255333608	0.000394041370238024\\
1104.77730040174	0.000394022098452867\\
1105.05204746741	0.000394002934281935\\
1105.32679453307	0.00039398354642139\\
1105.60154159874	0.00039396393583236\\
1105.8762886644	0.000393943937823558\\
1106.15103573007	0.000393925038673686\\
1106.42578279573	0.000393905585700028\\
1106.7005298614	0.00039388558056631\\
1106.97527692706	0.000393865679242116\\
1107.25002399272	0.000393845561706935\\
1107.52477105839	0.000393825545749593\\
1107.79951812405	0.000393805630668857\\
1108.07426518972	0.000393785649200502\\
1108.34901225538	0.000393765601206747\\
1108.62375932105	0.000393745167616317\\
1108.89850638671	0.00039372515594381\\
1109.17325345238	0.000393704763079757\\
1109.44800051804	0.000393684303573577\\
1109.72274758371	0.000393663777521042\\
1109.99749464937	0.000393643185984495\\
1110.27224171504	0.00039362236552337\\
1110.5469887807	0.000393602140879766\\
1110.82173584637	0.0003935818518706\\
1111.09648291203	0.000393560999543436\\
1111.3712299777	0.000393540087995774\\
1111.64597704336	0.000393518945733767\\
1111.92072410903	0.000393498065248117\\
1112.19547117469	0.000393477282568025\\
1112.47021824036	0.00039345610430319\\
1112.74496530602	0.000393435355460145\\
1113.01971237169	0.000393414376556857\\
1113.29445943735	0.000393393164711111\\
1113.56920650302	0.000393371719392301\\
1113.84395356868	0.000393350212856744\\
1114.11870063435	0.000393328964284878\\
1114.39344770001	0.000393307480114646\\
1114.66819476568	0.000393286091011004\\
1114.94294183134	0.000393264306133419\\
1115.21768889701	0.000393243111066892\\
1115.49243596267	0.000393221688148329\\
1115.76718302834	0.000393199864521302\\
1116.041930094	0.000393178134542203\\
1116.31667715966	0.00039315617579099\\
1116.59142422533	0.000393133978919017\\
1116.86617129099	0.000393111702321414\\
1117.14091835666	0.000393089366413415\\
1117.41566542232	0.000393067122249726\\
1117.69041248799	0.000393044639325563\\
1117.96515955365	0.000393022084582654\\
1118.23990661932	0.000392999624730595\\
1118.51465368498	0.000392977421943462\\
1118.78940075065	0.000392954497812025\\
1119.06414781631	0.000392931337460094\\
1119.33889488198	0.000392908433538155\\
1119.61364194764	0.000392885626816965\\
1119.88838901331	0.000392862415581576\\
1120.16313607897	0.000392839624144029\\
1120.43788314464	0.000392816431420437\\
1120.7126302103	0.00039279349711605\\
1120.98737727597	0.000392770319406867\\
1121.26212434163	0.000392746909837833\\
1121.5368714073	0.000392723750232757\\
1121.81161847296	0.000392700353587563\\
1122.08636553863	0.000392677209910188\\
1122.36111260429	0.000392653995300893\\
1122.63585966996	0.000392630545075098\\
1122.91060673562	0.000392606691181554\\
1123.18535380129	0.000392582768700795\\
1123.46010086695	0.000392559101214302\\
1123.73484793262	0.000392535033351397\\
1124.00959499828	0.000392510734967845\\
1124.28434206395	0.000392486852208338\\
1124.55908912961	0.000392462729879393\\
1124.83383619527	0.00039243836794055\\
1125.10858326094	0.000392413938650427\\
1125.3833303266	0.000392389605119111\\
1125.65807739227	0.000392365201131488\\
1125.93282445793	0.000392340889439499\\
1126.2075715236	0.000392316504096444\\
1126.48231858926	0.000392291882074195\\
1126.75706565493	0.000392267184904772\\
1127.03181272059	0.000392242252070491\\
1127.30655978626	0.000392217406491814\\
1127.58130685192	0.000392192645602062\\
1127.85605391759	0.000392167805960329\\
1128.13080098325	0.000392142720485333\\
1128.40554804892	0.000392117077990583\\
1128.68029511458	0.000392091524227879\\
1128.95504218025	0.000392066058971923\\
1129.22978924591	0.000392040192406627\\
1129.50453631158	0.000392014420502571\\
1129.77928337724	0.000391989231945737\\
1130.05403044291	0.000391963310881852\\
1130.32877750857	0.000391937158464845\\
1130.60352457424	0.000391911584994436\\
1130.8782716399	0.000391885605565237\\
1131.15301870557	0.000391860372995834\\
1131.42776577123	0.000391834241480044\\
1131.7025128369	0.000391808202433424\\
1131.97725990256	0.000391782076172599\\
1132.25200696823	0.00039175571658631\\
1132.52675403389	0.000391729114404054\\
1132.80150109956	0.000391702599496944\\
1133.07624816522	0.000391676006914288\\
1133.35099523088	0.000391649998157096\\
1133.62574229655	0.000391623256333468\\
1133.90048936221	0.00039159676160968\\
1134.17523642788	0.000391569706195058\\
1134.44998349354	0.000391542573510473\\
1134.72473055921	0.000391515365002173\\
1134.99947762487	0.000391488407749633\\
1135.27422469054	0.000391461050305629\\
1135.5489717562	0.000391433615207383\\
1135.82371882187	0.000391406596395942\\
1136.09846588753	0.000391379012166768\\
1136.3732129532	0.000391351684676679\\
1136.64796001886	0.000391324445769401\\
1136.92270708453	0.000391296799310475\\
1137.19745415019	0.000391269083749155\\
1137.47220121586	0.000391241782870952\\
1137.74694828152	0.000391214074240916\\
1138.02169534719	0.000391186294476613\\
1138.29644241285	0.000391158108378619\\
1138.57118947852	0.00039113018045246\\
1138.84593654418	0.000391102010361029\\
1139.12068360985	0.000391073602636594\\
1139.39543067551	0.000391045609588709\\
1139.67017774118	0.000391016882052592\\
1139.94492480684	0.000390988086492574\\
1140.21967187251	0.000390960036508481\\
1140.49441893817	0.000390931581319637\\
1140.76916600384	0.000390903210710769\\
1141.0439130695	0.000390874431965562\\
1141.31866013517	0.000390845257454867\\
1141.59340720083	0.000390816497119599\\
1141.86815426649	0.000390787662985429\\
1142.14290133216	0.000390758748593398\\
1142.41764839782	0.000390729597894627\\
1142.69239546349	0.00039070036634469\\
1142.96714252915	0.000390671063859975\\
1143.24188959482	0.000390641846207538\\
1143.51663666048	0.000390612389629062\\
1143.79138372615	0.00039058284069866\\
1144.06613079181	0.00039055323459334\\
1144.34087785748	0.000390523386423213\\
1144.61562492314	0.000390493790212218\\
1144.89037198881	0.000390463459040828\\
1145.16511905447	0.00039043371219697\\
1145.43986612014	0.000390403722126486\\
1145.7146131858	0.000390373326075706\\
1145.98936025147	0.000390343677369518\\
1146.26410731713	0.000390312972338794\\
1146.5388543828	0.000390282846515065\\
1146.81360144846	0.000390252654020659\\
1147.08834851413	0.00039022221822354\\
1147.36309557979	0.000390192035605181\\
1147.63784264546	0.000390161450223533\\
1147.91258971112	0.000390131122302124\\
1148.18733677679	0.00039010023511382\\
1148.46208384245	0.000390069436909161\\
1148.73683090812	0.000390038890606448\\
1149.01157797378	0.000390007778546616\\
1149.28632503945	0.000389976923570448\\
1149.56107210511	0.000389946161309607\\
1149.83581917078	0.000389915004241547\\
1150.11056623644	0.000389883773736017\\
1150.38531330211	0.0003898524772164\\
1150.66006036777	0.000389821279723488\\
1150.93480743343	0.000389790175000696\\
1151.2095544991	0.000389758508921265\\
1151.48430156476	0.000389726933060514\\
1151.75904863043	0.000389695288351798\\
1152.03379569609	0.000389663900754208\\
1152.30854276176	0.000389632276546694\\
1152.58328982742	0.000389600913326368\\
1152.85803689309	0.000389568988195939\\
1153.13278395875	0.000389537488755263\\
1153.40753102442	0.000389505424585726\\
1153.68227809008	0.000389473458335825\\
1153.95702515575	0.000389441584670547\\
1154.23177222141	0.000389409308799427\\
1154.50651928708	0.000389376967126669\\
1154.78126635274	0.000389344883395056\\
1155.05601341841	0.000389312729685925\\
1155.33076048407	0.000389280177011197\\
1155.60550754974	0.000389248215320886\\
1155.8802546154	0.000389215527828136\\
1156.15500168107	0.000389182938350341\\
1156.42974874673	0.000389150117851907\\
1156.7044958124	0.000389117557516211\\
1156.97924287806	0.000389085088200007\\
1157.25398994373	0.000389052224613315\\
1157.52873700939	0.000389019452927081\\
1157.80348407506	0.000388986612573539\\
1158.07823114072	0.000388953535901094\\
1158.35297820639	0.00038892023541608\\
1158.62772527205	0.000388887029380342\\
1158.90247233771	0.000388853746799745\\
1159.17721940338	0.000388820239736389\\
1159.45196646904	0.000388786339435945\\
1159.72671353471	0.000388752860831588\\
1160.00146060037	0.000388718826345622\\
1160.27620766604	0.000388685222885607\\
1160.5509547317	0.000388652045160264\\
1160.82570179737	0.000388618310930854\\
1161.10044886303	0.000388584012447093\\
1161.3751959287	0.000388550159202208\\
1161.64994299436	0.00038851623750156\\
1161.92469006003	0.000388481920018875\\
1162.19943712569	0.000388447702676886\\
1162.47418419136	0.000388412930286565\\
1162.74893125702	0.00038837842555456\\
1163.02367832269	0.000388344350797432\\
1163.29842538835	0.000388309886808988\\
1163.57317245402	0.000388275527687776\\
1163.84791951968	0.000388241273160331\\
1164.12266658535	0.000388207120396638\\
1164.39741365101	0.000388172578041472\\
1164.67216071668	0.00038813830404123\\
1164.94690778234	0.000388103639928644\\
1165.22165484801	0.000388068747556141\\
1165.49640191367	0.000388033636992781\\
1165.77114897934	0.000387998794620808\\
1166.045896045	0.000387964056850394\\
1166.32064311067	0.000387928934441087\\
1166.59539017633	0.00038789425055622\\
1166.870137242	0.000387859344013709\\
1167.14488430766	0.000387823883962594\\
1167.41963137333	0.000387788535800555\\
1167.69437843899	0.000387753463211026\\
1167.96912550465	0.000387718006073598\\
1168.24387257032	0.000387682820866782\\
1168.51861963598	0.000387647749663495\\
1168.79336670165	0.000387612460711089\\
1169.06811376731	0.000387576624140929\\
1169.34286083298	0.00038754138445436\\
1169.61760789864	0.000387505271447883\\
1169.89235496431	0.000387469598907444\\
1170.16710202997	0.000387433704729899\\
1170.44184909564	0.000387397922442799\\
1170.7165961613	0.000387362748684907\\
1170.99134322697	0.000387326535991387\\
1171.26609029263	0.00038729092807246\\
1171.5408373583	0.000387254611575595\\
1171.81558442396	0.000387218898748254\\
1172.09033148963	0.000387182479496616\\
1172.36507855529	0.000387146336487743\\
1172.63982562096	0.00038711014152992\\
1172.91457268662	0.000387074059817918\\
1173.18931975229	0.000387038090115198\\
1173.46406681795	0.000387001741034498\\
1173.73881388362	0.000386965181308816\\
1174.01356094928	0.000386928738742497\\
1174.28830801495	0.000386892413954386\\
1174.56305508061	0.000386856042173022\\
1174.83780214628	0.000386819127812242\\
1175.11254921194	0.000386782169764589\\
1175.38729627761	0.000386745327485417\\
1175.66204334327	0.000386708603138243\\
1175.93679040894	0.000386671669028159\\
1176.2115374746	0.000386635181970474\\
1176.48628454026	0.000386598489132642\\
1176.76103160593	0.00038656191209163\\
1177.03577867159	0.000386524310598892\\
1177.31052573726	0.000386487326608914\\
1177.58527280292	0.000386449967582154\\
1177.86001986859	0.000386413050319422\\
1178.13476693425	0.00038637576397993\\
1178.40951399992	0.000386338595801552\\
1178.68426106558	0.000386301534511816\\
1178.95900813125	0.000386263951185575\\
1179.23375519691	0.000386226653460127\\
1179.50850226258	0.000386189309687069\\
1179.78324932824	0.000386151920282309\\
1180.05799639391	0.000386114322588849\\
1180.33274345957	0.000386076843490215\\
1180.60749052524	0.000386038997916785\\
1180.8822375909	0.000386001438495823\\
1181.15698465657	0.000385963519201692\\
1181.43173172223	0.000385925724620134\\
1181.7064787879	0.000385887885753391\\
1181.98122585356	0.000385850020373708\\
1182.25597291923	0.000385811949060667\\
1182.53071998489	0.000385773842191908\\
1182.80546705056	0.000385736520386418\\
1183.08021411622	0.000385698670050875\\
1183.35496118189	0.000385660948906679\\
1183.62970824755	0.00038562401819482\\
1183.90445531322	0.000385586072241714\\
1184.17920237888	0.000385548091614092\\
1184.45394944455	0.000385509919588567\\
1184.72869651021	0.000385472037030839\\
1185.00344357587	0.000385433633948569\\
1185.27819064154	0.000385395361801007\\
1185.5529377072	0.000385357381787663\\
1185.82768477287	0.000385319204978698\\
1186.10243183853	0.000385280830215724\\
1186.3771789042	0.000385242261444527\\
1186.65192596986	0.000385203834698296\\
1186.92667303553	0.000385165543185947\\
1187.20142010119	0.000385127058494441\\
1187.47616716686	0.000385088374978715\\
1187.75091423252	0.000385049344114829\\
1188.02566129819	0.000385010783596157\\
1188.30040836385	0.000384972689308274\\
1188.57515542952	0.000384933748720533\\
1188.84990249518	0.000384895275022172\\
1189.12464956085	0.000384856443881828\\
1189.39939662651	0.000384817922465495\\
1189.67414369218	0.000384778879712553\\
1189.94889075784	0.000384740304285798\\
1190.22363782351	0.000384701704884194\\
1190.49838488917	0.00038466373868778\\
1190.77313195484	0.00038462508824196\\
1191.0478790205	0.000384585923929841\\
1191.32262608617	0.000384546903538426\\
1191.59737315183	0.000384508189381162\\
1191.8721202175	0.000384469290345356\\
1192.14686728316	0.000384430199726793\\
1192.42161434883	0.000384390932085596\\
1192.69636141449	0.000384351971100367\\
1192.97110848016	0.000384312668089468\\
1193.24585554582	0.000384274000289734\\
1193.52060261148	0.000384234985296179\\
1193.79534967715	0.000384196278871813\\
1194.07009674281	0.000384157222953102\\
1194.34484380848	0.000384118318694323\\
1194.61959087414	0.00038407922752254\\
1194.89433793981	0.000384039794754501\\
1195.16908500547	0.000384001009068632\\
1195.44383207114	0.000383962198512356\\
1195.7185791368	0.000383923048186262\\
1195.99332620247	0.00038388404471766\\
1196.26807326813	0.000383844859489808\\
1196.5428203338	0.000383805818101287\\
1196.81756739946	0.000383766606585358\\
1197.09231446513	0.000383726887552611\\
1197.36706153079	0.000383687814401078\\
1197.64180859646	0.000383648731407838\\
1197.91655566212	0.000383609794598885\\
1198.19130272779	0.000383570354589247\\
1198.46604979345	0.000383531399645315\\
1198.74079685912	0.000383492273326834\\
1199.01554392478	0.000383453297360905\\
1199.29029099045	0.000383413986096969\\
1199.56503805611	0.000383374826575498\\
1199.83978512178	0.000383335505226074\\
1200.11453218744	0.000383296338396441\\
1200.38927925311	0.000383256996711379\\
1200.66402631877	0.000383217973868618\\
1200.93877338444	0.000383177804230799\\
1201.2135204501	0.000383138621021269\\
1201.48826751577	0.000383099428203294\\
1201.76301458143	0.000383060555824062\\
1202.0377616471	0.000383021514679113\\
1202.31250871276	0.000382982803569006\\
1202.58725577842	0.000382943755222881\\
1202.86200284409	0.000382904378845457\\
1203.13674990975	0.000382865162072237\\
1203.41149697542	0.000382825934434344\\
1203.68624404108	0.00038278655133482\\
1203.96099110675	0.00038274700247032\\
1204.23573817241	0.000382706961430731\\
1204.51048523808	0.000382667747059682\\
1204.78523230374	0.000382628694564953\\
1205.05997936941	0.000382589643847138\\
1205.33472643507	0.00038255026077098\\
1205.60947350074	0.000382510886571226\\
1205.8842205664	0.000382471513843141\\
1206.15896763207	0.00038243263667273\\
1206.43371469773	0.000382392940156428\\
1206.7084617634	0.000382353582973642\\
1206.98320882906	0.000382314234152812\\
1207.25795589473	0.000382274729564731\\
1207.53270296039	0.000382235557826636\\
1207.80745002606	0.000382196230290757\\
1208.08219709172	0.000382157234648744\\
1208.35694415739	0.000382118243618258\\
1208.63169122305	0.000382078935457016\\
1208.90643828872	0.000382039959414317\\
1209.18118535438	0.000382000501333882\\
1209.45593242005	0.000381961043599038\\
1209.73067948571	0.000381921609261159\\
1210.00542655138	0.000381882350185458\\
1210.28017361704	0.000381843263718889\\
1210.5549206827	0.000381804020919178\\
1210.82966774837	0.000381765282004082\\
1211.10441481403	0.000381726394126497\\
1211.3791618797	0.000381687518492968\\
1211.65390894536	0.000381648654211824\\
1211.92865601103	0.000381608986452085\\
1212.20340307669	0.000381570156630162\\
1212.47815014236	0.000381531011121587\\
1212.75289720802	0.00038149204770913\\
1213.02764427369	0.000381453264782603\\
1213.30239133935	0.000381414164230172\\
1213.57713840502	0.000381375252136147\\
1213.85188547068	0.00038133618917015\\
1214.12663253635	0.000381297141895482\\
1214.40137960201	0.000381258441204851\\
1214.67612666768	0.00038121943389399\\
1214.95087373334	0.000381180443397682\\
1215.22562079901	0.000381141472766074\\
1215.50036786467	0.000381102687433449\\
1215.77511493034	0.000381063758476625\\
1216.049861996	0.000381025345990939\\
1216.32460906167	0.00038098662919304\\
1216.59935612733	0.00038094793150712\\
1216.874103193	0.000380909422411457\\
1217.14885025866	0.000380870766067453\\
1217.42359732433	0.000380831979219604\\
1217.69834438999	0.000380793378403658\\
1217.97309145566	0.000380754798783574\\
1218.24783852132	0.000380715913395681\\
1218.52258558699	0.000380677548963815\\
1218.79733265265	0.000380639046435261\\
1219.07207971832	0.000380600404684723\\
1219.34682678398	0.000380561140965972\\
1219.62157384964	0.000380522732353713\\
1219.89632091531	0.000380484674143887\\
1220.17106798097	0.000380445829372775\\
1220.44581504664	0.000380407343668278\\
1220.7205621123	0.000380369212140611\\
1220.99530917797	0.000380330787678921\\
1221.27005624363	0.00038029239307429\\
1221.5448033093	0.000380253866844103\\
1221.81955037496	0.000380216186578551\\
1222.09429744063	0.000380177893637058\\
1222.36904450629	0.000380139795824826\\
1222.64379157196	0.000380101727655\\
1222.91853863762	0.00038006352666335\\
1223.19328570329	0.000380025192256745\\
1223.46803276895	0.00037998722265973\\
1223.74277983462	0.000379949284773693\\
1224.01752690028	0.000379911223651735\\
1224.29227396595	0.000379873525864303\\
1224.56702103161	0.000379835371409966\\
1224.84176809728	0.000379797591389602\\
1225.11651516294	0.000379760174337648\\
1225.39126222861	0.000379722467378831\\
1225.66600929427	0.000379684797095703\\
1225.94075635994	0.000379647171008542\\
1226.2155034256	0.000379609753546343\\
1226.49025049127	0.000379572203970173\\
1226.76499755693	0.000379534706386302\\
1227.0397446226	0.000379497246406564\\
1227.31449168826	0.000379459661469875\\
1227.58923875393	0.000379422442918129\\
1227.86398581959	0.000379384615805677\\
1228.13873288525	0.000379347327130094\\
1228.41347995092	0.000379310086841967\\
1228.68822701658	0.000379272721586412\\
1228.96297408225	0.000379235573794105\\
1229.23772114791	0.000379198303141476\\
1229.51246821358	0.000379161569612528\\
1229.78721527924	0.000379124554183554\\
1230.06196234491	0.00037908741489834\\
1230.33670941057	0.000379050165862994\\
1230.61145647624	0.000379013127452344\\
1230.8862035419	0.000378975966822725\\
1231.16095060757	0.000378939029495797\\
1231.43569767323	0.000378901975613649\\
1231.7104447389	0.000378865120797304\\
1231.98519180456	0.000378828168691518\\
1232.25993887023	0.0003787914309352\\
1232.53468593589	0.000378754743149\\
1232.80943300156	0.000378717939554521\\
1233.08418006722	0.000378681190384357\\
1233.35892713289	0.000378644656351432\\
1233.63367419855	0.000378608338481232\\
1233.90842126422	0.000378572071992067\\
1234.18316832988	0.000378535689889843\\
1234.45791539555	0.000378499207106295\\
1234.73266246121	0.000378462942919937\\
1235.00740952688	0.000378426730049599\\
1235.28215659254	0.000378391060333525\\
1235.55690365821	0.000378354956547141\\
1235.83165072387	0.000378319072017301\\
1236.10639778954	0.000378283076382756\\
1236.3811448552	0.00037824730030567\\
1236.65589192087	0.000378211258263392\\
1236.93063898653	0.000378175764478356\\
1237.20538605219	0.000378140164639748\\
1237.48013311786	0.000378104784488069\\
1237.75488018352	0.000378069294900419\\
1238.02962724919	0.000378033867177548\\
1238.30437431485	0.000377998655548056\\
1238.57912138052	0.00037796285135643\\
1238.85386844618	0.000377927110307961\\
1239.12861551185	0.000377892085225918\\
1239.40336257751	0.000377856959521382\\
1239.67810964318	0.000377821891905482\\
1239.95285670884	0.000377787049125427\\
1240.22760377451	0.000377752103848278\\
1240.50235084017	0.000377717224743888\\
1240.77709790584	0.000377682573229854\\
1241.0518449715	0.000377647332837104\\
1241.32659203717	0.000377612649078595\\
1241.60133910283	0.000377578033323192\\
1241.8760861685	0.000377543473716375\\
1242.15083323416	0.000377508821401786\\
1242.42558029983	0.000377474397695071\\
1242.70032736549	0.000377440040401412\\
1242.97507443116	0.000377405754508968\\
1243.24982149682	0.000377371704952235\\
1243.52456856249	0.000377338053668294\\
1243.79931562815	0.000377303977378742\\
1244.07406269382	0.000377269807804205\\
1244.34880975948	0.00037723586296504\\
1244.62355682515	0.000377201993848465\\
1244.89830389081	0.000377167863234842\\
1245.17305095648	0.000377134130623981\\
1245.44779802214	0.000377100466083164\\
1245.7225450878	0.000377067036380986\\
1245.99729215347	0.000377033679161198\\
1246.27203921913	0.000377000226108392\\
1246.5467862848	0.000376966848051005\\
1246.82153335046	0.000376933535587902\\
1247.09628041613	0.000376900299321281\\
1247.37102748179	0.000376867294768471\\
1247.64577454746	0.000376834036566757\\
1247.92052161312	0.000376800848563161\\
1248.19526867879	0.000376767739307713\\
1248.47001574445	0.000376734694201548\\
1248.74476281012	0.000376701734283189\\
1249.01950987578	0.000376668848397763\\
1249.29425694145	0.00037663620015516\\
1249.56900400711	0.000376603461375973\\
1249.84375107278	0.000376571130190103\\
1250.11849813844	0.000376539041522161\\
1250.39324520411	0.0003765068643664\\
1250.66799226977	0.000376474430030995\\
1250.94273933544	0.000376442080407096\\
1251.2174864011	0.000376409809186199\\
1251.49223346677	0.00037637793935354\\
1251.76698053243	0.000376345502225322\\
1252.0417275981	0.00037631347462408\\
1252.31647466376	0.00037628152082662\\
1252.59122172943	0.00037624981588993\\
1252.86596879509	0.00037621818636699\\
1253.14071586076	0.000376187133465556\\
1253.41546292642	0.000376156331436282\\
1253.69020999209	0.00037612511676125\\
1253.96495705775	0.000376094147404566\\
1254.23970412341	0.000376063264029821\\
1254.51445118908	0.000376032135369018\\
1254.78919825474	0.00037600092809042\\
1255.06394532041	0.000375970132773318\\
1255.33869238607	0.000375939253874701\\
1255.61343945174	0.000375908628868026\\
1255.8881865174	0.000375878414265744\\
1256.16293358307	0.000375847628778448\\
1256.43768064873	0.000375816762053351\\
1256.7124277144	0.000375786637754593\\
1256.98717478006	0.000375756431528596\\
1257.26192184573	0.000375726150058152\\
1257.53666891139	0.000375695949818808\\
1257.81141597706	0.000375665830192636\\
1258.08616304272	0.000375635961855508\\
1258.36091010839	0.000375606343838699\\
1258.63565717405	0.000375576649334786\\
1258.91040423972	0.000375547043019905\\
1259.18515130538	0.000375517029726039\\
1259.45989837105	0.00037548711130187\\
1259.73464543671	0.000375457769071864\\
1260.00939250238	0.000375428352089466\\
1260.28413956804	0.000375398691725181\\
1260.55888663371	0.000375369281451997\\
1260.83363369937	0.000375340292099767\\
1261.10838076504	0.00037531122659751\\
1261.3831278307	0.000375282578084911\\
1261.65787489637	0.000375253852745574\\
1261.93262196203	0.000375225217598159\\
1262.2073690277	0.000375196839471178\\
1262.48211609336	0.000375168386761643\\
1262.75686315902	0.00037513969387603\\
1263.03161022469	0.000375111100062195\\
1263.30635729035	0.000375082924850251\\
1263.58110435602	0.0003750548391937\\
1263.85585142168	0.000375026675482617\\
1264.13059848735	0.000374998118657237\\
1264.40534555301	0.000374970145941912\\
1264.68009261868	0.000374942255905972\\
1264.95483968434	0.000374914297976628\\
1265.22958675001	0.000374886594284974\\
1265.50433381567	0.000374858975607898\\
1265.77908088134	0.000374831289166603\\
1266.053827947	0.000374803694495713\\
1266.32857501267	0.000374776521456199\\
1266.60332207833	0.000374749113291319\\
1266.878069144	0.000374722294437733\\
1267.15281620966	0.00037469491044567\\
1267.42756327533	0.000374667295788053\\
1267.70231034099	0.000374640923920637\\
1267.97705740666	0.000374613997048281\\
1268.25180447232	0.000374587325612379\\
1268.52655153799	0.000374560416630762\\
1268.80129860365	0.000374533928626946\\
1269.07604566932	0.000374507530282834\\
1269.35079273498	0.000374480728774734\\
1269.62553980065	0.000374454356651934\\
1269.90028686631	0.000374429074183937\\
1270.17503393198	0.000374403228989581\\
1270.44978099764	0.000374377313193856\\
1270.72452806331	0.00037435133236535\\
1270.99927512897	0.000374325767164095\\
1271.27402219464	0.000374299976535703\\
1271.5487692603	0.000374274278713997\\
1271.82351632596	0.000374248507589585\\
1272.09826339163	0.000374222997377786\\
1272.37301045729	0.000374197585094305\\
1272.64775752296	0.000374172419402222\\
1272.92250458862	0.000374147194249975\\
1273.19725165429	0.000374122227359551\\
1273.47199871995	0.000374097024143251\\
1273.74674578562	0.000374071915547375\\
1274.02149285128	0.000374047066182188\\
1274.29623991695	0.000374022640097828\\
1274.57098698261	0.000373998311966462\\
1274.84573404828	0.000373973917911251\\
1275.12048111394	0.000373949947025811\\
1275.39522817961	0.000373925744103768\\
1275.66997524527	0.000373901631260122\\
1275.94472231094	0.000373877452580056\\
1276.2194693766	0.000373853528952175\\
1276.49421644227	0.000373829540745671\\
1276.76896350793	0.000373805973211489\\
1277.0437105736	0.000373782171561365\\
1277.31845763926	0.00037375847232112\\
1277.59320470493	0.000373735194465584\\
1277.86795177059	0.000373711354831684\\
1278.14269883626	0.000373688101547369\\
1278.41744590192	0.000373664456499296\\
1278.69219296759	0.000373641238339696\\
1278.96694003325	0.000373618112855536\\
1279.24168709892	0.000373594922513899\\
1279.51643416458	0.00037357215680592\\
1279.79118123025	0.000373549643662953\\
1280.06592829591	0.000373526580127081\\
1280.34067536157	0.000373503778535197\\
1280.61542242724	0.000373481400030865\\
1280.8901694929	0.000373458783203241\\
1281.16491655857	0.000373436425369201\\
1281.43966362423	0.000373414331299493\\
1281.7144106899	0.000373392497072497\\
1281.98915775556	0.000373370921494189\\
1282.26390482123	0.000373349112387581\\
1282.53865188689	0.000373327398693038\\
1282.81339895256	0.000373305943734615\\
1283.08814601822	0.00037328442179404\\
1283.36289308389	0.000373262827733373\\
1283.63764014955	0.000373241979552853\\
1283.91238721522	0.000373220574737463\\
1284.18713428088	0.000373199262150552\\
1284.46188134655	0.000373177878520826\\
1284.73662841221	0.000373156917918413\\
1285.01137547788	0.000373136216327292\\
1285.28612254354	0.000373115604344375\\
1285.56086960921	0.00037309475917832\\
1285.83561667487	0.000373073841904467\\
1286.11036374054	0.000373053352254367\\
1286.3851108062	0.000373033119360377\\
1286.65985787187	0.000373012648495994\\
1286.93460493753	0.00037299195063366\\
1287.2093520032	0.000372971840995651\\
1287.48409906886	0.00037295198586529\\
1287.75884613453	0.000372932372672861\\
1288.03359320019	0.000372912377023154\\
1288.30834026586	0.000372892472652673\\
1288.58308733152	0.000372872493640425\\
1288.85783439718	0.000372852770912815\\
1289.13258146285	0.000372833302712865\\
1289.40732852851	0.000372813922269573\\
1289.68207559418	0.000372794629405954\\
1289.95682265984	0.000372775095144543\\
1290.23156972551	0.000372755979298902\\
1290.50631679117	0.000372736629303353\\
1290.78106385684	0.000372717535085926\\
1291.0558109225	0.00037269837426121\\
1291.33055798817	0.000372679794378711\\
1291.60530505383	0.000372661298683512\\
1291.8800521195	0.000372642731955039\\
1292.15479918516	0.000372623596747963\\
1292.42954625083	0.000372605374857541\\
1292.70429331649	0.00037258740821598\\
1292.97904038216	0.000372569365592503\\
1293.25378744782	0.000372551407948478\\
1293.52853451349	0.000372533212436601\\
1293.80328157915	0.000372515099016774\\
1294.07802864482	0.000372497243858616\\
1294.35277571048	0.000372479801054859\\
1294.62752277615	0.000372462280516643\\
1294.90226984181	0.000372444196235747\\
1295.17701690748	0.000372426691390358\\
1295.45176397314	0.000372409432606761\\
1295.72651103881	0.000372391935768983\\
1296.00125810447	0.000372374523786726\\
1296.27600517014	0.000372357359486013\\
1296.5507522358	0.000372340109204083\\
1296.82549930147	0.00037232311411759\\
1297.10024636713	0.000372306034501108\\
1297.3749934328	0.000372289371399634\\
1297.64974049846	0.000372272626794567\\
1297.92448756412	0.00037225629338288\\
1298.19923462979	0.000372239388194396\\
1298.47398169545	0.000372222893401541\\
1298.74872876112	0.000372206157382132\\
1299.02347582678	0.000372189829396044\\
1299.29822289245	0.000372173584440216\\
1299.57296995811	0.000372157253640388\\
1299.84771702378	0.000372140848012642\\
1300.12246408944	0.0003721248518411\\
1300.39721115511	0.000372109095190132\\
1300.67195822077	0.000372092930549108\\
1300.94670528644	0.00037207767160602\\
1301.2214523521	0.000372062330236178\\
1301.49619941777	0.000372046575930607\\
1301.77094648343	0.000372031225583329\\
1302.0456935491	0.000372015625134471\\
1302.32044061476	0.00037200009945658\\
1302.59518768043	0.000371984820626395\\
1302.86993474609	0.000371969785913886\\
1303.14468181176	0.000371954990371711\\
1303.41942887742	0.000371939781731528\\
1303.69417594309	0.000371924974997553\\
1303.96892300875	0.000371910081033769\\
1304.24367007442	0.00037189525729874\\
1304.51841714008	0.000371880350529933\\
1304.79316420575	0.000371865517820681\\
1305.06791127141	0.000371850755632364\\
1305.34265833708	0.000371836230023112\\
1305.61740540274	0.000371821942862529\\
1305.89215246841	0.000371807718118027\\
1306.16689953407	0.000371793242515125\\
1306.44164659973	0.000371778672785315\\
1306.7163936654	0.000371764502995746\\
1306.99114073106	0.000371750235141218\\
1307.26588779673	0.000371736366914884\\
1307.54063486239	0.000371721912396537\\
1307.81538192806	0.000371708186346545\\
1308.09012899372	0.000371694353837526\\
1308.36487605939	0.00037168059822439\\
1308.63962312505	0.00037166674259521\\
1308.91437019072	0.000371652951298572\\
1309.18911725638	0.000371639061978049\\
1309.46386432205	0.000371625398427458\\
1309.73861138771	0.000371611970870322\\
1310.01335845338	0.000371598604171975\\
1310.28810551904	0.000371585462707712\\
1310.56285258471	0.000371572220255229\\
1310.83759965037	0.000371559375613394\\
1311.11234671604	0.000371546102280934\\
1311.3870937817	0.000371532725883354\\
1311.66184084737	0.000371519744524165\\
1311.93658791303	0.00037150715455676\\
1312.2113349787	0.000371494296547718\\
1312.48608204436	0.000371481500747362\\
1312.76082911003	0.000371468928010877\\
1313.03557617569	0.000371456085847151\\
1313.31032324136	0.000371443456098662\\
1313.58507030702	0.000371430897703758\\
1313.85981737269	0.000371418066071579\\
1314.13456443835	0.000371405454011811\\
1314.40931150402	0.00037139289550983\\
1314.68405856968	0.000371380554419484\\
1314.95880563534	0.000371368427677405\\
1315.23355270101	0.000371355535126645\\
1315.50829976667	0.000371342865290636\\
1315.78304683234	0.000371330586020141\\
1316.057793898	0.000371318521305077\\
1316.33254096367	0.000371306339931333\\
1316.60728802933	0.000371294378718895\\
1316.882035095	0.000371282462667077\\
1317.15678216066	0.000371270433537859\\
1317.43152922633	0.00037125811943868\\
1317.70627629199	0.000371246355614376\\
1317.98102335766	0.000371234471915397\\
1318.25577042332	0.000371222797126421\\
1318.53051748899	0.000371211328221721\\
1318.80526455465	0.000371199746488405\\
1319.08001162032	0.000371187876462808\\
1319.35475868598	0.000371176545623071\\
1319.62950575165	0.000371165585651721\\
1319.90425281731	0.000371153687461937\\
1320.17899988298	0.000371142326054935\\
1320.45374694864	0.000371131163989553\\
1320.72849401431	0.000371119718215448\\
1321.00324107997	0.000371108473207243\\
1321.27798814564	0.000371097269237172\\
1321.5527352113	0.000371085767673112\\
1321.82748227697	0.000371074476099948\\
1322.10222934263	0.000371063056443355\\
1322.3769764083	0.000371052001548696\\
1322.65172347396	0.000371040656258304\\
1322.92647053963	0.000371029346071274\\
1323.20121760529	0.000371018563397053\\
1323.47596467095	0.000371007316965765\\
1323.75071173662	0.000370996274354697\\
1324.02545880228	0.000370985426095834\\
1324.30020586795	0.000370974604153294\\
1324.57495293361	0.000370963816220497\\
1324.84969999928	0.000370952727592649\\
1325.12444706494	0.000370941993477132\\
1325.39919413061	0.000370931447823849\\
1325.67394119627	0.000370920280017019\\
1325.94868826194	0.000370909467022411\\
1326.2234353276	0.000370898841988008\\
1326.49818239327	0.00037088840088807\\
1326.77292945893	0.000370877983020048\\
1327.0476765246	0.000370867262125601\\
1327.32242359026	0.00037085656281853\\
1327.59717065593	0.000370845887705339\\
1327.87191772159	0.000370835716559289\\
1328.14666478726	0.000370825250030134\\
1328.42141185292	0.000370814142191843\\
1328.69615891859	0.000370803713081684\\
1328.97090598425	0.000370793296922168\\
1329.24565304992	0.000370782892890163\\
1329.52040011558	0.000370772500158206\\
1329.79514718125	0.000370761790917992\\
1330.06989424691	0.000370751751946538\\
1330.34464131258	0.000370741066410321\\
1330.61938837824	0.000370730399930626\\
1330.89413544391	0.000370720399945576\\
1331.16888250957	0.000370709920925984\\
1331.44362957524	0.000370699774154552\\
1331.7183766409	0.000370689311903385\\
1331.99312370657	0.000370678529552771\\
1332.26787077223	0.000370668412106216\\
1332.54261783789	0.000370657967099127\\
1332.81736490356	0.000370647207904443\\
1333.09211196922	0.000370636619663492\\
1333.36685903489	0.000370626034851595\\
1333.64160610055	0.000370615944664721\\
1333.91635316622	0.000370605530225049\\
1334.19110023188	0.000370595444292766\\
1334.46584729755	0.000370585358417484\\
1334.74059436321	0.000370575096056411\\
1335.01534142888	0.000370564678715871\\
1335.29008849454	0.000370554255180289\\
1335.56483556021	0.000370543827253646\\
1335.83958262587	0.000370533392055792\\
1336.11432969154	0.000370523112298102\\
1336.3890767572	0.000370512978826655\\
1336.66382382287	0.000370502186778018\\
1336.93857088853	0.000370491717367941\\
1337.2133179542	0.00037048090796405\\
1337.48806501986	0.000370470742333047\\
1337.76281208553	0.000370460402577562\\
1338.03755915119	0.00037045004618726\\
1338.31230621686	0.000370439669171514\\
1338.58705328252	0.000370428790629112\\
1338.86180034819	0.000370418220151282\\
1339.13654741385	0.00037040764159446\\
1339.41129447952	0.000370397043686303\\
1339.68604154518	0.000370386249602532\\
1339.96078861085	0.000370375617705765\\
1340.23553567651	0.000370364471912674\\
1340.51028274217	0.000370353800925406\\
1340.78502980784	0.000370343268922514\\
1341.0597768735	0.000370332708253628\\
1341.33452393917	0.000370321964528458\\
1341.60927100483	0.000370310704564315\\
1341.8840180705	0.000370299920693839\\
1342.15876513616	0.00037028926225265\\
1342.43351220183	0.00037027858587348\\
1342.70825926749	0.000370267876019854\\
1342.98300633316	0.000370256968314294\\
1343.25775339882	0.00037024586224298\\
1343.53250046449	0.000370234889079969\\
1343.80724753015	0.000370224210083457\\
1344.08199459582	0.000370213664645064\\
1344.35674166148	0.000370202757427454\\
1344.63148872715	0.000370191474919568\\
1344.90623579281	0.000370180330068423\\
1345.18098285848	0.000370169145061093\\
1345.45572992414	0.00037015808329272\\
1345.73047698981	0.00037014714041966\\
1346.00522405547	0.000370136152982061\\
1346.27997112114	0.000370124959077794\\
1346.5547181868	0.00037011388137623\\
1346.82946525247	0.00037010243171316\\
1347.10421231813	0.000370091265255998\\
1347.3789593838	0.000370079890852462\\
1347.65370644946	0.000370068135449835\\
1347.92845351513	0.00037005666109701\\
1348.20320058079	0.000370044979335956\\
1348.47794764646	0.000370033243296689\\
1348.75269471212	0.000370021453399204\\
1349.02744177779	0.000370009603845158\\
1349.30218884345	0.000369997873192983\\
1349.57693590911	0.000369986083647402\\
1349.85168297478	0.000369974396530331\\
1350.12643004044	0.000369962324408348\\
1350.40117710611	0.000369950519504269\\
1350.67592417177	0.00036993864922179\\
1350.95067123744	0.000369926066731826\\
1351.2254183031	0.000369913589068057\\
1351.50016536877	0.000369901052357059\\
1351.77491243443	0.000369889113027914\\
1352.0496595001	0.000369876619946386\\
1352.32440656576	0.000369864388872634\\
1352.59915363143	0.00036985208947394\\
1352.87390069709	0.000369839551702066\\
1353.14864776276	0.000369826785325394\\
1353.42339482842	0.000369814112397352\\
1353.69814189409	0.000369801201437393\\
1353.97288895975	0.000369788548022112\\
1354.24763602542	0.000369775656051542\\
1354.52238309108	0.000369762522552768\\
1354.79713015675	0.000369748994429917\\
1355.07187722241	0.000369735721027918\\
1355.34662428808	0.000369722529531222\\
1355.62137135374	0.000369709262578962\\
1355.89611841941	0.000369696075682613\\
1356.17086548507	0.000369682643446603\\
1356.44561255074	0.000369668961805242\\
1356.7203596164	0.000369654876725366\\
1356.99510668207	0.000369641037656009\\
1357.26985374773	0.000369627437704642\\
1357.5446008134	0.000369613425480711\\
1357.81934787906	0.000369599488780711\\
1358.09409494472	0.000369585460215134\\
1358.36884201039	0.00036957134325435\\
1358.64358907605	0.000369557300958337\\
1358.91833614172	0.000369542661901966\\
1359.19308320738	0.000369528274819553\\
1359.46783027305	0.000369513461937564\\
1359.74257733871	0.000369498883606385\\
1360.01732440438	0.000369484205773225\\
1360.29207147004	0.000369469097720899\\
1360.56681853571	0.000369454389535622\\
1360.84156560137	0.000369439744304738\\
1361.11631266704	0.000369424344672559\\
1361.3910597327	0.00036940917264357\\
1361.66580679837	0.000369393897389201\\
1361.94055386403	0.000369378195266811\\
1362.2153009297	0.000369362550686004\\
1362.49004799536	0.000369346640306435\\
1362.76479506103	0.0003693309478451\\
1363.03954212669	0.00036931515725932\\
1363.31428919236	0.000369299254117937\\
1363.58903625802	0.000369282746562703\\
1363.86378332369	0.000369266952861736\\
1364.13853038935	0.000369250719133235\\
1364.41327745502	0.000369234537721815\\
1364.68802452068	0.000369218077867399\\
1364.96277158635	0.000369201661970181\\
1365.23751865201	0.000369184799842187\\
1365.51226571768	0.000369167661179269\\
1365.78701278334	0.000369150730716655\\
1366.06175984901	0.00036913351905356\\
1366.33650691467	0.000369116515747024\\
1366.61125398033	0.000369099392550569\\
1366.886001046	0.000369082480074069\\
1367.16074811166	0.000369064942747268\\
1367.43549517733	0.00036904695998537\\
1367.71024224299	0.000369029016419328\\
1367.98498930866	0.000369011109149666\\
1368.25973637432	0.000368993070075509\\
1368.53448343999	0.000368975055901508\\
1368.80923050565	0.000368956263673191\\
1369.08397757132	0.000368937835995076\\
1369.35872463698	0.000368918946044581\\
1369.63347170265	0.000368899921671837\\
1369.90821876831	0.000368881091216563\\
1370.18296583398	0.000368862443831979\\
1370.45771289964	0.000368843010181609\\
1370.73245996531	0.000368823602799317\\
1371.00720703097	0.000368804219491667\\
1371.28195409664	0.000368785021655012\\
1371.5567011623	0.000368765352242994\\
1371.83144822797	0.000368745219524735\\
1372.10619529363	0.000368725273244531\\
1372.3809423593	0.00036870517642055\\
1372.65568942496	0.000368684774133531\\
1372.93043649063	0.000368664055981122\\
1373.20518355629	0.000368643364139598\\
1373.47993062196	0.000368622521278677\\
1373.75467768762	0.000368601856329599\\
1374.02942475329	0.000368580704889259\\
1374.30417181895	0.00036855924758856\\
1374.57891888462	0.000368537965699694\\
1374.85366595028	0.000368516526427173\\
1375.12841301594	0.00036849525871422\\
1375.40316008161	0.000368473509989313\\
1375.67790714727	0.000368451436462781\\
1375.95265421294	0.000368429377465128\\
1376.2274012786	0.000368406992738468\\
1376.50214834427	0.0003683846099497\\
1376.77689540993	0.000368362227072228\\
1377.0516424756	0.000368339514975192\\
1377.32638954126	0.000368316479237993\\
1377.60113660693	0.00036829343681096\\
1377.87588367259	0.000368270069681365\\
1378.15063073826	0.000368246535752047\\
1378.42537780392	0.000368222996811086\\
1378.70012486959	0.000368199454606361\\
1378.97487193525	0.00036817573069911\\
1379.24961900092	0.000368151353685327\\
1379.52436606658	0.000368126808043317\\
1379.79911313225	0.000368102255582557\\
1380.07386019791	0.000368077692183367\\
1380.34860726358	0.000368052951043698\\
1380.62335432924	0.000368027374818492\\
1380.89810139491	0.000368002280743218\\
1381.17284846057	0.000367976690095415\\
1381.44759552624	0.000367951250072403\\
1381.7223425919	0.000367925947191851\\
1381.99708965757	0.000367900149397093\\
1382.27183672323	0.00036787400246422\\
1382.5465837889	0.000367847836504758\\
1382.82133085456	0.000367821155508735\\
1383.09607792023	0.000367794618717289\\
1383.37082498589	0.00036776805999698\\
1383.64557205156	0.000367741152574934\\
1383.92031911722	0.00036771421336778\\
1384.19506618288	0.000367686926112782\\
1384.46981324855	0.000367659283398673\\
1384.74456031421	0.000367631284485697\\
1385.01930737988	0.00036760359065305\\
1385.29405444554	0.000367575545777861\\
1385.56880151121	0.000367547142198692\\
1385.84354857687	0.000367518868771929\\
1386.11829564254	0.000367490076705474\\
1386.3930427082	0.000367461252404878\\
1386.66778977387	0.000367432228153904\\
1386.94253683953	0.000367403162651779\\
1387.2172839052	0.000367373418862759\\
1387.49203097086	0.000367343804767622\\
1387.76677803653	0.000367313660111302\\
1388.04152510219	0.000367283477617872\\
1388.31627216786	0.000367253419550521\\
1388.59101923352	0.000367223158227443\\
1388.86576629919	0.000367192685322244\\
1389.14051336485	0.000367161689379724\\
1389.41526043052	0.000367130323864865\\
1389.69000749618	0.000367098915112368\\
1389.96475456185	0.000367067462210246\\
1390.23950162751	0.000367035962418063\\
1390.51424869318	0.000367004089494988\\
1390.78899575884	0.000366972174542279\\
1391.06374282451	0.000366940044624007\\
1391.33848989017	0.000366907380055906\\
1391.61323695584	0.000366874666117334\\
1391.8879840215	0.000366841891828484\\
1392.16273108717	0.000366808091632182\\
1392.43747815283	0.000366774570723798\\
1392.71222521849	0.000366740668858064\\
1392.98697228416	0.000366706877857915\\
1393.26171934982	0.000366672871715206\\
1393.53646641549	0.00036663880406719\\
1393.81121348115	0.000366604027612228\\
1394.08596054682	0.000366569030324668\\
1394.36070761248	0.000366534138919146\\
1394.63545467815	0.000366498865189938\\
1394.91020174381	0.000366463687485293\\
1395.18494880948	0.000366427801271936\\
1395.45969587514	0.000366391527309241\\
1395.73444294081	0.000366354702440702\\
1396.00919000647	0.00036631799023558\\
1396.28393707214	0.000366281703879749\\
1396.5586841378	0.000366244864384403\\
1396.83343120347	0.000366207794864522\\
1397.10817826913	0.000366170988813297\\
1397.3829253348	0.000366133946830684\\
1397.65767240046	0.000366095861401043\\
1397.93241946613	0.000366057874978349\\
1398.20716653179	0.000366019818121863\\
1398.48191359746	0.000365981525066325\\
1398.75666066312	0.00036594299940874\\
1399.03140772879	0.00036590407485766\\
1399.30615479445	0.000365865245431861\\
1399.58090186012	0.00036582616063996\\
1399.85564892578	0.000365786529055143\\
1400.13039599145	0.000365746822130016\\
1400.40514305711	0.000365706708678864\\
1400.67989012278	0.000365666191859575\\
1400.95463718844	0.000365625600558922\\
1401.2293842541	0.000365584766965363\\
1401.50413131977	0.000365543691605301\\
1401.77887838543	0.000365502373594513\\
1402.0536254511	0.000365460974743175\\
1402.32837251676	0.000365419655467715\\
1402.60311958243	0.000365377607261136\\
1402.87786664809	0.000365335312170583\\
1403.15261371376	0.000365292771959014\\
1403.42736077942	0.00036525032000434\\
1403.70210784509	0.000365207293592667\\
1403.97685491075	0.000365164350967766\\
1404.25160197642	0.000365121160104201\\
1404.52634904208	0.000365077394989412\\
1404.80109610775	0.000365033707097916\\
1405.07584317341	0.000364989443712239\\
1405.35059023908	0.000364945093710967\\
1405.62533730474	0.000364900819913086\\
1405.90008437041	0.000364855957565749\\
1406.17483143607	0.000364810851211386\\
1406.44957850174	0.000364765323762673\\
1406.7243255674	0.000364719378672286\\
1406.99907263307	0.000364673503688283\\
1407.27381969873	0.000364626888601394\\
1407.5485667644	0.000364580184854801\\
1407.82331383006	0.000364533379762548\\
1408.09806089573	0.000364486331469286\\
1408.37280796139	0.00036443902423756\\
1408.64755502706	0.000364391459338332\\
1408.92230209272	0.000364343634770474\\
1409.19704915839	0.000364295714532791\\
1409.47179622405	0.000364247532581194\\
1409.74654328971	0.000364198771312161\\
1410.02129035538	0.000364149587153716\\
1410.29603742104	0.000364100634633819\\
1410.57078448671	0.00036405142301276\\
1410.84553155237	0.000364002097107126\\
1411.12027861804	0.000363952192259532\\
1411.3950256837	0.000363902022618581\\
1411.66977274937	0.000363851589492445\\
1411.94451981503	0.00036380105392332\\
1412.2192668807	0.000363750250313317\\
1412.49401394636	0.000363699178189679\\
1412.76876101203	0.000363647672632404\\
1413.04350807769	0.000363596229675541\\
1413.31825514336	0.000363544358072058\\
1413.59300220902	0.000363492052981827\\
1413.86774927469	0.000363439318151162\\
1414.14249634035	0.000363386320892155\\
1414.41724340602	0.000363333389531947\\
1414.69199047168	0.000363280184919481\\
1414.96673753735	0.000363226548993598\\
1415.24148460301	0.00036317263474993\\
1415.51623166868	0.000363118625092168\\
1415.79097873434	0.000363064177496171\\
1416.06572580001	0.000363009457934201\\
1416.34047286567	0.000362954466498556\\
1416.61521993134	0.000362899522124035\\
1416.889966997	0.000362843655157169\\
1417.16471406267	0.000362787846501782\\
1417.43946112833	0.000362731928460373\\
1417.714208194	0.000362675406578352\\
1417.98895525966	0.000362618942119159\\
1418.26370232533	0.000362562205992036\\
1418.53844939099	0.000362505194194691\\
1418.81319645665	0.000362447910919124\\
1419.08794352232	0.000362390019765457\\
1419.36269058798	0.000362331841745929\\
1419.63743765365	0.000362273084252174\\
1419.91218471931	0.000362214546229565\\
1420.18693178498	0.000362155732027243\\
1420.46167885064	0.000362096638152467\\
1420.73642591631	0.000362037430706385\\
1421.01117298197	0.000361977942300531\\
1421.28592004764	0.00036191817265487\\
1421.5606671133	0.000361858121494949\\
1421.83541417897	0.000361797788551896\\
1422.11016124463	0.000361737168172171\\
1422.3849083103	0.000361675947198864\\
1422.65965537596	0.000361614776125592\\
1422.93440244163	0.000361553157439921\\
1423.20914950729	0.000361491258304027\\
1423.48389657296	0.000361429074452487\\
1423.75864363862	0.000361366785908565\\
1424.03339070429	0.000361304049671731\\
1424.30813776995	0.000361241195390809\\
1424.58288483562	0.000361177884086845\\
1424.85763190128	0.000361114467516898\\
1425.13237896695	0.000361050766200878\\
1425.40712603261	0.000360986779947784\\
1425.68187309828	0.000360922344737728\\
1425.95662016394	0.000360857783155973\\
1426.23136722961	0.000360792457167384\\
1426.50611429527	0.000360726850270923\\
1426.78086136094	0.000360661291860998\\
1427.0556084266	0.000360595778511074\\
1427.33035549226	0.000360529815517485\\
1427.60510255793	0.000360463411094775\\
1427.87984962359	0.000360397047226171\\
1428.15459668926	0.00036033007473864\\
1428.42934375492	0.000360262652375468\\
1428.70409082059	0.000360195109106464\\
1428.97883788625	0.000360127289437828\\
1429.25358495192	0.000360059342368948\\
1429.52833201758	0.000359991118066644\\
1429.80307908325	0.000359921786113028\\
1430.07782614891	0.000359852507598136\\
1430.35257321458	0.000359783763962927\\
1430.62732028024	0.000359714415349883\\
1430.90206734591	0.000359644778754045\\
1431.17681441157	0.000359574852299852\\
1431.45156147724	0.00035950447991602\\
1431.7263085429	0.000359434146705166\\
1432.00105560857	0.00035936320298973\\
1432.27580267423	0.000359292300849735\\
1432.5505497399	0.000359220946532525\\
1432.82529680556	0.000359149310284095\\
1433.10004387123	0.000359077551149544\\
1433.37479093689	0.000359005340935913\\
1433.64953800256	0.000358932523315196\\
1433.92428506822	0.000358859585093587\\
1434.19903213389	0.00035878618714844\\
1434.47377919955	0.000358712682390605\\
1434.74852626522	0.000358638729227063\\
1435.02327333088	0.000358564654401647\\
1435.29802039655	0.000358490458252235\\
1435.57276746221	0.000358415808720852\\
1435.84751452787	0.000358340874649184\\
1436.12226159354	0.000358265826556416\\
1436.3970086592	0.000358189996997904\\
1436.67175572487	0.000358114211912455\\
1436.94650279053	0.000358038316498322\\
1437.2212498562	0.00035796196848547\\
1437.49599692186	0.000357885498812153\\
1437.77074398753	0.000357808741403159\\
1438.04549105319	0.000357731700241252\\
1438.32023811886	0.000357654542509688\\
1438.59498518452	0.000357576935979618\\
1438.86973225019	0.000357498881935728\\
1439.14447931585	0.000357421367904606\\
1439.41922638152	0.000357343079378458\\
1439.69397344718	0.000357264505216371\\
1439.96872051285	0.000357185481404505\\
1440.24346757851	0.000357106180088829\\
1440.51821464418	0.000357026754311669\\
1440.79296170984	0.000356946387675651\\
1441.06770877551	0.000356866233891538\\
1441.34245584117	0.000356785799437796\\
1441.61720290684	0.000356705221253892\\
1441.8919499725	0.000356624053313892\\
1442.16669703817	0.000356542601851644\\
1442.44144410383	0.000356460373987317\\
1442.7161911695	0.000356378197508197\\
1442.99093823516	0.00035629590352492\\
1443.26568530083	0.000356213490503385\\
1443.54043236649	0.00035613079230346\\
1443.81517943216	0.000356047645753787\\
1444.08992649782	0.000355964051812184\\
1444.36467356348	0.000355880341738762\\
1444.63942062915	0.000355796513393633\\
1444.91416769481	0.000355712076482828\\
1445.18891476048	0.000355628017755407\\
1445.46366182614	0.000355543187956334\\
1445.73840889181	0.000355458236204158\\
1446.01315595747	0.000355372846415805\\
1446.28790302314	0.000355287338815711\\
1446.5626500888	0.000355201546115523\\
1446.83739715447	0.000355115313370214\\
1447.11214422013	0.000355028793660612\\
1447.3868912858	0.000354941675963372\\
1447.66163835146	0.000354854449961749\\
1447.93638541713	0.000354766617945055\\
1448.21113248279	0.000354679005708722\\
1448.48587954846	0.000354591278987661\\
1448.76062661412	0.000354503437271867\\
1449.03537367979	0.000354414828317794\\
1449.31012074545	0.000354326435002513\\
1449.58486781112	0.000354237760434974\\
1449.85961487678	0.000354148484425362\\
1450.13436194245	0.000354058925753835\\
1450.40910900811	0.0003539692637345\\
1450.68385607378	0.000353879160231593\\
1450.95860313944	0.000353788940534709\\
1451.23335020511	0.000353698287106981\\
1451.50809727077	0.000353607194270366\\
1451.78284433644	0.000353516152634016\\
1452.0575914021	0.000353424188432312\\
1452.33233846777	0.000353332445416221\\
1452.60708553343	0.000353240265875013\\
1452.8818325991	0.0003531483105933\\
1453.15657966476	0.000353055581909229\\
1453.43132673042	0.000352962594954141\\
1453.70607379609	0.000352869501024867\\
1453.98082086175	0.000352776135189545\\
1454.25556792742	0.000352682823115384\\
1454.53031499308	0.000352588258826974\\
1454.80506205875	0.000352493928141089\\
1455.07980912441	0.000352399824157174\\
1455.35455619008	0.000352305290698999\\
1455.62930325574	0.000352210158986033\\
1455.90405032141	0.000352114918311712\\
1456.17879738707	0.000352019580525012\\
1456.45354445274	0.000351923973317178\\
1456.7282915184	0.000351828254053725\\
1457.00303858407	0.00035173178823108\\
1457.27778564973	0.000351634893805406\\
1457.5525327154	0.000351538227755255\\
1457.82727978106	0.000351441292900045\\
1458.10202684673	0.000351344254672875\\
1458.37677391239	0.000351246950169927\\
1458.65152097806	0.000351149051152719\\
1458.92626804372	0.00035105089359049\\
1459.20101510939	0.00035095214125969\\
1459.47576217505	0.000350853461627475\\
1459.75050924072	0.000350754689543124\\
1460.02525630638	0.000350655640280231\\
1460.30000337205	0.000350556015339143\\
1460.57475043771	0.000350456294772468\\
1460.84949750338	0.000350356312441119\\
1461.12424456904	0.000350256068644771\\
1461.39899163471	0.00035015572888281\\
1461.67373870037	0.000350054630419876\\
1461.94848576603	0.00034995345041309\\
1462.2232328317	0.000349852012279284\\
1462.49797989736	0.000349750483461019\\
1462.77272696303	0.000349649186356312\\
1463.04747402869	0.00034954731199064\\
1463.32222109436	0.000349445178519078\\
1463.59696816002	0.000349342948900387\\
1463.87171522569	0.000349240301077233\\
1464.14646229135	0.000349137232392353\\
1464.42120935702	0.000349034060678919\\
1464.69595642268	0.000348930490253601\\
1464.97070348835	0.000348826501874814\\
1465.24545055401	0.000348722425856304\\
1465.52019761968	0.000348618261333901\\
1465.79494468534	0.000348513677618841\\
1466.06969175101	0.000348409007525102\\
1466.34443881667	0.000348303756174129\\
1466.61918588234	0.000348198584965744\\
1466.893932948	0.000348093311669009\\
1467.16868001367	0.000347987475633179\\
1467.44342707933	0.00034788155462756\\
1467.718174145	0.000347775220380665\\
1467.99292121066	0.000347668638375055\\
1468.26766827633	0.000347561808995756\\
1468.54241534199	0.000347454733991772\\
1468.81716240766	0.000347347576702646\\
1469.09190947332	0.000347240332233189\\
1469.36665653899	0.000347132355131993\\
1469.64140360465	0.000347024464436981\\
1469.91615067032	0.000346916489496093\\
1470.19089773598	0.000346807950871581\\
1470.46564480164	0.000346699334437726\\
1470.74039186731	0.00034659030970589\\
1471.01513893297	0.00034648136092125\\
1471.28988599864	0.000346371856849774\\
1471.5646330643	0.000346262112099691\\
1471.83938012997	0.000346152128278325\\
1472.11412719563	0.000346042070288437\\
1472.3888742613	0.000345931282554939\\
1472.66362132696	0.000345820587557516\\
1472.93836839263	0.000345709810397164\\
1473.21311545829	0.000345598809436327\\
1473.48786252396	0.000345487077377716\\
1473.76260958962	0.000345375608744922\\
1474.03735665529	0.000345263738863893\\
1474.31210372095	0.000345151310348308\\
1474.58685078662	0.000345038980687093\\
1474.86159785228	0.000344927078811038\\
1475.13634491795	0.000344814297012088\\
1475.41109198361	0.000344701449116901\\
1475.68583904928	0.000344588192818236\\
1475.96058611494	0.000344474887261753\\
1476.23533318061	0.000344361024027589\\
1476.51008024627	0.000344246933447356\\
1476.78482731194	0.000344132618310669\\
1477.0595743776	0.000344018241589501\\
1477.33432144327	0.000343903471004733\\
1477.60906850893	0.000343788484380373\\
1477.8838155746	0.000343673440042128\\
1478.15856264026	0.000343558332576095\\
1478.43330970593	0.000343443168770274\\
1478.70805677159	0.000343327618602604\\
1478.98280383726	0.000343211524120523\\
1479.25755090292	0.000343095204535058\\
1479.53229796858	0.000342978515855655\\
1479.80704503425	0.000342861777171208\\
1480.08179209991	0.000342745147330107\\
1480.35653916558	0.000342628296025131\\
1480.63128623124	0.000342511548761757\\
1480.90603329691	0.000342394096399324\\
1481.18078036257	0.000342276591584241\\
1481.45552742824	0.000342158861162201\\
1481.7302744939	0.000342040760119049\\
1482.00502155957	0.000341922444365566\\
1482.27976862523	0.000341804077878621\\
1482.5545156909	0.000341685488477434\\
1482.82926275656	0.0003415666952413\\
1483.10400982223	0.000341447523615682\\
1483.37875688789	0.000341328139562316\\
1483.65350395356	0.000341208867519899\\
1483.92825101922	0.000341089226999375\\
1484.20299808489	0.000340969209842185\\
1484.47774515055	0.000340848985929033\\
1484.75249221622	0.00034072887955223\\
1485.02723928188	0.000340608238929944\\
1485.30198634755	0.000340487718577203\\
1485.57673341321	0.000340366987212639\\
1485.85148047888	0.000340245401451256\\
1486.12622754454	0.000340124106655936\\
1486.40097461021	0.000340002442733934\\
1486.67572167587	0.000339880893883884\\
1486.95046874154	0.000339758991526054\\
1487.2252158072	0.00033963688515712\\
1487.49996287287	0.000339514575487656\\
1487.77470993853	0.000339391899226313\\
1488.04945700419	0.000339269187296387\\
1488.32420406986	0.000339146435034068\\
1488.59895113552	0.000339022832729431\\
1488.87369820119	0.000338899673780858\\
1489.14844526685	0.000338776172105314\\
1489.42319233252	0.000338652144564474\\
1489.69793939818	0.000338527921214827\\
1489.97268646385	0.000338403835307736\\
1490.24743352951	0.000338279553030983\\
1490.52218059518	0.000338154746720788\\
1490.79692766084	0.000338029913286509\\
1491.07167472651	0.0003379047238966\\
1491.34642179217	0.000337779347011474\\
1491.62116885784	0.00033765410149649\\
1491.8959159235	0.000337528511119609\\
1492.17066298917	0.000337402566917951\\
1492.44541005483	0.000337276764821749\\
1492.7201571205	0.000337150601672596\\
1492.99490418616	0.000337024261904935\\
1493.26965125183	0.000336897899854194\\
1493.54439831749	0.000336771020341992\\
1493.81914538316	0.000336644122453789\\
1494.09389244882	0.00033651736539922\\
1494.36863951449	0.000336390263473607\\
1494.64338658015	0.000336263126741599\\
1494.91813364582	0.000336135493372406\\
1495.19288071148	0.000336007678207703\\
1495.46762777715	0.000335879681166182\\
1495.74237484281	0.000335751667739781\\
1496.01712190848	0.000335623472323823\\
1496.29186897414	0.000335494930697473\\
1496.5666160398	0.000335366702506909\\
1496.84136310547	0.000335237479178087\\
1497.11611017113	0.000335108413693924\\
1497.3908572368	0.000334979499697645\\
1497.66560430246	0.000334849758115322\\
1497.94035136813	0.000334720168315943\\
1498.21509843379	0.000334590249898681\\
1498.48984549946	0.000334460487221264\\
1498.76459256512	0.000334330879673866\\
1499.03933963079	0.000334200426305814\\
1499.31408669645	0.000334069825173494\\
1499.58883376212	0.000333939055493922\\
1499.86358082778	0.000333808281957792\\
1500.13832789345	0.000333677175262067\\
1500.41307495911	0.000333546228534878\\
1500.68782202478	0.000333415111156533\\
1500.96256909044	0.000333283823809731\\
1501.23731615611	0.000333152202753042\\
1501.51206322177	0.000333020410187204\\
1501.78681028744	0.00033288862578446\\
1502.0615573531	0.00033275667406958\\
1502.33630441877	0.000332624391892637\\
1502.61105148443	0.000332492431827249\\
1502.8857985501	0.000332359988793809\\
1503.16054561576	0.000332226888965283\\
1503.43529268143	0.000332093960760991\\
1503.71003974709	0.000331961034683353\\
1503.98478681276	0.000331827941963968\\
1504.25953387842	0.000331694532509292\\
1504.53428094409	0.000331561126883629\\
1504.80902800975	0.000331427561187255\\
1505.08377507541	0.000331293998337914\\
1505.35852214108	0.000331159954059991\\
1505.63326920674	0.000331025755124869\\
1505.90801627241	0.00033089139236047\\
1506.18276333807	0.000330756885221098\\
1506.45751040374	0.000330622223438904\\
1506.7322574694	0.000330486917352136\\
1507.00700453507	0.000330351955490716\\
1507.28175160073	0.000330216993541205\\
1507.5564986664	0.000330081400536764\\
1507.83124573206	0.000329946151687183\\
1508.10599279773	0.000329810585773517\\
1508.38073986339	0.000329675033085607\\
1508.65548692906	0.000329539163366414\\
1508.93023399472	0.000329403142821167\\
1509.20498106039	0.000329267144502414\\
1509.47972812605	0.000329130832634081\\
1509.75447519172	0.000328994867282633\\
1510.02922225738	0.000328858418342795\\
1510.30396932305	0.0003287216693233\\
1510.57871638871	0.000328584774558477\\
1510.85346345438	0.000328447409265095\\
1511.12821052004	0.000328310394086863\\
1511.40295758571	0.000328173397663069\\
1511.67770465137	0.000328036087440338\\
1511.95245171704	0.00032789847893142\\
1512.2271987827	0.000327760891986945\\
1512.50194584837	0.000327623489936977\\
1512.77669291403	0.000327485783875663\\
1513.0514399797	0.000327347933748985\\
1513.32618704536	0.000327209621351726\\
1513.60093411102	0.000327071333374089\\
1513.87568117669	0.000326932902627174\\
1514.15042824235	0.000326794013987174\\
1514.42517530802	0.000326655481137124\\
1514.69992237368	0.000326516649661324\\
1514.97466943935	0.000326377842374549\\
1515.24941650501	0.000326238244326515\\
1515.52416357068	0.000326099161661275\\
1515.79891063634	0.000325959795659695\\
1516.07365770201	0.000325819967357585\\
1516.34840476767	0.000325680004443162\\
1516.62315183334	0.000325540077696772\\
1516.897898899	0.000325400181614391\\
1517.17264596467	0.000325259662272403\\
1517.44739303033	0.000325119501997494\\
1517.722140096	0.000324979051938913\\
1517.99688716166	0.000324838471280093\\
1518.27163422733	0.000324698247890606\\
1518.54638129299	0.000324557251654575\\
1518.82112835866	0.000324416457414767\\
1519.09587542432	0.000324275532863913\\
1519.37062248999	0.000324133984773348\\
1519.64536955565	0.000323992806538979\\
1519.92011662132	0.000323851505234113\\
1520.19486368698	0.000323709416970067\\
1520.46961075265	0.000323567545007158\\
1520.74435781831	0.000323425713979691\\
1521.01910488398	0.000323283591732203\\
1521.29385194964	0.000323140862062088\\
1521.56859901531	0.000322998999561767\\
1521.84334608097	0.000322856688749097\\
1522.11809314663	0.000322714579434624\\
1522.3928402123	0.000322571694542595\\
1522.66758727796	0.000322429348658525\\
1522.94233434363	0.000322286715013035\\
1523.21708140929	0.00032214380159825\\
1523.49182847496	0.000322000923060674\\
1523.76657554062	0.000321857934658115\\
1524.04132260629	0.000321714661743109\\
1524.31606967195	0.000321571268260004\\
1524.59081673762	0.000321427920709013\\
1524.86556380328	0.000321284290057395\\
1525.14031086895	0.000321140539161275\\
1525.41505793461	0.000320996845441949\\
1525.68980500028	0.000320852867766045\\
1525.96455206594	0.000320709089455281\\
1526.23929913161	0.000320565044618907\\
1526.51404619727	0.000320420718229835\\
1526.78879326294	0.000320276112940395\\
1527.0635403286	0.000320131723438781\\
1527.33828739427	0.000319986891023932\\
1527.61303445993	0.00031984227382\\
1527.8877815256	0.000319697212356749\\
1528.16252859126	0.000319552536258386\\
1528.43727565693	0.000319407585073142\\
1528.71202272259	0.00031926248983863\\
1528.98676978826	0.00031911698633058\\
1529.26151685392	0.000318971538941622\\
1529.53626391959	0.000318825815814737\\
1529.81101098525	0.000318680312982787\\
1530.08575805092	0.000318534534700574\\
1530.36050511658	0.000318388646233075\\
1530.63525218225	0.000318242649443615\\
1530.90999924791	0.000318096541943425\\
1531.18474631357	0.000317950327957717\\
1531.45949337924	0.000317804004180148\\
1531.7342404449	0.000317657572484918\\
1532.00898751057	0.000317510705636422\\
1532.28373457623	0.000317364228791546\\
1532.5584816419	0.00031721747932347\\
1532.83322870756	0.000317070623181306\\
1533.10797577323	0.000316923821241416\\
1533.38272283889	0.000316776919815361\\
1533.65746990456	0.00031662957941493\\
1533.93221697022	0.000316482144581939\\
1534.20696403589	0.000316334933183324\\
1534.48171110155	0.000316187457721131\\
1534.75645816722	0.000316039711143147\\
1535.03120523288	0.000315892034285737\\
1535.30595229855	0.000315744252525737\\
1535.58069936421	0.000315596538390799\\
1535.85544642988	0.000315448883079013\\
1536.13019349554	0.000315300609647464\\
1536.40494056121	0.000315152264607148\\
1536.67968762687	0.000315003984651902\\
1536.95443469254	0.000314855602089184\\
1537.2291817582	0.000314707283458333\\
1537.50392882387	0.000314558697051139\\
1537.77867588953	0.000314409847384725\\
1538.0534229552	0.000314260897793419\\
1538.32817002086	0.000314112180185233\\
1538.60291708653	0.000313963195868535\\
1538.87766415219	0.000313814112240759\\
1539.15241121786	0.000313665094488406\\
1539.42715828352	0.000313515975901204\\
1539.70190534918	0.000313366757921235\\
1539.97665241485	0.00031321728414915\\
1540.25139948051	0.00031306787540102\\
1540.52614654618	0.000312918039215514\\
1540.80089361184	0.000312768439694373\\
1541.07564067751	0.000312619073630125\\
1541.35038774317	0.000312469285041054\\
1541.62513480884	0.000312319399582754\\
1541.8998818745	0.000312169427274596\\
1542.17462894017	0.000312019194738235\\
1542.44937600583	0.000311868866195572\\
1542.7241230715	0.000311718287055234\\
1542.99887013716	0.000311567777635367\\
1543.27361720283	0.000311417501564211\\
1543.54836426849	0.000311266955448516\\
1543.82311133416	0.000311116000706038\\
1544.09785839982	0.00031096495259494\\
1544.37260546549	0.000310813978055727\\
1544.64735253115	0.000310662908685735\\
1544.92209959682	0.000310511910077125\\
1545.19684666248	0.000310360814225886\\
1545.47159372815	0.000310209633593306\\
1545.74634079381	0.000310058194519487\\
1546.02108785948	0.000309906977728174\\
1546.29583492514	0.000309755352642626\\
1546.57058199081	0.000309603799505793\\
1546.84532905647	0.000309451987673114\\
1547.12007612214	0.000309299920870765\\
1547.3948231878	0.000309148092386689\\
1547.66957025347	0.000308996006601609\\
1547.94431731913	0.000308843825985474\\
1548.21906438479	0.000308691397805998\\
1548.49381145046	0.000308538880627379\\
1548.76855851612	0.000308386608819991\\
1549.04330558179	0.000308234246054105\\
1549.31805264745	0.000308081633954919\\
1549.59279971312	0.000307929094610818\\
1549.86754677878	0.000307776301260472\\
1550.14229384445	0.000307623423883295\\
1550.41704091011	0.000307470298166652\\
1550.69178797578	0.00030731755871059\\
1550.96653504144	0.000307164420452775\\
1551.24128210711	0.000307011356800457\\
1551.51602917277	0.000306858037480905\\
1551.79077623844	0.000306704793793059\\
1552.0655233041	0.000306551459212258\\
1552.34027036977	0.000306397706527434\\
1552.61501743543	0.000306244030983966\\
1552.8897645011	0.000306090267245543\\
1553.16451156676	0.000305935926093617\\
1553.43925863243	0.000305782150529783\\
1553.71400569809	0.000305628300035118\\
1553.98875276376	0.000305474031459285\\
1554.26349982942	0.00030532000556856\\
1554.53824689509	0.000305165232624339\\
1554.81299396075	0.000305011198682041\\
1555.08774102642	0.000304856914978089\\
1555.36248809208	0.000304702692634038\\
1555.63723515775	0.000304548068033984\\
1555.91198222341	0.000304393029544601\\
1556.18672928908	0.000304238236876991\\
1556.46147635474	0.000304083191334414\\
1556.7362234204	0.000303928061708496\\
1557.01097048607	0.000303773012513506\\
1557.28571755173	0.000303618533414731\\
1557.5604646174	0.000303463317151133\\
1557.83521168306	0.000303308338381139\\
1558.10995874873	0.000303153112868065\\
1558.38470581439	0.000302997471384378\\
1558.65945288006	0.000302842078516144\\
1558.93419994572	0.000302686743094851\\
1559.20894701139	0.000302531176412758\\
1559.48369407705	0.000302375522015125\\
1559.75844114272	0.000302219452434623\\
1560.03318820838	0.00030206330023473\\
1560.30793527405	0.000301907227825477\\
1560.58268233971	0.00030175107096643\\
1560.85742940538	0.000301594992480208\\
1561.13217647104	0.000301438663719921\\
1561.40692353671	0.000301281921179208\\
1561.68167060237	0.000301125754396081\\
1561.95641766804	0.000300969332703374\\
1562.2311647337	0.000300812505948281\\
1562.50591179937	0.000300655595270382\\
1562.78065886503	0.00030049876780454\\
1563.0554059307	0.000300342171389583\\
1563.33015299636	0.00030018517508766\\
1563.60490006203	0.00030002825759501\\
1563.87964712769	0.000299871088939542\\
1564.15439419336	0.000299714000460561\\
1564.42914125902	0.000299556498353884\\
1564.70388832469	0.000299399241320614\\
1564.97863539035	0.000299242065264768\\
1565.25338245602	0.000299084118544607\\
1565.52812952168	0.000298926456345135\\
1565.80287658734	0.000298768709370349\\
1566.07762365301	0.000298611042522281\\
1566.35237071867	0.000298453289459617\\
1566.62711778434	0.000298295450244387\\
1566.90186485	0.000298137034309543\\
1567.17661191567	0.000297978703352837\\
1567.45135898133	0.000297820289108866\\
1567.726106047	0.000297662121069027\\
1568.00085311266	0.000297503866966106\\
1568.27560017833	0.000297345198208989\\
1568.55034724399	0.00029718661365581\\
1568.82509430966	0.000297027943740891\\
1569.09984137532	0.000296869354599923\\
1569.37458844099	0.000296710516157159\\
1569.64933550665	0.000296551757414523\\
1569.92408257232	0.000296393059398405\\
1570.19882963798	0.000296234130988488\\
1570.47357670365	0.000296074789240571\\
1570.74832376931	0.000295915693780727\\
1571.02307083498	0.000295756349016653\\
1571.29781790064	0.000295596920017817\\
1571.57256496631	0.000295437571872641\\
1571.84731203197	0.000295277973217197\\
1572.12205909764	0.000295118291844923\\
1572.3968061633	0.000294958851271671\\
1572.67155322897	0.000294799003755448\\
1572.94630029463	0.000294639236268998\\
1573.2210473603	0.000294479218746268\\
1573.49579442596	0.000294319113220064\\
1573.77054149163	0.000294158938095823\\
1574.04528855729	0.000293998514672014\\
1574.32003562295	0.000293838339249433\\
1574.59478268862	0.000293678078249977\\
1574.86952975428	0.000293517731699205\\
1575.14427681995	0.000293357298875967\\
1575.41902388561	0.000293196932543643\\
1575.69377095128	0.000293036171173418\\
1575.96851801694	0.000292875326535841\\
1576.24326508261	0.000292714563500447\\
1576.51801214827	0.00029255355077429\\
1576.79275921394	0.000292392454669859\\
1577.0675062796	0.000292231275378118\\
1577.34225334527	0.000292070013041509\\
1577.61700041093	0.00029190883164536\\
1577.8917474766	0.000291747732117914\\
1578.16649454226	0.000291586545423704\\
1578.44124160793	0.000291425116594507\\
1578.71598867359	0.000291263602404704\\
1578.99073573926	0.000291101998828733\\
1579.26548280492	0.000290940154467912\\
1579.54022987059	0.000290777899988506\\
1579.81497693625	0.000290616058505119\\
1580.08972400192	0.000290454449781195\\
1580.36447106758	0.000290292281739157\\
1580.63921813325	0.000290129867448709\\
1580.91396519891	0.000289967699100284\\
1581.18871226458	0.000289805281767586\\
1581.46345933024	0.000289642781055308\\
1581.73820639591	0.000289480361031137\\
1582.01295346157	0.000289317691173359\\
1582.28770052724	0.000289155261952263\\
1582.5624475929	0.000288992590192541\\
1582.83719465856	0.000288829668977469\\
1583.11194172423	0.000288666665501196\\
1583.38668878989	0.00028850374279907\\
1583.66143585556	0.000288340742918074\\
1583.93618292122	0.000288177493840474\\
1584.21092998689	0.000288014490264601\\
1584.48567705255	0.000287851240384361\\
1584.76042411822	0.000287687906680623\\
1585.03517118388	0.000287524328195014\\
1585.30991824955	0.000287360673483238\\
1585.58466531521	0.000287196752534957\\
1585.85941238088	0.000287033101050003\\
1586.13415944654	0.000286869036898363\\
1586.40890651221	0.000286705219182773\\
1586.68365357787	0.000286541316118059\\
1586.95840064354	0.000286377327723127\\
1587.2331477092	0.000286212433647696\\
1587.50789477487	0.000286047958445809\\
1587.78264184053	0.000285883729496696\\
1588.0573889062	0.000285719577252689\\
1588.33213597186	0.000285554851634708\\
1588.60688303753	0.000285390538038134\\
1588.88163010319	0.000285226133650824\\
1589.15637716886	0.000285061325010413\\
1589.43112423452	0.000284896596982341\\
1589.70587130019	0.000284731620718694\\
1589.98061836585	0.00028456688780737\\
1590.25536543152	0.00028440207470311\\
1590.53011249718	0.000284237181711668\\
1590.80485956285	0.00028407154793278\\
1591.07960662851	0.000283906174375354\\
1591.35435369417	0.000283740552264271\\
1591.62910075984	0.000283575177910636\\
1591.9038478255	0.000283409883477502\\
1592.17859489117	0.000283244302651578\\
1592.45334195683	0.000283078355919039\\
1592.7280890225	0.000282912656044316\\
1593.00283608816	0.000282746707673633\\
1593.27758315383	0.000282580677045648\\
1593.55233021949	0.000282414726407993\\
1593.82707728516	0.000282248526367706\\
1594.10182435082	0.000282082407943064\\
1594.37657141649	0.000281915875929012\\
1594.65131848215	0.000281749428994913\\
1594.92606554782	0.000281582899137166\\
1595.20081261348	0.000281416286212191\\
1595.47555967915	0.000281249426709635\\
1595.75030674481	0.000281082814193142\\
1596.02505381048	0.000280915952892454\\
1596.29980087614	0.000280748844955006\\
1596.57454794181	0.000280581657534433\\
1596.84929500747	0.000280414553665579\\
1597.12404207314	0.000280247530179953\\
1597.3987891388	0.000280079928740348\\
1597.67353620447	0.000279912742357129\\
1597.94828327013	0.000279745471204447\\
1598.2230303358	0.000279577788526092\\
1598.49777740146	0.00027941035241625\\
1598.77252446713	0.000279242995062947\\
1599.04727153279	0.000279075376890185\\
1599.32201859846	0.00027890753043856\\
1599.59676566412	0.00027873943778475\\
1599.87151272979	0.000278571592248663\\
1600.14625979545	0.000278403662336485\\
1600.42100686111	0.000278235320739488\\
1600.69575392678	0.000278067063522126\\
1600.97050099244	0.000277898394756763\\
1601.24524805811	0.000277729810746646\\
1601.51999512377	0.000277561317334824\\
1601.79474218944	0.000277392904362472\\
1602.0694892551	0.000277224248920525\\
1602.34423632077	0.000277055835522806\\
1602.61898338643	0.000276886529369181\\
1602.8937304521	0.000276717804624013\\
1603.16847751776	0.000276548831114669\\
1603.44322458343	0.000276379775444132\\
1603.71797164909	0.000276210489851262\\
1603.99271871476	0.00027604112335638\\
1604.26746578042	0.000275872006217295\\
1604.54221284609	0.000275702975228512\\
1604.81695991175	0.000275533692199968\\
1605.09170697742	0.000275364008022287\\
1605.36645404308	0.000275194410133172\\
1605.64120110875	0.000275024894266713\\
1605.91594817441	0.000274855616704215\\
1606.19069524008	0.000274685774821286\\
1606.46544230574	0.000274515688213308\\
1606.74018937141	0.00027434585172716\\
1607.01493643707	0.000274175769214424\\
1607.28968350274	0.000274005766387499\\
1607.5644305684	0.000273835688789641\\
1607.83917763407	0.00027366536568914\\
1608.11392469973	0.000273494952623885\\
1608.3886717654	0.000273324472246345\\
1608.66341883106	0.000273153913186051\\
1608.93816589672	0.00027298343818938\\
1609.21291296239	0.000272813043326615\\
1609.48766002805	0.00027264208043655\\
1609.76240709372	0.000272471358779696\\
1610.03715415938	0.000272300405280683\\
1610.31190122505	0.000272129371091075\\
1610.58664829071	0.000271958258516963\\
1610.86139535638	0.000271787231116253\\
1611.13614242204	0.000271616287479753\\
1611.41088948771	0.000271445084820585\\
1611.68563655337	0.000271273497835758\\
1611.96038361904	0.000271101834528792\\
1612.2351306847	0.00027093025930362\\
1612.50987775037	0.000270758440684172\\
1612.78462481603	0.000270586549513427\\
1613.0593718817	0.00027041458080899\\
1613.33411894736	0.000270242700401234\\
1613.60886601303	0.000270070907127879\\
1613.88361307869	0.000269899034200228\\
1614.15836014436	0.000269726917222251\\
1614.43310721002	0.000269555047296465\\
1614.70785427569	0.000269382945273483\\
1614.98260134135	0.000269210435791614\\
1615.25734840702	0.000269038180471893\\
1615.53209547268	0.000268865679369034\\
1615.80684253835	0.000268692947215211\\
1616.08158960401	0.000268520467487608\\
1616.35633666968	0.000268347909694604\\
1616.63108373534	0.000268175273338833\\
1616.90583080101	0.000268002404706338\\
1617.18057786667	0.00026782962451799\\
1617.45532493233	0.00026765692844141\\
1617.730071998	0.0002674835058127\\
1618.00481906366	0.000267310822090399\\
1618.27956612933	0.000267137746661465\\
1618.55431319499	0.000266964594654706\\
1618.82906026066	0.000266791203019802\\
1619.10380732632	0.000266617737701432\\
1619.37855439199	0.000266444361614248\\
1619.65330145765	0.000266270896839143\\
1619.92804852332	0.000266097210762512\\
1620.20279558898	0.000265923288405837\\
1620.47754265465	0.000265749621426543\\
1620.75228972031	0.0002655755525795\\
1621.02703678598	0.000265401246491301\\
1621.30178385164	0.000265227366341319\\
1621.57653091731	0.000265053249150446\\
1621.85127798297	0.000264879061123271\\
1622.12602504864	0.000264704644410953\\
1622.4007721143	0.000264530322315722\\
1622.67551917997	0.000264355930288051\\
1622.95026624563	0.000264181470843567\\
1623.2250133113	0.000264007101178616\\
1623.49976037696	0.000263832506324093\\
1623.77450744263	0.000263657675800535\\
1624.04925450829	0.000263482941217127\\
1624.32400157396	0.000263308295653885\\
1624.59874863962	0.000263133424021369\\
1624.87349570529	0.000262958647121573\\
1625.14824277095	0.000262783798812099\\
1625.42298983662	0.000262608715006197\\
1625.69773690228	0.00026243373131103\\
1625.97248396794	0.000262258821169949\\
1626.24723103361	0.000262083702916016\\
1626.52197809927	0.000261908023619481\\
1626.79672516494	0.000261732280175975\\
1627.0714722306	0.000261556965888911\\
1627.34621929627	0.000261381582281809\\
1627.62096636193	0.000261206129649221\\
1627.8957134276	0.000261030608287793\\
1628.17046049326	0.00026085485359524\\
1628.44520755893	0.000260678870518218\\
1628.71995462459	0.000260503151375212\\
1628.99470169026	0.00026032720172961\\
1629.26944875592	0.000260151186415317\\
1629.54419582159	0.000259975094184947\\
1629.81894288725	0.00025979895382998\\
1630.09368995292	0.000259622749475828\\
1630.36843701858	0.000259446481126258\\
1630.64318408425	0.000259270312333721\\
1630.91793114991	0.000259093750224807\\
1631.19267821558	0.000258917455043277\\
1631.46742528124	0.000258740922457939\\
1631.74217234691	0.000258564504171162\\
1632.01691941257	0.000258387528723981\\
1632.29166647824	0.000258210988743122\\
1632.5664135439	0.000258034385016803\\
1632.84116060957	0.000257857717885469\\
1633.11590767523	0.000257681136949881\\
1633.3906547409	0.000257504346086768\\
1633.66540180656	0.000257327492878675\\
1633.94014887223	0.000257150577674585\\
1634.21489593789	0.000256973108879231\\
1634.48964300356	0.000256795914495386\\
1634.76439006922	0.000256618331611335\\
1635.03913713488	0.000256440858999792\\
1635.31388420055	0.000256263655470855\\
1635.58863126621	0.000256086235979443\\
1635.86337833188	0.000255909084845976\\
1636.13812539754	0.000255731056946744\\
1636.41287246321	0.000255553795888099\\
1636.68761952887	0.000255375989728359\\
1636.96236659454	0.00025519829340683\\
1637.2371136602	0.000255020700109736\\
1637.51186072587	0.000254843059660659\\
1637.78660779153	0.000254665034645842\\
1638.0613548572	0.000254487284450805\\
1638.33610192286	0.000254309479100129\\
1638.61084898853	0.000254131450800147\\
1638.88559605419	0.000253953543162683\\
1639.16034311986	0.000253775580126576\\
1639.43509018552	0.000253597397410989\\
1639.70983725119	0.000253419484049469\\
1639.98458431685	0.000253241197252588\\
1640.25933138252	0.0002530628599328\\
1640.53407844818	0.000252883980437865\\
1640.80882551385	0.000252705708564503\\
1641.08357257951	0.000252527390738964\\
1641.35831964518	0.000252349019252626\\
1641.63306671084	0.000252170601638125\\
1641.90781377651	0.000251991968537007\\
1642.18256084217	0.000251813291354703\\
1642.45730790784	0.000251634732587761\\
1642.7320549735	0.000251456245449473\\
1643.00680203916	0.000251277430592759\\
1643.28154910483	0.000251098568635745\\
1643.55629617049	0.000250919823128542\\
1643.83104323616	0.000250741029055082\\
1644.10579030182	0.000250562022024541\\
1644.38053736749	0.000250382804323569\\
1644.65528443315	0.000250203709088591\\
1644.93003149882	0.000250024568947381\\
1645.20477856448	0.000249845385938163\\
1645.47952563015	0.000249666322676162\\
1645.75427269581	0.000249487050273826\\
1646.02901976148	0.000249307406597884\\
1646.30376682714	0.000249127887215456\\
1646.57851389281	0.000248948657292787\\
1646.85326095847	0.000248768892225212\\
1647.12800802414	0.000248588763384196\\
1647.4027550898	0.000248409420482234\\
1647.67750215547	0.000248229542411474\\
1647.95224922113	0.000248050120995477\\
1648.2269962868	0.0002478704944363\\
1648.50174335246	0.000247690991786393\\
1648.77649041813	0.000247511121597028\\
1649.05123748379	0.000247331378319355\\
1649.32598454946	0.000247151760545622\\
1649.60073161512	0.000246972103088722\\
1649.87547868079	0.000246792396859033\\
1650.15022574645	0.000246612670271026\\
1650.42497281212	0.000246432742184323\\
1650.69971987778	0.000246252778409215\\
1650.97446694345	0.000246072942960683\\
1651.24921400911	0.000245892743387423\\
1651.52396107478	0.000245712673466335\\
1651.79870814044	0.000245532740030422\\
1652.0734552061	0.000245352935423252\\
1652.34820227177	0.000245172933348907\\
1652.62294933743	0.000244992718901086\\
1652.8976964031	0.000244812659966316\\
1653.17244346876	0.000244632567614216\\
1653.44719053443	0.000244452442351048\\
1653.72193760009	0.000244272284684441\\
1653.99668466576	0.000244092095123377\\
1654.27143173142	0.00024391170990294\\
1654.54617879709	0.000243731460298294\\
1654.82092586275	0.000243551180306174\\
1655.09567292842	0.00024337054213793\\
1655.37041999408	0.000243190206502931\\
1655.64516705975	0.000243010007686081\\
1655.91991412541	0.000242829457638297\\
1656.19466119108	0.000242648883497204\\
1656.46940825674	0.000242468589268213\\
1656.74415532241	0.000242287964678355\\
1657.01890238807	0.000242107151459139\\
1657.29364945374	0.000241926481762622\\
1657.5683965194	0.000241745952134265\\
1657.84314358507	0.000241565071113014\\
1658.11789065073	0.000241384167563511\\
1658.3926377164	0.000241203573882132\\
1658.66738478206	0.000241022628223847\\
1658.94213184773	0.000240841992863133\\
1659.21687891339	0.000240661335199262\\
1659.49162597906	0.000240480491377795\\
1659.76637304472	0.00024029995751985\\
1660.04112011039	0.000240119403232819\\
1660.31586717605	0.000239938668415032\\
1660.59061424172	0.000239757913091014\\
1660.86536130738	0.00023957730736402\\
1661.14010837304	0.000239396678651173\\
1661.41485543871	0.000239215882159884\\
1661.68960250437	0.000239034741138384\\
1661.96434957004	0.000238853751763674\\
1662.2390966357	0.000238672747081652\\
1662.51384370137	0.00023849188962135\\
1662.78859076703	0.000238310697989538\\
1663.0633378327	0.000238129981073791\\
1663.33808489836	0.000237948766828817\\
1663.61283196403	0.000237767870862374\\
1663.88757902969	0.0002375867984083\\
1664.16232609536	0.000237406024874392\\
1664.43707316102	0.000237225095582755\\
1664.71182022669	0.000237043826281128\\
1664.98656729235	0.00023686271518475\\
1665.26131435802	0.000236681597202602\\
1665.53606142368	0.000236500469795838\\
1665.81080848935	0.000236319336379447\\
1666.08555555501	0.000236138197141495\\
1666.36030262068	0.000235957050387559\\
1666.63504968634	0.000235775898613475\\
1666.90979675201	0.000235594905574915\\
1667.18454381767	0.000235414073307043\\
1667.45929088334	0.00023523291170252\\
1667.734037949	0.000235052057298825\\
1668.00878501467	0.000234871053317308\\
1668.28353208033	0.000234690045167553\\
1668.558279146	0.000234509033428345\\
1668.83302621166	0.000234327690710924\\
1669.10777327732	0.000234146351520508\\
1669.38252034299	0.000233965015235715\\
1669.65726740865	0.000233784007983066\\
1669.93201447432	0.000233602507643948\\
1670.20676153998	0.000233421642844508\\
1670.48150860565	0.000233240479120109\\
1670.75625567131	0.000233059319474678\\
1671.03100273698	0.000232877834645397\\
1671.30574980264	0.000232696850265481\\
1671.58049686831	0.000232515540501166\\
1671.85524393397	0.000232334566057722\\
1672.12999099964	0.000232153432154727\\
1672.4047380653	0.000231972467676138\\
1672.67948513097	0.000231791507939923\\
1672.95423219663	0.000231610225673379\\
1673.2289792623	0.000231428953832708\\
1673.50372632796	0.000231247857446934\\
1673.77847339363	0.000231066933466854\\
1674.05322045929	0.00023088598363056\\
1674.32796752496	0.000230704916133401\\
1674.60271459062	0.000230523695858245\\
1674.87746165629	0.000230342651280067\\
1675.15220872195	0.000230161619921266\\
1675.42695578762	0.000229980602180846\\
1675.70170285328	0.000229799597371902\\
1675.97644991895	0.000229618769503197\\
1676.25119698461	0.000229437625593141\\
1676.52594405028	0.000229256663507061\\
1676.80069111594	0.000229075717352682\\
1677.07543818161	0.000228894950316771\\
1677.35018524727	0.000228713869419035\\
1677.62493231294	0.000228533136255451\\
1677.8996793786	0.000228352089774293\\
1678.17442644426	0.000228171392780596\\
1678.44917350993	0.000227990220966558\\
1678.72392057559	0.000227809562666176\\
1678.99866764126	0.000227628928564596\\
1679.27341470692	0.000227448291520234\\
1679.54816177259	0.000227267204616017\\
1679.82290883825	0.000227086475633208\\
1680.09765590392	0.000226905931830903\\
1680.37240296958	0.000226725408764311\\
1680.64715003525	0.000226544578975926\\
1680.92189710091	0.0002263639404901\\
1681.19664416658	0.000226183489818112\\
1681.47139123224	0.000226003062214825\\
1681.74613829791	0.000225822658284692\\
1682.02088536357	0.000225641785249236\\
1682.29563242924	0.000225461427761349\\
1682.5703794949	0.000225281109339485\\
1682.84512656057	0.000225100490075194\\
1683.11987362623	0.000224920228671699\\
1683.3946206919	0.000224739829692611\\
1683.66936775756	0.000224559462431862\\
1683.94411482323	0.000224379441390482\\
1684.21886188889	0.000224198970772989\\
1684.49360895456	0.00022401869886199\\
1684.76835602022	0.000223838130168948\\
1685.04310308589	0.000223658091010087\\
1685.31785015155	0.000223478083043143\\
1685.59259721722	0.000223298105649372\\
1685.86734428288	0.000223118166289527\\
1686.14209134855	0.000222938419811721\\
1686.41683841421	0.000222758389064811\\
1686.69158547987	0.000222578395806046\\
1686.96633254554	0.00022239860316224\\
1687.2410796112	0.000222218835405749\\
1687.51582667687	0.000222038789300743\\
1687.79057374253	0.000221858949004716\\
1688.0653208082	0.00022167931245473\\
1688.34006787386	0.000221499550308804\\
1688.61481493953	0.000221319992005406\\
1688.88956200519	0.000221140464168481\\
1689.16430907086	0.000220960661075565\\
1689.43905613652	0.000220780738927522\\
1689.71380320219	0.000220600699921986\\
1689.98855026785	0.000220421202850632\\
1690.26329733352	0.000220241747120743\\
1690.53804439918	0.000220062342046412\\
1690.81279146485	0.000219883301599054\\
1691.08753853051	0.000219704151020432\\
1691.36228559618	0.000219524716427219\\
1691.63703266184	0.000219345658880453\\
1691.91177972751	0.000219166155229448\\
1692.18652679317	0.000218987177810496\\
1692.46127385884	0.000218808268294417\\
1692.7360209245	0.000218628913676408\\
1693.01076799017	0.000218450105116457\\
1693.28551505583	0.000218270854046682\\
1693.5602621215	0.000218091984447803\\
1693.83500918716	0.000217913165686192\\
1694.10975625283	0.000217734234950601\\
1694.38450331849	0.000217555687390636\\
1694.65925038416	0.000217377191947913\\
1694.93399744982	0.000217198759873413\\
1695.20874451549	0.000217020051810177\\
1695.48349158115	0.000216841889692608\\
1695.75823864681	0.000216662963649202\\
1696.03298571248	0.000216484755645564\\
1696.30773277814	0.000216306609018153\\
1696.58247984381	0.000216128354714585\\
1696.85722690947	0.000215950157053523\\
1697.13197397514	0.000215772025660401\\
1697.4067210408	0.000215594117151575\\
1697.68146810647	0.000215416260029892\\
1697.95621517213	0.000215238470656076\\
1698.2309622378	0.000215060739552374\\
1698.50570930346	0.000214883067309131\\
1698.78045636913	0.00021470560720529\\
1699.05520343479	0.00021452806099693\\
1699.32995050046	0.000214350410494851\\
1699.60469756612	0.000214172988399124\\
1699.87944463179	0.000213995628099823\\
1700.15419169745	0.000213818330181664\\
1700.42893876312	0.000213641095228173\\
1700.70368582878	0.000213464084599912\\
1700.97843289445	0.000213286814860041\\
1701.25317996011	0.000213109766875596\\
1701.52792702578	0.000212932634235275\\
1701.80267409144	0.000212755405362426\\
1702.07742115711	0.000212578575741879\\
1702.35216822277	0.000212401813055861\\
1702.62691528844	0.000212225117880328\\
1702.9016623541	0.000212048487582333\\
1703.17640941977	0.000211871933574843\\
1703.45115648543	0.000211695611529652\\
1703.72590355109	0.000211519198776795\\
1704.00065061676	0.000211342688847675\\
1704.27539768242	0.000211166424074022\\
1704.55014474809	0.000210989739744677\\
1704.82489181375	0.000210813596018042\\
1705.09963887942	0.000210637231852885\\
1705.37438594508	0.000210460944882479\\
1705.64913301075	0.000210284899320429\\
1705.92388007641	0.000210108764575836\\
1706.19862714208	0.000209932874257729\\
1706.47337420774	0.000209757061711718\\
1706.74812127341	0.000209581327206183\\
1707.02286833907	0.000209405507924622\\
1707.29761540474	0.000209229766770564\\
1707.5723624704	0.000209054272287033\\
1707.84710953607	0.000208879021961301\\
1708.12185660173	0.00020870384994778\\
1708.3966036674	0.000208528592760758\\
1708.67135073306	0.000208353089549172\\
1708.94609779873	0.000208178322239158\\
1709.22084486439	0.000208003479864938\\
1709.49559193006	0.000207828718482167\\
1709.77033899572	0.000207653874895048\\
1710.04508606139	0.000207479440793861\\
1710.31983312705	0.000207304934525291\\
1710.59458019272	0.000207130347085081\\
1710.86932725838	0.000206955998073494\\
1711.14407432405	0.000206781586462163\\
1711.41882138971	0.000206607261943315\\
1711.69356845538	0.000206433024269749\\
1711.96831552104	0.000206258548248999\\
1712.24306258671	0.000206084490435812\\
1712.51780965237	0.000205910521548547\\
1712.79255671803	0.000205736797418721\\
1713.0673037837	0.000205562845860046\\
1713.34205084936	0.00020538915251184\\
1713.61679791503	0.000205215057110607\\
1713.89154498069	0.000205041226632589\\
1714.16629204636	0.000204867817335412\\
1714.44103911202	0.000204694321869453\\
1714.71578617769	0.000204521106215553\\
1714.99053324335	0.000204348146775925\\
1715.26528030902	0.000204175115809075\\
1715.54002737468	0.000204002342351003\\
1715.81477444035	0.000203829333643235\\
1716.08952150601	0.00020365675042505\\
1716.36426857168	0.000203483604158638\\
1716.63901563734	0.000203311051219066\\
1716.91376270301	0.000203138437677104\\
1717.18850976867	0.000202966245701195\\
1717.46325683434	0.000202793832382806\\
1717.7380039	0.000202621681908703\\
1718.01275096567	0.000202449631872519\\
1718.28749803133	0.000202277673536485\\
1718.562245097	0.000202105989803432\\
1718.83699216266	0.000201934404003261\\
1719.11173922833	0.000201762589111586\\
1719.38648629399	0.000201591192165209\\
1719.66123335966	0.00020141974790335\\
1719.93598042532	0.000201248078419426\\
1720.21072749099	0.000201077007205508\\
1720.48547455665	0.00020090570874688\\
1720.76022162232	0.000200734844033606\\
1721.03496868798	0.000200563917522639\\
1721.30971575364	0.000200393250529529\\
1721.58446281931	0.000200222370319283\\
1721.85920988497	0.000200051926421208\\
1722.13395695064	0.000199881256690454\\
1722.4087040163	0.000199710532672357\\
1722.68345108197	0.000199540568739413\\
1722.95819814763	0.000199369899631592\\
1723.2329452133	0.000199199672017493\\
1723.50769227896	0.000199029883121916\\
1723.78243934463	0.000198860197007177\\
1724.05718641029	0.000198690625497922\\
1724.33193347596	0.00019852116121425\\
1724.60668054162	0.000198351936176877\\
1724.88142760729	0.000198182689598448\\
1725.15617467295	0.000198013387716417\\
1725.43092173862	0.000197844196879507\\
1725.70566880428	0.000197675282151313\\
1725.98041586995	0.000197506148662299\\
1726.25516293561	0.000197336966281515\\
1726.52991000128	0.000197168229260462\\
1726.80465706694	0.000196999441132727\\
1727.07940413261	0.000196831098603987\\
1727.35415119827	0.000196662703431408\\
1727.62889826394	0.000196493931587275\\
1727.9036453296	0.000196325937869053\\
1728.17839239527	0.000196157892788403\\
1728.45313946093	0.000195989801475535\\
1728.7278865266	0.000195821826585596\\
1729.00263359226	0.000195654300004503\\
1729.27738065793	0.00019548705123923\\
1729.55212772359	0.000195319431607008\\
1729.82687478925	0.000195152255818906\\
1730.10162185492	0.000194985203103121\\
1730.37636892058	0.000194818570855579\\
1730.65111598625	0.000194651592449409\\
1730.92586305191	0.000194484734007227\\
1731.20061011758	0.000194317995893037\\
1731.47535718324	0.000194151379506397\\
1731.75010424891	0.00019398488258672\\
1732.02485131457	0.000193818506753035\\
1732.29959838024	0.000193652254058618\\
1732.5743454459	0.000193486124232983\\
1732.84909251157	0.000193320280507547\\
1733.12383957723	0.000193154230601958\\
1733.3985866429	0.000192988141655595\\
1733.67333370856	0.000192822179891527\\
1733.94808077423	0.000192656185328339\\
1734.22282783989	0.000192490803937678\\
1734.49757490556	0.000192325724448516\\
1734.77232197122	0.000192160768051806\\
1735.04706903689	0.000191995771606781\\
1735.32181610255	0.000191830900667609\\
1735.59656316822	0.000191666320410337\\
1735.87131023388	0.00019150199748237\\
1736.14605729955	0.000191337345677501\\
1736.42080436521	0.000191172987269958\\
1736.69555143088	0.000191008921043492\\
1736.97029849654	0.000190844653224586\\
1737.24504556221	0.000190680680791049\\
1737.51979262787	0.000190516836800121\\
1737.79453969354	0.000190353286568164\\
1738.0692867592	0.000190189699278438\\
1738.34403382487	0.00019002624351372\\
1738.61878089053	0.000189863081943436\\
1738.89352795619	0.000189700049181007\\
1739.16827502186	0.000189536981356131\\
1739.44302208752	0.000189374046104661\\
1739.71776915319	0.000189211241594133\\
1739.99251621885	0.000189048734432605\\
1740.26726328452	0.000188886357892901\\
1740.54201035018	0.000188723948622543\\
1740.81675741585	0.000188561834534988\\
1741.09150448151	0.000188399695683082\\
1741.36625154718	0.000188237524378955\\
1741.64099861284	0.000188075332916006\\
1741.91574567851	0.000187913764998645\\
1742.19049274417	0.00018775217663849\\
1742.46523980984	0.000187590725431957\\
1742.7399868755	0.000187429409942763\\
1743.01473394117	0.000187268065340536\\
1743.28948100683	0.000187107192571519\\
1743.5642280725	0.000186946618612781\\
1743.83897513816	0.000186786008678749\\
1744.11372220383	0.000186625547469486\\
1744.38846926949	0.0001864650574982\\
1744.66321633516	0.000186304705625972\\
1744.93796340082	0.000186144657005968\\
1745.21271046649	0.000185984566287349\\
1745.48745753215	0.000185824797881906\\
1745.76220459782	0.000185664838965772\\
1746.03695166348	0.000185505349983968\\
1746.31169872915	0.000185345672531623\\
1746.58644579481	0.00018518613757596\\
1746.86119286047	0.000185027072171766\\
1747.13593992614	0.00018486829004019\\
1747.4106869918	0.000184709504592128\\
1747.68543405747	0.000184550364399487\\
1747.96018112313	0.000184391864919306\\
1748.2349281888	0.000184233340899689\\
1748.50967525446	0.000184075124051948\\
1748.78442232013	0.000183917047856242\\
1749.05916938579	0.000183758947923038\\
1749.33391645146	0.000183600829108936\\
1749.60866351712	0.000183443183329152\\
1749.88341058279	0.000183285775131986\\
1750.15815764845	0.000183128414609497\\
1750.43290471412	0.000182971033124641\\
1750.70765177978	0.000182813795703011\\
1750.98239884545	0.000182656867731195\\
1751.25714591111	0.000182499591743236\\
1751.53189297678	0.00018234246498639\\
1751.80664004244	0.000182185652104438\\
1752.08138710811	0.000182029153113446\\
1752.35613417377	0.000181872798904267\\
1752.63088123944	0.000181716592089338\\
1752.9056283051	0.000181560368111945\\
1753.18037537077	0.000181404457619486\\
1753.45512243643	0.000181248695170772\\
1753.7298695021	0.000181092916124161\\
1754.00461656776	0.000180937451175107\\
1754.27936363343	0.000180782132956342\\
1754.55411069909	0.000180627128043957\\
1754.82885776476	0.000180471779497225\\
1755.10360483042	0.000180316746750774\\
1755.37835189609	0.000180161865036611\\
1755.65309896175	0.000180007297539322\\
1755.92784602741	0.000179853044327016\\
1756.20259309308	0.000179698938566198\\
1756.47734015874	0.000179544980512523\\
1756.75208722441	0.000179391170419101\\
1757.02683429007	0.000179237180522628\\
1757.30158135574	0.000179083671518131\\
1757.5763284214	0.000178930147204305\\
1757.85107548707	0.000178776775081653\\
1758.12582255273	0.00017862371712703\\
1758.4005696184	0.000178470808488605\\
1758.67531668406	0.000178318049402971\\
1758.95006374973	0.000178165275922397\\
1759.22481081539	0.00017801249006002\\
1759.49955788106	0.000177859858996699\\
1759.77430494672	0.000177707547988502\\
1760.04905201239	0.000177555553146313\\
1760.32379907805	0.000177403709058\\
1760.59854614372	0.000177251852102289\\
1760.87329320938	0.000177100266807997\\
1761.14804027505	0.000176948718193501\\
1761.42278734071	0.000176797159697806\\
1761.69753440638	0.000176645757556617\\
1761.97228147204	0.000176494675426578\\
1762.24702853771	0.000176343911580825\\
1762.52177560337	0.000176193137120602\\
1762.79652266904	0.000176042516265789\\
1763.0712697347	0.000175892050477405\\
1763.34601680037	0.000175741739073614\\
1763.62076386603	0.000175591747148643\\
1763.8955109317	0.000175441250859088\\
1764.17025799736	0.000175291247518334\\
1764.44500506303	0.000175141565341713\\
1764.71975212869	0.000174991873560809\\
1764.99449919435	0.000174842668597525\\
1765.26924626002	0.000174693617449784\\
1765.54399332568	0.000174544556716262\\
1765.81874039135	0.000174395815944675\\
1766.09348745701	0.000174247229496299\\
1766.36823452268	0.00017409879754472\\
1766.64298158834	0.000173950520260895\\
1766.91772865401	0.000173802069460389\\
1767.19247571967	0.000173653615292574\\
1767.46722278534	0.000173505975506899\\
1767.741969851	0.000173358331603589\\
1768.01671691667	0.00017321083433998\\
1768.29146398233	0.000173063505657095\\
1768.566211048	0.000172916332659258\\
1768.84095811366	0.000172769152417285\\
1769.11570517933	0.000172622128907321\\
1769.39045224499	0.000172475401894251\\
1769.66519931066	0.00017232869855479\\
1769.93994637632	0.000172182316776552\\
1770.21469344199	0.000172035926586038\\
1770.48944050765	0.00017188986011892\\
1770.76418757332	0.000171743622334743\\
1771.03893463898	0.000171597381131686\\
1771.31368170465	0.000171451468746662\\
1771.58842877031	0.000171305880937564\\
1771.86317583598	0.000171160616682726\\
1772.13792290164	0.000171015017156409\\
1772.41266996731	0.000170870073945242\\
1772.68741703297	0.000170725125122143\\
1772.96216409863	0.000170580334714099\\
1773.2369111643	0.000170435851930647\\
1773.51165822996	0.000170291219512166\\
1773.78640529563	0.000170146912155558\\
1774.06115236129	0.000170002436481127\\
1774.33589942696	0.000169858454610676\\
1774.61064649262	0.000169714467571896\\
1774.88539355829	0.000169570807418911\\
1775.16014062395	0.000169426980807234\\
1775.43488768962	0.000169283645707242\\
1775.70963475528	0.000169140779392095\\
1775.98438182095	0.000168997928891768\\
1776.25912888661	0.000168855071989492\\
1776.53387595228	0.000168712376655844\\
1776.80862301794	0.000168569675642943\\
1777.08337008361	0.00016842746742967\\
1777.35811714927	0.000168285417692578\\
1777.63286421494	0.000168143361761869\\
1777.9076112806	0.000168001467324152\\
1778.18235834627	0.000167859897888736\\
1778.45710541193	0.000167718154103033\\
1778.7318524776	0.000167576746697803\\
1779.00659954326	0.00016743566462887\\
1779.28134660893	0.000167294737420988\\
1779.55609367459	0.000167153977507825\\
1779.83084074026	0.00016701271918567\\
1780.10558780592	0.000166872277333299\\
1780.38033487159	0.000166732007579941\\
1780.65508193725	0.000166591896854565\\
1780.92982900292	0.000166451782300459\\
1781.20457606858	0.000166311991982697\\
1781.47932313425	0.000166172507792597\\
1781.75407019991	0.000166032873866216\\
1782.02881726557	0.000165893075193293\\
1782.30356433124	0.000165753932596758\\
1782.5783113969	0.000165614458063099\\
1782.85305846257	0.000165475640581276\\
1783.12780552823	0.000165336982315254\\
1783.4025525939	0.000165198483296543\\
1783.67729965956	0.00016505981695985\\
1783.95204672523	0.000164921802629958\\
1784.22679379089	0.000164783788069377\\
1784.50154085656	0.00016464574907961\\
1784.77628792222	0.000164507895150231\\
1785.05103498789	0.000164370039054269\\
1785.32578205355	0.000164232510552721\\
1785.60052911922	0.000164094979429957\\
1785.87527618488	0.000163957940356666\\
1786.15002325055	0.000163820730911291\\
1786.42477031621	0.000163683686413953\\
1786.69951738188	0.000163547133765486\\
1786.97426444754	0.000163410410949702\\
1787.24901151321	0.000163274141751539\\
1787.52375857887	0.000163137911111565\\
1787.79850564454	0.000163002005285847\\
1788.0732527102	0.000162866095159021\\
1788.34799977587	0.000162730509507397\\
1788.62274684153	0.0001625949199117\\
1788.8974939072	0.000162459326457918\\
1789.17224097286	0.000162324224165415\\
1789.44698803853	0.00016218895346435\\
1789.72173510419	0.000162054008272066\\
1789.99648216985	0.00016191938889483\\
1790.27122923552	0.000161784928162825\\
1790.54597630118	0.000161650463164688\\
1790.82072336685	0.000161515994123637\\
1791.09547043251	0.00016138201572269\\
1791.37021749818	0.000161248031002067\\
1791.64496456384	0.000161114371636692\\
1791.91971162951	0.000160980705720222\\
1792.19445869517	0.000160847365168477\\
1792.46920576084	0.000160714019006177\\
1792.7439528265	0.00016058099701101\\
1793.01869989217	0.0001604482969411\\
1793.29344695783	0.000160315758074155\\
1793.5681940235	0.000160183373174155\\
1793.84294108916	0.000160050989861107\\
1794.11768815483	0.000159918764470957\\
1794.39243522049	0.000159786861459015\\
1794.66718228616	0.000159654768507671\\
1794.94192935182	0.000159523025166837\\
1795.21667641749	0.000159391276271012\\
1795.49142348315	0.00015925968746237\\
1795.76617054882	0.000159128421215307\\
1796.04091761448	0.000158996984633526\\
1796.31566468015	0.000158865874461325\\
1796.59041174581	0.000158734594476539\\
1796.86515881148	0.000158603970307471\\
1797.13990587714	0.000158473503001941\\
1797.41465294281	0.000158343028491515\\
1797.68940000847	0.000158212856954998\\
1797.96414707414	0.000158082702628011\\
1798.2388941398	0.000157952377698608\\
1798.51364120547	0.000157822543791964\\
1798.78838827113	0.000157692869857248\\
1799.06313533679	0.000157563023387345\\
1799.33788240246	0.000157433669226446\\
1799.61262946812	0.000157304145131697\\
1799.88737653379	0.00015717527510991\\
1800.16212359945	0.000157046395657675\\
1800.43687066512	0.000156917979731486\\
1800.71161773078	0.000156789251678273\\
1800.98636479645	0.000156660847507408\\
1801.26111186211	0.000156532271542791\\
1801.53585892778	0.000156404185217558\\
1801.81060599344	0.000156275928726856\\
1802.08535305911	0.00015614815908943\\
1802.36010012477	0.000156020546019645\\
1802.63484719044	0.000155893090371417\\
1802.9095942561	0.000155765789951116\\
1803.18434132177	0.000155638809670312\\
1803.45908838743	0.000155511654977197\\
1803.7338354531	0.000155384896875802\\
1804.00858251876	0.000155258220447357\\
1804.28332958443	0.000155131534195421\\
1804.55807665009	0.000155005007622686\\
1804.83282371576	0.00015487879949425\\
1805.10757078142	0.000154752747596701\\
1805.38231784709	0.000154626850404244\\
1805.65706491275	0.000154501107439059\\
1805.93181197842	0.000154375518798274\\
1806.20655904408	0.000154250082250656\\
1806.48130610975	0.000154124800051936\\
1806.75605317541	0.000153999507963606\\
1807.03080024108	0.000153874698757281\\
1807.30554730674	0.000153750040436166\\
1807.5802943724	0.000153625368950569\\
1807.85504143807	0.000153501014099841\\
1808.12978850373	0.00015337680959325\\
1808.4045355694	0.000153252755255006\\
1808.67928263506	0.000153128687987762\\
1808.95402970073	0.000153004935644642\\
1809.22877676639	0.000152881005070211\\
1809.50352383206	0.000152757065659802\\
1809.77827089772	0.000152633608294886\\
1810.05301796339	0.00015251046626544\\
1810.32776502905	0.000152387472911973\\
1810.60251209472	0.000152264628040971\\
1810.87725916038	0.000152141931456437\\
1811.15200622605	0.000152019382959893\\
1811.42675329171	0.000151896982350396\\
1811.70150035738	0.00015177440049099\\
1811.97624742304	0.000151652299473802\\
1812.25099448871	0.000151529854536433\\
1812.52574155437	0.000151407889786465\\
1812.80048862004	0.00015128590984522\\
1813.0752356857	0.000151164245161453\\
1813.34998275137	0.000151042565177301\\
1813.62472981703	0.000150921200142964\\
1813.8994768827	0.00015079998304803\\
1814.17422394836	0.000150679076953333\\
1814.44897101403	0.000150558152489376\\
1814.72371807969	0.00015043721089607\\
1814.99846514536	0.000150316581281886\\
1815.27321221102	0.000150195770152135\\
1815.54795927669	0.000150075439188026\\
1815.82270634235	0.000149955253933951\\
1816.09745340801	0.000149835215133975\\
1816.37220047368	0.000149715156418267\\
1816.64694753934	0.000149595410764924\\
1816.92169460501	0.000149475809366728\\
1817.19644167067	0.000149355695624409\\
1817.47118873634	0.000149236392177884\\
1817.745935802	0.000149117396509541\\
1818.02068286767	0.000148998214418985\\
1818.29542993333	0.000148879505962677\\
1818.570176999	0.000148760938469666\\
1818.84492406466	0.000148642347213094\\
1819.11967113033	0.000148523898841929\\
1819.39441819599	0.000148405102233109\\
1819.66916526166	0.000148287253725514\\
1819.94391232732	0.000148169078945337\\
1820.21865939299	0.000148051375500453\\
1820.49340645865	0.000147933646364846\\
1820.76815352432	0.000147816059902333\\
1821.04290058998	0.000147698613220759\\
1821.31764765565	0.000147581307290255\\
1821.59239472131	0.000147464141556695\\
1821.86714178698	0.000147346950021597\\
1822.14188885264	0.000147230064000205\\
1822.41663591831	0.000147113153800801\\
1822.69138298397	0.000146996548239073\\
1822.96613004964	0.000146880244724475\\
1823.2408771153	0.000146763911781098\\
1823.51562418097	0.000146647728011769\\
1823.79037124663	0.000146531844880125\\
1824.0651183123	0.000146415933021873\\
1824.33986537796	0.000146300308785982\\
1824.61461244362	0.000146184836124744\\
1824.88935950929	0.000146069005524142\\
1825.16410657495	0.00014595347886184\\
1825.43885364062	0.000145837928040421\\
1825.71360070628	0.000145723007251389\\
1825.98834777195	0.000145608219098958\\
1826.26309483761	0.000145493692906577\\
1826.53784190328	0.000145379006851659\\
1826.81258896894	0.000145264618816462\\
1827.08733603461	0.000145150362192444\\
1827.36208310027	0.00014503590776001\\
1827.63683016594	0.000144921917456978\\
1827.9115772316	0.000144808057572987\\
1828.18632429727	0.000144694327792493\\
1828.46107136293	0.000144580727797889\\
1828.7358184286	0.000144467257269514\\
1829.01056549426	0.000144353751052761\\
1829.28531255993	0.00014424054070979\\
1829.56005962559	0.000144127129663248\\
1829.83480669126	0.000144014180360752\\
1830.10955375692	0.000143901194226514\\
1830.38430082259	0.00014378833728656\\
1830.65904788825	0.000143675774663803\\
1830.93379495392	0.000143563338741286\\
1831.20854201958	0.000143451165560022\\
1831.48328908525	0.000143338490214062\\
1831.75803615091	0.000143226276565429\\
1832.03278321658	0.000143114190535934\\
1832.30753028224	0.000143001411047268\\
1832.58227734791	0.00014288942628454\\
1832.85702441357	0.000142777568998884\\
1833.13177147924	0.000142666168654621\\
1833.4065185449	0.000142554562339994\\
1833.68126561056	0.000142443412533665\\
1833.95601267623	0.000142332385199423\\
1834.23075974189	0.000142221316415049\\
1834.50550680756	0.000142110535174303\\
1834.78025387322	0.000141999712361558\\
1835.05500093889	0.000141888991677876\\
1835.32974800455	0.000141778417982474\\
1835.60449507022	0.000141667966331107\\
1835.87924213588	0.000141557801287193\\
1836.15398920155	0.000141447592173601\\
1836.42873626721	0.000141337667922039\\
1836.70348333288	0.000141227534032633\\
1836.97823039854	0.000141117522435206\\
1837.25297746421	0.000141007962039561\\
1837.52772452987	0.000140898354447825\\
1837.80247159554	0.000140788866361357\\
1838.0772186612	0.000140679812681242\\
1838.35196572687	0.000140570726494662\\
1838.62671279253	0.00014046159234829\\
1838.9014598582	0.000140352739963203\\
1839.17620692386	0.00014024334779235\\
1839.45095398953	0.000140134569523478\\
1839.72570105519	0.000140026072540178\\
1840.00044812086	0.000139917839904863\\
1840.27519518652	0.000139809408739352\\
1840.54994225219	0.000139701256670626\\
1840.82468931785	0.000139593054051069\\
1841.09943638352	0.00013948512933617\\
1841.37418344918	0.000139377152549721\\
1841.64893051485	0.000139269453590552\\
1841.92367758051	0.000139162183316469\\
1842.19842464617	0.000139054715054893\\
1842.47317171184	0.000138947029483271\\
1842.7479187775	0.000138839622903284\\
1843.02266584317	0.000138732491310455\\
1843.29741290883	0.000138625303852834\\
1843.5721599745	0.000138518392410196\\
1843.84690704016	0.000138411416612498\\
1844.12165410583	0.000138304564770533\\
1844.39640117149	0.000138197985850786\\
1844.67114823716	0.0001380915140116\\
1844.94589530282	0.000137985148850554\\
1845.22064236849	0.000137878714509708\\
1845.49538943415	0.000137772568004206\\
1845.77013649982	0.00013766652693239\\
1846.04488356548	0.000137560427381344\\
1846.31963063115	0.000137454598158342\\
1846.59437769681	0.000137348708455781\\
1846.86912476248	0.000137243243454843\\
1847.14387182814	0.000137137237947928\\
1847.41861889381	0.000137031833909595\\
1847.69336595947	0.000136926369293031\\
1847.96811302514	0.000136821009028919\\
1848.2428600908	0.000136715424002265\\
1848.51760715647	0.000136610596940277\\
1848.79235422213	0.000136505713957313\\
1849.0671012878	0.000136400931232754\\
1849.34184835346	0.000136295884900783\\
1849.61659541913	0.000136191309825108\\
1849.89134248479	0.000136086833690289\\
1850.16608955046	0.000135982456076353\\
1850.44083661612	0.00013587784915405\\
1850.71558368178	0.000135773342574281\\
1850.99033074745	0.000135668773661343\\
1851.26507781311	0.000135564638214489\\
1851.53982487878	0.000135460764191196\\
1851.81457194444	0.000135356985926679\\
1852.08931901011	0.000135252974948255\\
1852.36406607577	0.000135149391345304\\
1852.63881314144	0.00013504590216752\\
1852.9135602071	0.000134942506987624\\
1853.18830727277	0.000134839205377349\\
1853.46305433843	0.000134735668583195\\
1853.7378014041	0.000134632393182976\\
1854.01254846976	0.000134529211659507\\
1854.28729553543	0.000134426288737573\\
1854.56204260109	0.000134323615714387\\
1854.83678966676	0.000134220548007254\\
1855.11153673242	0.000134117740864633\\
1855.38628379809	0.000134015186741762\\
1855.66103086375	0.000133912564955514\\
1855.93577792942	0.000133810334822136\\
1856.21052499508	0.000133707895836752\\
1856.48527206075	0.00013360571046516\\
1856.76001912641	0.000133503449617392\\
1857.03476619208	0.000133401112485054\\
1857.30951325774	0.000133299195630785\\
1857.58426032341	0.000133197200334159\\
1857.85900738907	0.000133095457954287\\
1858.13375445474	0.000132993636350065\\
1858.4085015204	0.000132892066633745\\
1858.68324858607	0.00013279058145332\\
1858.95799565173	0.000132689180368958\\
1859.2327427174	0.000132587862940152\\
1859.50748978306	0.00013248613601413\\
1859.78223684872	0.000132384826854536\\
1860.05698391439	0.000132283751196633\\
1860.33173098005	0.000132182616670955\\
1860.60647804572	0.00013208173036449\\
1860.88122511138	0.000131980760035469\\
1861.15597217705	0.000131879873270327\\
1861.43071924271	0.000131779068567662\\
1861.70546630838	0.00013167834460742\\
1861.98021337404	0.000131577702335908\\
1862.25496043971	0.000131477305372346\\
1862.52970750537	0.000131377141312558\\
1862.80445457104	0.000131276739076726\\
1863.0792016367	0.00013117641487395\\
1863.35394870237	0.000131076334078073\\
1863.62869576803	0.000130976329545645\\
1863.9034428337	0.00013087639628537\\
1864.17818989936	0.000130776384317428\\
1864.45293696503	0.000130676448878342\\
1864.72768403069	0.000130576425103112\\
1865.00243109636	0.000130476926319738\\
1865.27717816202	0.000130377219915088\\
1865.55192522769	0.000130277426342573\\
1865.82667229335	0.000130177872830474\\
1866.10141935902	0.000130078557757467\\
1866.37616642468	0.000129979314300335\\
1866.65091349035	0.00012988014201497\\
1866.92566055601	0.000129781040456991\\
1867.20040762168	0.000129681681320938\\
1867.47515468734	0.000129582560593821\\
1867.74990175301	0.000129483675116558\\
1868.02464881867	0.000129384695575253\\
1868.29939588433	0.000129285949578226\\
1868.57414295	0.000129186943077965\\
1868.84889001566	0.000129088336955927\\
1869.12363708133	0.000128989797836077\\
1869.39838414699	0.000128890997101979\\
1869.67313121266	0.000128792595075098\\
1869.94787827832	0.000128694094304007\\
1870.22262534399	0.000128595825274075\\
1870.49737240965	0.000128497129587325\\
1870.77211947532	0.000128398996129322\\
1871.04686654098	0.000128300926397739\\
1871.32161360665	0.000128202919949768\\
1871.59636067231	0.000128104976342569\\
1871.87110773798	0.000128007254759765\\
1872.14585480364	0.000127909437687361\\
1872.42060186931	0.000127811355530051\\
1872.69534893497	0.000127713817253362\\
1872.97009600064	0.000127616025728677\\
1873.2448430663	0.000127518132428282\\
1873.51959013197	0.000127420631610792\\
1873.79433719763	0.000127322863188626\\
1874.0690842633	0.000127225484772149\\
1874.34383132896	0.000127127837199317\\
1874.61857839463	0.00012703038738516\\
1874.89332546029	0.000126933029574675\\
1875.16807252596	0.000126835566626525\\
1875.44281959162	0.000126738492351272\\
1875.71756665729	0.000126641473997146\\
1875.99231372295	0.000126544511128337\\
1876.26706078862	0.000126447274305897\\
1876.54180785428	0.00012635042553288\\
1876.81655491994	0.00012625330251642\\
1877.09130198561	0.000126156565979149\\
1877.36604905127	0.000126059555000919\\
1877.64079611694	0.00012596292884305\\
1877.9155431826	0.000125866354962479\\
1878.19029024827	0.000125769505537241\\
1878.46503731393	0.00012567318727744\\
1878.7397843796	0.000125576610861961\\
1879.01453144526	0.000125480250332905\\
1879.28927851093	0.000125383939869412\\
1879.56402557659	0.000125287515702942\\
1879.83877264226	0.000125191306285597\\
1880.11351970792	0.000125095145620576\\
1880.38826677359	0.000124998870301192\\
1880.66301383925	0.00012490280805586\\
1880.93776090492	0.000124806947434595\\
1881.21250797058	0.000124710981763137\\
1881.48725503625	0.000124615062711655\\
1881.76200210191	0.000124519189855716\\
1882.03674916758	0.000124423362771389\\
1882.31149623324	0.000124327416618236\\
1882.58624329891	0.000124231839636373\\
1882.86099036457	0.000124135991473787\\
1883.13573743024	0.000124040354232126\\
1883.4104844959	0.000123944760624345\\
1883.68523156157	0.000123849210231469\\
1883.95997862723	0.000123753523020067\\
1884.2347256929	0.000123658064256105\\
1884.50947275856	0.000123562647424141\\
1884.78421982423	0.000123467107497147\\
1885.05896688989	0.000123371611787401\\
1885.33371395556	0.000123276322656163\\
1885.60846102122	0.000123180908724264\\
1885.88320808688	0.000123085701840013\\
1886.15795515255	0.000122990490975424\\
1886.43270221821	0.00012289520328504\\
1886.70744928388	0.000122800119329425\\
1886.98219634954	0.000122704909065196\\
1887.25694341521	0.000122609903126178\\
1887.53169048087	0.00012251493445002\\
1887.80643754654	0.000122419839103565\\
1888.0811846122	0.000122324945941515\\
1888.35593167787	0.000122230088800879\\
1888.63067874353	0.000122134940590614\\
1888.9054258092	0.000122039994905204\\
1889.18017287486	0.000121945085199661\\
1889.45491994053	0.000121850049531571\\
1889.72966700619	0.000121755379466565\\
1890.00441407186	0.000121660578094783\\
1890.27916113752	0.000121565976762327\\
1890.55390820319	0.000121471243605076\\
1890.82865526885	0.000121376709367635\\
1891.10340233452	0.000121282206953053\\
1891.37814940018	0.000121187407186823\\
1891.65289646585	0.000121092645164313\\
1891.92764353151	0.000120997916779513\\
1892.20239059718	0.000120903058975769\\
1892.47713766284	0.000120808564881021\\
1892.75188472851	0.000120714380206723\\
1893.02663179417	0.000120619948947785\\
1893.30137885984	0.000120525711894911\\
1893.5761259255	0.000120431502907613\\
1893.85087299117	0.000120337321607556\\
1894.12562005683	0.00012024300381931\\
1894.40036712249	0.00012014871567867\\
1894.67511418816	0.000120054292802187\\
1894.94986125382	0.000119960063772273\\
1895.22460831949	0.000119866026089745\\
1895.49935538515	0.000119771849787003\\
1895.77410245082	0.000119677864548584\\
1896.04884951648	0.000119583903847291\\
1896.32359658215	0.00011948980444651\\
1896.59834364781	0.000119395730624142\\
1896.87309071348	0.00011930184570685\\
1897.14783777914	0.000119207820689276\\
1897.42258484481	0.000119113983653044\\
1897.69733191047	0.000119020168889346\\
1897.97207897614	0.000118926376039441\\
1898.2468260418	0.000118832440085095\\
1898.52157310747	0.000118738364033251\\
1898.79632017313	0.00011864431241184\\
1899.0710672388	0.000118550450510862\\
1899.34581430446	0.000118456610380198\\
1899.62056137013	0.000118362956629901\\
1899.89530843579	0.000118269322125263\\
1900.17005550146	0.000118175214765169\\
1900.44480256712	0.000118081777706848\\
1900.71954963279	0.000117988207507894\\
1900.99429669845	0.000117894655119992\\
1901.26904376412	0.000117800957292791\\
1901.54379082978	0.00011770727796049\\
1901.81853789545	0.00011761328830508\\
1902.09328496111	0.000117519279918824\\
1902.36803202678	0.000117425669349503\\
1902.64277909244	0.00011733174838357\\
1902.9175261581	0.000117238012000975\\
1903.19227322377	0.000117144296304316\\
1903.46702028943	0.000117050761811189\\
1903.7417673551	0.000116957245056075\\
1904.01651442076	0.000116863743110641\\
1904.29126148643	0.000116770093999104\\
1904.56600855209	0.000116676298353455\\
1904.84075561776	0.000116583013257485\\
1905.11550268342	0.000116489740385626\\
1905.39024974909	0.000116396151481864\\
1905.66499681475	0.000116302579143797\\
1905.93974388042	0.00011620934985501\\
1906.21449094608	0.000116116131441821\\
1906.48923801175	0.000116022760406246\\
1906.76398507741	0.000115929400153971\\
1907.03873214308	0.000115835888296153\\
1907.31347920874	0.000115742391787357\\
1907.58822627441	0.000115648909584593\\
1907.86297334007	0.00011555576738032\\
1908.13772040574	0.000115462470904151\\
1908.4124674714	0.0001153691827595\\
1908.68721453707	0.000115276234505215\\
1908.96196160273	0.000115183261174812\\
1909.2367086684	0.000115090004131254\\
1909.51145573406	0.000114996754833773\\
1909.78620279973	0.000114903515632235\\
1910.06094986539	0.000114810121550501\\
1910.33569693106	0.000114716900556461\\
1910.61044399672	0.000114623522464592\\
1910.88519106239	0.000114530156565149\\
1911.15993812805	0.000114436799799979\\
1911.43468519372	0.000114343451819149\\
1911.70943225938	0.000114250441417692\\
1911.98417932504	0.000114157270827609\\
1912.25892639071	0.000114063942877436\\
1912.53367345637	0.000113970788064501\\
1912.80842052204	0.00011387780175928\\
1913.0831675877	0.00011378481853426\\
1913.35791465337	0.000113691673666964\\
1913.63266171903	0.000113598698048015\\
1913.9074087847	0.000113505876995737\\
1914.18215585036	0.000113412907783569\\
1914.45690291603	0.000113319777464766\\
1914.73164998169	0.000113226649854831\\
1915.00639704736	0.000113133526680379\\
1915.28114411302	0.000113040243273401\\
1915.55589117869	0.000112947036079408\\
1915.83063824435	0.000112853925154428\\
1916.10538531002	0.000112760981656469\\
1916.38013237568	0.000112667546776371\\
1916.65487944135	0.000112574609467932\\
1916.92962650701	0.000112481671511149\\
1917.20437357268	0.000112388732676155\\
1917.47912063834	0.000112295792734754\\
1917.75386770401	0.000112202851460422\\
1918.02861476967	0.000112109745463834\\
1918.30336183534	0.000112016803049809\\
1918.578108901	0.000111922876382129\\
1918.85285596667	0.00011182977937574\\
1919.12760303233	0.000111736518069983\\
1919.402350098	0.000111643585715432\\
1919.67709716366	0.000111550321995326\\
1919.95184422932	0.000111457387564489\\
1920.22659129499	0.000111364449515955\\
1920.50133836065	0.000111271507644196\\
1920.77608542632	0.000111178397950136\\
1921.05083249198	0.000111085450029204\\
1921.32557955765	0.000110992334414934\\
1921.60032662331	0.000110899379580931\\
1921.87507368898	0.000110806419866972\\
1922.14982075464	0.000110713455078619\\
1922.42456782031	0.000110620485023195\\
1922.69931488597	0.000110527509509787\\
1922.97406195164	0.000110434200331898\\
1923.2488090173	0.000110341053576711\\
1923.52355608297	0.000110247740632249\\
1923.79830314863	0.000110154424243717\\
1924.0730502143	0.000110061433813249\\
1924.34779727996	0.000109968273432198\\
1924.62254434563	0.000109875271508725\\
1924.89729141129	0.000109782098749275\\
1925.17203847696	0.000109689084586653\\
1925.44678554262	0.000109595899812624\\
1925.72153260829	0.000109502543580088\\
1925.99627967395	0.00010940951299734\\
1926.27102673962	0.000109316310729105\\
1926.54577380528	0.000109223266009758\\
1926.82052087095	0.000109130049868824\\
1927.09526793661	0.000109036990309752\\
1927.37001500228	0.00010894392196977\\
1927.64476206794	0.000108850679824715\\
1927.91950913361	0.000108757595683756\\
1928.19425619927	0.000108664337799017\\
1928.46900326494	0.000108570910446671\\
1928.7437503306	0.000108477804659488\\
1929.01849739626	0.000108384689085538\\
1929.29324446193	0.000108291563583375\\
1929.56799152759	0.000108198428013409\\
1929.84273859326	0.000108105117654602\\
1930.11748565892	0.000108011799431041\\
1930.39223272459	0.000107918472482668\\
1930.66697979025	0.000107824974489771\\
1930.94172685592	0.000107731632806361\\
1931.21647392158	0.000107638283509231\\
1931.49122098725	0.000107544762307391\\
1931.76596805291	0.000107451560615126\\
1932.04071511858	0.000107358442013061\\
1932.31546218424	0.000107264726451552\\
1932.59020924991	0.000107171497939453\\
1932.86495631557	0.00010707825752995\\
1933.13970338124	0.000106985005111982\\
1933.4144504469	0.00010689157743467\\
1933.68919751257	0.000106798138847142\\
1933.96394457823	0.000106704691080388\\
1934.2386916439	0.000106611396078604\\
1934.51343870956	0.000106517924951576\\
1934.78818577523	0.000106424442479208\\
1935.06293284089	0.000106330950558363\\
1935.33767990656	0.000106237284000689\\
1935.61242697222	0.000106143935797174\\
1935.88717403789	0.000106050574363476\\
1936.16192110355	0.000105956873028613\\
1936.43666816922	0.000105863324745468\\
1936.71141523488	0.000105769766836497\\
1936.98616230054	0.000105676360298378\\
1937.26090936621	0.000105582940001194\\
1937.53565643187	0.000105489505868137\\
1937.81040349754	0.000105396057824301\\
1938.0851505632	0.000105302595796679\\
1938.35989762887	0.000105209119714176\\
1938.63464469453	0.0001051156295076\\
1938.9093917602	0.00010502196089933\\
1939.18413882586	0.0001049284444569\\
1939.45888589153	0.000104834586751743\\
1939.73363295719	0.000104741045957245\\
1940.00838002286	0.000104647326699145\\
1940.28312708852	0.000104553594107587\\
1940.55787415419	0.00010446001397376\\
1940.83262121985	0.000104366419095609\\
1941.10736828552	0.000104272480268857\\
1941.38211535118	0.000104178860186779\\
1941.65686241685	0.000104084734847143\\
1941.93160948251	0.000103990926615483\\
1942.20635654818	0.000103897270310932\\
1942.48110361384	0.000103803434277899\\
1942.75585067951	0.000103709749789013\\
1943.03059774517	0.000103615720354383\\
1943.30534481084	0.000103521810665412\\
1943.5800918765	0.00010342809086051\\
1943.85483894217	0.000103334191018247\\
1944.12958600783	0.000103240278730295\\
1944.4043330735	0.000103146516860101\\
1944.67908013916	0.000103052739401876\\
1944.95382720483	0.000102958946336641\\
1945.22857427049	0.000102864974504876\\
1945.50332133616	0.000102771152375007\\
1945.77806840182	0.000102677150793095\\
1946.05281546748	0.000102583135014689\\
1946.32756253315	0.000102489106267206\\
1946.60230959881	0.0001023948997283\\
1946.87705666448	0.000102301009991561\\
1947.15180373014	0.000102206940141368\\
1947.42655079581	0.000102113020909098\\
1947.70129786147	0.000102018922817289\\
1947.97604492714	0.000101924809486464\\
1948.2507919928	0.000101830784572207\\
1948.52553905847	0.000101736809841512\\
1948.80028612413	0.000101642819195095\\
1949.0750331898	0.000101548483726881\\
1949.34978025546	0.000101454465720612\\
1949.62452732113	0.000101359938665993\\
1949.89927438679	0.00010126540447763\\
1950.17402145246	0.000101171351924688\\
1950.44876851812	0.000101077120093132\\
1950.72351558379	0.000100983037706901\\
1950.99826264945	0.000100888775966133\\
1951.27300971512	0.000100794498865445\\
1951.54775678078	0.000100700044681711\\
1951.82250384645	0.000100605579711945\\
1952.09725091211	0.000100511430706932\\
1952.37199797778	0.000100417265707706\\
1952.64674504344	0.000100322593760895\\
1952.92149210911	0.000100228240224266\\
1953.19623917477	0.000100134036235836\\
1953.47098624044	0.000100039816322506\\
1953.7457333061	9.9945580535127e-05\\
1954.02048037177	9.98511651037162e-05\\
1954.29522743743	9.97567351904776e-05\\
1954.56997450309	9.9662293490597e-05\\
1954.84472156876	9.9568001291799e-05\\
1955.11946863442	9.94736934178087e-05\\
1955.39421570009	9.93793699313954e-05\\
1955.66896276575	9.92848660724849e-05\\
1955.94370983142	9.91905137767528e-05\\
1956.21845689708	9.90959831649987e-05\\
1956.49320396275	9.9001438708398e-05\\
1956.76795102841	9.89068815007374e-05\\
1957.04269809408	9.88124743842577e-05\\
1957.31744515974	9.8718052036245e-05\\
1957.59219222541	9.86236145352424e-05\\
1957.86693929107	9.85293098479429e-05\\
1958.14168635674	9.84346804670227e-05\\
1958.4164334224	9.83402022610397e-05\\
1958.69118048807	9.82457092327179e-05\\
1958.96592755373	9.8151037865572e-05\\
1959.2406746194	9.80561899857404e-05\\
1959.51542168506	9.796116836499e-05\\
1959.79016875073	9.78666300741589e-05\\
1960.06491581639	9.77719137848659e-05\\
1960.33966288206	9.76773311945302e-05\\
1960.61440994772	9.75824268347669e-05\\
1960.88915701339	9.74876784940614e-05\\
1961.16390407905	9.73927544281389e-05\\
1961.43865114472	9.72979830185836e-05\\
1961.71339821038	9.72033644354513e-05\\
1961.98814527605	9.7108403102183e-05\\
1962.26289234171	9.70137610822329e-05\\
1962.53763940738	9.69191052465602e-05\\
1962.81238647304	9.6824563875437e-05\\
1963.0871335387	9.67298825406586e-05\\
1963.36188060437	9.66351877733026e-05\\
1963.63662767003	9.65403158585302e-05\\
1963.9113747357	9.64454326455365e-05\\
1964.18612180136	9.63505384615532e-05\\
1964.46086886703	9.62557977667105e-05\\
1964.73561593269	9.61608796793455e-05\\
1965.01036299836	9.60659518939506e-05\\
1965.28511006402	9.59711773549484e-05\\
1965.55985712969	9.58762261540873e-05\\
1965.83460419535	9.57814289043426e-05\\
1966.10935126102	9.56864558321486e-05\\
1966.38409832668	9.55916363712524e-05\\
1966.65884539235	9.54968048906797e-05\\
1966.93359245801	9.54019615367546e-05\\
1967.20833952368	9.53067799830109e-05\\
1967.48308658934	9.52117548251713e-05\\
1967.75783365501	9.51168832855933e-05\\
1968.03258072067	9.5021795682185e-05\\
1968.30732778634	9.49264150922795e-05\\
1968.582074852	9.48315226964016e-05\\
1968.85682191767	9.47366193014623e-05\\
1969.13156898333	9.46413768036292e-05\\
1969.406316049	9.45464563218395e-05\\
1969.68106311466	9.44513611542741e-05\\
1969.95581018033	9.43564219256357e-05\\
1970.23055724599	9.42613087032828e-05\\
1970.50530431166	9.41663513453765e-05\\
1970.78005137732	9.40712193444651e-05\\
1971.05479844299	9.39762445060985e-05\\
1971.32954550865	9.38810972142799e-05\\
1971.60429257432	9.37861055221818e-05\\
1971.87903963998	9.36909413311328e-05\\
1972.15378670564	9.35957696919412e-05\\
1972.42853377131	9.35007550768717e-05\\
1972.70328083697	9.34057316739122e-05\\
1972.97802790264	9.33103730489931e-05\\
1973.2527749683	9.32153370351269e-05\\
1973.52752203397	9.31202927439998e-05\\
1973.80226909963	9.30250765667085e-05\\
1974.0770161653	9.29299909553022e-05\\
1974.35176323096	9.28347664027753e-05\\
1974.62651029663	9.27396993668858e-05\\
1974.90125736229	9.26446249735193e-05\\
1975.17600442796	9.25493803853292e-05\\
1975.45075149362	9.2453965169496e-05\\
1975.72549855929	9.23588765180997e-05\\
1976.00024562495	9.22636179181861e-05\\
1976.27499269062	9.21681904523464e-05\\
1976.54973975628	9.20727615367706e-05\\
1976.82448682195	9.19776585578497e-05\\
1977.09923388761	9.18825496563849e-05\\
1977.37398095328	9.17874350482061e-05\\
1977.64872801894	9.16923149505849e-05\\
1977.92347508461	9.15971895822315e-05\\
1978.19822215027	9.1502205893018e-05\\
1978.47296921594	9.14070726611852e-05\\
1978.7477162816	9.13116070446346e-05\\
1979.02246334727	9.12163047057982e-05\\
1979.29721041293	9.11211649872038e-05\\
1979.5719574786	9.10258562461096e-05\\
1979.84670454426	9.09305480939861e-05\\
1980.12145160992	9.08350732764041e-05\\
1980.39619867559	9.07399277606818e-05\\
1980.67094574125	9.06449109824397e-05\\
1980.94569280692	9.05494344886635e-05\\
1981.22043987258	9.04542864331425e-05\\
1981.49518693825	9.03589729626124e-05\\
1981.76993400391	9.02638223756048e-05\\
1982.04468106958	9.01686697611375e-05\\
1982.31942813524	9.00731864594044e-05\\
1982.59417520091	8.99780349563598e-05\\
1982.86892226657	8.98825541982286e-05\\
1983.14366933224	8.97872408820929e-05\\
1983.4184163979	8.96919277436651e-05\\
1983.69316346357	8.95967810697343e-05\\
1983.96791052923	8.95016339463052e-05\\
1984.2426575949	8.94064866293424e-05\\
1984.51740466056	8.93113393760858e-05\\
1984.79215172623	8.92161924450451e-05\\
1985.06689879189	8.91208820723376e-05\\
1985.34164585756	8.90258651222252e-05\\
1985.61639292322	8.89307246554944e-05\\
1985.89113998889	8.88355855528035e-05\\
1986.16588705455	8.87402838760273e-05\\
1986.44063412022	8.86451505109006e-05\\
1986.71538118588	8.85500192816825e-05\\
1986.99012825155	8.84547274820646e-05\\
1987.26487531721	8.83596035306687e-05\\
1987.53962238288	8.82643180719493e-05\\
1987.81436944854	8.81692024487652e-05\\
1988.08911651421	8.80740902654398e-05\\
1988.36386357987	8.79789817980375e-05\\
1988.63861064554	8.78837144197753e-05\\
1988.9133577112	8.77886164128158e-05\\
1989.18810477686	8.76935229292506e-05\\
1989.46285184253	8.75984342501504e-05\\
1989.73759890819	8.7503020881742e-05\\
1990.01234597386	8.74077820753858e-05\\
1990.28709303952	8.73125512769948e-05\\
1990.56184010519	8.72174933938628e-05\\
1990.83658717085	8.71224264458461e-05\\
1991.11133423652	8.70273853246647e-05\\
1991.38608130218	8.6932350897603e-05\\
1991.66082836785	8.68373234566685e-05\\
1991.93557543351	8.67423032949537e-05\\
1992.21032249918	8.66472907066301e-05\\
1992.48506956484	8.65517923569222e-05\\
1992.75981663051	8.64566382089424e-05\\
1993.03456369617	8.63615804039337e-05\\
1993.30931076184	8.62664489925873e-05\\
1993.5840578275	8.61714928708695e-05\\
1993.85880489317	8.60765459622184e-05\\
1994.13355195883	8.59816085702884e-05\\
1994.4082990245	8.58865179174437e-05\\
1994.68304609016	8.57914381207796e-05\\
1994.95779315583	8.56965354976101e-05\\
1995.23254022149	8.56014799649438e-05\\
1995.50728728716	8.55064369542805e-05\\
1995.78203435282	8.54112444440642e-05\\
1996.05678141849	8.53163951673943e-05\\
1996.33152848415	8.52213933193851e-05\\
1996.60627554982	8.51264052250775e-05\\
1996.88102261548	8.50315967547518e-05\\
1997.15576968115	8.49368009924848e-05\\
1997.43051674681	8.4842018256027e-05\\
1997.70526381248	8.47470857423294e-05\\
1997.98001087814	8.46523322127375e-05\\
1998.2547579438	8.45575926397099e-05\\
1998.52950500947	8.44627028786018e-05\\
1998.80425207513	8.43679943997688e-05\\
1999.0789991408	8.42731374137671e-05\\
1999.35374620646	8.41782963563376e-05\\
1999.62849327213	8.40836379344337e-05\\
1999.90324033779	8.39889953287033e-05\\
2000.17798740346	8.38943688667254e-05\\
2000.45273446912	8.37991047349922e-05\\
2000.72748153479	8.37045203611351e-05\\
2001.00222860045	8.36097893991401e-05\\
2001.27697566612	8.35150785371474e-05\\
2001.55172273178	8.34203874307574e-05\\
2001.82646979745	8.33258796492318e-05\\
2002.10121686311	8.32313901375663e-05\\
2002.37596392878	8.31367559780145e-05\\
2002.65071099444	8.30421420470508e-05\\
2002.92545806011	8.29475490087251e-05\\
2003.20020512577	8.28529786559916e-05\\
2003.47495219144	8.27585942189793e-05\\
2003.7496992571	8.26642299682286e-05\\
2004.02444632277	8.25697212621351e-05\\
2004.29919338843	8.24754006306269e-05\\
2004.5739404541	8.23809377699577e-05\\
2004.84868751976	8.22864974055739e-05\\
2005.12343458543	8.21920812508285e-05\\
2005.39818165109	8.20978539725672e-05\\
2005.67292871676	8.2003649173684e-05\\
2005.94767578242	8.19093023455098e-05\\
2006.22242284809	8.18151457672632e-05\\
2006.49716991375	8.17210496254551e-05\\
2006.77191697941	8.16267779221143e-05\\
2007.04666404508	8.15325324693917e-05\\
2007.32141111074	8.14383121702562e-05\\
2007.59615817641	8.13441201599135e-05\\
2007.87090524207	8.12501190241341e-05\\
2008.14565230774	8.11561433827797e-05\\
2008.4203993734	8.10621935893349e-05\\
2008.69514643907	8.09682699977564e-05\\
2008.96989350473	8.08743729624632e-05\\
2009.2446405704	8.07801747032479e-05\\
2009.51938763606	8.06860084683972e-05\\
2009.79413470173	8.05922020143992e-05\\
2010.06888176739	8.04982583854016e-05\\
2010.34362883306	8.04043455198362e-05\\
2010.61837589872	8.03106281552117e-05\\
2010.89312296439	8.02167754256827e-05\\
2011.16787003005	8.01231184061764e-05\\
2011.44261709572	8.00291618714724e-05\\
2011.71736416138	7.99355688018311e-05\\
2011.99211122705	7.98420059760519e-05\\
2012.26685829271	7.97484737561691e-05\\
2012.54160535838	7.96546461078868e-05\\
2012.81635242404	7.95608531994846e-05\\
2013.09109948971	7.94672602753786e-05\\
2013.36584655537	7.93738649517907e-05\\
2013.64059362104	7.92803388746338e-05\\
2013.9153406867	7.91868456432623e-05\\
2014.19008775237	7.90935526575906e-05\\
2014.46483481803	7.90001299246646e-05\\
2014.7395818837	7.8906906152023e-05\\
2015.01432894936	7.88133883774836e-05\\
2015.28907601502	7.87199087350469e-05\\
2015.56382308069	7.86263047593094e-05\\
2015.83857014635	7.85330708818499e-05\\
2016.11331721202	7.84400382583316e-05\\
2016.38806427768	7.8347041283769e-05\\
2016.66281134335	7.82540803288294e-05\\
2016.93755840901	7.81609913217576e-05\\
2017.21230547468	7.80679426415507e-05\\
2017.48705254034	7.79750963555439e-05\\
2017.76179960601	7.78822875171584e-05\\
2018.03654667167	7.77895164980628e-05\\
2018.31129373734	7.76967836699578e-05\\
2018.586040803	7.76040894045661e-05\\
2018.86078786867	7.75112702043414e-05\\
2019.13553493433	7.7418493527477e-05\\
2019.410282	7.73259215557359e-05\\
2019.68502906566	7.7233389578251e-05\\
2019.95977613133	7.71405691382596e-05\\
2020.23452319699	7.70477940498846e-05\\
2020.50927026266	7.69553930897273e-05\\
2020.78401732832	7.68630335014774e-05\\
2021.05876439399	7.67705526251153e-05\\
2021.33351145965	7.6678279098007e-05\\
2021.60825852532	7.65858831144736e-05\\
2021.88300559098	7.64936971096679e-05\\
2022.15775265665	7.64015542872309e-05\\
2022.43249972231	7.6309291671607e-05\\
2022.70724678798	7.62170749610214e-05\\
2022.98199385364	7.61250683029331e-05\\
2023.25674091931	7.60331062643024e-05\\
2023.53148798497	7.59411892184258e-05\\
2023.80623505063	7.58488241746099e-05\\
2024.0809821163	7.57568405379267e-05\\
2024.35572918196	7.56649048842089e-05\\
2024.63047624763	7.55731824196099e-05\\
2024.90522331329	7.54815066730199e-05\\
2025.17997037896	7.53898780178761e-05\\
2025.45471744462	7.52981333047291e-05\\
2025.72946451029	7.52064386429779e-05\\
2026.00421157595	7.51149578787444e-05\\
2026.27895864162	7.50233621460696e-05\\
2026.55370570728	7.49319810107821e-05\\
2026.82845277295	7.48406491188541e-05\\
2027.10319983861	7.47492038177982e-05\\
2027.37794690428	7.46578098335395e-05\\
2027.65269396994	7.45664675141461e-05\\
2027.92744103561	7.44753428536013e-05\\
2028.20218810127	7.43842692098378e-05\\
2028.47693516694	7.42932469538016e-05\\
2028.7516822326	7.42022764559791e-05\\
2029.02642929827	7.41113580863863e-05\\
2029.30117636393	7.40204922145565e-05\\
2029.5759234296	7.39293529872272e-05\\
2029.85067049526	7.38385976141582e-05\\
2030.12541756093	7.37478957888335e-05\\
2030.40016462659	7.36567533509107e-05\\
2030.67491169226	7.35661663855287e-05\\
2030.94965875792	7.34756339864723e-05\\
2031.22440582359	7.33851565211883e-05\\
2031.49915288925	7.32947343565305e-05\\
2031.77389995492	7.32043678587477e-05\\
2032.04864702058	7.31137302346929e-05\\
2032.32339408625	7.30234805753851e-05\\
2032.59814115191	7.29331231544655e-05\\
2032.87288821757	7.28429894872338e-05\\
2033.14763528324	7.27529132261833e-05\\
2033.4223823489	7.26627298663449e-05\\
2033.69712941457	7.25724434588005e-05\\
2033.97187648023	7.2482548208463e-05\\
2034.2466235459	7.23925468777198e-05\\
2034.52137061156	7.23026086245848e-05\\
2034.79611767723	7.22128951636017e-05\\
2035.07086474289	7.21232415008244e-05\\
2035.34561180856	7.20336479960568e-05\\
2035.62035887422	7.19437096622847e-05\\
2035.89510593989	7.18542453861458e-05\\
2036.16985300555	7.17648422751753e-05\\
2036.44460007122	7.16755006870951e-05\\
2036.71934713688	7.15862209787733e-05\\
2036.99409420255	7.14968386781014e-05\\
2037.26884126821	7.1407521108382e-05\\
2037.54358833388	7.13181048084729e-05\\
2037.81833539954	7.12290849564401e-05\\
2038.09308246521	7.11401286441476e-05\\
2038.36782953087	7.10512362243924e-05\\
2038.64257659654	7.09624080490206e-05\\
2038.9173236622	7.08734804761621e-05\\
2039.19207072787	7.0784457237615e-05\\
2039.46681779353	7.06956670950209e-05\\
2039.7415648592	7.06067799413923e-05\\
2040.01631192486	7.05182912465423e-05\\
2040.29105899053	7.0429868721739e-05\\
2040.56580605619	7.03413493245813e-05\\
2040.84055312186	7.02530623809979e-05\\
2041.11530018752	7.01648426167717e-05\\
2041.39004725318	7.00766903757835e-05\\
2041.66479431885	6.99884410509859e-05\\
2041.93954138451	6.99004271052294e-05\\
2042.21428845018	6.98123171465346e-05\\
2042.48903551584	6.97244427970274e-05\\
2042.76378258151	6.96366376179103e-05\\
2043.03852964717	6.95487389420809e-05\\
2043.31327671284	6.9461075312695e-05\\
2043.5880237785	6.93733172064296e-05\\
2043.86277084417	6.92857964334207e-05\\
2044.13751790983	6.91983464522998e-05\\
2044.4122649755	6.91109675954058e-05\\
2044.68701204116	6.9023660193829e-05\\
2044.96175910683	6.89362608675689e-05\\
2045.23650617249	6.88490995707258e-05\\
2045.51125323816	6.87620106869732e-05\\
2045.78600030382	6.86746661058277e-05\\
2046.06074736949	6.85873985130645e-05\\
2046.33549443515	6.85005376207885e-05\\
2046.61024150082	6.84137503276357e-05\\
2046.88498856648	6.83270369559082e-05\\
2047.15973563215	6.82402337723386e-05\\
2047.43448269781	6.81535074093449e-05\\
2047.70922976348	6.8066857841242e-05\\
2047.98397682914	6.79802849696745e-05\\
2048.25872389481	6.78937896880254e-05\\
2048.53347096047	6.78075373355086e-05\\
2048.80821802614	6.77214047019619e-05\\
2049.0829650918	6.76353053098589e-05\\
2049.35771215747	6.75492825522647e-05\\
2049.63245922313	6.74633367389529e-05\\
2049.90720628879	6.73774681781982e-05\\
2050.18195335446	6.72916771767638e-05\\
2050.45670042012	6.72059640398918e-05\\
2050.73144748579	6.71203290712907e-05\\
2051.00619455145	6.70344435951883e-05\\
2051.28094161712	6.69488074729188e-05\\
2051.55568868278	6.68634153860296e-05\\
2051.83043574845	6.67777733168226e-05\\
2052.10518281411	6.6692544530741e-05\\
2052.37992987978	6.66070676665906e-05\\
2052.65467694544	6.65218392437345e-05\\
2052.92942401111	6.64365290689104e-05\\
2053.20417107677	6.63514680917507e-05\\
2053.47891814244	6.62664905472005e-05\\
2053.7536652081	6.61817600917598e-05\\
2054.02841227377	6.60971110072544e-05\\
2054.30315933943	6.60122150819138e-05\\
2054.5779064051	6.59275708441149e-05\\
2054.85265347076	6.5843174153216e-05\\
2055.12740053643	6.57588598858232e-05\\
2055.40214760209	6.56746283224611e-05\\
2055.67689466776	6.55903154624599e-05\\
2055.95164173342	6.55062523546892e-05\\
2056.22638879909	6.54222727538383e-05\\
2056.50113586475	6.53382124540085e-05\\
2056.77588293042	6.52544029018187e-05\\
2057.05062999608	6.51706776441041e-05\\
2057.32537706175	6.50870369493762e-05\\
2057.60012412741	6.5003481084291e-05\\
2057.87487119308	6.49198454697451e-05\\
2058.14961825874	6.48361354265378e-05\\
2058.42436532441	6.47528422283217e-05\\
2058.69911239007	6.46696348241893e-05\\
2058.97385945573	6.45863499831958e-05\\
2059.2486065214	6.45033171430258e-05\\
2059.52335358706	6.44202076350876e-05\\
2059.79810065273	6.43373503172664e-05\\
2060.07284771839	6.42545800102141e-05\\
2060.34759478406	6.41718969630719e-05\\
2060.62234184972	6.40893014230023e-05\\
2060.89708891539	6.40067936351813e-05\\
2061.17183598105	6.39243738427886e-05\\
2061.44658304672	6.38418784174554e-05\\
2061.72133011238	6.37594739996281e-05\\
2061.99607717805	6.36773240162848e-05\\
2062.27082424371	6.35950980631812e-05\\
2062.54557130938	6.35129648329617e-05\\
2062.82031837504	6.34309234240603e-05\\
2063.09506544071	6.33491369441802e-05\\
2063.36981250637	6.32672771611834e-05\\
2063.64455957204	6.31856725695835e-05\\
2063.9193066377	6.31039945201682e-05\\
2064.19405370337	6.30225726660916e-05\\
2064.46880076903	6.29412413822267e-05\\
2064.7435478347	6.28600008871676e-05\\
2065.01829490036	6.27788513973665e-05\\
2065.29304196603	6.26976288847535e-05\\
2065.56778903169	6.26166642588281e-05\\
2065.84253609736	6.25357912466176e-05\\
2066.11728316302	6.24550100562563e-05\\
2066.39203022869	6.23741562667785e-05\\
2066.66677729435	6.22933979201083e-05\\
2066.94152436002	6.22127333119366e-05\\
2067.21627142568	6.21321642006284e-05\\
2067.49101849135	6.20518538435763e-05\\
2067.76576555701	6.19714716667918e-05\\
2068.04051262267	6.18911845292836e-05\\
2068.31525968834	6.18111579090606e-05\\
2068.590006754	6.17312247228322e-05\\
2068.86475381967	6.16513851584525e-05\\
2069.13950088533	6.15716394014963e-05\\
2069.414247951	6.14919876352546e-05\\
2069.68899501666	6.14124300407258e-05\\
2069.96374208233	6.13329667966103e-05\\
2070.23848914799	6.12535980793034e-05\\
2070.51323621366	6.11743240628895e-05\\
2070.78798327932	6.1094981667887e-05\\
2071.06273034499	6.10158997663154e-05\\
2071.33747741065	6.0936913046132e-05\\
2071.61222447632	6.08580216724386e-05\\
2071.88697154198	6.0779062164576e-05\\
2072.16171860765	6.0700200030903e-05\\
2072.43646567331	6.062160005232e-05\\
2072.71121273898	6.05430960016011e-05\\
2072.98595980464	6.04646880327107e-05\\
2073.26070687031	6.03862118263551e-05\\
2073.53545393597	6.03079986889773e-05\\
2073.81020100164	6.02298820530034e-05\\
2074.0849480673	6.01518620631669e-05\\
2074.35969513297	6.00739388617948e-05\\
2074.63444219863	5.9995948667649e-05\\
2074.9091892643	5.9918294776089e-05\\
2075.18393632996	5.98406659976155e-05\\
2075.45868339563	5.97631345392892e-05\\
2075.73343046129	5.96855357039776e-05\\
2076.00817752696	5.96078745114539e-05\\
2076.28292459262	5.95306423863601e-05\\
2076.55767165828	5.94533432147158e-05\\
2076.83241872395	5.93763089210025e-05\\
2077.10716578961	5.92993725855826e-05\\
2077.38191285528	5.92223708066622e-05\\
2077.65665992094	5.91456329407982e-05\\
2077.93140698661	5.90688286710641e-05\\
2078.20615405227	5.89922896834897e-05\\
2078.48090111794	5.89158491607954e-05\\
2078.7556481836	5.88395072085427e-05\\
2079.03039524927	5.876293617851e-05\\
2079.30514231493	5.86866331807351e-05\\
2079.5798893806	5.86105941040284e-05\\
2079.85463644626	5.8534490344484e-05\\
2080.12938351193	5.84586513352681e-05\\
2080.40413057759	5.83827474372844e-05\\
2080.67887764326	5.83071088106564e-05\\
2080.95362470892	5.82314046485368e-05\\
2081.22837177459	5.81558029015109e-05\\
2081.50311884025	5.80804664949625e-05\\
2081.77786590592	5.80050647880051e-05\\
2082.05261297158	5.79296021849811e-05\\
2082.32736003725	5.78545707507035e-05\\
2082.60210710291	5.77796388387687e-05\\
2082.87685416858	5.7704643217751e-05\\
2083.15160123424	5.76299127612847e-05\\
2083.42634829991	5.75552820042462e-05\\
2083.70109536557	5.74807510101284e-05\\
2083.97584243124	5.74063198398485e-05\\
2084.2505894969	5.73318244469405e-05\\
2084.52533656257	5.72574308338766e-05\\
2084.80008362823	5.71833038733033e-05\\
2085.07483069389	5.71092768988413e-05\\
2085.34957775956	5.70350224218269e-05\\
2085.62432482522	5.69611999762925e-05\\
2085.89907189089	5.68874775993597e-05\\
2086.17381895655	5.68138553326094e-05\\
2086.44856602222	5.67401689433123e-05\\
2086.72331308788	5.66665846762487e-05\\
2086.99806015355	5.65929404311621e-05\\
2087.27280721921	5.65195640763174e-05\\
2087.54755428488	5.64464537392185e-05\\
2087.82230135054	5.63727873476035e-05\\
2088.09704841621	5.62997212227142e-05\\
2088.37179548187	5.62267592197886e-05\\
2088.64654254754	5.61537345400187e-05\\
2088.9212896132	5.6081142535596e-05\\
2089.19603667887	5.60084863421115e-05\\
2089.47078374453	5.59360966617779e-05\\
2089.7455308102	5.58636436304839e-05\\
2090.02027787586	5.57914562559678e-05\\
2090.29502494153	5.57192047553733e-05\\
2090.56977200719	5.56472196692443e-05\\
2090.84451907286	5.55753345980925e-05\\
2091.11926613852	5.55035958357221e-05\\
2091.39401320419	5.54319114716188e-05\\
2091.66876026985	5.53603271236648e-05\\
2091.94350733552	5.52888427839931e-05\\
2092.21825440118	5.5217458442128e-05\\
2092.49300146685	5.51461740849877e-05\\
2092.76774853251	5.50749896968879e-05\\
2093.04249559817	5.50039052595454e-05\\
2093.31724266384	5.4932756096988e-05\\
2093.5919897295	5.48618737137562e-05\\
2093.86673679517	5.47907639188991e-05\\
2094.14148386083	5.47200856217774e-05\\
2094.4162309265	5.46493439164065e-05\\
2094.69097799216	5.45788672599001e-05\\
2094.96572505783	5.45084902447048e-05\\
2095.24047212349	5.44380483662359e-05\\
2095.51521918916	5.4367872733573e-05\\
2095.78996625482	5.42976325125978e-05\\
2096.06471332049	5.42276580307163e-05\\
2096.33946038615	5.41577829208748e-05\\
2096.61420745182	5.40880071332417e-05\\
2096.88895451748	5.40183306154298e-05\\
2097.16370158315	5.39485898599113e-05\\
2097.43844864881	5.38789502955325e-05\\
2097.71319571448	5.38095756269096e-05\\
2097.98794278014	5.37401361815637e-05\\
2098.26268984581	5.36707980847616e-05\\
2098.53743691147	5.36017245406896e-05\\
2098.81218397714	5.35325847815776e-05\\
2099.0869310428	5.34637108337252e-05\\
2099.36167810847	5.33949354356666e-05\\
2099.63642517413	5.33260938282633e-05\\
2099.9111722398	5.32575175173224e-05\\
2100.18591930546	5.31890394924877e-05\\
2100.46066637113	5.3120496573171e-05\\
2100.73541343679	5.30522170665554e-05\\
2101.01016050246	5.29840355611464e-05\\
2101.28490756812	5.29157884190831e-05\\
2101.55965463379	5.28476416378854e-05\\
2101.83440169945	5.27795943041266e-05\\
2102.10914876511	5.27118104009459e-05\\
2102.38389583078	5.26441239401263e-05\\
2102.65864289644	5.25765348200462e-05\\
2102.93338996211	5.25088788910551e-05\\
2103.20813702777	5.24414863479643e-05\\
2103.48288409344	5.23736977748945e-05\\
2103.7576311591	5.23063424572838e-05\\
2104.03237822477	5.22392493169998e-05\\
2104.30712529043	5.21722527190399e-05\\
2104.5818723561	5.21051879404002e-05\\
2104.85661942176	5.20383862912227e-05\\
2105.13136648743	5.19716807984963e-05\\
2105.40611355309	5.1904581028919e-05\\
2105.68086061876	5.18380740828473e-05\\
2105.95560768442	5.17716628314263e-05\\
2106.23035475009	5.17053471448446e-05\\
2106.50510181575	5.16391268909415e-05\\
2106.77984888142	5.15728368972275e-05\\
2107.05459594708	5.15066463110169e-05\\
2107.32934301275	5.14405522598541e-05\\
2107.60409007841	5.13747189099481e-05\\
2107.87883714408	5.1308817178752e-05\\
2108.15358420974	5.12430102011243e-05\\
2108.42833127541	5.11774647680263e-05\\
2108.70307834107	5.11118498514447e-05\\
2108.97782540674	5.10464947854178e-05\\
2109.2525724724	5.09812335434431e-05\\
2109.52731953807	5.09160659701143e-05\\
2109.80206660373	5.08509919077871e-05\\
2110.0768136694	5.07860111965919e-05\\
2110.35156073506	5.07209587295156e-05\\
2110.62630780073	5.06561664536151e-05\\
2110.90105486639	5.05914670079254e-05\\
2111.17580193205	5.05268602241221e-05\\
2111.45054899772	5.04620180995022e-05\\
2111.72529606338	5.03974376792287e-05\\
2112.00004312905	5.03331144058113e-05\\
2112.27479019471	5.02687194272753e-05\\
2112.54953726038	5.02045819482659e-05\\
2112.82428432604	5.01403726206297e-05\\
2113.09903139171	5.00762563289926e-05\\
2113.37377845737	5.00123971766066e-05\\
2113.64852552304	4.99486289145715e-05\\
2113.9232725887	4.98849513553346e-05\\
2114.19801965437	4.98213643092838e-05\\
2114.47276672003	4.97578675847605e-05\\
2114.7475137857	4.96944609880743e-05\\
2115.02226085136	4.96311443235161e-05\\
2115.29700791703	4.95679173933725e-05\\
2115.57175498269	4.9504616375183e-05\\
2115.84650204836	4.94415705175573e-05\\
2116.12124911402	4.93786725334402e-05\\
2116.39599617969	4.93158055257462e-05\\
2116.67074324535	4.92530272187558e-05\\
2116.94549031102	4.91903374034949e-05\\
2117.22023737668	4.91277358690829e-05\\
2117.49498444235	4.9065222402748e-05\\
2117.76973150801	4.90026336348051e-05\\
2118.04447857368	4.89402978576824e-05\\
2118.31922563934	4.887804947131e-05\\
2118.59397270501	4.88155612799747e-05\\
2118.86871977067	4.87534914198408e-05\\
2119.14346683633	4.86915082297027e-05\\
2119.418213902	4.86296114849358e-05\\
2119.69296096766	4.85676370345089e-05\\
2119.96770803333	4.85059147080955e-05\\
2120.24245509899	4.84442781146311e-05\\
2120.51720216466	4.83827270227887e-05\\
2120.79194923032	4.83212611995187e-05\\
2121.06669629599	4.82598804100665e-05\\
2121.34144336165	4.81985844179875e-05\\
2121.61619042732	4.81372099846865e-05\\
2121.89093749298	4.80760850682531e-05\\
2122.16568455865	4.80150442020938e-05\\
2122.44043162431	4.79540871434917e-05\\
2122.71517868998	4.78932136481183e-05\\
2122.98992575564	4.78324234700492e-05\\
2123.26467282131	4.77715529994952e-05\\
2123.53941988697	4.771027570738e-05\\
2123.81416695264	4.76497449917537e-05\\
2124.0889140183	4.75892964549655e-05\\
2124.36366108397	4.75287651760886e-05\\
2124.63840814963	4.7468482457389e-05\\
2124.9131552153	4.74082811288986e-05\\
2125.18790228096	4.73481609344346e-05\\
2125.46264934663	4.7288121616349e-05\\
2125.73739641229	4.72281629155467e-05\\
2126.01214347796	4.71682845715023e-05\\
2126.28689054362	4.71083233978181e-05\\
2126.56163760929	4.70486071760304e-05\\
2126.83638467495	4.69888073566064e-05\\
2127.11113174062	4.69290875115245e-05\\
2127.38587880628	4.68692842887967e-05\\
2127.66062587195	4.68098933903827e-05\\
2127.93537293761	4.67502537580362e-05\\
2128.21012000327	4.66910237116937e-05\\
2128.48486706894	4.66318714259739e-05\\
2128.7596141346	4.65727966292463e-05\\
2129.03436120027	4.65137990486219e-05\\
2129.30910826593	4.64545515893493e-05\\
2129.5838553316	4.63957120215e-05\\
2129.85860239726	4.63369487880139e-05\\
2130.13334946293	4.62782616117994e-05\\
2130.40809652859	4.62194873012159e-05\\
2130.68284359426	4.61609535993986e-05\\
2130.95759065992	4.61024950888278e-05\\
2131.23233772559	4.60439467810645e-05\\
2131.50708479125	4.5985640026299e-05\\
2131.78183185692	4.59274075873825e-05\\
2132.05657892258	4.5869249180033e-05\\
2132.33132598825	4.58110003184143e-05\\
2132.60607305391	4.5752991329244e-05\\
2132.88082011958	4.56950554845943e-05\\
2133.15556718524	4.56371924964174e-05\\
2133.43031425091	4.55792388664217e-05\\
2133.70506131657	4.55215229226981e-05\\
2133.97980838224	4.54637159611728e-05\\
2134.2545554479	4.54059808013767e-05\\
2134.52930251357	4.53484842497793e-05\\
2134.80404957923	4.52910587240571e-05\\
2135.0787966449	4.52337039306692e-05\\
2135.35354371056	4.51760913945309e-05\\
2135.62829077623	4.51188816031526e-05\\
2135.90303784189	4.50617416025221e-05\\
2136.17778490756	4.500467109644e-05\\
2136.45253197322	4.49477121771558e-05\\
2136.72727903888	4.48907803941839e-05\\
2137.00202610455	4.48337533042712e-05\\
2137.27677317021	4.4776960650761e-05\\
2137.55152023588	4.47200710647913e-05\\
2137.82626730154	4.46634162985551e-05\\
2138.10101436721	4.46068288863713e-05\\
2138.37576143287	4.4550308527053e-05\\
2138.65050849854	4.4493854918775e-05\\
2138.9252555642	4.44374677590938e-05\\
2139.20000262987	4.43811467449666e-05\\
2139.47474969553	4.43248915727711e-05\\
2139.7494967612	4.42687019383245e-05\\
2140.02424382686	4.42125775369033e-05\\
2140.29899089253	4.41563546616751e-05\\
2140.57373795819	4.4100362012049e-05\\
2140.84848502386	4.40444336503783e-05\\
2141.12323208952	4.39885692702958e-05\\
2141.39797915519	4.39326048532183e-05\\
2141.67272622085	4.38768697213437e-05\\
2141.94747328652	4.38211976214557e-05\\
2142.22222035218	4.3765588245841e-05\\
2142.49696741785	4.37100412864123e-05\\
2142.77171448351	4.36543934810085e-05\\
2143.04646154918	4.35989726246153e-05\\
2143.32120861484	4.35434483467891e-05\\
2143.59595568051	4.34881523262397e-05\\
2143.87070274617	4.34329171213195e-05\\
2144.14544981184	4.33777424227548e-05\\
2144.4201968775	4.33226279210276e-05\\
2144.69494394317	4.32675733063942e-05\\
2144.96969100883	4.32125782689045e-05\\
2145.24443807449	4.31574779596157e-05\\
2145.51918514016	4.31024404424005e-05\\
2145.79393220582	4.30474627228311e-05\\
2146.06867927149	4.29925461878643e-05\\
2146.34342633715	4.29378530356775e-05\\
2146.61817340282	4.2883052697587e-05\\
2146.89292046848	4.28284766645444e-05\\
2147.16766753415	4.27739575888298e-05\\
2147.44241459981	4.27191674379941e-05\\
2147.71716166548	4.26646023651837e-05\\
2147.99190873114	4.26102588669082e-05\\
2148.26665579681	4.2555971005388e-05\\
2148.54140286247	4.25017384722244e-05\\
2148.81614992814	4.24475609590516e-05\\
2149.0908969938	4.23934381575575e-05\\
2149.36564405947	4.23393697595008e-05\\
2149.64039112513	4.22853554567313e-05\\
2149.9151381908	4.22313949412077e-05\\
2150.18988525646	4.21773239410743e-05\\
2150.46463232213	4.21234722855412e-05\\
2150.73937938779	4.20696734660446e-05\\
2151.01412645346	4.20159271754954e-05\\
2151.28887351912	4.19619064612622e-05\\
2151.56362058479	4.19082687101307e-05\\
2151.83836765045	4.18546825122784e-05\\
2152.11311471612	4.18011475621774e-05\\
2152.38786178178	4.17476635545659e-05\\
2152.66260884745	4.16940663879919e-05\\
2152.93735591311	4.16406855577059e-05\\
2153.21210297877	4.15873547280012e-05\\
2153.48685004444	4.15339104516474e-05\\
2153.7615971101	4.14805169140162e-05\\
2154.03634417577	4.14273386457958e-05\\
2154.31109124143	4.1374209110193e-05\\
2154.5858383071	4.13209636101087e-05\\
2154.86058537276	4.12679328501729e-05\\
2155.13533243843	4.12149498916046e-05\\
2155.41007950409	4.11620144345335e-05\\
2155.68482656976	4.1109126179558e-05\\
2155.95957363542	4.10561210852281e-05\\
2156.23432070109	4.10033285444992e-05\\
2156.50906776675	4.09505822827586e-05\\
2156.78381483242	4.08978820029864e-05\\
2157.05856189808	4.08452274087162e-05\\
2157.33330896375	4.07926182040534e-05\\
2157.60805602941	4.07398898741858e-05\\
2157.88280309508	4.06873727757431e-05\\
2158.15755016074	4.06349001548539e-05\\
2158.43229722641	4.0582144418669e-05\\
2158.70704429207	4.05296007585585e-05\\
2158.98179135774	4.04772663723182e-05\\
2159.2565384234	4.04249752121924e-05\\
2159.53128548907	4.03727269887563e-05\\
2159.80603255473	4.03205214132953e-05\\
2160.0807796204	4.02681938255968e-05\\
2160.35552668606	4.02159117492635e-05\\
2160.63027375173	4.01638367778194e-05\\
2160.90502081739	4.01118032514478e-05\\
2161.17976788306	4.00594816263721e-05\\
2161.45451494872	4.00073697486038e-05\\
2161.72926201439	3.99553006066348e-05\\
2162.00400908005	3.99034387616695e-05\\
2162.27875614571	3.98516168380665e-05\\
2162.55350321138	3.97995057524025e-05\\
2162.82825027704	3.97477672652325e-05\\
2163.10299734271	3.96960340988862e-05\\
2163.37774440837	3.9644377413166e-05\\
2163.65249147404	3.95925953447632e-05\\
2163.9272385397	3.95410175476774e-05\\
2164.20198560537	3.94891497555803e-05\\
2164.47673267103	3.94376519131776e-05\\
2164.7514797367	3.93861913780507e-05\\
2165.02622680236	3.9334767881953e-05\\
2165.30097386803	3.92833811576259e-05\\
2165.57572093369	3.9232030938814e-05\\
2165.85046799936	3.91807169602801e-05\\
2166.12521506502	3.91292748225556e-05\\
2166.39996213069	3.90780347442662e-05\\
2166.67470919635	3.90268300888417e-05\\
2166.94945626202	3.89751688999014e-05\\
2167.22420332768	3.89240409340068e-05\\
2167.49895039335	3.88729475284156e-05\\
2167.77369745901	3.88218884267212e-05\\
2168.04844452468	3.87709682225371e-05\\
2168.32319159034	3.87198135859114e-05\\
2168.59793865601	3.86688595787823e-05\\
2168.87268572167	3.86177754887209e-05\\
2169.14743278734	3.85668899953217e-05\\
2169.422179853	3.8516037256224e-05\\
2169.69692691867	3.84650535375982e-05\\
2169.97167398433	3.84142677692005e-05\\
2170.24642105	3.83633502078049e-05\\
2170.52116811566	3.83126303786278e-05\\
2170.79591518133	3.82617779382475e-05\\
2171.07066224699	3.82111230280913e-05\\
2171.34540931265	3.81603343837482e-05\\
2171.62015637832	3.81095796891038e-05\\
2171.89490344398	3.80590213633136e-05\\
2172.16965050965	3.80083302036028e-05\\
2172.44439757531	3.79578342458593e-05\\
2172.71914464098	3.79073679973001e-05\\
2172.99389170664	3.78569312277792e-05\\
2173.26863877231	3.78065237084911e-05\\
2173.54338583797	3.7755981953678e-05\\
2173.81813290364	3.77054714605708e-05\\
2174.0928799693	3.76551547019021e-05\\
2174.36762703497	3.7604866236601e-05\\
2174.64237410063	3.7554441208079e-05\\
2174.9171211663	3.75042108846558e-05\\
2175.19186823196	3.74540081669575e-05\\
2175.46661529763	3.74038328382775e-05\\
2175.74136236329	3.73536846833418e-05\\
2176.01610942896	3.73035634883202e-05\\
2176.29085649462	3.72534690408365e-05\\
2176.56560356029	3.72032365174391e-05\\
2176.84035062595	3.71530333364e-05\\
2177.11509769162	3.71030222684335e-05\\
2177.38984475728	3.70528739780854e-05\\
2177.66459182295	3.70029166180815e-05\\
2177.93933888861	3.69529846803625e-05\\
2178.21408595428	3.69030779639809e-05\\
2178.48883301994	3.6853196269513e-05\\
2178.76358008561	3.68033393990689e-05\\
2179.03832715127	3.67535071563017e-05\\
2179.31307421694	3.6703699346416e-05\\
2179.5878212826	3.66539157761786e-05\\
2179.86256834827	3.66041562539255e-05\\
2180.13731541393	3.65544205895721e-05\\
2180.41206247959	3.65047085946209e-05\\
2180.68680954526	3.64550200821709e-05\\
2180.96155661092	3.6405354866925e-05\\
2181.23630367659	3.63557127651992e-05\\
2181.51105074225	3.63060935949296e-05\\
2181.78579780792	3.62563334421959e-05\\
2182.06054487358	3.62067618013474e-05\\
2182.33529193925	3.61570477340289e-05\\
2182.61003900491	3.61075228719192e-05\\
2182.88478607058	3.60578569881379e-05\\
2183.15953313624	3.60083781323957e-05\\
2183.43428020191	3.5958756414623e-05\\
2183.70902726757	3.59093228491385e-05\\
2183.98377433324	3.58597465341777e-05\\
2184.2585213989	3.58103575325189e-05\\
2184.53326846457	3.57608257829753e-05\\
2184.80801553023	3.57113163264806e-05\\
2185.0827625959	3.56619939495253e-05\\
2185.35750966156	3.56126919452807e-05\\
2185.63225672723	3.55634101590329e-05\\
2185.90700379289	3.55141484377758e-05\\
2186.18175085856	3.54649066302161e-05\\
2186.45649792422	3.54156845867801e-05\\
2186.73124498989	3.53664821596179e-05\\
2187.00599205555	3.531729920261e-05\\
2187.28073912122	3.52681355713723e-05\\
2187.55548618688	3.52189911232608e-05\\
2187.83023325255	3.51698657173771e-05\\
2188.10498031821	3.51205949288413e-05\\
2188.37972738387	3.50713459137146e-05\\
2188.65447444954	3.50222811949419e-05\\
2188.9292215152	3.49732349168406e-05\\
2189.20396858087	3.49240439499762e-05\\
2189.47871564653	3.48748719020219e-05\\
2189.7534627122	3.4825555055531e-05\\
2190.02820977786	3.47765900101634e-05\\
2190.30295684353	3.47271506339433e-05\\
2190.57770390919	3.46782274765922e-05\\
2190.85245097486	3.46291568393393e-05\\
2191.12719804052	3.45801066753462e-05\\
2191.40194510619	3.45312396659624e-05\\
2191.67669217185	3.44823896215508e-05\\
2191.95143923752	3.44333931049355e-05\\
2192.22618630318	3.43844153441176e-05\\
2192.50093336885	3.43356199135592e-05\\
2192.77568043451	3.42868409600425e-05\\
2193.05042750018	3.4238078380516e-05\\
2193.32517456584	3.4189332073702e-05\\
2193.59992163151	3.41406019400993e-05\\
2193.87466869717	3.40915611273662e-05\\
2194.14941576284	3.40428674521478e-05\\
2194.4241628285	3.39941896066276e-05\\
2194.69890989417	3.39453632090934e-05\\
2194.97365695983	3.38967189627898e-05\\
2195.2484040255	3.38480902483763e-05\\
2195.52315109116	3.37994769798005e-05\\
2195.79789815683	3.37508790727905e-05\\
2196.07264522249	3.37022964448565e-05\\
2196.34739228816	3.36537290152913e-05\\
2196.62213935382	3.36050131921589e-05\\
2196.89688641948	3.35564781277374e-05\\
2197.17163348515	3.35077939121195e-05\\
2197.44638055081	3.34591260320611e-05\\
2197.72112761648	3.34104760379098e-05\\
2197.99587468214	3.3362006853329e-05\\
2198.27062174781	3.33133887925461e-05\\
2198.54536881347	3.32647867969597e-05\\
2198.82011587914	3.32163655258718e-05\\
2199.0948629448	3.31677957409233e-05\\
2199.36961001047	3.31192421983417e-05\\
2199.64435707613	3.30708681013213e-05\\
2199.9191041418	3.30225081051804e-05\\
2200.19385120746	3.2974162156699e-05\\
2200.46859827313	3.2925830204414e-05\\
2200.74334533879	3.28775121986198e-05\\
2201.01809240446	3.28292080913661e-05\\
2201.29283947012	3.27809178364581e-05\\
2201.56758653579	3.27326413894543e-05\\
2201.84233360145	3.26843787076661e-05\\
2202.11708066712	3.26358019445816e-05\\
2202.39182773278	3.25875710887942e-05\\
};
\addplot [color=mycolor1,solid,line width=2.0pt,forget plot]
  table[row sep=crcr]{%
2202.39182773278	3.25875710887942e-05\\
2202.66657479845	3.25393538248131e-05\\
2202.94132186411	3.24911501175983e-05\\
2203.21606892978	3.24429599338473e-05\\
2203.49081599544	3.2394619700605e-05\\
2203.76556306111	3.23464586745632e-05\\
2204.04031012677	3.22981477176537e-05\\
2204.31505719244	3.22500156789148e-05\\
2204.5898042581	3.22018969680591e-05\\
2204.86455132377	3.21537915628091e-05\\
2205.13929838943	3.21056994426018e-05\\
2205.4140454551	3.2057620588586e-05\\
2205.68879252076	3.20095549836193e-05\\
2205.96353958642	3.19610087902789e-05\\
2206.23828665209	3.19129762904619e-05\\
2206.51303371775	3.18647924208415e-05\\
2206.78778078342	3.18167883749859e-05\\
2207.06252784908	3.17687974078621e-05\\
2207.33727491475	3.17206566073548e-05\\
2207.61202198041	3.16725297230401e-05\\
2207.88676904608	3.1624582345524e-05\\
2208.16151611174	3.15766479786748e-05\\
2208.43626317741	3.15287266241425e-05\\
2208.71101024307	3.14808182852273e-05\\
2208.98575730874	3.14329229668749e-05\\
2209.2605043744	3.1385040675673e-05\\
2209.53525144007	3.13371714198467e-05\\
2209.80999850573	3.1289315209254e-05\\
2210.0847455714	3.1241472055382e-05\\
2210.35949263706	3.11934783517653e-05\\
2210.63423970273	3.11455000322407e-05\\
2210.90898676839	3.1097700514066e-05\\
2211.18373383406	3.10499140593336e-05\\
2211.45848089972	3.10021406882523e-05\\
2211.73322796539	3.09542172429552e-05\\
2212.00797503105	3.09064723053457e-05\\
2212.28272209672	3.08587404937355e-05\\
2212.55746916238	3.08110218349994e-05\\
2212.83221622805	3.07633163575808e-05\\
2213.10696329371	3.07156240914857e-05\\
2213.38171035938	3.06679450682777e-05\\
2213.65645742504	3.06199528533696e-05\\
2213.93120449071	3.05721407986404e-05\\
2214.20595155637	3.05245082628094e-05\\
2214.48069862203	3.04768890286435e-05\\
2214.7554456877	3.04292831363771e-05\\
2215.03019275336	3.03816906277563e-05\\
2215.30493981903	3.03341115460326e-05\\
2215.57968688469	3.02865459359566e-05\\
2215.85443395036	3.0238993843772e-05\\
2216.12918101602	3.01914553172087e-05\\
2216.40392808169	3.01439304054771e-05\\
2216.67867514735	3.00964191592606e-05\\
2216.95342221302	3.00489216307095e-05\\
2217.22816927868	3.00014378734341e-05\\
2217.50291634435	2.99539679424982e-05\\
2217.77766341001	2.99065118944115e-05\\
2218.05241047568	2.98590697871236e-05\\
2218.32715754134	2.98116416800158e-05\\
2218.60190460701	2.97640634681163e-05\\
2218.87665167267	2.97166657568471e-05\\
2219.15139873834	2.96692822044637e-05\\
2219.426145804	2.96219128763319e-05\\
2219.70089286967	2.95745578392097e-05\\
2219.97563993533	2.95272171612398e-05\\
2220.250387001	2.94798909119425e-05\\
2220.52513406666	2.94325791622073e-05\\
2220.79988113233	2.93852819842861e-05\\
2221.07462819799	2.93378349235881e-05\\
2221.34937526366	2.92905693276782e-05\\
2221.62412232932	2.92433185004163e-05\\
2221.89886939499	2.91959190429841e-05\\
2222.17361646065	2.91485362441018e-05\\
2222.44836352632	2.91013345796926e-05\\
2222.72311059198	2.9054147948146e-05\\
2222.99785765764	2.90069764332868e-05\\
2223.27260472331	2.89598201202198e-05\\
2223.54735178897	2.89126790953201e-05\\
2223.82209885464	2.88653897338213e-05\\
2224.0968459203	2.88181172420693e-05\\
2224.37159298597	2.87710270065577e-05\\
2224.64634005163	2.87239523616786e-05\\
2224.9210871173	2.86768934006877e-05\\
2225.19583418296	2.86298502180534e-05\\
2225.47058124863	2.85828229094493e-05\\
2225.74532831429	2.85358115717446e-05\\
2226.02007537996	2.84888163029964e-05\\
2226.29482244562	2.84418372024397e-05\\
2226.56956951129	2.83945457329439e-05\\
2226.84431657695	2.83474407524126e-05\\
2227.11906364262	2.83005173308966e-05\\
2227.39381070828	2.82536104030297e-05\\
2227.66855777395	2.82065553738517e-05\\
2227.94330483961	2.81596839707041e-05\\
2228.21805190528	2.81128293544053e-05\\
2228.49279897094	2.80658277385818e-05\\
2228.76754603661	2.80186817387666e-05\\
2229.04229310227	2.79718847380411e-05\\
2229.31704016794	2.79251048856922e-05\\
2229.5917872336	2.78783422965911e-05\\
2229.86653429927	2.78315970866455e-05\\
2230.14128136493	2.77848693727917e-05\\
2230.4160284306	2.77381592729839e-05\\
2230.69077549626	2.76914669061863e-05\\
2230.96552256193	2.7644627868494e-05\\
2231.24026962759	2.75979735447647e-05\\
2231.51501669326	2.75513372888871e-05\\
2231.78976375892	2.75045549693545e-05\\
2232.06451082458	2.74577933569332e-05\\
2232.33925789025	2.74112164431495e-05\\
2232.61400495591	2.73646580402868e-05\\
2232.88875202158	2.73179543524093e-05\\
2233.16349908724	2.72712723511119e-05\\
2233.43824615291	2.72246112889527e-05\\
2233.71299321857	2.71779703551464e-05\\
2233.98774028424	2.71315150979348e-05\\
2234.2624873499	2.70849146185163e-05\\
2234.53723441557	2.7038500071886e-05\\
2234.81198148123	2.69921049475221e-05\\
2235.0867285469	2.69457293817282e-05\\
2235.36147561256	2.68993735116293e-05\\
2235.63622267823	2.68530374751618e-05\\
2235.91096974389	2.6806721411065e-05\\
2236.18571680956	2.67604254588704e-05\\
2236.46046387522	2.67141497588932e-05\\
2236.73521094089	2.66677301317389e-05\\
2237.00995800655	2.66214975738212e-05\\
2237.28470507222	2.65751220578669e-05\\
2237.55945213788	2.6528770136727e-05\\
2237.83419920355	2.64826043336801e-05\\
2238.10894626921	2.64364595582969e-05\\
2238.38369333488	2.63903359570714e-05\\
2238.65844040054	2.63440695802792e-05\\
2238.93318746621	2.62976615590738e-05\\
2239.20793453187	2.62512807854607e-05\\
2239.48268159754	2.62050894705219e-05\\
2239.7574286632	2.61590870009218e-05\\
2240.03217572887	2.61131064336579e-05\\
2240.30692279453	2.60671479218295e-05\\
2240.58166986019	2.60208830907184e-05\\
2240.85641692586	2.59748094085417e-05\\
2241.13116399152	2.59289245691447e-05\\
2241.40591105719	2.5883062320311e-05\\
2241.68065812285	2.5837222819062e-05\\
2241.95540518852	2.57914062229691e-05\\
2242.23015225418	2.57456126901456e-05\\
2242.50489931985	2.56998423792364e-05\\
2242.77964638551	2.56540954494092e-05\\
2243.05439345118	2.56083720603445e-05\\
2243.32914051684	2.55626723722271e-05\\
2243.60388758251	2.55169965457358e-05\\
2243.87863464817	2.5471344742035e-05\\
2244.15338171384	2.54257171227646e-05\\
2244.4281287795	2.53801138500311e-05\\
2244.70287584517	2.53345350863978e-05\\
2244.97762291083	2.52889809948759e-05\\
2245.2523699765	2.52434517389147e-05\\
2245.52711704216	2.51979474823933e-05\\
2245.80186410783	2.51524683896098e-05\\
2246.07661117349	2.51070146252731e-05\\
2246.35135823916	2.5061586354493e-05\\
2246.62610530482	2.50161837427715e-05\\
2246.90085237049	2.49708069559928e-05\\
2247.17559943615	2.49254561604146e-05\\
2247.45034650182	2.48799681361189e-05\\
2247.72509356748	2.48345074511434e-05\\
2247.99984063315	2.4789240021049e-05\\
2248.27458769881	2.47439991951929e-05\\
2248.54933476448	2.46987851422114e-05\\
2248.82408183014	2.46535980310513e-05\\
2249.0988288958	2.4608438030962e-05\\
2249.37357596147	2.45633053114852e-05\\
2249.64832302713	2.4518200042447e-05\\
2249.9230700928	2.44729590270876e-05\\
2250.19781715846	2.44279113710369e-05\\
2250.47256422413	2.43827267691634e-05\\
2250.74731128979	2.43377373723393e-05\\
2251.02205835546	2.42924469098427e-05\\
2251.29680542112	2.42475186171955e-05\\
2251.57155248679	2.4202618860918e-05\\
2251.84629955245	2.41577478140727e-05\\
2252.12104661812	2.41124150689818e-05\\
2252.39579368378	2.40676085716342e-05\\
2252.67054074945	2.4022831220408e-05\\
2252.94528781511	2.39779190146137e-05\\
2253.22003488078	2.39330387785257e-05\\
2253.49478194644	2.38881912253786e-05\\
2253.76952901211	2.38435387421303e-05\\
2254.04427607777	2.37989161973854e-05\\
2254.31902314344	2.37543237674961e-05\\
2254.5937702091	2.37094342351665e-05\\
2254.86851727477	2.36649069758299e-05\\
2255.14326434043	2.36204103051395e-05\\
2255.4180114061	2.35759444004176e-05\\
2255.69275847176	2.35315094390379e-05\\
2255.96750553743	2.34869418484069e-05\\
2256.24225260309	2.34425715123857e-05\\
2256.51699966876	2.33982326241583e-05\\
2256.79174673442	2.3353761542948e-05\\
2257.06649380009	2.33094882910477e-05\\
2257.34124086575	2.32652469925263e-05\\
2257.61598793142	2.32210378255367e-05\\
2257.89073499708	2.31766963387997e-05\\
2258.16548206274	2.31325541867463e-05\\
2258.44022912841	2.30882808567218e-05\\
2258.71497619407	2.30442063658715e-05\\
2258.98972325974	2.30001648415868e-05\\
2259.2644703254	2.29561564624362e-05\\
2259.53921739107	2.29121814069156e-05\\
2259.81396445673	2.28679105522209e-05\\
2260.0887115224	2.28238422590985e-05\\
2260.36345858806	2.27799748444276e-05\\
2260.63820565373	2.2736141383023e-05\\
2260.91295271939	2.26923420538827e-05\\
2261.18769978506	2.26485770358776e-05\\
2261.46244685072	2.26048465077429e-05\\
2261.73719391639	2.25611506480724e-05\\
2262.01194098205	2.25174896353104e-05\\
2262.28668804772	2.24738636477448e-05\\
2262.56143511338	2.24302728635003e-05\\
2262.83618217905	2.23867174605309e-05\\
2263.11092924471	2.23428696735809e-05\\
2263.38567631038	2.22993899846773e-05\\
2263.66042337604	2.22559461539536e-05\\
2263.93517044171	2.22125383592663e-05\\
2264.20991750737	2.21690031010966e-05\\
2264.48466457304	2.21256701166601e-05\\
2264.7594116387	2.208237367264e-05\\
2265.03415870437	2.20389502450163e-05\\
2265.30890577003	2.19955649965792e-05\\
2265.5836528357	2.19523836318873e-05\\
2265.85839990136	2.19092394597131e-05\\
2266.13314696702	2.18659680246497e-05\\
2266.40789403269	2.18229009855802e-05\\
2266.68264109835	2.17795430136229e-05\\
2266.95738816402	2.17365559669213e-05\\
2267.23213522968	2.16936069112608e-05\\
2267.50688229535	2.16506960231148e-05\\
2267.78162936101	2.16078234786396e-05\\
2268.05637642668	2.15649894536689e-05\\
2268.33112349234	2.15220303241844e-05\\
2268.60587055801	2.14792760714135e-05\\
2268.88061762367	2.14365608357439e-05\\
2269.15536468934	2.13938847919853e-05\\
2269.430111755	2.13512481145909e-05\\
2269.70485882067	2.13086509776527e-05\\
2269.97960588633	2.12660935548952e-05\\
2270.254352952	2.12234123930063e-05\\
2270.52910001766	2.11807734288277e-05\\
2270.80384708333	2.11383405698298e-05\\
2271.07859414899	2.10959480606794e-05\\
2271.35334121466	2.10534331069988e-05\\
2271.62808828032	2.10109610636673e-05\\
2271.90283534599	2.0968531726616e-05\\
2272.17758241165	2.0926308847378e-05\\
2272.45232947732	2.08841270937054e-05\\
2272.72707654298	2.08419866369269e-05\\
2273.00182360865	2.07997243423013e-05\\
2273.27657067431	2.07576691922859e-05\\
2273.55131773998	2.07156558226336e-05\\
2273.82606480564	2.06736844031931e-05\\
2274.10081187131	2.06315914773077e-05\\
2274.37555893697	2.05895419556654e-05\\
2274.65030600264	2.05477017967727e-05\\
2274.9250530683	2.05057413459329e-05\\
2275.19980013396	2.04639886888806e-05\\
2275.47454719963	2.04222789090795e-05\\
2275.74929426529	2.0380612173969e-05\\
2276.02404133096	2.0338988650478e-05\\
2276.29878839662	2.02974085050205e-05\\
2276.57353546229	2.02558719034912e-05\\
2276.84828252795	2.0214379011261e-05\\
2277.12302959362	2.01729299931729e-05\\
2277.39777665928	2.01315250135375e-05\\
2277.67252372495	2.00901642361292e-05\\
2277.94727079061	2.00488478241814e-05\\
2278.22201785628	2.00074124916537e-05\\
2278.49676492194	1.99661875007221e-05\\
2278.77151198761	1.99250073338117e-05\\
2279.04625905327	1.98838721522783e-05\\
2279.32100611894	1.98427821169136e-05\\
2279.5957531846	1.9801737387941e-05\\
2279.87050025027	1.97607381250113e-05\\
2280.14524731593	1.97196195211623e-05\\
2280.4199943816	1.96787138887155e-05\\
2280.69474144726	1.96378541697152e-05\\
2280.96948851293	1.95970405218146e-05\\
2281.24423557859	1.95562731020726e-05\\
2281.51898264426	1.95155520669501e-05\\
2281.79372970992	1.9474714079879e-05\\
2282.06847677559	1.94340884840999e-05\\
2282.34322384125	1.93935097108401e-05\\
2282.61797090692	1.93529779144601e-05\\
2282.89271797258	1.93124932487023e-05\\
2283.16746503825	1.92720558666868e-05\\
2283.44221210391	1.9231665920909e-05\\
2283.71695916958	1.91913235632356e-05\\
2283.99170623524	1.91508651345382e-05\\
2284.2664533009	1.91106206132934e-05\\
2284.54120036657	1.9070260032834e-05\\
2284.81594743223	1.90301138956724e-05\\
2285.0906944979	1.89898527480159e-05\\
2285.36544156356	1.89498055255394e-05\\
2285.64018862923	1.89098068546283e-05\\
2285.91493569489	1.88696931731067e-05\\
2286.18968276056	1.88297942536023e-05\\
2286.46442982622	1.87899442988329e-05\\
2286.73917689189	1.87501434548077e-05\\
2287.01392395755	1.87102286083276e-05\\
2287.28867102322	1.86703653363741e-05\\
2287.56341808888	1.86307170667021e-05\\
2287.83816515455	1.85911184296796e-05\\
2288.11291222021	1.85515695685516e-05\\
2288.38765928588	1.85119076527623e-05\\
2288.66240635154	1.84724609669347e-05\\
2288.93715341721	1.84330644547628e-05\\
2289.21190048287	1.83937182570164e-05\\
2289.48664754854	1.83542581240412e-05\\
2289.7613946142	1.83150151890851e-05\\
2290.03614167987	1.82756584013322e-05\\
2290.31088874553	1.82365192301624e-05\\
2290.5856358112	1.81972677716718e-05\\
2290.86038287686	1.81582328420267e-05\\
2291.13512994253	1.81189223751204e-05\\
2291.40987700819	1.80798299947688e-05\\
2291.68462407386	1.80409556280323e-05\\
2291.95937113952	1.80021327819891e-05\\
2292.23411820519	1.79633615913956e-05\\
2292.50886527085	1.79246421902569e-05\\
2292.78361233651	1.7885974711825e-05\\
2293.05835940218	1.78473592885972e-05\\
2293.33310646784	1.78087960523148e-05\\
2293.60785353351	1.77702851339609e-05\\
2293.88260059917	1.773182666376e-05\\
2294.15734766484	1.76934207711753e-05\\
2294.4320947305	1.76550675849083e-05\\
2294.70684179617	1.76167672328971e-05\\
2294.98158886183	1.75785198423147e-05\\
2295.2563359275	1.7540325539568e-05\\
2295.53108299316	1.75021844502964e-05\\
2295.80583005883	1.74640966993706e-05\\
2296.08057712449	1.74260624108915e-05\\
2296.35532419016	1.73880817081883e-05\\
2296.63007125582	1.73501547138181e-05\\
2296.90481832149	1.73122815495643e-05\\
2297.17956538715	1.72744623364358e-05\\
2297.45431245282	1.72366971946653e-05\\
2297.72905951848	1.71989862437091e-05\\
2298.00380658415	1.71613296022449e-05\\
2298.27855364981	1.7123727388172e-05\\
2298.55330071548	1.7085850413106e-05\\
2298.82804778114	1.70483618399776e-05\\
2299.10279484681	1.7010927987188e-05\\
2299.37754191247	1.69735489701665e-05\\
2299.65228897814	1.69362249035518e-05\\
2299.9270360438	1.68989559011908e-05\\
2300.20178310946	1.68615787840791e-05\\
2300.47653017513	1.68244224491941e-05\\
2300.75127724079	1.67871582593249e-05\\
2301.02602430646	1.67499512571229e-05\\
2301.30077137212	1.67128021353575e-05\\
2301.57551843779	1.66758744834396e-05\\
2301.85026550345	1.66390025613078e-05\\
2302.12501256912	1.66021864777312e-05\\
2302.39975963478	1.6565426340668e-05\\
2302.67450670045	1.6528722257264e-05\\
2302.94925376611	1.64920743338534e-05\\
2303.22400083178	1.64554826759571e-05\\
2303.49874789744	1.64189473882836e-05\\
2303.77349496311	1.63824685747277e-05\\
2304.04824202877	1.63458827970314e-05\\
2304.32298909444	1.63093546273545e-05\\
2304.5977361601	1.62730502471237e-05\\
2304.87248322577	1.62368026928498e-05\\
2305.14723029143	1.6200448940677e-05\\
2305.4219773571	1.61643175373941e-05\\
2305.69672442276	1.61282432312169e-05\\
2305.97147148843	1.60922261204783e-05\\
2306.24621855409	1.60562663026849e-05\\
2306.52096561976	1.60203638745176e-05\\
2306.79571268542	1.59845189318308e-05\\
2307.07045975109	1.59487315696528e-05\\
2307.34520681675	1.59130018821856e-05\\
2307.61995388242	1.58773299628048e-05\\
2307.89470094808	1.58417159040598e-05\\
2308.16944801375	1.58061597976738e-05\\
2308.44419507941	1.57704986516045e-05\\
2308.71894214508	1.57348975219056e-05\\
2308.99368921074	1.56995201867597e-05\\
2309.2684362764	1.56640379625115e-05\\
2309.54318334207	1.56286153007072e-05\\
2309.81793040773	1.55930877063384e-05\\
2310.0926774734	1.55579526867976e-05\\
2310.36742453906	1.55228761599784e-05\\
2310.64217160473	1.54878582120895e-05\\
2310.91691867039	1.54528989284944e-05\\
2311.19166573606	1.54178345187372e-05\\
2311.46641280172	1.53829950244025e-05\\
2311.74115986739	1.53480495827267e-05\\
2312.01590693305	1.53133301655822e-05\\
2312.29065399872	1.52786697715449e-05\\
2312.56540106438	1.52440684815589e-05\\
2312.84014813005	1.52093628600693e-05\\
2313.11489519571	1.51748822210416e-05\\
2313.38964226138	1.51404608959343e-05\\
2313.66438932704	1.51060989626204e-05\\
2313.93913639271	1.50717964981215e-05\\
2314.21388345837	1.50375535786083e-05\\
2314.48863052404	1.50033702794012e-05\\
2314.7633775897	1.49692466749707e-05\\
2315.03812465537	1.49351828389381e-05\\
2315.31287172103	1.49011788440762e-05\\
2315.5876187867	1.48670715731648e-05\\
2315.86236585236	1.48330246778962e-05\\
2316.13711291803	1.47992050283271e-05\\
2316.41185998369	1.47652818153342e-05\\
2316.68660704936	1.47315845762699e-05\\
2316.96135411502	1.46979475150633e-05\\
2317.23610118069	1.46643706995275e-05\\
2317.51084824635	1.46308541966212e-05\\
2317.78559531202	1.45973980724501e-05\\
2318.06034237768	1.45640023922675e-05\\
2318.33508944334	1.45306672204752e-05\\
2318.60983650901	1.44973926206237e-05\\
2318.88458357467	1.44640143878394e-05\\
2319.15933064034	1.44308633320485e-05\\
2319.434077706	1.43977730057768e-05\\
2319.70882477167	1.43647434695065e-05\\
2319.98357183733	1.4331774782869e-05\\
2320.258318903	1.42988670046462e-05\\
2320.53306596866	1.42660201927709e-05\\
2320.80781303433	1.42332344043282e-05\\
2321.08256009999	1.42005096955555e-05\\
2321.35730716566	1.4167846121844e-05\\
2321.63205423132	1.41352437377392e-05\\
2321.90680129699	1.41027025969418e-05\\
2322.18154836265	1.40702227523087e-05\\
2322.45629542832	1.40378042558536e-05\\
2322.73104249398	1.40052826856753e-05\\
2323.00578955965	1.39729892537721e-05\\
2323.28053662531	1.39407572930314e-05\\
2323.55528369098	1.39085868524306e-05\\
2323.83003075664	1.38764779801047e-05\\
2324.10477782231	1.38444307233472e-05\\
2324.37952488797	1.3812445128611e-05\\
2324.65427195364	1.37805212415093e-05\\
2324.9290190193	1.37486591068163e-05\\
2325.20376608497	1.37168587684685e-05\\
2325.47851315063	1.36851202695653e-05\\
2325.7532602163	1.36534436523696e-05\\
2326.02800728196	1.36218289583094e-05\\
2326.30275434763	1.3590276227978e-05\\
2326.57750141329	1.35587855011353e-05\\
2326.85224847896	1.35273568167087e-05\\
2327.12699554462	1.34959902127936e-05\\
2327.40174261028	1.34646857266547e-05\\
2327.67648967595	1.34332805302664e-05\\
2327.95123674161	1.34019380968604e-05\\
2328.22598380728	1.33708245624584e-05\\
2328.50073087294	1.33397732311488e-05\\
2328.77547793861	1.3308784136705e-05\\
2329.05022500427	1.32778573120676e-05\\
2329.32497206994	1.32469927893459e-05\\
2329.5997191356	1.32161905998186e-05\\
2329.87446620127	1.31854507739353e-05\\
2330.14921326693	1.31547733413163e-05\\
2330.4239603326	1.31241583307548e-05\\
2330.69870739826	1.3093605770217e-05\\
2330.97345446393	1.30629509449898e-05\\
2331.24820152959	1.30325255840154e-05\\
2331.52294859526	1.30021627239669e-05\\
2331.79769566092	1.29718623898415e-05\\
2332.07244272659	1.29416246058109e-05\\
2332.34718979225	1.29114493952222e-05\\
2332.62193685792	1.28811732676041e-05\\
2332.89668392358	1.28511254740379e-05\\
2333.17143098925	1.28211402911538e-05\\
2333.44617805491	1.27912177393369e-05\\
2333.72092512058	1.27613578381495e-05\\
2333.99567218624	1.27315606063317e-05\\
2334.27041925191	1.27016615089364e-05\\
2334.54516631757	1.2671991885326e-05\\
2334.81991338324	1.26420561365671e-05\\
2335.0946604489	1.2612516343468e-05\\
2335.36940751456	1.25830392177831e-05\\
2335.64415458023	1.25536247743173e-05\\
2335.91890164589	1.25241097595199e-05\\
2336.19364871156	1.24948229218925e-05\\
2336.46839577722	1.24655987781214e-05\\
2336.74314284289	1.24364373400417e-05\\
2337.01788990855	1.24073386186639e-05\\
2337.29263697422	1.23783026241748e-05\\
2337.56738403988	1.23491655440526e-05\\
2337.84213110555	1.23202572379076e-05\\
2338.11687817121	1.22914116563721e-05\\
2338.39162523688	1.22626288066767e-05\\
2338.66637230254	1.22339086952304e-05\\
2338.94111936821	1.22050881825754e-05\\
2339.21586643387	1.21764957623134e-05\\
2339.49061349954	1.2147801031827e-05\\
2339.7653605652	1.21193362871445e-05\\
2340.04010763087	1.20909342452222e-05\\
2340.31485469653	1.20625949082113e-05\\
2340.5896017622	1.20343182774421e-05\\
2340.86434882786	1.20059407058044e-05\\
2341.13909589353	1.19777916933261e-05\\
2341.41384295919	1.19497053582855e-05\\
2341.68859002486	1.19216816990691e-05\\
2341.96333709052	1.18937207132452e-05\\
2342.23808415619	1.18658223975654e-05\\
2342.51283122185	1.18379867479653e-05\\
2342.78757828752	1.18102137595659e-05\\
2343.06232535318	1.17825034266749e-05\\
2343.33707241885	1.17543640830969e-05\\
2343.61181948451	1.17267856652329e-05\\
2343.88656655017	1.16992697971731e-05\\
2344.16131361584	1.16716534022343e-05\\
2344.4360606815	1.16442648068777e-05\\
2344.71080774717	1.161677394569e-05\\
2344.98555481283	1.15895125508067e-05\\
2345.2603018785	1.15621492821836e-05\\
2345.53504894416	1.15348511275862e-05\\
2345.80979600983	1.15077814310685e-05\\
2346.08454307549	1.14807740890443e-05\\
2346.35929014116	1.14536651857879e-05\\
2346.63403720682	1.14267847206907e-05\\
2346.90878427249	1.139996654015e-05\\
2347.18353133815	1.13732106288489e-05\\
2347.45827840382	1.13461887136865e-05\\
2347.73302546948	1.13195617139668e-05\\
2348.00777253515	1.12929968767064e-05\\
2348.28251960081	1.12664941839688e-05\\
2348.55726666648	1.12400536169951e-05\\
2348.83201373214	1.12136751562062e-05\\
2349.10676079781	1.11873587812037e-05\\
2349.38150786347	1.11609410517593e-05\\
2349.65625492914	1.11347509860691e-05\\
2349.9310019948	1.11084579072537e-05\\
2350.20574906047	1.10823940241577e-05\\
2350.48049612613	1.10563920573373e-05\\
2350.7552431918	1.1030451982159e-05\\
2351.02999025746	1.10045737731734e-05\\
2351.30473732313	1.09787574041175e-05\\
2351.57948438879	1.09528392502325e-05\\
2351.85423145446	1.09271486834577e-05\\
2352.12897852012	1.09013555777367e-05\\
2352.40372558578	1.08757906514944e-05\\
2352.67847265145	1.08501243403758e-05\\
2352.95321971711	1.0824684863403e-05\\
2353.22796678278	1.07993069381412e-05\\
2353.50271384844	1.07738272867509e-05\\
2353.77746091411	1.07485745688287e-05\\
2354.05220797977	1.07233832770746e-05\\
2354.32695504544	1.06982533776251e-05\\
2354.6017021111	1.06731848358073e-05\\
2354.87644917677	1.06481776161409e-05\\
2355.15119624243	1.06232316823398e-05\\
2355.4259433081	1.05983469973134e-05\\
2355.70069037376	1.05735235231689e-05\\
2355.97543743943	1.05487612212124e-05\\
2356.25018450509	1.05240600519508e-05\\
2356.52493157076	1.04994199750934e-05\\
2356.79967863642	1.04748409495537e-05\\
2357.07442570209	1.04503229334507e-05\\
2357.34917276775	1.0425865884111e-05\\
2357.62391983342	1.04013048725084e-05\\
2357.89866689908	1.03769718462473e-05\\
2358.17341396475	1.0352699625923e-05\\
2358.44816103041	1.03284881660366e-05\\
2358.72290809608	1.03040095526773e-05\\
2358.99765516174	1.02799238914386e-05\\
2359.27240222741	1.02558987935686e-05\\
2359.54714929307	1.02317707061862e-05\\
2359.82189635874	1.02078687935963e-05\\
2360.0966434244	1.01840272701931e-05\\
2360.37139049007	1.01602460859592e-05\\
2360.64613755573	1.0136525190091e-05\\
2360.9208846214	1.01128645310012e-05\\
2361.19563168706	1.00892640563201e-05\\
2361.47037875272	1.0065560668918e-05\\
2361.74512581839	1.00420826002872e-05\\
2362.01987288405	1.00186645264868e-05\\
2362.29461994972	9.99530639236231e-06\\
2362.56936701538	9.97200814198288e-06\\
2362.84411408105	9.94876971864318e-06\\
2363.11886114671	9.92542733888026e-06\\
2363.39360821238	9.90231060249801e-06\\
2363.66835527804	9.87925349051589e-06\\
2363.94310234371	9.85625594347633e-06\\
2364.21784940937	9.83331790115322e-06\\
2364.49259647504	9.81043930255421e-06\\
2364.7673435407	9.78762008592232e-06\\
2365.04209060637	9.76486018873836e-06\\
2365.31683767203	9.74199522555756e-06\\
2365.5915847377	9.71935599028759e-06\\
2365.86633180336	9.6967758543684e-06\\
2366.14107886903	9.67409190615438e-06\\
2366.41582593469	9.65163196747845e-06\\
2366.69057300036	9.62923090245608e-06\\
2366.96532006602	9.60672496418082e-06\\
2367.24006713169	9.58444364973452e-06\\
2367.51481419735	9.56222097731953e-06\\
2367.78956126302	9.54005687785046e-06\\
2368.06430832868	9.51778738109072e-06\\
2368.33905539435	9.49574242554442e-06\\
2368.61380246001	9.47375580327246e-06\\
2368.88854952568	9.4518274425278e-06\\
2369.16329659134	9.42995727081814e-06\\
2369.43804365701	9.40814521490834e-06\\
2369.71279072267	9.38639120082264e-06\\
2369.98753778833	9.36453122169809e-06\\
2370.262284854	9.34289527506938e-06\\
2370.53703191966	9.32131711598932e-06\\
2370.81177898533	9.29979666786885e-06\\
2371.08652605099	9.27833385338603e-06\\
2371.36127311666	9.25676485277322e-06\\
2371.63602018232	9.23541927696123e-06\\
2371.91076724799	9.21413107054929e-06\\
2372.18551431365	9.192900153634e-06\\
2372.46026137932	9.17156159154528e-06\\
2372.73500844498	9.15044723135386e-06\\
2373.00975551065	9.12938988854895e-06\\
2373.28450257631	9.10822544275417e-06\\
2373.55924964198	9.08728409665203e-06\\
2373.83399670764	9.06639949046554e-06\\
2374.10874377331	9.04557153991544e-06\\
2374.38349083897	9.02480016000814e-06\\
2374.65823790464	9.0040852650383e-06\\
2374.9329849703	8.98342676859129e-06\\
2375.20773203597	8.96282458354599e-06\\
2375.48247910163	8.94227862207709e-06\\
2375.7572261673	8.92178879565796e-06\\
2376.03197323296	8.90135501506307e-06\\
2376.30672029863	8.88097719037074e-06\\
2376.58146736429	8.86065523096559e-06\\
2376.85621442996	8.84038904554143e-06\\
2377.13096149562	8.82017854210369e-06\\
2377.40570856129	8.80002362797229e-06\\
2377.68045562695	8.77992420978424e-06\\
2377.95520269262	8.75971587424474e-06\\
2378.22994975828	8.73972937871557e-06\\
2378.50469682395	8.71963508978551e-06\\
2378.77944388961	8.69976106273342e-06\\
2379.05419095527	8.67961368992779e-06\\
2379.32893802094	8.65985391684404e-06\\
2379.6036850866	8.64014885751823e-06\\
2379.87843215227	8.62049841450901e-06\\
2380.15317921793	8.60090248970304e-06\\
2380.4279262836	8.58136098431842e-06\\
2380.70267334926	8.56187379890712e-06\\
2380.97742041493	8.54244083335838e-06\\
2381.25216748059	8.52306198690142e-06\\
2381.52691454626	8.50373715810871e-06\\
2381.80166161192	8.48446624489869e-06\\
2382.07640867759	8.46524914453921e-06\\
2382.35115574325	8.4460857536501e-06\\
2382.62590280892	8.42697596820666e-06\\
2382.90064987458	8.40775662792858e-06\\
2383.17539694025	8.38875593634688e-06\\
2383.45014400591	8.36980850650249e-06\\
2383.72489107158	8.35091423214822e-06\\
2383.99963813724	8.3320730064044e-06\\
2384.27438520291	8.31328472176185e-06\\
2384.54913226857	8.29422312638873e-06\\
2384.82387933424	8.27554479454661e-06\\
2385.0986263999	8.25691902193593e-06\\
2385.37337346557	8.23834569921306e-06\\
2385.64812053123	8.21982471641445e-06\\
2385.9228675969	8.20135596295964e-06\\
2386.19761466256	8.18293932765487e-06\\
2386.47236172823	8.16457469869634e-06\\
2386.74710879389	8.14626196367363e-06\\
2387.02185585956	8.12800100957305e-06\\
2387.29660292522	8.10979172278126e-06\\
2387.57134999088	8.09163398908843e-06\\
2387.84609705655	8.07352769369198e-06\\
2388.12084412221	8.05547272119987e-06\\
2388.39559118788	8.03746895563423e-06\\
2388.67033825354	8.01951628043467e-06\\
2388.94508531921	8.00161457846206e-06\\
2389.21983238487	7.98376373200185e-06\\
2389.49457945054	7.96579859673234e-06\\
2389.7693265162	7.94805132837892e-06\\
2390.04407358187	7.93019148957006e-06\\
2390.31882064753	7.91254724072185e-06\\
2390.5935677132	7.89478848668178e-06\\
2390.86831477886	7.8772467422981e-06\\
2391.14306184453	7.85975493163447e-06\\
2391.41780891019	7.84231293346481e-06\\
2391.69255597586	7.82475645156542e-06\\
2391.96730304152	7.80741592424723e-06\\
2392.24205010719	7.78996162191829e-06\\
2392.51679717285	7.77255820124378e-06\\
2392.79154423852	7.75536993537518e-06\\
2393.06629130418	7.73823066077699e-06\\
2393.34103836985	7.72114025339902e-06\\
2393.61578543551	7.70409858865501e-06\\
2393.89053250118	7.68694053911853e-06\\
2394.16527956684	7.66999820596815e-06\\
2394.44002663251	7.65310421043151e-06\\
2394.71477369817	7.63625842614656e-06\\
2394.98952076384	7.61946072623171e-06\\
2395.2642678295	7.6027109832902e-06\\
2395.53901489517	7.58600906941404e-06\\
2395.81376196083	7.56918999207244e-06\\
2396.08850902649	7.5525855710431e-06\\
2396.36325609216	7.53602856426618e-06\\
2396.63800315782	7.51951884215934e-06\\
2396.91275022349	7.50305627464649e-06\\
2397.18749728915	7.48664073116235e-06\\
2397.46224435482	7.47027208065639e-06\\
2397.73699142048	7.45395019159741e-06\\
2398.01173848615	7.43767493197765e-06\\
2398.28648555181	7.42144616931735e-06\\
2398.56123261748	7.40509907326733e-06\\
2398.83597968314	7.38896512357704e-06\\
2399.11072674881	7.37287724257805e-06\\
2399.38547381447	7.3568352962738e-06\\
2399.66022088014	7.34083915021391e-06\\
2399.9349679458	7.32488866949885e-06\\
2400.20971501147	7.30898371878431e-06\\
2400.48446207713	7.29312416228573e-06\\
2400.7592091428	7.27730986378271e-06\\
2401.03395620846	7.2615406866237e-06\\
2401.30870327413	7.24581649373035e-06\\
2401.58345033979	7.23013714760221e-06\\
2401.85819740546	7.21450251032125e-06\\
2402.13294447112	7.19891244355633e-06\\
2402.40769153679	7.18336680856797e-06\\
2402.68243860245	7.16786546621288e-06\\
2402.95718566812	7.15240827694848e-06\\
2403.23193273378	7.13699510083774e-06\\
2403.50667979945	7.1216257975537e-06\\
2403.78142686511	7.1063002263841e-06\\
2404.05617393078	7.09101824623623e-06\\
2404.33092099644	7.07577971564141e-06\\
2404.6056680621	7.06058449275989e-06\\
2404.88041512777	7.04543243538546e-06\\
2405.15516219343	7.03032340095025e-06\\
2405.4299092591	7.01525724652936e-06\\
2405.70465632476	7.00023382884582e-06\\
2405.97940339043	6.98525300427518e-06\\
2406.25415045609	6.97031462885043e-06\\
2406.52889752176	6.95541855826674e-06\\
2406.80364458742	6.94056464788634e-06\\
2407.07839165309	6.92575275274323e-06\\
2407.35313871875	6.91098272754818e-06\\
2407.62788578442	6.89625442669344e-06\\
2407.90263285008	6.88156770425774e-06\\
2408.17737991575	6.86692241401112e-06\\
2408.45212698141	6.8523184094198e-06\\
2408.72687404708	6.83775554365109e-06\\
2409.00162111274	6.82323366957841e-06\\
2409.27636817841	6.80875263978603e-06\\
2409.55111524407	6.79414828747185e-06\\
2409.82586230974	6.77975071264844e-06\\
2410.1006093754	6.76539351023444e-06\\
2410.37535644107	6.75107653203465e-06\\
2410.65010350673	6.73679962958613e-06\\
2410.9248505724	6.72256265416339e-06\\
2411.19959763806	6.70836545678335e-06\\
2411.47434470373	6.69420788821043e-06\\
2411.74909176939	6.68008979896156e-06\\
2412.02383883506	6.66601103931139e-06\\
2412.29858590072	6.65197145929719e-06\\
2412.57333296639	6.6379709087242e-06\\
2412.84808003205	6.62400923717044e-06\\
2413.12282709771	6.61008629399213e-06\\
2413.39757416338	6.59620192832856e-06\\
2413.67232122904	6.58235598910745e-06\\
2413.94706829471	6.56854832504987e-06\\
2414.22181536037	6.5547787846757e-06\\
2414.49656242604	6.54104721630844e-06\\
2414.7713094917	6.5273534680807e-06\\
2415.04605655737	6.51369738793916e-06\\
2415.32080362303	6.50007882364991e-06\\
2415.5955506887	6.4864976228036e-06\\
2415.87029775436	6.47295363282058e-06\\
2416.14504482003	6.45928206897939e-06\\
2416.41979188569	6.44581425986013e-06\\
2416.69453895136	6.43221891645465e-06\\
2416.96928601702	6.41882661551677e-06\\
2417.24403308269	6.40547070482158e-06\\
2417.51878014835	6.39215103152618e-06\\
2417.79352721402	6.37886744264736e-06\\
2418.06827427968	6.36545490342704e-06\\
2418.34302134535	6.3522452445813e-06\\
2418.61776841101	6.33907118234635e-06\\
2418.89251547668	6.3259325635589e-06\\
2419.16726254234	6.31282923493835e-06\\
2419.44200960801	6.29976104309239e-06\\
2419.71675667367	6.2867278345222e-06\\
2419.99150373934	6.273729455628e-06\\
2420.266250805	6.26076575271436e-06\\
2420.54099787067	6.24783657199564e-06\\
2420.81574493633	6.23494175960136e-06\\
2421.090492002	6.22208116158171e-06\\
2421.36523906766	6.20925462391283e-06\\
2421.63998613333	6.19646199250235e-06\\
2421.91473319899	6.18353859236773e-06\\
2422.18948026465	6.17081552707629e-06\\
2422.46422733032	6.15812587739566e-06\\
2422.73897439598	6.14546948934118e-06\\
2423.01372146165	6.13284620888338e-06\\
2423.28846852731	6.12025588195332e-06\\
2423.56321559298	6.10769835444811e-06\\
2423.83796265864	6.0951734722363e-06\\
2424.11270972431	6.08268108116348e-06\\
2424.38745678997	6.07005787524233e-06\\
2424.66220385564	6.05763220274365e-06\\
2424.9369509213	6.04523853102606e-06\\
2425.21169798697	6.03287670622327e-06\\
2425.48644505263	6.02054657446976e-06\\
2425.7611921183	6.00824798190642e-06\\
2426.03593918396	5.99598077468589e-06\\
2426.31068624963	5.98374479897826e-06\\
2426.58543331529	5.9715399009763e-06\\
2426.86018038096	5.95936592690117e-06\\
2427.13492744662	5.94705785123474e-06\\
2427.40967451229	5.9349474843775e-06\\
2427.68442157795	5.92286755226988e-06\\
2427.95916864362	5.91065413488986e-06\\
2428.23391570928	5.8986368190835e-06\\
2428.50866277495	5.88664945043946e-06\\
2428.78340984061	5.87452850323494e-06\\
2429.05815690628	5.86260277225224e-06\\
2429.33290397194	5.85070650244118e-06\\
2429.60765103761	5.83883954155723e-06\\
2429.88239810327	5.82700173743428e-06\\
2430.15714516894	5.81502867066943e-06\\
2430.4318922346	5.80325093683434e-06\\
2430.70663930027	5.79150187581117e-06\\
2430.98138636593	5.77978133612987e-06\\
2431.25613343159	5.76808916642269e-06\\
2431.53088049726	5.75642521543003e-06\\
2431.80562756292	5.74478933200572e-06\\
2432.08037462859	5.73318136512274e-06\\
2432.35512169425	5.7216011638785e-06\\
2432.62986875992	5.70988448080065e-06\\
2432.90461582558	5.69836156941946e-06\\
2433.17936289125	5.68670188755533e-06\\
2433.45410995691	5.67523560797382e-06\\
2433.72885702258	5.66379628633433e-06\\
2434.00360408824	5.65238377324407e-06\\
2434.27835115391	5.64099791946619e-06\\
2434.55309821957	5.62963857592536e-06\\
2434.82784528524	5.61830559371319e-06\\
2435.1025923509	5.60699882409365e-06\\
2435.37733941657	5.59571811850869e-06\\
2435.65208648223	5.58446332858351e-06\\
2435.9268335479	5.57306939189107e-06\\
2436.20158061356	5.56186821001827e-06\\
2436.47632767923	5.55069247176359e-06\\
2436.75107474489	5.53954202986814e-06\\
2437.02582181056	5.52841673727985e-06\\
2437.30056887622	5.51731644715899e-06\\
2437.57531594189	5.50624101288338e-06\\
2437.85006300755	5.49519028805397e-06\\
2438.12481007322	5.4841641265e-06\\
2438.39955713888	5.47316238228446e-06\\
2438.67430420455	5.46218490970939e-06\\
2438.94905127021	5.45123156332117e-06\\
2439.22379833588	5.44030219791574e-06\\
2439.49854540154	5.42939666854406e-06\\
2439.77329246721	5.41851483051713e-06\\
2440.04803953287	5.40765653941148e-06\\
2440.32278659853	5.39682165107414e-06\\
2440.5975336642	5.38601002162806e-06\\
2440.87228072986	5.37522150747715e-06\\
2441.14702779553	5.36445596531154e-06\\
2441.42177486119	5.35371325211267e-06\\
2441.69652192686	5.34282818259227e-06\\
2441.97126899252	5.33196880327641e-06\\
2442.24601605819	5.32113495608873e-06\\
2442.52076312385	5.31048868360917e-06\\
2442.79551018952	5.29986444667509e-06\\
2443.07025725518	5.28926210509405e-06\\
2443.34500432085	5.27835520651281e-06\\
2443.61975138651	5.26780063409181e-06\\
2443.89449845218	5.25726748274279e-06\\
2444.16924551784	5.24675561422119e-06\\
2444.44399258351	5.23626489061224e-06\\
2444.71873964917	5.22579517433624e-06\\
2444.99348671484	5.21534632815334e-06\\
2445.2682337805	5.20491821516876e-06\\
2445.54298084617	5.19451069883765e-06\\
2445.81772791183	5.18412364297021e-06\\
2446.0924749775	5.17375691173646e-06\\
2446.36722204316	5.16324665881327e-06\\
2446.64196910883	5.15292237671241e-06\\
2446.91671617449	5.14261798607042e-06\\
2447.19146324016	5.13233335287246e-06\\
2447.46621030582	5.12206834348459e-06\\
2447.74095737148	5.11182282465848e-06\\
2448.01570443715	5.10159666353625e-06\\
2448.29045150281	5.09138972765529e-06\\
2448.56519856848	5.08120188495294e-06\\
2448.83994563414	5.07103300377127e-06\\
2449.11469269981	5.06071797875927e-06\\
2449.38943976547	5.05058884914037e-06\\
2449.66418683114	5.04047826046311e-06\\
2449.9389338968	5.03038608306359e-06\\
2450.21368096247	5.0203121877023e-06\\
2450.48842802813	5.01025644556884e-06\\
2450.7631750938	5.00021872828629e-06\\
2451.03792215946	4.99019890791602e-06\\
2451.31266922513	4.98019685696188e-06\\
2451.58741629079	4.97021244837493e-06\\
2451.86216335646	4.9602455555577e-06\\
2452.13691042212	4.95029605236877e-06\\
2452.41165748779	4.94019992750867e-06\\
2452.68640455345	4.93028703509744e-06\\
2452.96115161912	4.92039112875011e-06\\
2453.23589868478	4.91051208451943e-06\\
2453.51064575045	4.90064977893222e-06\\
2453.78539281611	4.89080408899371e-06\\
2454.06013988178	4.88097489219179e-06\\
2454.33488694744	4.87116206650106e-06\\
2454.60963401311	4.86136549038722e-06\\
2454.88438107877	4.85158504281098e-06\\
2455.15912814444	4.84182060323237e-06\\
2455.4338752101	4.83207205161464e-06\\
2455.70862227577	4.82233926842843e-06\\
2455.98336934143	4.81262213465564e-06\\
2456.25811640709	4.80292053179354e-06\\
2456.53286347276	4.79323434185858e-06\\
2456.80761053842	4.78356344739041e-06\\
2457.08235760409	4.77390773145565e-06\\
2457.35710466975	4.76426707765181e-06\\
2457.63185173542	4.754641370111e-06\\
2457.90659880108	4.74503049350379e-06\\
2458.18134586675	4.73543433304284e-06\\
2458.45609293241	4.72585277448675e-06\\
2458.73083999808	4.71628570414351e-06\\
2459.00558706374	4.70673300887436e-06\\
2459.28033412941	4.69719457609715e-06\\
2459.55508119507	4.68767029379011e-06\\
2459.82982826074	4.6781600504952e-06\\
2460.1045753264	4.66866373532171e-06\\
2460.37932239207	4.65918123794965e-06\\
2460.65406945773	4.64971244863315e-06\\
2460.9288165234	4.64025725820389e-06\\
2461.20356358906	4.63081555807441e-06\\
2461.47831065473	4.62138724024139e-06\\
2461.75305772039	4.61197219728895e-06\\
2462.02780478606	4.60257032239187e-06\\
2462.30255185172	4.59318150931874e-06\\
2462.57729891739	4.5838056524352e-06\\
2462.85204598305	4.57444264670695e-06\\
2463.12679304872	4.56509238770295e-06\\
2463.40154011438	4.55575477159832e-06\\
2463.67628718005	4.54642969517748e-06\\
2463.95103424571	4.53711705583699e-06\\
2464.22578131138	4.52781675158859e-06\\
2464.50052837704	4.51852868106198e-06\\
2464.77527544271	4.5092527435078e-06\\
2465.05002250837	4.49998883880028e-06\\
2465.32476957403	4.49073686744016e-06\\
2465.5995166397	4.4814967305573e-06\\
2465.87426370536	4.4722683299135e-06\\
2466.14901077103	4.46305156790504e-06\\
2466.42375783669	4.45384634756539e-06\\
2466.69850490236	4.44465257256766e-06\\
2466.97325196802	4.4354701472273e-06\\
2467.24799903369	4.42629897650447e-06\\
2467.52274609935	4.41713896600655e-06\\
2467.79749316502	4.40799002199057e-06\\
2468.07224023068	4.39885205136553e-06\\
2468.34698729635	4.38939761404877e-06\\
2468.62173436201	4.38028572439743e-06\\
2468.89648142768	4.37118447702298e-06\\
2469.17122849334	4.36209378211832e-06\\
2469.44597555901	4.35301355053272e-06\\
2469.72072262467	4.343943693774e-06\\
2469.99546969034	4.33488412401076e-06\\
2470.270216756	4.32583475407441e-06\\
2470.54496382167	4.3167954974613e-06\\
2470.81971088733	4.30776626833477e-06\\
2471.094457953	4.2987469815271e-06\\
2471.36920501866	4.28973755254146e-06\\
2471.64395208433	4.28057419930101e-06\\
2471.91869914999	4.27142246702369e-06\\
2472.19344621566	4.2624464983158e-06\\
2472.46819328132	4.25348000069159e-06\\
2472.74294034699	4.24452289368658e-06\\
2473.01768741265	4.23541099239675e-06\\
2473.29243447832	4.22647463881147e-06\\
2473.56718154398	4.21754741055603e-06\\
2473.84192860965	4.2086292302027e-06\\
2474.11667567531	4.19972002100181e-06\\
2474.39142274097	4.19081970688329e-06\\
2474.66616980664	4.1819282124583e-06\\
2474.9409168723	4.17288140694775e-06\\
2475.21566393797	4.16400953872921e-06\\
2475.49041100363	4.15514624019479e-06\\
2475.7651580693	4.14629143902198e-06\\
2476.03990513496	4.13744506357304e-06\\
2476.31465220063	4.12860704289641e-06\\
2476.58939926629	4.11977730672786e-06\\
2476.86414633196	4.11095578549191e-06\\
2477.13889339762	4.10214241030277e-06\\
2477.41364046329	4.09333711296573e-06\\
2477.68838752895	4.08453982597809e-06\\
2477.96313459462	4.07575048253038e-06\\
2478.23788166028	4.06696901650717e-06\\
2478.51262872595	4.05819536248833e-06\\
2478.78737579161	4.04942945574966e-06\\
2479.06212285728	4.04067123226403e-06\\
2479.33686992294	4.03192062870204e-06\\
2479.61161698861	4.02317758243297e-06\\
2479.88636405427	4.01427756228092e-06\\
2480.16111111994	4.00555166097843e-06\\
2480.4358581856	3.99683310527e-06\\
2480.71060525127	3.9881218356542e-06\\
2480.98535231693	3.97941779332818e-06\\
2481.2600993826	3.97072092018811e-06\\
2481.53484644826	3.96203115882993e-06\\
2481.80959351393	3.95334845254959e-06\\
2482.08434057959	3.94467274534369e-06\\
2482.35908764525	3.93600398190978e-06\\
2482.63383471092	3.92734210764683e-06\\
2482.90858177658	3.91868706865539e-06\\
2483.18332884225	3.9100388117381e-06\\
2483.45807590791	3.90139728439968e-06\\
2483.73282297358	3.8927624348473e-06\\
2484.00757003924	3.88397128899103e-06\\
2484.28231710491	3.87535183837395e-06\\
2484.55706417057	3.86673888660289e-06\\
2484.83181123624	3.85813238502091e-06\\
2485.1065583019	3.84953228567032e-06\\
2485.38130536757	3.84093854129274e-06\\
2485.65605243323	3.83235110532889e-06\\
2485.9307994989	3.82376993191863e-06\\
2486.20554656456	3.8151949759007e-06\\
2486.48029363023	3.80662619281252e-06\\
2486.75504069589	3.79790029625247e-06\\
2487.02978776156	3.78934592840234e-06\\
2487.30453482722	3.78079757645963e-06\\
2487.57928189289	3.77225519907959e-06\\
2487.85402895855	3.76371875561117e-06\\
2488.12877602422	3.75518820609651e-06\\
2488.40352308988	3.74666351127058e-06\\
2488.67827015555	3.7379803922692e-06\\
2488.95301722121	3.7294695043759e-06\\
2489.22776428688	3.72096432955576e-06\\
2489.50251135254	3.7124648316368e-06\\
2489.77725841821	3.70397097513344e-06\\
2490.05200548387	3.69548272524588e-06\\
2490.32675254954	3.68700004785932e-06\\
2490.6014996152	3.67802812744932e-06\\
2490.87624668087	3.66956315924419e-06\\
2491.15099374653	3.66110358107797e-06\\
2491.42574081219	3.65264936255327e-06\\
2491.70048787786	3.64420047394318e-06\\
2491.97523494352	3.63575688619034e-06\\
2492.24998200919	3.62731857090608e-06\\
2492.52472907485	3.61872072727154e-06\\
2492.79947614052	3.61029509374212e-06\\
2493.07422320618	3.60187462346716e-06\\
2493.34897027185	3.59345929105112e-06\\
2493.62371733751	3.58504907175853e-06\\
2493.89846440318	3.57647952811532e-06\\
2494.17321146884	3.56808167826732e-06\\
2494.44795853451	3.55968884328838e-06\\
2494.72270560017	3.55130100139969e-06\\
2494.99745266584	3.54291813147309e-06\\
2495.2721997315	3.53454021302997e-06\\
2495.54694679717	3.52600260382685e-06\\
2495.82169386283	3.51763674691585e-06\\
2496.0964409285	3.50927575590738e-06\\
2496.37118799416	3.50091961323506e-06\\
2496.64593505983	3.49256830197149e-06\\
2496.92068212549	3.48422180582695e-06\\
2497.19542919116	3.47571700840653e-06\\
2497.47017625682	3.4673822943521e-06\\
2497.74492332249	3.45905232255412e-06\\
2498.01967038815	3.45056220897158e-06\\
2498.29441745382	3.44224390257023e-06\\
2498.56916451948	3.43393027194921e-06\\
2498.84391158515	3.42562130588104e-06\\
2499.11865865081	3.41731699375596e-06\\
2499.39340571648	3.40901732558005e-06\\
2499.66815278214	3.40072229197353e-06\\
2499.9428998478	3.39243188416896e-06\\
2500.21764691347	3.38414609400928e-06\\
2500.49239397913	3.3758649139461e-06\\
2500.7671410448	3.36758833703758e-06\\
2501.04188811046	3.35931635694666e-06\\
2501.31663517613	3.3510489679389e-06\\
2501.59138224179	3.34278616488063e-06\\
2501.86612930746	3.3345279432367e-06\\
2502.14087637312	3.32627429906854e-06\\
2502.41562343879	3.3180252290319e-06\\
2502.69037050445	3.3097807303748e-06\\
2502.96511757012	3.30154080093525e-06\\
2503.23986463578	3.29314072979355e-06\\
2503.51461170145	3.28491215316031e-06\\
2503.78935876711	3.27668811473511e-06\\
2504.06410583278	3.26830359510023e-06\\
2504.33885289844	3.26009085760827e-06\\
2504.61359996411	3.25171871391439e-06\\
2504.88834702977	3.2435172142095e-06\\
2505.16309409544	3.23532020559874e-06\\
2505.4378411611	3.22712769208535e-06\\
2505.71258822677	3.21893967822852e-06\\
2505.98733529243	3.21075616914106e-06\\
2506.2620823581	3.20257717048684e-06\\
2506.53682942376	3.19440268847825e-06\\
2506.81157648943	3.18623272987372e-06\\
2507.08632355509	3.17806730197505e-06\\
2507.36107062076	3.16990641262488e-06\\
2507.63581768642	3.16175007020399e-06\\
2507.91056475209	3.15359828362866e-06\\
2508.18531181775	3.1454510623479e-06\\
2508.46005888341	3.13730841634082e-06\\
2508.73480594908	3.12884260381402e-06\\
2509.00955301474	3.12071355635555e-06\\
2509.28430008041	3.11258906142437e-06\\
2509.55904714607	3.10446913174537e-06\\
2509.83379421174	3.09635378055239e-06\\
2510.1085412774	3.08824302158555e-06\\
2510.38328834307	3.0801368690882e-06\\
2510.65803540873	3.0720353378042e-06\\
2510.9327824744	3.06393844297485e-06\\
2511.20752954006	3.05568269127464e-06\\
2511.48227660573	3.04759732039421e-06\\
2511.75702367139	3.03951660678403e-06\\
2512.03177073706	3.03144056797189e-06\\
2512.30651780272	3.02336922196463e-06\\
2512.58126486839	3.01513809456124e-06\\
2512.85601193405	3.0070784058561e-06\\
2513.13075899972	2.99902343881089e-06\\
2513.40550606538	2.99097321364561e-06\\
2513.68025313105	2.98276478398363e-06\\
2513.95500019671	2.97472630156428e-06\\
2514.22974726238	2.96669259664948e-06\\
2514.50449432804	2.95866369160607e-06\\
2514.77924139371	2.95063960924495e-06\\
2515.05398845937	2.94262037281767e-06\\
2515.32873552504	2.93460600601346e-06\\
2515.6034825907	2.92659653295575e-06\\
2515.87822965637	2.91859197819906e-06\\
2516.15297672203	2.91059236672562e-06\\
2516.4277237877	2.90259772394214e-06\\
2516.70247085336	2.89460807567642e-06\\
2516.97721791902	2.88662344817403e-06\\
2517.25196498469	2.87864386809491e-06\\
2517.52671205035	2.87066936251e-06\\
2517.80145911602	2.86269995889781e-06\\
2518.07620618168	2.854735685141e-06\\
2518.35095324735	2.84677656952296e-06\\
2518.62570031301	2.83882264072423e-06\\
2518.90044737868	2.83087392781915e-06\\
2519.17519444434	2.82293046027223e-06\\
2519.44994151001	2.81499226793471e-06\\
2519.72468857567	2.80705938104098e-06\\
2519.99943564134	2.79913183020502e-06\\
2520.274182707	2.7912096464168e-06\\
2520.54892977267	2.78312880244222e-06\\
2520.82367683833	2.77521965749123e-06\\
2521.098423904	2.76731594683288e-06\\
2521.37317096966	2.75925374743847e-06\\
2521.64791803533	2.75136321306145e-06\\
2521.92266510099	2.74347818446155e-06\\
2522.19741216666	2.73559869580284e-06\\
2522.47215923232	2.72772478158931e-06\\
2522.74690629799	2.71985647666131e-06\\
2523.02165336365	2.71199381619178e-06\\
2523.29640042932	2.70397350796574e-06\\
2523.57114749498	2.69612444429216e-06\\
2523.84589456065	2.68828110471781e-06\\
2524.12064162631	2.68044352603504e-06\\
2524.39538869198	2.67261174534688e-06\\
2524.67013575764	2.66478580006323e-06\\
2524.94488282331	2.65696572789717e-06\\
2525.21962988897	2.64915156686118e-06\\
2525.49437695463	2.64134335526347e-06\\
2525.7691240203	2.63354113170408e-06\\
2526.04387108596	2.62574493507124e-06\\
2526.31861815163	2.61795480453751e-06\\
2526.59336521729	2.61017077955596e-06\\
2526.86811228296	2.60239289985646e-06\\
2527.14285934862	2.59445768899279e-06\\
2527.41760641429	2.58669442356844e-06\\
2527.69235347995	2.57893739637289e-06\\
2527.96710054562	2.57118664853087e-06\\
2528.24184761128	2.56344222142114e-06\\
2528.51659467695	2.55570415667279e-06\\
2528.79134174261	2.54797249616129e-06\\
2529.06608880828	2.54024728200477e-06\\
2529.34083587394	2.53252855656001e-06\\
2529.61558293961	2.52465333801923e-06\\
2529.89033000527	2.51694991521761e-06\\
2530.16507707094	2.50925308179986e-06\\
2530.4398241366	2.50156288136449e-06\\
2530.71457120227	2.49387935772592e-06\\
2530.98931826793	2.48620255491063e-06\\
2531.2640653336	2.47820579721272e-06\\
2531.53881239926	2.47054697186307e-06\\
2531.81355946493	2.46289494496637e-06\\
2532.08830653059	2.45524976201054e-06\\
2532.36305359626	2.44761146866928e-06\\
2532.63780066192	2.43998011079829e-06\\
2532.91254772759	2.43219283973924e-06\\
2533.18729479325	2.42457768665812e-06\\
2533.46204185892	2.41696957989482e-06\\
2533.73678892458	2.40936856631983e-06\\
2534.01153599025	2.40177469296291e-06\\
2534.28628305591	2.3941880070094e-06\\
2534.56103012157	2.38660855579637e-06\\
2534.83577718724	2.3788714916309e-06\\
2535.1105242529	2.37130887335377e-06\\
2535.38527131857	2.36375360463191e-06\\
2535.66001838423	2.35620573369947e-06\\
2535.9347654499	2.34866530891985e-06\\
2536.20951251556	2.34113237878177e-06\\
2536.48425958123	2.3336069918956e-06\\
2536.75900664689	2.32608919698951e-06\\
2537.03375371256	2.3185790429057e-06\\
2537.30850077822	2.31107657859659e-06\\
2537.58324784389	2.30358185312111e-06\\
2537.85799490955	2.29609491564081e-06\\
2538.13274197522	2.28861581541616e-06\\
2538.40748904088	2.28114460180277e-06\\
2538.68223610655	2.27368132424752e-06\\
2538.95698317221	2.26622603228492e-06\\
2539.23173023788	2.2587787755332e-06\\
2539.50647730354	2.25133960369066e-06\\
2539.78122436921	2.24390856653181e-06\\
2540.05597143487	2.23648571390368e-06\\
2540.33071850054	2.22907109572199e-06\\
2540.6054655662	2.22166476196746e-06\\
2540.88021263187	2.214266762682e-06\\
2541.15495969753	2.206877147965e-06\\
2541.4297067632	2.19933227069477e-06\\
2541.70445382886	2.19196178134343e-06\\
2541.97920089453	2.18459979927454e-06\\
2542.25394796019	2.17724637510651e-06\\
2542.52869502586	2.16990155949306e-06\\
2542.80344209152	2.16256540311968e-06\\
2543.07818915719	2.15523795669975e-06\\
2543.35293622285	2.14791927097109e-06\\
2543.62768328851	2.1406093966921e-06\\
2543.90243035418	2.13330838463829e-06\\
2544.17717741984	2.12601628559847e-06\\
2544.45192448551	2.11873315037125e-06\\
2544.72667155117	2.11145902976134e-06\\
2545.00141861684	2.10419397457589e-06\\
2545.2761656825	2.09693803562099e-06\\
2545.55091274817	2.08969126369792e-06\\
2545.82565981383	2.08245370959973e-06\\
2546.1004068795	2.07522542410748e-06\\
2546.37515394516	2.06800645798679e-06\\
2546.64990101083	2.06079686198419e-06\\
2546.92464807649	2.05359668682364e-06\\
2547.19939514216	2.04640598320288e-06\\
2547.47414220782	2.03922480179004e-06\\
2547.74888927349	2.03205319321996e-06\\
2548.02363633915	2.0248912080908e-06\\
2548.29838340482	2.01773889696044e-06\\
2548.57313047048	2.01059631034307e-06\\
2548.84787753615	2.00346349870567e-06\\
2549.12262460181	1.99617685291347e-06\\
2549.39737166748	1.98906594767332e-06\\
2549.67211873314	1.98196494061018e-06\\
2549.94686579881	1.97487388229406e-06\\
2550.22161286447	1.96779282322712e-06\\
2550.49635993014	1.96072181384028e-06\\
2550.7711069958	1.95349707927442e-06\\
2551.04585406147	1.94644852757505e-06\\
2551.32060112713	1.93941014847695e-06\\
2551.59534819279	1.93238199241925e-06\\
2551.87009525846	1.92536410974975e-06\\
2552.14484232412	1.91835655072163e-06\\
2552.41958938979	1.91135936549016e-06\\
2552.69433645545	1.90437260410957e-06\\
2552.96908352112	1.89739631652976e-06\\
2553.24383058678	1.89043055259313e-06\\
2553.51857765245	1.88347536203139e-06\\
2553.79332471811	1.87653079446242e-06\\
2554.06807178378	1.86959689938703e-06\\
2554.34281884944	1.86267372618596e-06\\
2554.61756591511	1.85576132411661e-06\\
2554.89231298077	1.84885974231006e-06\\
2555.16706004644	1.84196902976791e-06\\
2555.4418071121	1.83508923535923e-06\\
2555.71655417777	1.82822040781747e-06\\
2555.99130124343	1.82136259573752e-06\\
2556.2660483091	1.81451584757254e-06\\
2556.54079537476	1.80768021163109e-06\\
2556.81554244043	1.80085573607406e-06\\
2557.09028950609	1.79404246891176e-06\\
2557.36503657176	1.78707693229599e-06\\
2557.63978363742	1.78012522230406e-06\\
2557.91453070309	1.77335024203855e-06\\
2558.18927776875	1.76658660528484e-06\\
2558.46402483442	1.75983435990473e-06\\
2558.73877190008	1.753093553585e-06\\
2559.01351896575	1.74636423383478e-06\\
2559.28826603141	1.73964644798269e-06\\
2559.56301309708	1.7329402431742e-06\\
2559.83776016274	1.72624566636878e-06\\
2560.1125072284	1.7195627643373e-06\\
2560.38725429407	1.71289158365929e-06\\
2560.66200135973	1.70623217072031e-06\\
2560.9367484254	1.69958457170926e-06\\
2561.21149549106	1.69294883261585e-06\\
2561.48624255673	1.68632499922784e-06\\
2561.76098962239	1.6797131171286e-06\\
2562.03573668806	1.67311323169444e-06\\
2562.31048375372	1.66652538809216e-06\\
2562.58523081939	1.65978540324955e-06\\
2562.85997788505	1.65322399038724e-06\\
2563.13472495072	1.64667472560445e-06\\
2563.40947201638	1.64013765353193e-06\\
2563.68421908205	1.63361281857499e-06\\
2563.95896614771	1.62710026491107e-06\\
2564.23371321338	1.62043504721365e-06\\
2564.50846027904	1.61394940978944e-06\\
2564.78320734471	1.60747615671734e-06\\
2565.05795441037	1.60101533156837e-06\\
2565.33270147604	1.59456697767063e-06\\
2565.6074485417	1.58813113810695e-06\\
2565.88219560737	1.58170785571283e-06\\
2566.15694267303	1.57529717307409e-06\\
2566.4316897387	1.56889913252481e-06\\
2566.70643680436	1.56251377614513e-06\\
2566.98118387003	1.55614114575916e-06\\
2567.25593093569	1.54978128293286e-06\\
2567.53067800136	1.54343422897196e-06\\
2567.80542506702	1.53710002491985e-06\\
2568.08017213269	1.53077871155567e-06\\
2568.35491919835	1.52447032939213e-06\\
2568.62966626402	1.51817491867367e-06\\
2568.90441332968	1.51189251937437e-06\\
2569.17916039534	1.50562317119609e-06\\
2569.45390746101	1.49936691356645e-06\\
2569.72865452667	1.49312378563704e-06\\
2570.00340159234	1.48689382628143e-06\\
2570.278148658	1.48067707409337e-06\\
2570.55289572367	1.47447356738497e-06\\
2570.82764278933	1.46828334418482e-06\\
2571.102389855	1.46210644223626e-06\\
2571.37713692066	1.45594289899558e-06\\
2571.65188398633	1.44979275163031e-06\\
2571.92663105199	1.44365603701744e-06\\
2572.20137811766	1.43753279174181e-06\\
2572.47612518332	1.43142305209432e-06\\
2572.75087224899	1.42532685407039e-06\\
2573.02561931465	1.41924423336825e-06\\
2573.30036638032	1.41317522538738e-06\\
2573.57511344598	1.40711986522691e-06\\
2573.84986051165	1.4010781876841e-06\\
2574.12460757731	1.3950502272527e-06\\
2574.39935464298	1.38903601812158e-06\\
2574.67410170864	1.38303559417312e-06\\
2574.94884877431	1.37704898898186e-06\\
2575.22359583997	1.37107623581292e-06\\
2575.49834290564	1.36511736762072e-06\\
2575.7730899713	1.3591724170475e-06\\
2576.04783703697	1.35324141642198e-06\\
2576.32258410263	1.347324397758e-06\\
2576.5973311683	1.34142139275323e-06\\
2576.87207823396	1.33553243278784e-06\\
2577.14682529963	1.32965754892323e-06\\
2577.42157236529	1.3237967719008e-06\\
2577.69631943096	1.31795013214068e-06\\
2577.97106649662	1.31211765974058e-06\\
2578.24581356228	1.30629938447457e-06\\
2578.52056062795	1.30049533579194e-06\\
2578.79530769361	1.29470554281607e-06\\
2579.07005475928	1.28893003434331e-06\\
2579.34480182494	1.28316883884191e-06\\
2579.61954889061	1.27742198445098e-06\\
2579.89429595627	1.27168949897937e-06\\
2580.16904302194	1.26597140990478e-06\\
2580.4437900876	1.26026774437265e-06\\
2580.71853715327	1.25441544949124e-06\\
2580.99328421893	1.24874290905954e-06\\
2581.2680312846	1.24308484380368e-06\\
2581.54277835026	1.23744127985637e-06\\
2581.81752541593	1.23181224301012e-06\\
2582.09227248159	1.22619775871636e-06\\
2582.36701954726	1.22059785208471e-06\\
2582.64176661292	1.21501254788216e-06\\
2582.91651367859	1.20944187053233e-06\\
2583.19126074425	1.20388584411476e-06\\
2583.46600780992	1.19834449236426e-06\\
2583.74075487558	1.19281783867012e-06\\
2584.01550194125	1.18730590607561e-06\\
2584.29024900691	1.18180871727723e-06\\
2584.56499607258	1.17632629462414e-06\\
2584.83974313824	1.17085866011765e-06\\
2585.11449020391	1.16540583541054e-06\\
2585.38923726957	1.15996784180662e-06\\
2585.66398433524	1.15454470026022e-06\\
2585.9387314009	1.14913643137563e-06\\
2586.21347846656	1.14374305540668e-06\\
2586.48822553223	1.13836459225633e-06\\
2586.76297259789	1.13300106147618e-06\\
2587.03771966356	1.12765248226615e-06\\
2587.31246672922	1.12231887347404e-06\\
2587.58721379489	1.11700025359523e-06\\
2587.86196086055	1.1116966407723e-06\\
2588.13670792622	1.1064080527948e-06\\
2588.41145499188	1.10113450709888e-06\\
2588.68620205755	1.09587602076713e-06\\
2588.96094912321	1.09063261052823e-06\\
2589.23569618888	1.08540429275684e-06\\
2589.51044325454	1.08019108347331e-06\\
2589.78519032021	1.07499299834362e-06\\
2590.05993738587	1.06981005267912e-06\\
2590.33468445154	1.06464226143649e-06\\
2590.6094315172	1.0594896392176e-06\\
2590.88417858287	1.0543522002694e-06\\
2591.15892564853	1.04922995848394e-06\\
2591.4336727142	1.04412292739823e-06\\
2591.70841977986	1.03903112019433e-06\\
2591.98316684553	1.03379074129098e-06\\
2592.25791391119	1.02873162710074e-06\\
2592.53266097686	1.02368774630045e-06\\
2592.80740804252	1.01865911097907e-06\\
2593.08215510819	1.0136457328673e-06\\
2593.35690217385	1.00864762333778e-06\\
2593.63164923952	1.00366479340523e-06\\
2593.90639630518	9.98697253726657e-07\\
2594.18114337085	9.93745014601507e-07\\
2594.45589043651	9.88808085971984e-07\\
2594.73063750218	9.83886477423199e-07\\
2595.00538456784	9.78980198183548e-07\\
2595.2801316335	9.74089257124883e-07\\
2595.55487869917	9.69213662762919e-07\\
2595.82962576483	9.64353423257494e-07\\
2596.1043728305	9.59508546412972e-07\\
2596.37911989616	9.54679039678591e-07\\
2596.65386696183	9.49864910148827e-07\\
2596.92861402749	9.44902615851439e-07\\
2597.20336109316	9.40121464439072e-07\\
2597.47810815882	9.35355681520896e-07\\
2597.75285522449	9.30605273102957e-07\\
2598.02760229015	9.25870244835402e-07\\
2598.30234935582	9.21150602012906e-07\\
2598.57709642148	9.16446349575328e-07\\
2598.85184348715	9.1175749210821e-07\\
2599.12659055281	9.07084033843435e-07\\
2599.40133761848	9.02425978659803e-07\\
2599.67608468414	8.97783330083681e-07\\
2599.95083174981	8.93156091289621e-07\\
2600.22557881547	8.88544265101087e-07\\
2600.50032588114	8.83947853991073e-07\\
2600.7750729468	8.79366860082867e-07\\
2601.04982001247	8.74801285150742e-07\\
2601.32456707813	8.70251130620708e-07\\
2601.5993141438	8.65716397571244e-07\\
2601.87406120946	8.61197086734125e-07\\
2602.14880827513	8.56693198495151e-07\\
2602.42355534079	8.52204732895015e-07\\
2602.69830240646	8.47731689630104e-07\\
2602.97304947212	8.43274068053321e-07\\
2603.24779653778	8.38831867174999e-07\\
2603.52254360345	8.34405085663722e-07\\
2603.79729066911	8.29993721847263e-07\\
2604.07203773478	8.25597773713475e-07\\
2604.34678480044	8.21217238911228e-07\\
2604.62153186611	8.16852114751322e-07\\
2604.89627893177	8.125023982075e-07\\
2605.17102599744	8.08168085917365e-07\\
2605.4457730631	8.03849174183421e-07\\
2605.72052012877	7.99545658974054e-07\\
2605.99526719443	7.95257535924564e-07\\
2606.2700142601	7.90984800338179e-07\\
2606.54476132576	7.86562998169816e-07\\
2606.81950839143	7.82323237315612e-07\\
2607.09425545709	7.78098819849792e-07\\
2607.36900252276	7.73889740085698e-07\\
2607.64374958842	7.69695992006601e-07\\
2607.91849665409	7.65517569266796e-07\\
2608.19324371975	7.61354465192831e-07\\
2608.46799078542	7.57206672784642e-07\\
2608.74273785108	7.53074184716802e-07\\
2609.01748491675	7.48793899234368e-07\\
2609.29223198241	7.44694194670267e-07\\
2609.56697904808	7.40609742727696e-07\\
2609.84172611374	7.36540535117508e-07\\
2610.11647317941	7.32486563227941e-07\\
2610.39122024507	7.28447818125906e-07\\
2610.66596731074	7.24424290558364e-07\\
2610.9407143764	7.20415970953609e-07\\
2611.21546144207	7.16422849422678e-07\\
2611.49020850773	7.12280562167423e-07\\
2611.7649555734	7.08320019868336e-07\\
2612.03970263906	7.04374616097458e-07\\
2612.31444970472	7.00444340037949e-07\\
2612.58919677039	6.96529180559132e-07\\
2612.86394383605	6.92629126218001e-07\\
2613.13869090172	6.88744165260628e-07\\
2613.41343796738	6.84874285623722e-07\\
2613.68818503305	6.81019474936029e-07\\
2613.96293209871	6.77179720519904e-07\\
2614.23767916438	6.73355009392774e-07\\
2614.51242623004	6.69545328268693e-07\\
2614.78717329571	6.65750663559856e-07\\
2615.06192036137	6.61971001378153e-07\\
2615.33666742704	6.58206327536691e-07\\
2615.6114144927	6.54456627551407e-07\\
2615.88616155837	6.50721886642583e-07\\
2616.16090862403	6.47002089736476e-07\\
2616.4356556897	6.43297221466888e-07\\
2616.71040275536	6.3944318637391e-07\\
2616.98514982103	6.35770338670668e-07\\
2617.25989688669	6.32112343818435e-07\\
2617.53464395236	6.28469185624355e-07\\
2617.80939101802	6.24840847607735e-07\\
2618.08413808369	6.21227313001735e-07\\
2618.35888514935	6.17628564755025e-07\\
2618.63363221502	6.14044585533525e-07\\
2618.90837928068	6.10475357722054e-07\\
2619.18312634635	6.06920863426084e-07\\
2619.45787341201	6.03381084473427e-07\\
2619.73262047768	5.99856002415977e-07\\
2620.00736754334	5.96345598531411e-07\\
2620.28211460901	5.92849853824975e-07\\
2620.55686167467	5.89368749031177e-07\\
2620.83160874034	5.85902264615582e-07\\
2621.106355806	5.82450380776549e-07\\
2621.38110287166	5.79013077447005e-07\\
2621.65584993733	5.75590334296198e-07\\
2621.93059700299	5.72182130731513e-07\\
2622.20534406866	5.68788445900209e-07\\
2622.48009113432	5.65409258691256e-07\\
2622.75483819999	5.61881064735592e-07\\
2623.02958526565	5.58533011424932e-07\\
2623.30433233132	5.55199363011679e-07\\
2623.57907939698	5.51880097695698e-07\\
2623.85382646265	5.48575193424141e-07\\
2624.12857352831	5.4528462789328e-07\\
2624.40332059398	5.42008378550379e-07\\
2624.67806765964	5.38746422595537e-07\\
2624.95281472531	5.35498736983596e-07\\
2625.22756179097	5.32265298425967e-07\\
2625.50230885664	5.2904608339257e-07\\
2625.7770559223	5.25841068113647e-07\\
2626.05180298797	5.22650228581709e-07\\
2626.32655005363	5.19473540553376e-07\\
2626.6012971193	5.16310979551321e-07\\
2626.87604418496	5.12999135010255e-07\\
2627.15079125063	5.09866955485307e-07\\
2627.42553831629	5.06748800324161e-07\\
2627.70028538196	5.03644644458588e-07\\
2627.97503244762	5.00554462592426e-07\\
2628.24977951329	4.97478229203567e-07\\
2628.52452657895	4.94415918545884e-07\\
2628.79927364462	4.91367504651171e-07\\
2629.07402071028	4.88332961331125e-07\\
2629.34876777594	4.85147890884934e-07\\
2629.62351484161	4.82143223515709e-07\\
2629.89826190727	4.79152318863789e-07\\
2630.17300897294	4.76175150207513e-07\\
2630.4477560386	4.7321169061139e-07\\
2630.72250310427	4.70261912928152e-07\\
2630.99725016993	4.67325789800707e-07\\
2631.2719972356	4.64403293664175e-07\\
2631.54674430126	4.61494396747841e-07\\
2631.82149136693	4.58599071077212e-07\\
2632.09623843259	4.5571728847595e-07\\
2632.37098549826	4.52849020567926e-07\\
2632.64573256392	4.49994238779176e-07\\
2632.92047962959	4.4715291433994e-07\\
2633.19522669525	4.44325018286637e-07\\
2633.46997376092	4.41510521463877e-07\\
2633.74472082658	4.38709394526439e-07\\
2634.01946789225	4.35757734857464e-07\\
2634.29421495791	4.32985466847097e-07\\
2634.56896202358	4.30226451649884e-07\\
2634.84370908924	4.27480659508095e-07\\
2635.11845615491	4.24748060480697e-07\\
2635.39320322057	4.22028624445413e-07\\
2635.66795028624	4.19322321100716e-07\\
2635.9426973519	4.16629119967886e-07\\
2636.21744441757	4.13948990393009e-07\\
2636.49219148323	4.11281901549018e-07\\
2636.7669385489	4.08627822437685e-07\\
2637.04168561456	4.0598672189168e-07\\
2637.31643268023	4.03358568576545e-07\\
2637.59117974589	4.00743330992745e-07\\
2637.86592681156	3.98140977477657e-07\\
2638.14067387722	3.95551476207591e-07\\
2638.41542094288	3.92974795199784e-07\\
2638.69016800855	3.90410902314427e-07\\
2638.96491507421	3.87859765256637e-07\\
2639.23966213988	3.85321351578501e-07\\
2639.51440920554	3.82631468366786e-07\\
2639.78915627121	3.80120615073066e-07\\
2640.06390333687	3.77622358956711e-07\\
2640.33865040254	3.75136667304707e-07\\
2640.6133974682	3.72663507257447e-07\\
2640.88814453387	3.7020284581078e-07\\
2641.16289159953	3.67754649817999e-07\\
2641.4376386652	3.65318885991857e-07\\
2641.71238573086	3.62730693613854e-07\\
2641.98713279653	3.60321913705363e-07\\
2642.26187986219	3.57925437219824e-07\\
2642.53662692786	3.55541230656019e-07\\
2642.81137399352	3.53169260378676e-07\\
2643.08612105919	3.50809492620461e-07\\
2643.36086812485	3.48461893484023e-07\\
2643.63561519052	3.46126428943956e-07\\
2643.91036225618	3.43803064848841e-07\\
2644.18510932185	3.41491766923191e-07\\
2644.45985638751	3.39192500769481e-07\\
2644.73460345318	3.36905231870097e-07\\
2645.00935051884	3.34629925589344e-07\\
2645.28409758451	3.3236654717538e-07\\
2645.55884465017	3.30115061762235e-07\\
2645.83359171584	3.27875434371713e-07\\
2646.1083387815	3.25647629915392e-07\\
2646.38308584717	3.23431613196544e-07\\
2646.65783291283	3.21227348912102e-07\\
2646.9325799785	3.19034801654582e-07\\
2647.20732704416	3.16853935914023e-07\\
2647.48207410982	3.14684716079899e-07\\
2647.75682117549	3.12527106443061e-07\\
2648.03156824115	3.10381071197625e-07\\
2648.30631530682	3.08246574442901e-07\\
2648.58106237248	3.06123580185285e-07\\
2648.85580943815	3.0401205234014e-07\\
2649.13055650381	3.01911954733711e-07\\
2649.40530356948	2.99823251104975e-07\\
2649.68005063514	2.97745905107541e-07\\
2649.95479770081	2.95679880311505e-07\\
2650.22954476647	2.93625140205314e-07\\
2650.50429183214	2.91581648197611e-07\\
2650.7790388978	2.89549367619099e-07\\
2651.05378596347	2.8752826172436e-07\\
2651.32853302913	2.85518293693707e-07\\
2651.6032800948	2.83519426634997e-07\\
2651.87802716046	2.81531623585457e-07\\
2652.15277422613	2.7955484751348e-07\\
2652.42752129179	2.77589061320451e-07\\
2652.70226835746	2.7563422784252e-07\\
2652.97701542312	2.73690309852409e-07\\
2653.25176248879	2.71757270061182e-07\\
2653.52650955445	2.69835071120026e-07\\
2653.80125662012	2.67923675622001e-07\\
2654.07600368578	2.66023046103821e-07\\
2654.35075075145	2.6413314504758e-07\\
2654.62549781711	2.62253934882513e-07\\
2654.90024488278	2.60385377986725e-07\\
2655.17499194844	2.58363340899806e-07\\
2655.4497390141	2.56518188236728e-07\\
2655.72448607977	2.54683547737086e-07\\
2655.99923314543	2.5285938191873e-07\\
2656.2739802111	2.51045653253454e-07\\
2656.54872727676	2.49242324168724e-07\\
2656.82347434243	2.47449357049372e-07\\
2657.09822140809	2.4566671423932e-07\\
2657.37296847376	2.43894358043241e-07\\
2657.64771553942	2.42132250728279e-07\\
2657.92246260509	2.40380354525684e-07\\
2658.19720967075	2.38638631632509e-07\\
2658.47195673642	2.36907044213242e-07\\
2658.74670380208	2.35185554401479e-07\\
2659.02145086775	2.33474124301532e-07\\
2659.29619793341	2.31772715990095e-07\\
2659.57094499908	2.30081291517828e-07\\
2659.84569206474	2.28399812910998e-07\\
2660.12043913041	2.26728242173059e-07\\
2660.39518619607	2.25066541286267e-07\\
2660.66993326174	2.23414672213243e-07\\
2660.9446803274	2.21772596898565e-07\\
2661.21942739307	2.2014027727033e-07\\
2661.49417445873	2.18517675241698e-07\\
2661.7689215244	2.1690475271246e-07\\
2662.04366859006	2.15301471570551e-07\\
2662.31841565573	2.13707793693615e-07\\
2662.59316272139	2.12123680950479e-07\\
2662.86790978706	2.10549095202702e-07\\
2663.14265685272	2.08983998306048e-07\\
2663.41740391838	2.07428352111998e-07\\
2663.69215098405	2.05882118469208e-07\\
2663.96689804971	2.04345259225014e-07\\
2664.24164511538	2.02817736226853e-07\\
2664.51639218104	2.01299511323753e-07\\
2664.79113924671	1.99790546367747e-07\\
2665.06588631237	1.98290803215328e-07\\
2665.34063337804	1.96800243728862e-07\\
2665.6153804437	1.95318829778005e-07\\
2665.89012750937	1.93846523241099e-07\\
2666.16487457503	1.92383286006583e-07\\
2666.4396216407	1.90929079974358e-07\\
2666.71436870636	1.89483867057183e-07\\
2666.98911577203	1.88047609182024e-07\\
2667.26386283769	1.86620268291413e-07\\
2667.53860990336	1.85201806344801e-07\\
2667.81335696902	1.83792185319882e-07\\
2668.08810403469	1.8239136721393e-07\\
2668.36285110035	1.80999314045108e-07\\
2668.63759816602	1.79615987853779e-07\\
2668.91234523168	1.78241350703793e-07\\
2669.18709229735	1.76875364683788e-07\\
2669.46183936301	1.75517991908451e-07\\
2669.73658642868	1.74169194519795e-07\\
2670.01133349434	1.72828934688414e-07\\
2670.28608056001	1.71497174614726e-07\\
2670.56082762567	1.70173876530209e-07\\
2670.83557469134	1.68859002698637e-07\\
2671.110321757	1.6755251541728e-07\\
2671.38506882267	1.66254377018128e-07\\
2671.65981588833	1.6496454986908e-07\\
2671.934562954	1.6368299637513e-07\\
2672.20931001966	1.62409678979542e-07\\
2672.48405708532	1.61144560165026e-07\\
2672.75880415099	1.59887602454881e-07\\
2673.03355121665	1.58638768414156e-07\\
2673.30829828232	1.57398020650776e-07\\
2673.58304534798	1.56165321816679e-07\\
2673.85779241365	1.54940634608917e-07\\
2674.13253947931	1.53723921770784e-07\\
2674.40728654498	1.52515146092891e-07\\
2674.68203361064	1.51314270414271e-07\\
2674.95678067631	1.50121257623449e-07\\
2675.23152774197	1.48936070659501e-07\\
2675.50627480764	1.47758672513128e-07\\
2675.7810218733	1.46589026227682e-07\\
2676.05576893897	1.45427094900228e-07\\
2676.33051600463	1.44272841682546e-07\\
2676.6052630703	1.43126229782167e-07\\
2676.88001013596	1.41987222463365e-07\\
2677.15475720163	1.40855783048173e-07\\
2677.42950426729	1.39731874917346e-07\\
2677.70425133296	1.38615461511367e-07\\
2677.97899839862	1.37506506331387e-07\\
2678.25374546429	1.36404972940203e-07\\
2678.52849252995	1.35310824963191e-07\\
2678.80323959562	1.34224026089258e-07\\
2679.07798666128	1.33144540071759e-07\\
2679.35273372695	1.32072330729422e-07\\
2679.62748079261	1.31007361947247e-07\\
2679.90222785828	1.29949597677422e-07\\
2680.17697492394	1.2889900194019e-07\\
2680.45172198961	1.27855538824748e-07\\
2680.72646905527	1.26819172490104e-07\\
2681.00121612094	1.25789867165938e-07\\
2681.2759631866	1.2476758715346e-07\\
2681.55071025226	1.23752296826238e-07\\
2681.82545731793	1.2274396063104e-07\\
2682.10020438359	1.21742543088644e-07\\
2682.37495144926	1.20748008794662e-07\\
2682.64969851492	1.19760322420329e-07\\
2682.92444558059	1.18779448713302e-07\\
2683.19919264625	1.1780535249844e-07\\
2683.47393971192	1.16837998678579e-07\\
2683.74868677758	1.15877352235295e-07\\
2684.02343384325	1.14923378229655e-07\\
2684.29818090891	1.13811531671806e-07\\
2684.57292797458	1.12873014037672e-07\\
2684.84767504024	1.11941036496942e-07\\
2685.12242210591	1.11015564763749e-07\\
2685.39716917157	1.10096564631869e-07\\
2685.67191623724	1.09184001975459e-07\\
2685.9466633029	1.08277842749765e-07\\
2686.22141036857	1.07378052991847e-07\\
2686.49615743423	1.06484598821266e-07\\
2686.7709044999	1.05597446440787e-07\\
2687.04565156556	1.04716562137053e-07\\
2687.32039863123	1.03841912281258e-07\\
2687.59514569689	1.02973463329817e-07\\
2687.86989276256	1.02111181825009e-07\\
2688.14463982822	1.0125503439563e-07\\
2688.41938689389	1.00404987757626e-07\\
2688.69413395955	9.95610087147168e-08\\
2688.96888102522	9.87230641590123e-08\\
2689.24362809088	9.78911210716251e-08\\
2689.51837515655	9.70651465232587e-08\\
2689.79312222221	9.62451076748124e-08\\
2690.06786928788	9.54309717779426e-08\\
2690.34261635354	9.46227061756497e-08\\
2690.6173634192	9.38202783028281e-08\\
2690.89211048487	9.30236556868298e-08\\
2691.16685755053	9.22328059479966e-08\\
2691.4416046162	9.1447696800211e-08\\
2691.71635168186	9.06682960514039e-08\\
2691.99109874753	8.9894571604088e-08\\
2692.26584581319	8.91264914558545e-08\\
2692.54059287886	8.83640236998831e-08\\
2692.81533994452	8.76071365254234e-08\\
2693.09008701019	8.68557982182856e-08\\
2693.36483407585	8.61099771613109e-08\\
2693.63958114152	8.53696418348327e-08\\
2693.91432820718	8.46347608171423e-08\\
2694.18907527285	8.39053027849252e-08\\
2694.46382233851	8.31812365137159e-08\\
2694.73856940418	8.24625308783119e-08\\
2695.01331646984	8.17491548532109e-08\\
2695.28806353551	8.10410775130143e-08\\
2695.56281060117	8.03382680328425e-08\\
2695.83755766684	7.9640695688723e-08\\
2696.1123047325	7.89483298579926e-08\\
2696.38705179817	7.82611400196634e-08\\
2696.66179886383	7.75790957548095e-08\\
2696.9365459295	7.69021667469216e-08\\
2697.21129299516	7.6230322782272e-08\\
2697.48604006083	7.53999603367013e-08\\
2697.76078712649	7.47404003839674e-08\\
2698.03553419216	7.40858075800664e-08\\
2698.31028125782	7.34361524520699e-08\\
2698.58502832348	7.27914056280289e-08\\
2698.85977538915	7.21515378373227e-08\\
2699.13452245481	7.15165199109889e-08\\
2699.40926952048	7.08863227820459e-08\\
2699.68401658614	7.02609174858165e-08\\
2699.95876365181	6.96402751602343e-08\\
2700.23351071747	6.90243670461431e-08\\
2700.50825778314	6.8413164487596e-08\\
2700.7830048488	6.78066389321398e-08\\
2701.05775191447	6.72047619310899e-08\\
2701.33249898013	6.66075051398083e-08\\
2701.6072460458	6.60148403179596e-08\\
2701.88199311146	6.54267393297732e-08\\
2702.15674017713	6.48431741442881e-08\\
2702.43148724279	6.42641168355956e-08\\
2702.70623430846	6.36895395830692e-08\\
2702.98098137412	6.3119414671597e-08\\
2703.25572843979	6.25537144917935e-08\\
2703.53047550545	6.19924115402178e-08\\
2703.80522257112	6.14354784195737e-08\\
2704.07996963678	6.08828878389095e-08\\
2704.35471670245	6.0334612613804e-08\\
2704.62946376811	5.97906256665561e-08\\
2704.90421083378	5.92509000263537e-08\\
2705.17895789944	5.87154088294491e-08\\
2705.45370496511	5.81841253193175e-08\\
2705.72845203077	5.76570228468141e-08\\
2706.00319909644	5.71340748703192e-08\\
2706.2779461621	5.66152549558861e-08\\
2706.55269322777	5.61005367773686e-08\\
2706.82744029343	5.55898941165563e-08\\
2707.10218735909	5.50833008632929e-08\\
2707.37693442476	5.45807310155895e-08\\
2707.65168149042	5.40821586797384e-08\\
2707.92642855609	5.35875580704083e-08\\
2708.20117562175	5.30969035107499e-08\\
2708.47592268742	5.26101694324731e-08\\
2708.75066975308	5.21273303759408e-08\\
2709.02541681875	5.16483609902377e-08\\
2709.30016388441	5.11732360332473e-08\\
2709.57491095008	5.07019303717094e-08\\
2709.84965801574	5.02344189812872e-08\\
2710.12440508141	4.97706769466093e-08\\
2710.39915214707	4.93106794613242e-08\\
2710.67389921274	4.88544018281343e-08\\
2710.9486462784	4.84018194588357e-08\\
2711.22339334407	4.795290787434e-08\\
2711.49814040973	4.75076427047054e-08\\
2711.7728874754	4.70659996891437e-08\\
2712.04763454106	4.66279546760393e-08\\
2712.32238160673	4.61934836229484e-08\\
2712.59712867239	4.5762562596604e-08\\
2712.87187573806	4.53351677729071e-08\\
2713.14662280372	4.49112754369137e-08\\
2713.42136986939	4.44908619828226e-08\\
2713.69611693505	4.40739039139475e-08\\
2713.97086400072	4.36603778426948e-08\\
2714.24561106638	4.32502604905243e-08\\
2714.52035813205	4.28435286879191e-08\\
2714.79510519771	4.24401593743325e-08\\
2715.06985226338	4.20401295981471e-08\\
2715.34459932904	4.16434165166153e-08\\
2715.61934639471	4.12499973958045e-08\\
2715.89409346037	4.08598496105307e-08\\
2716.16884052603	4.04729506442907e-08\\
2716.4435875917	4.00892780891854e-08\\
2716.71833465736	3.97088096458457e-08\\
2716.99308172303	3.93315231233434e-08\\
2717.26782878869	3.89573964391074e-08\\
2717.54257585436	3.85864076188277e-08\\
2717.81732292002	3.82185347963589e-08\\
2718.09206998569	3.7853756213615e-08\\
2718.36681705135	3.74920502204665e-08\\
2718.64156411702	3.71333952746241e-08\\
2718.91631118268	3.67777699415259e-08\\
2719.19105824835	3.6425152894216e-08\\
2719.46580531401	3.60755229132165e-08\\
2719.74055237968	3.57288588864029e-08\\
2720.01529944534	3.53851398088653e-08\\
2720.29004651101	3.50443447827747e-08\\
2720.56479357667	3.47064530172379e-08\\
2720.83954064234	3.43714438281528e-08\\
2721.114287708	3.4039296638055e-08\\
2721.38903477367	3.37099909759678e-08\\
2721.66378183933	3.33835064772381e-08\\
2721.938528905	3.30598228833787e-08\\
2722.21327597066	3.27389200418987e-08\\
2722.48802303633	3.24207779061343e-08\\
2722.76277010199	3.21053765350726e-08\\
2723.03751716766	3.17926960931771e-08\\
2723.31226423332	3.14827168502009e-08\\
2723.58701129899	3.10119175178768e-08\\
2723.86175836465	3.0709485149092e-08\\
2724.13650543032	3.04096675556456e-08\\
2724.41125249598	3.0112445752732e-08\\
2724.68599956164	2.98178008564172e-08\\
2724.96074662731	2.95257140834633e-08\\
2725.23549369297	2.92361667511572e-08\\
2725.51024075864	2.89491402771297e-08\\
2725.7849878243	2.86646161791725e-08\\
2726.05973488997	2.83825760750549e-08\\
2726.33448195563	2.81030016823323e-08\\
2726.6092290213	2.78258748181547e-08\\
2726.88397608696	2.75511773990686e-08\\
2727.15872315263	2.72788914408208e-08\\
2727.43347021829	2.70089990581516e-08\\
2727.70821728396	2.67414824645927e-08\\
2727.98296434962	2.64763239722535e-08\\
2728.25771141529	2.62135059916142e-08\\
2728.53245848095	2.5953011031304e-08\\
2728.80720554662	2.56948216978875e-08\\
2729.08195261228	2.54389206956391e-08\\
2729.35669967795	2.51852908263215e-08\\
2729.63144674361	2.49339149889558e-08\\
2729.90619380928	2.46847761795923e-08\\
2730.18094087494	2.44378574910749e-08\\
2730.45568794061	2.41931421128079e-08\\
2730.73043500627	2.39506133305135e-08\\
2731.00518207194	2.37102545259932e-08\\
2731.2799291376	2.34720491768813e-08\\
2731.55467620327	2.32359808563973e-08\\
2731.82942326893	2.30020332330977e-08\\
2732.1041703346	2.27701900706233e-08\\
2732.37891740026	2.25404352274421e-08\\
2732.65366446593	2.23127526565951e-08\\
2732.92841153159	2.20871264054347e-08\\
2733.20315859725	2.18635406153614e-08\\
2733.47790566292	2.1641979521562e-08\\
2733.75265272858	2.1422427452739e-08\\
2734.02739979425	2.12048688308448e-08\\
2734.30214685991	2.0989288170808e-08\\
2734.57689392558	2.07756700802603e-08\\
2734.85164099124	2.05639992592598e-08\\
2735.12638805691	2.03542605000147e-08\\
2735.40113512257	2.01464386866008e-08\\
2735.67588218824	1.99405187946813e-08\\
2735.9506292539	1.97364858912216e-08\\
2736.22537631957	1.95343251342034e-08\\
2736.50012338523	1.93340217723352e-08\\
2736.7748704509	1.91355611447649e-08\\
2737.04961751656	1.89389286807844e-08\\
2737.32436458223	1.87441098995401e-08\\
2737.59911164789	1.85510904097325e-08\\
2737.87385871356	1.83598559093229e-08\\
2738.14860577922	1.81703921852315e-08\\
2738.42335284489	1.79826851130387e-08\\
2738.69809991055	1.77967206566811e-08\\
2738.97284697622	1.76124848681488e-08\\
2739.24759404188	1.74299638871791e-08\\
2739.52234110755	1.72491439409487e-08\\
2739.79708817321	1.69056888303417e-08\\
2740.07183523888	1.67304436040746e-08\\
2740.34658230454	1.65568306315551e-08\\
2740.62132937021	1.63848368140683e-08\\
2740.89607643587	1.62144491353124e-08\\
2741.17082350154	1.60456546611127e-08\\
2741.4455705672	1.58784405391391e-08\\
2741.72031763287	1.57127939986161e-08\\
2741.99506469853	1.55487023500387e-08\\
2742.26981176419	1.53861529848787e-08\\
2742.54455882986	1.52251333752967e-08\\
2742.81930589552	1.50656310738479e-08\\
2743.09405296119	1.490763371319e-08\\
2743.36880002685	1.47511290057862e-08\\
2743.64354709252	1.45961047436118e-08\\
2743.91829415818	1.44425487978532e-08\\
2744.19304122385	1.42904491186122e-08\\
2744.46778828951	1.41397937346036e-08\\
2744.74253535518	1.39905707528566e-08\\
2745.01728242084	1.3842768358411e-08\\
2745.29202948651	1.36963748140139e-08\\
2745.56677655217	1.35513784598167e-08\\
2745.84152361784	1.34077677130681e-08\\
2746.1162706835	1.32655310678097e-08\\
2746.39101774917	1.31246570945665e-08\\
2746.66576481483	1.29851344400416e-08\\
2746.9405118805	1.28469518268037e-08\\
2747.21525894616	1.27100980529803e-08\\
2747.49000601183	1.25745619919444e-08\\
2747.76475307749	1.24403325920047e-08\\
2748.03950014316	1.23073988760925e-08\\
2748.31424720882	1.21757499414484e-08\\
2748.58899427449	1.20453749593084e-08\\
2748.86374134015	1.191626317459e-08\\
2749.13848840582	1.17884039055759e-08\\
2749.41323547148	1.16617865435993e-08\\
2749.68798253715	1.15364005527266e-08\\
2749.96272960281	1.14122354694407e-08\\
2750.23747666848	1.12892809023228e-08\\
2750.51222373414	1.11675265317353e-08\\
2750.7869707998	1.10469621095016e-08\\
2751.06171786547	1.09275774585886e-08\\
2751.33646493113	1.08093624727862e-08\\
2751.6112119968	1.06923071163865e-08\\
};
\addplot[fill=mycolor2,fill opacity=0.5,draw=black,ybar interval,area legend] plot table[row sep=crcr] 
\end{subfigure}
\caption{Density of $\Phi(Y)$.}
\label{fig:densities}
\end{figure}

\begin{table}[t]
	\centering
	\begin{tabular}{cccc}
		\toprule
		$\frac{1}{h_{\max} - h_{\min}} \int_{h_{\min}}^{h_{\max}} \Phi(U_N(s))\dd s$ & $\int_{\R^3} \Phi(y)\dd \mu^{h_{\max}}(y)$ & $\frac{1}{M} \sum_{i=1}^M \Phi(U_N^{\xi^{(i)}}(h))$ & $\int_{\R^3} \Phi(y)\dd \nu^{h,M}(y)$ \\ 
		\midrule
		769.9 & 759.0 & 763.1 & 763.3 \\
		\bottomrule
	\end{tabular}
	\caption{Numerical results.}
	\label{tab:numResults}
\end{table}

\end{document}