\documentclass{siamart1116}

% basics
\usepackage[left=3cm,right=3cm,top=2.5cm,bottom=2.5cm]{geometry}
\usepackage[utf8x]{inputenc}
\usepackage[title,titletoc]{appendix}
\usepackage{afterpage}
\usepackage{enumitem}   
\setlist[enumerate]{topsep=3pt,itemsep=3pt,label=(\roman*)}

% maths
\usepackage{mathtools}
\usepackage{amsmath}
\usepackage{amssymb}
\newsiamremark{assumption}{Assumption}
\newsiamremark{remark}{Remark}
\newsiamremark{example}{Example}
\numberwithin{theorem}{section}

% tables
\usepackage{booktabs}

% plots
\usepackage{graphicx}
\usepackage{pgfplots}
\usepackage{tikz}
\usetikzlibrary{arrows,decorations.pathmorphing,backgrounds,positioning,fit,matrix}
\usepackage[labelfont=bf]{caption}
\setlength{\belowcaptionskip}{-5pt}
\usepackage{here}
\usepackage[font=normal]{subcaption}

% title and authors
%\newcommand{\TheTitle}{Probabilistic numerical methods with random time steps for chaotic and geometric integration} 
\newcommand{\TheTitle}{Report -- A posteriori error estimator for ODE} 
\newcommand{\TheAuthors}{A. Abdulle, G. Garegnani}
%\headers{Probabilistic Runge-Kutta method based on random time steps}{\TheAuthors}
\headers{}{\TheAuthors}
\title{{\TheTitle}}
\author{Assyr Abdulle\thanks{Institute of Mathematics, \'Ecole Polytechnique F\'ed\'erale de Lausanne (\email{assyr.abdulle@epfl.ch})}
		\and
		Giacomo Garegnani\thanks{Institute of Mathematics, \'Ecole Polytechnique F\'ed\'erale de Lausanne (\email{giacomo.garegnani@epfl.ch})}}

% my commands 
\DeclarePairedDelimiter{\ceil}{\left\lceil}{\right\rceil}
\DeclarePairedDelimiter{\floor}{\lfloor}{\rfloor}
\DeclarePairedDelimiter{\abs}{\lvert}{\rvert}
\DeclarePairedDelimiter{\norm}{\|}{\|}
\renewcommand{\phi}{\varphi}
\renewcommand{\theta}{\vartheta}
\renewcommand{\Pr}{\mathbb{P}}
\newcommand{\eqtext}[1]{\ensuremath{\stackrel{#1}{=}}}
\newcommand{\leqtext}[1]{\ensuremath{\stackrel{#1}{\leq}}}
\newcommand{\iid}{\ensuremath{\stackrel{\text{i.i.d.}}{\sim}}}
\newcommand{\totext}[1]{\ensuremath{\stackrel{#1}{\to}}}
\newcommand{\rightarrowtext}[1]{\ensuremath{\stackrel{#1}{\longrightarrow}}}
\newcommand{\leftrightarrowtext}[1]{\ensuremath{\stackrel{#1}{\longleftrightarrow}}}
\newcommand{\pdv}[2]{\ensuremath\partial_{#2}#1}
\newcommand{\N}{\mathbb{N}}
\newcommand{\R}{\mathbb{R}}
\newcommand{\C}{\mathbb{C}}
\newcommand{\OO}{\mathcal{O}}
\newcommand{\epl}{\varepsilon}
\newcommand{\diffL}{\mathcal{L}}
\newcommand{\prior}{\mathcal{Q}}
\newcommand{\defeq}{\coloneqq}
\newcommand{\eqdef}{\eqqcolon}
\newcommand{\Var}{\operatorname{Var}}
\newcommand{\E}{\operatorname{\mathbb{E}}}
\newcommand{\MSE}{\operatorname{MSE}}
\newcommand{\trace}{\operatorname{tr}}
\newcommand{\MH}{\mathrm{MH}}
\newcommand{\ttt}{\texttt}
\newcommand{\Hell}{d_{\mathrm{Hell}}}
\newcommand{\sksum}{{\textstyle\sum}}
\newcommand{\dd}{\mathrm{d}}
\definecolor{shade}{RGB}{100, 100, 100}
\definecolor{bordeaux}{RGB}{128, 0, 50}
\newcommand{\corr}[1]{{\color{red}#1}}

\ifpdf
\hypersetup{
	pdftitle={\TheTitle},
	pdfauthor={\TheAuthors}
}
\fi

\begin{document}
\maketitle	

\section{Theory} Consider $f \colon \R^d \to \R^d$ and the ODE
\begin{equation}\label{eq:ODE}
	\begin{cases}
	y'(t) = f(y(t)), & t \in (0, T],\\
    y(0) = y_0,
	\end{cases}
\end{equation}
whose flow is denoted by $y(t) = \phi_t(y_0)$. Consider $N > 0$, $h = T / N$ and the additive noise numerical method
\begin{equation}\label{eq:AdditiveNoise}
	Y_{k+1} = \Psi_h(Y_k) + \xi_k(h), \quad k = 0, \ldots, N,
\end{equation}
where $\xi_k(h)$ satisfies
\begin{align}\label{eq:AdditiveNoise_Assumption}
	\E \xi_k(h) &= 0,  \\
	\E \xi_k(h)\xi_k(h)^\top &= Q h^{2p+1},  \quad \forall k = 1, \ldots, N,
\end{align}
for some symmetric positive definite matrix $Q$ and exponent $p \geq 1$. 

\begin{theorem}[Mean square order] Report the result with precise constants. 
\end{theorem}

\begin{proof} From the proof of \cite[Theorem 2.2]{CGS16}, denoting the exact solution by $y_k = y(kh)$, the global error by $e_k = Y_k - y_k$ and the local error by $\epl_k = \Psi_h(Y_k) - \phi_h(Y_k)$, we have
	\begin{equation*}
		\E\abs{e_{k+1}}^2 \leq \E\abs{\phi_h(y_k) - \phi_h(y_k - e_k) - \epl_k}^2 + \abs{Q}h^{2p+1}.
	\end{equation*}
	Hence, developing the square in the first term of the right hand side, considering that $\phi_h$ is Lipschitz with constant $1 + Lh$ and that $\E\abs{\epl_k}^2 \leq C_{\mathrm{loc}}^2 h^{2q + 2}$, we have
	\begin{equation*}
		\E\abs{e_{k+1}}^2 \leq (1 + Lh)^2 \E \abs{e_k}^2 + \E\abs{\big(h^{1/2}(\phi_h(y_k) - \phi_h(y_k - e_k)), h^{-1/2}\epl_k\big)} + C_{\mathrm{loc}}^2 h^{2q + 2} + \abs{Q}h^{2p+1}.
	\end{equation*}
	Applying Cauchy-Schwarz and Young inequalities on the inner product we get
	\begin{equation*}
		\E\abs{e_{k+1}}^2 \leq (1 + h)(1 + Lh)^2 \E \abs{e_k}^2 + (1 + h)C_{\mathrm{loc}}^2 h^{2q + 1} + \abs{Q}h^{2p+1}.
	\end{equation*}
	Setting $p = q$, we rewrite this bound as
	\begin{equation*}
		\E\abs{e_{k+1}}^2 \leq \Big(1 + \big(1 + 2L + (L^2 + 2L)h + L^2h^2\big)h\Big) \E \abs{e_k}^2 + \big((1 + h)C_{\mathrm{loc}}^2 + \abs{Q}\big)h^{2q + 1}.
	\end{equation*}
	Denoting by $C_{\mathrm{err}} = C_{\mathrm{err}}(L, h) = 1 + 2L + (L^2 + 2L)h + L^2h^2$, we get by Gronwall's inequality
	\begin{equation*}
		\E\abs{e_k}^2 \leq \frac{(1 + h)C_{\mathrm{loc}}^2 + \abs{Q}}{C_{\mathrm{err}}}\Big(e^{C_{\mathrm{err}}T} - 1\Big)h^{2q},
	\end{equation*}
	which proves the desired result.
\end{proof}

\begin{theorem}[Variance] The bound for the variance.
\end{theorem}

\begin{proof} Thanks to independence of $Y_k$ and $\xi_k$ we have
	\begin{equation*}
		\Var Y_{k+1} = \Var \Psi_h(Y_k) + \abs{Q} h^{2q+1}.
	\end{equation*}
	The Lipschitz constant of $\Psi_h$ being $1 + L_\Psi h$, we have
	\begin{equation*}
		\Var Y_{k+1} \leq (1 + L_\Psi h)^2 \Var Y_k + \abs{Q}h^{2q+1}.
	\end{equation*}
	Hence, denoting by $C_{\mathrm{var}} = C_{\mathrm{var}}(L, h) = 1 + (2L_\Psi + L_\Psi^2 h) h$, we obtain by Gronwall's inequality
	\begin{equation}
		\Var Y_k \leq \frac{\abs{Q}}{C_{\mathrm{var}}}\Big(e^{C_{\mathrm{var}}T} - 1\Big)h^{2q}.
 	\end{equation}
\end{proof}
\clearpage

\begin{figure}
	\centering
	\includegraphics[]{EmbeddedLorenz} \\
	\includegraphics[]{VarianceMSE} \\
	\includegraphics[]{VarianceErrorRatio}
	\caption{Lorenz system with tuning of $Q$. Here $h = 0.01$, $T = 50$, Heun as deterministic integrator.}
\end{figure}

\begin{figure}
	\centering
	\includegraphics[]{EmbeddedLorenz2} \\
	\includegraphics[]{VarianceMSE2} \\
	\includegraphics[]{VarianceErrorRatio2}
	\caption{Lorenz system without tuning of $Q$. Here $h = 0.01$, $T = 50$, Heun as deterministic integrator.}
\end{figure}
\begin{figure}
	\centering
	\includegraphics[]{EmbeddedVdPol} \\
	\includegraphics[]{VarianceMSEVdPol} \\
	\includegraphics[]{VarianceErrorRatioVdPol}
	\caption{Van der Pol system ($\epl = 1$) with tuning of $Q$. Here $h = 0.1$, $T = 400$, Heun as deterministic integrator.}
\end{figure}

\begin{figure}
	\centering
	\includegraphics[]{EmbeddedVdPol2} \\
	\includegraphics[]{VarianceMSEVdPol2} \\
	\includegraphics[]{VarianceErrorRatioVdPol2}
	\caption{Van der Pol system ($\epl = 1$) without tuning of $Q$. Here $h = 0.1$, $T = 400$, Heun as deterministic integrator.}
\end{figure}
\end{document}